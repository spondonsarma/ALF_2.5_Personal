% !TEX root = doc.tex
% Copyright (c) 2017-2020 The ALF project.
% This is a part of the ALF project documentation.
% The ALF project documentation by the ALF contributors is licensed
% under a Creative Commons Attribution-ShareAlike 4.0 International License.
% For the licensing details of the documentation see license.CCBYSA.
%
%-----------------------------------------------------------------------------------
\subsection{Predefined observables} \label{sec:predefined_observales}
%-----------------------------------------------------------------------------------

The types \texttt{Obser\_Vec} and \texttt{Obser\_Latt} described in Section~\ref{sec:obs} handles arrays of scalar observables and correlation functions with lattice symmetry respectively. The module \texttt{Predefined\_Obs} provides a set of standard equal-time and time-displaced observables, as described bellow.

The predefined measurements methods take as input Green functions \texttt{GR}, \texttt{GT0}, \texttt{G0T}, \texttt{G00}, and \texttt{GTT}, defined in Sec.~\ref{sec:EqualTimeobs} and~\ref{sec:TimeDispObs}, as well as \texttt{N\_SUN}, time slice \texttt{Ntau}, lattice information, and so on -- see Table~\ref{table:predefined_obs}.
%
\begin{table}[h]
	\begin{center}
		\begin{tabular}{@{} p{0.21\columnwidth}  p{0.1\columnwidth} p{0.06\columnwidth} p{0.55\columnwidth}  @{}}
			\toprule
			Argument                      & Type       & Role   & Description \\
			\midrule
			\texttt{Latt}                 & Lattice    & Input  & Lattice as a variable of type \texttt{Lattice}, see Sec.~\ref{sec:latt}\\
			\texttt{Latt\_Unit}           & Unit\_cell & Input  & Unit cell as a variable of type \texttt{Unit\_cell}, see Sec.~\ref{sec:latt}\\
			\texttt{List(Ndim,2)}         & Integer    & Input  & For every site index $\texttt{I}$, stores the corresponding lattice position, \texttt{List(I,1)}, and the (local) orbital index, \texttt{List(I,2)}\\
			\texttt{NT}                   & Integer    & Input  & Imaginary time\\
			\texttt{GR(Ndim,Ndim,N\_FL)}  & Complex    & Input  & Equal-time Green function $\texttt{GR(i,j,s)}  = \langle c^{\phantom{\dagger}}_{i,s} c^{\dagger}_{j,s}  \rangle$\\
			\texttt{GRC(Ndim,Ndim,N\_FL)} & Complex    & Input  & $\texttt{GRC(i,j,s)}  = \langle c^{\dagger}_{i,s} c^{\phantom{\dagger}}_{j,s}  \rangle  =  \delta_{i,j} - \texttt{GR(j,i,s)}$\\
			\texttt{GT0(Ndim,Ndim,N\_FL)} & Complex    & Input  & Time-displaced Green function $\langle \langle \mathcal{T} \hat{c}^{\phantom\dagger}_{i,s}(\tau) \hat{c}^{\dagger}_{j,s}(0) \rangle \rangle$\\
			\texttt{G0T(Ndim,Ndim,N\_FL)} & Complex    & Input  & Time-displaced Green function $\langle \langle \mathcal{T} \hat{c}^{\phantom\dagger}_{i,s}(0) \hat{c}^{\dagger}_{j,s}(\tau) \rangle \rangle $\\
			\texttt{G00(Ndim,Ndim,N\_FL)} & Complex    & Input  & Time-displaced Green function $\langle \langle \mathcal{T} \hat{c}^{\phantom\dagger}_{i,s}(0) \hat{c}^{\dagger}_{j,s}(0) \rangle \rangle $\\
			\texttt{GTT(Ndim,Ndim,N\_FL)} & Complex    & Input  & Time-displaced Green function $\langle \langle \mathcal{T} \hat{c}^{\phantom\dagger}_{i,s}(\tau) \hat{c}^{\dagger}_{j,s}(\tau) \rangle \rangle $\\
			\texttt{N\_SUN}               & Integer    & Input  & Number of fermion colors $N_{\mathrm{col}}$\\
			\texttt{ZS}                   & Complex    & Input  & $\texttt{ZS} = \text{sign}(C)$, see Sec.~\ref{sec:obs}\\
			\texttt{ZP}                   & Complex    & Input  & $\texttt{ZP} = e^{-S(C)}/\Re \left[e^{-S(C)} \right]$, see Sec.~\ref{sec:obs}\\
			\texttt{Obs}                  & Obser\_Latt& Output & One or more measurement result\\
			\bottomrule
		\end{tabular}
		\caption{Arguments taken by the subroutines in the module \texttt{Predefined\_Obs}. Note that a given method makes use of only a subset of this list, as specified in their calls described bellow.}		\label{table:predefined_obs}
	\end{center}
\end{table}


\subsubsection{Equal-time $SU(N)$ spin-spin correlations}

A measurement of $SU(N)$ spin-spin correlations can be obtained by:

\begin{lstlisting}[style=fortran]
Call Predefined_Obs_eq_SpinSUN_measure(Latt, Latt_unit, List,  GR, GRC, N_SUN, ZS, ZP,  Obs)
\end{lstlisting}

If \texttt{N\_FL = 1} then  this routine returns
\red{FFA wil do this}
\begin{align}
\texttt{Obs} = \frac{2N}{N^2-1}\sum_{a=1}^{N^2 - 1}  \left\langle c^{\dagger}_{\vec{i}} T^a c_{\vec{i}}  c^{\dagger}_{\vec{j}} T^a  c_{\vec{j}}\right\rangle,
\end{align}
where $T^a$ are the generators of $SU(N)$ satisfying the normalization condition:  $\Tr [ T^a  T^b ]= \delta_{a,b}/2$. Note that for $SU(N)$ symmetry:
\begin{align}
\sum_{a=1}{N^2 - 1 }  \left\langle c^{\dagger}_i T^a c_i  c^{\dagger}_j T^a  c_j\right\rangle   = \sum_{a} \Tr{T^a T^a} \langle c^{\dagger}_i c_j\rangle \langle c_i c^{\dagger}_j\rangle = \frac{N^2 -1}{2} \langle c^{\dagger}_i c_j\rangle \langle c_i c^{\dagger}_j\rangle.
\end{align}


\subsubsection{Equal-time spin correlations}

A measurement of the equal-time spin correlations can be obtained by:

\begin{lstlisting}[style=fortran]
Call Predefined_Obs_eq_SpinMz_measure(Latt, Latt_unit, List,  GR, GRC, N_SUN, ZS, ZP,
                                      ObsZ, ObsXY, ObsXYZ)
\end{lstlisting}

If \texttt{N\_FL=2} and \texttt{N\_SUN=1}, then the routine returns:
\begin{align}
\texttt{ObsZ}   &=  4 \langle c^{\dagger}_i S^z c_i  c^{\dagger}_j S^z  c_j\rangle \\  
\texttt{ObsXY}  &=  2 \left( \langle c^{\dagger}_i S^x c_i  c^{\dagger}_j S^x  c_j\rangle  +   \langle c^{\dagger}_i S^y c_i  c^{\dagger}_j S^y  c_j\rangle \right)\\
\texttt{ObsXYZ} &=  \frac{2\cdot\texttt{ObsXY} + \texttt{ObsZ}}{3}.
\end{align}


\subsubsection{Equal-time Green function}

A measurement of the qual-time Green function can be obtained by:

\begin{lstlisting}[style=fortran]
Call Predefined_Obs_eq_Green_measure(Latt, Latt_unit, List,  GR, GRC, N_SUN, ZS, ZP, Obs)
\end{lstlisting}

Which returns:
\begin{align}
\texttt{Obs} = \sum_{\sigma=1}^{N_\text{col}} \sum_{s=1}^{N_\text{fl}} \langle  c^{\dagger}_{i,\sigma,s} c_{j,\sigma,s} \rangle
\end{align}


\subsubsection{}

A measurement of  can be obtained by:

\begin{lstlisting}[style=fortran]
Call 
\end{lstlisting}

\subsubsection{}

A measurement of  can be obtained by:

\begin{lstlisting}[style=fortran]
Call 
\end{lstlisting}

\subsubsection{}

A measurement of  can be obtained by:

\begin{lstlisting}[style=fortran]
Call 
\end{lstlisting}

\subsubsection{}

A measurement of  can be obtained by:

\begin{lstlisting}[style=fortran]
Call 
\end{lstlisting}

\subsubsection{}

A measurement of  can be obtained by:

\begin{lstlisting}[style=fortran]
Call 
\end{lstlisting}
