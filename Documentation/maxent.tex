% Copyright (c) 2016-2019 The ALF project.
% This is a part of the ALF project documentation.
% The ALF project documentation by the ALF contributors is licensed
% under a Creative Commons Attribution-ShareAlike 4.0 International License.
% For the licensing details of the documentation see license.CCBYSA.

% !TEX root = doc.tex

\red{Include an introduction like the one from Beach~\cite{Beach04a}}

\subsection{General setup}
Generically, the maximum entropy code (MaxEnt) computes the  image  $A(\omega) $ for a given  data  set $g(\tau) $  and kernel $K(\tau,\omega) $:
\begin{equation}
g(\tau) =  \int_{\omega_\text{start}}^{\omega_\text{end}} d {\omega} K(\tau,\omega) A(\omega).
\end{equation} 
The  ALF package includes a standard implementation of the stochastic MaxEnt as formulated in the article of K. Beach~\cite{Beach04a}. Here we will comment on the workflow.  The module 
\texttt{Libraries/Modules/\allowbreak{}maxent\_stoch\_mod.F90} contains a general implementation and its wrapper is in \texttt{Analysis/Max\_SAC.F90}. 

The stochastic MaxEnt is essentially a parallel tempering Monte Carlo simulation.  For a discrete set of $\tau_i$ points,  $i \in 1 \cdots n $ the energy reads
\begin{equation}
  \chi^{2}(A) =  \sum_{i,j=1}^{n}   \left[ g(\tau_i)  –    \overline{g(\tau_i)} \right] C^{-1}(\tau_i,\tau_j) \left[    g(\tau_j)  –  \overline{g(\tau_j)} \right] 
\end{equation} with $ \overline{g(i)} =\int d{\omega} K(\tau_{i},\omega)  A(\omega)$ and  $C$ the covariance matrix. 
The set  of inverse temperatures  we will consider  in the parallel tempering reads:
$ \alpha_m = \alpha_{st}  R^{m}, \; \; m = 1 \cdots N_{\alpha} $.   The phase space corresponds to all possible spectral functions with given sum rule and required positivity.  Finally,  the partition function reads
$Z =  \int{DA} e^{-\alpha \chi^{2}(A)}$.  

In the code, the spectral function is parametrized  by a  set of Dirac $\delta$ functions: 
\begin{equation}
      A(\omega)  = \sum_{i=1}^{N_{\gamma}} a_{i} \delta \left( \omega - \omega_i \right).
\end{equation}
In order to produce a histogram of  $ A(\omega) $ we divide  the frequency range in \texttt{Ndis} intervals. 
The Green function is read from the file \texttt{g\_dat}  corresponding to the  output of the  \texttt{cov\_tau.F90} analysis program.

Besides the parameters included in the namelist \texttt{VAR\_Max\_Stoch} set in the file \texttt{parameters} (see Sec.~\ref{sec:files}), also the variable \texttt{N\_cov}, from the namelist \texttt{VAR\_errors}, is required to run the maxent code. Recalling: \texttt{N\_cov = 1} (\texttt{N\_cov = 0}) sets that the covariance will (will not) be taken into account.


\noindent
\textbf{Output files} \\
The code produces  the following output files.
\begin{itemize}
\item The files  \texttt{Aom\_n}  correspond to the average spectral function at inverse  temperature  $ \alpha_n $. This corresponds to
$  \langle A_n(\omega) \rangle =   \frac{1}{Z}   \int DA(\omega)    e^{-\alpha_n \chi^{2}(A)  } A(\omega). $
The file contains three columns  $ \omega, \;  \langle A_n(\omega) \rangle , \;  \Delta \langle A_n(\omega) \rangle $.

\item The files \texttt{Aom\_ps\_n}   contain the average image over  the  inverse   temperatures  $ \alpha_n $ to $ \alpha_{N_\gamma} $  see Ref.~\cite{Beach04a} for more details.   
 The first three columns have the same meaning as for the files \texttt{Aom\_n}

\item The file \texttt{Green} contains the Green function. Its three columns correspond to $ \omega, \;   \Re G(\omega),$ and $\Im G(\omega)  $.  This is obtained from the spectral function through:
\begin{equation}
 G(\omega) =  -\frac{1}{\pi} \int d \Omega   \frac{A(\Omega)}{\omega – \Omega + i \delta}
 \end{equation}
where  $ \delta =  \Delta \omega$ with $ \Delta \omega = (\omega_\text{end} -  \omega_\text{start})/\text{Ndis}$ and the image corresponds to that of the file \texttt{Aom\_ps\_m} with $ m = N_{\alpha} -10 $. 
The first column of the  \texttt{Green}  file is a place holder for post-processing. The last three columns   correspond to $\omega, \text{Re} G(\omega) ,   - \text{Im} G(\omega)/\pi $. 

\item  One of the most important files is the file  \texttt{energies}. It contains there columns:  $ \alpha_n, \langle \chi^2 \rangle, \Delta \langle \chi^2 \rangle $.

\item   \texttt{best\_fit}  gives the values of $a_i$ and $\omega_i$   (recall that $ A(\omega)  = \sum_{i=1}^{N_{\gamma}} a_{i} \delta \left( \omega - \omega_i \right)$) corresponding to the last configuration of the  lowest temperature run.

\item  The File \texttt{data\_out}  is a crosscheck. It plots   $ \tau,  g(\tau),  \Delta g(\tau), \int d \omega  K(\tau, \omega) A(\omega) $ where the image  corresponds to the best fit (i.e. the lowest temperature). 
This file will give you  a feeling on how good the fit actually is.  Note that  \texttt{data\_out} contains only the data points that have  passed the tolerance test. 


\item There are two \texttt{dump} files which are generated. Since  the MaxEnt is a  Monte Carlo code, one  would like to be able to continue a simulation to improve. The data in the dump files will allow you to pursue the simulation without loosing the first run(s).   These files are  only generated if the variable  \texttt{checkpoint} is set to true. 
 \end{itemize}

The essential question is: which image should one use. There is no real answer to this question in the context of the stochastic MaxEnt. The only rule of thumb is to consider temperatures for which the \( \chi^2 \) is  comparable to the number of data points.


\subsection{Single particle quantities}
For the single-particle Green function, 

\begin{equation} 
	\langle \hat{c}^{\phantom\dagger}_{k} (\tau)  \hat{c}^{\dagger}_{k} (0)   \rangle   = \int d \omega  K_p(\tau,\omega)   A_p(k, \omega) 
\end{equation}
with 
\begin{equation}
K_{p}(\tau,\omega) =    \frac{1}{\pi} \frac{e^{-\tau \omega} }  {  1 + e^{-\beta\omega} }
\end{equation}
and in the Lehmann representation, 
 \begin{equation}
   A_p(k, \omega) = \frac{ \pi}{Z} \sum_{n,m} e^{-\beta E_n } \left( 1 + e^{-\beta \omega}\right) | \langle n | c_n | m  \rangle |^{2} \delta \left( E_m - E_n - \omega \right)  
\end{equation}  
Here $ \left( \hat{H} - \mu \hat{N} \right) | n \rangle = E_n | n \rangle  $.

Note that  $ A_p(k, \omega)  = - \text{Im} G^{\text{ret}} (k, \omega) $ with 
\begin{equation}
	G^{\text{ret}} (k, \omega)  = -i \int d t \Theta(t)  e^{i \omega t} \langle \left\{ \hat{c}^{\phantom\dagger}_{k} (t), \hat{c}^{\dagger}_{k} (0) \right\} \rangle
\end{equation}
Finally the sum rule reads:
\begin{equation}
	\int d \omega  A_p(k, \omega)  = \pi \langle  \left\{ \hat{c}^{\phantom\dagger}_{k} , \hat{c}^{\dagger}_{k}  \right\}   \rangle = \pi 
\end{equation}
Using the \texttt{Max\_Sac.F90}  with \texttt{Channel="P"}   will  load the above Kernel in the MaxEnt library.  Note that in this case the back  transformation is set to unity.  
Note that since for each  configuration of fields,  $ \langle  \langle \hat{c}^{\phantom\dagger}_{k} (\tau=0)  \hat{c}^{\dagger}_{k} (0)   \rangle  \rangle_{C} +   
\langle \langle \hat{c}^{\phantom\dagger}_{k} (\tau=\beta)  \hat{c}^{\dagger}_{k} (0)   \rangle \rangle_{C} = 
\langle \langle \left\{ \hat{c}^{\phantom\dagger}_{k},   \hat{c}^{\dagger}_{k}    \right\} \rangle \rangle_{C}   = 1$.  Hence if both  the $\tau=0$ anf $\tau=\beta$ data points are included, the covariance matrix will have a zero eigenvalue and the $\chi^{2}$. measure is not defined. Hence for the particle channel, the program omits the $\tau=\beta$ data point.     One should also not that there are special  particle-hole symmetric  cases where the $\tau=0$ data point shows no  fluctuations. In this case, the 
code equally omits the $\tau=0$ data point. 

\subsection{Particle-hole quantities }

\noindent
\textbf{Imaginary time formulation.}
 For particle-hole quantities such as spin-spin or charge-charge correlations, 
the  Kernel reads:
\begin{equation}
	\langle \hat{S}(q,\tau) \hat{S}(-q,0) \rangle  = \frac{1}{\pi} 
   \int {\text d} \omega  \frac{e^{- \tau \omega} }{ 1 - e^{-\beta  \omega} } \chi''(q,\omega).
\end{equation}
This follows directly from the  Lehmann representation: 
\begin{equation}
 \chi''(q,\omega)  = \frac{\pi}{Z} \sum_{n,m} e^{-\beta E_n} |\langle n | \hat{S}(q) | \rangle m |^2 
\delta ( \omega + E_n - E_m) \left( 1 - e^{-\beta  \omega} \right) 
\end{equation}
Since the linear response to a Hermitian perturbation  is real, $\chi''(q,\omega)  = - \chi''(-q,-\omega)$.  Hence for systems with inversion symmetry -- that 
we will consider here -- $\langle \hat{S}(q,\tau) \hat{S}(-q,0) \rangle $ is a symmetric function around $\beta= \tau/2$.  The analysis  file \texttt{cov\_tau\_ph.F90} produced at compilation
time will use this  to define an improved estimator. 

The  Stochastic MaxEnt requires a sum rule, such that   the Kernel and image have to be adequately redefined. 
Consider: 
\begin{equation}
	\text{coth}(\beta \omega/2) \chi''(q,\omega)
\end{equation}
For this quantitiy, we have the sum rule since: 
\begin{equation}
	\int {\text d} \omega 	\text{coth}(\beta \omega/2) \chi''(q,\omega) = 
  2 \pi \langle \hat{S}(q,\tau=0) \hat{S}(-q,0) \rangle
\end{equation}
which is just the first point in the data. 

Hence,
\begin{equation}
	\langle \hat{S}(q,\tau) \hat{S}(-q,0) \rangle  =  
       \int {\text d} \omega  \underbrace{ \frac{1}{\pi} \frac{e^{- \tau \omega} }
            { 1 - e^{-\beta  \omega} } \text{tanh}(\beta \omega/2)  }_{K_{pp}(\tau,\omega)} 
       \underbrace{ \text{coth}(\beta \omega/2)   \chi''(q,\omega) }_{A(\omega)} 
\label{Kpp.eq}
\end{equation}
and one  computes $A(\omega)$. Note that since $\chi'' $ is an odd function of $\omega$  one restricts the integration range  positive values of $\omega$. 
Hence: 
\begin{equation}
	\langle \hat{S}(q,\tau) \hat{S}(-q,0) \rangle  =  
       \int_{0}^{\infty}  {\text d} \omega \underbrace{\left( K(\tau,\omega)  + K(\tau,-\omega) \right)}_{K_{ph}(\tau,\omega)}  A(\omega).
\end{equation}
In the code, $\omega_\text{start}$ is set to zero by default and the Kernel $K_{ph}$ is used in the code and is defined in the  routine \texttt{XKER\_ph}. 
In general,  one would like to produce the  dynamical structure factor that relates to the susceptibility according to
\begin{equation}
 S(q,\omega)  = \chi''(q,\omega)/\left( 1 - e^{-\beta  \omega} \right). 
\end{equation}

In the code the routine \texttt{BACK\_TRANS\_ph}   transforms the image $A$ to the desired quantity.
\begin{equation}
	S(q,\omega) = \frac{A(\omega)}{1 + e^{-\beta \omega} }  
\end{equation}

\noindent
\textbf{Matsubara frequency formulation.}
The ALF  library uses  imaginary time. It is however possible to formulate the MaxEnt in  Matsubara frequencies.
Consider:
\begin{equation}
  \chi(q,i\Omega_m) = \int_0^{\beta} {\text d} \tau  e^{i \Omega_m \tau}
	\langle S(q,\tau) S(-q,0) \rangle  = \frac{1}{\pi}
   \int {\text d} \omega  \frac{\chi''(q,\omega)}{ \omega - i \Omega_m }.
\end{equation}
Using the fact that $\chi''(q,\omega) = -\chi''(-q,-\omega) = -\chi''(q,-\omega)$ one obtains:
\begin{equation}
\begin{gathered}
  \chi(q,i\Omega_m) = 
	\frac{1}{\pi}
   \int_0^{\infty} {\text d} \omega \left(\frac{1}{ \omega - i \Omega_m } - \frac{1}{ -\omega - i \Omega_m } \right)
         \chi''(q,\omega) \\
    = \frac{2}{\pi} \int_0^{\infty} {\text d} \omega \frac{\omega^2}{ \omega^2  + \Omega_m^2 } 
  \frac{\chi''(q,\omega)}{\omega} 
   \equiv \int_0^{\infty} {\text d} \omega K(\omega,i\Omega_m) A(q,\omega)
\end{gathered}
\end{equation}
with
\begin{equation}
   K(\omega,i\Omega_m) = \frac{\omega^2}{ \omega^2  + \Omega_m^2 } 
\end{equation}
and
\begin{equation}
A(q,\omega) =  \frac{2}{\pi}   \frac{\chi''(q,\omega)}{\omega} 
\end{equation}
The above definitions are useful since the image satisfies the sum rule:
\begin{equation}
\int_0^{\infty} {\text d} \omega A(q,\omega) =  \frac{1}{\pi}  \int_{-\infty}^{\infty} {\text d} \omega 
   \frac{\chi''(q,\omega)}{\omega}   \equiv \chi(q,i\Omega_m=0)
\end{equation}


\subsection{Particle-Particle quantities}

Similarly to the particle-hole channel  the particle-particle channel is also a bosonic correlation function. Here however we do not assume that the 
imaginary time data is symmetric around   the $\tau = \beta/2$ point.  We use the Kernel $K_{pp}$ define in Eq.~\ref{Kpp.eq}  and consider the whole frequency range. 
The back transformation  yields
\begin{equation}
 \frac{\chi''(\omega)} {\omega}   = \frac{\text{tanh} \left( \beta \omega/2 \right) }{ \omega }   A(\omega) 
\end{equation}



\subsection{Zero temperature, projective code}

 In the zero temperature limit,  the spectral function associated to an operator $\hat{O} $    reads:
 \begin{equation}
 	  A_o(\omega)    = \pi  \sum_{n}    | \langle n  | \hat{O} | 0 \rangle |^2 \delta( E_n - E_0 - \omega) 
 \end{equation}
 such that 
 \begin{equation}
 	\langle 0 | \hat{O}^{\dagger}(\tau) \hat{O}^{}(0) | 0 \rangle =  \int d  \omega  K_0(\tau,\omega) A_0(\omega) 
 \end{equation}
 with 
 \begin{equation}
 	K_0(\tau,\omega)  = \frac{1}{\pi}e^{-\tau \omega}.
 \end{equation}
 The zeroth moment of the spectral function reads, 
 \begin{equation}
  \int d \omega A_o(\omega) = \pi \langle 0 | \hat{O}^{\dagger}(0) \hat{O}^{}(0) | 0 \rangle, 
 \end{equation}
 and hence corresponds to the first data point. 
 In the zero-temperature limit one does not distinguish between  particle, particle-hole, or particle-particle channels.
 Using the \texttt{Max\_Sac.F90}  with \texttt{Channel="T0"}   will  load the above Kernel in the MaxEnt library. In this case the back  transformation is set to unity. 
 The code will also cut-off the tail of the  imaginary time correlation function  if the relative error is greater that the variable \texttt{Tolerance}. 
