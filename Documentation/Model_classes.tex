% Copyright (c) 2016-2020 The ALF project.
% This is a part of the ALF project documentation.
% The ALF project documentation by the ALF contributors is licensed
% under a Creative Commons Attribution-ShareAlike 4.0 International License.
% For the licensing details of the documentation see license.CCBYSA.

% !TEX root = doc.tex

\section{  Model Classes }

The ALF  library comes with five model classes. a) SU(N) Hubbard models, (b)  O(2N) t-V models, (c)   Kondo models, (d)    Models with long ranged coulomb and (e)  Generic $Z_2$ lattice gauges theories couples to $Z_2$ matter  and fermions.   Below we will 
detail the functioning of these classes.  


\subsection{ SU(N) Hubbard models   \texttt{Hamiltonian\_Hubbard\_mod.F90}}

The parameter space for this model class  reads: 

\begin{lstlisting}[style=fortran,escapechar=\#,breaklines=true]
&VAR_Hubbard               !! Variables for the Hubbard class
Mz        = .T.             ! If true , sets the M_z-Hubbard model: Nf=2, N_SUN has to be even.  
                            ! HS field couples to the z-component of magnetization
ham_T     = 1.d0            ! Hopping parameter
ham_chem  = 0.d0            ! Chemical potential
ham_U     = 4.d0            ! Hubbard interaction
ham_T2    = 1.d0            ! For bilayer systems
ham_U2    = 4.d0            ! For bilayer systems
ham_Tperp = 1.d0            ! For bilayer systems
/
               
\end{lstlisting}
In the above   \texttt{ham\_T} and \texttt{ham\_T2}   correspond to the hopping in the first and second layers respectively and  \texttt{ham\_Tperp}   is to the interlayer hopping.   The Hubbard $U$ term has an orbital index, 
 \texttt{ham\_U}  for the first and  \texttt{ham\_U2}   for the second layers.  Finally   \texttt{ham\_chem}  corresponds to the chemical potential.     If  the flag \texttt{Mz} is set to \texttt{.False.},   then the code will simulate the  following 
SU(N)   symmetry Hubbard model. 
\begin{eqnarray}
\hat{H}= & & \sum_{(\ve{i},\ve{\delta}), (\ve{j},\ve{\delta}')}  \sum_{\sigma =1}^{N}  T_{(\ve{i},\ve{\delta}), (\ve{j},\ve{\delta}')}    c^{\dagger}_{(\ve{i},\ve{\delta}), \sigma }   e^{\frac{2 \pi i}{\Phi_0} \int_{\ve{i} + \ve{\delta}}^{\ve{j} + \ve{\delta}'}  
     \vec{A}(\ve{l})  d \ve{l}} c^{}_{(\ve{j},\ve{\delta}'),\sigma} 
+ \sum_{\vec{i}} \sum_{\delta}   \frac{U_\delta}{N} \left(\sum_{\sigma=1}^{N}  \left[   \hat{c}^{\dagger}_{(\vec{i},\ve{\delta}),\sigma } 
    \hat{c}^{\phantom\dagger}_{(\vec{i}, \ve{\delta}),\sigma }  - 1/2  \right] \right)^2  \nonumber \\
    & & - \mu \sum_{(\ve{i},\ve{\delta})}  \sum_{\sigma =1}^{N} \hat{c}^{\dagger}_{(\vec{i},\ve{\delta}),\sigma } \hat{c}^{\phantom\dagger}_{(\vec{i},\ve{\delta}),\sigma } 
\end{eqnarray}
The  generic hopping is taken fron Eq.~\ref{generic_hopping.eq}   with appropriate boundary conditions given by Eq.~\ref{generic_boundary.eq}.    $\ve{i}$ runs over the unit cells, $\ve{\delta}$ over the orbitals in each unit cell and $\sigma$  from $1 \cdots N$  and encodes  the SU(N) symmetry.    Note that  $N$ corresponds to \texttt{N\_SUN}  in the code.  The flavor index is set to  unity such that it does not appear in the  Hamiltonian.  $\mu$ corresponds to the chemical potential  and is relevant only for the finite temperature code. 

If the variable \texttt{Mz} is set to \texttt{.True.}, then the code    will require  \texttt{N\_SUN}  to be even and will simulate the following Hamiltonian. 

\begin{eqnarray}
\hat{H}= & & \sum_{(\ve{i},\ve{\delta}), (\ve{j},\ve{\delta}')}  \sum_{\sigma =1}^{N/2}  \sum_{s=1,2} T_{(\ve{i},\ve{\delta}), (\ve{j},\ve{\delta}')}    c^{\dagger}_{(\ve{i},\ve{\delta}), \sigma,s }   e^{\frac{2 \pi i}{\Phi_0} \int_{\ve{i} + \ve{\delta}}^{\ve{j} + \ve{\delta}'}  
     \vec{A}(\ve{l})  d \ve{l}} c^{}_{(\ve{j},\ve{\delta}'),\sigma,s}     \nonumber   \\
    & & - \sum_{\vec{i}} \sum_{\delta}   \frac{U_\delta}{N} \left(\sum_{\sigma=1}^{N/2}  \left[   \hat{c}^{\dagger}_{(\vec{i},\ve{\delta}),\sigma, 2} 
    \hat{c}^{\phantom\dagger}_{(\vec{i}, \ve{\delta}),\sigma,2 }  -  \hat{c}^{\dagger}_{(\vec{i},\ve{\delta}),\sigma, 1} 
    \hat{c}^{\phantom\dagger}_{(\vec{i}, \ve{\delta}),\sigma,1} \right] \right)^2  \nonumber \\
    & & - \mu \sum_{(\ve{i},\ve{\delta})}  \sum_{\sigma =1}^{N/2}  \sum_{s=1,2}\hat{c}^{\dagger}_{(\vec{i},\ve{\delta}),\sigma, s} \hat{c}^{\phantom\dagger}_{(\vec{i},\ve{\delta}),\sigma, s} 
\end{eqnarray}
In this case, the flavor index \texttt{N\_FL}   takes the value 2. Cleary at $N=2$, both modes  correspond  to the Hubbard model.  For $N$  even and $N > 2$  the models differ.  In particular  in the latter  Hamiltonian the U(N) symmetry is broken down to  U(N/2) $\otimes $U(N/2).  

Since this model class  works for all predefined lattices  (see Fig.~\ref{fig_predefined_lattices})  is includes the SU(N) periodic Anderson model on   the square and Honeycomb lattices.     Finally,  we note that the executable for this class is given by \texttt{Hubbard.out}

{\color{red}  If we do a good job in the previous sections we actually do not need much more explanation for this.   We could also provide Juypyter notebooks to start a set of Hubbard  hamiltonians.  e.g.  
\begin{itemize}
\item   \texttt{Pam\_square.ipynb}            SU(N) Square lattice PAM 
\item  \texttt{Pam\_honeycomb.ipynb}     SU(N) Honeycomb lattice lattice PAM 
\item  \texttt{Bilayer\_Hubbard.ipynb}      SU(N) Hubbard model on bilayers.
\item  \texttt{N\_leg\_ladder\_Hubbard.ipynb}   SU(N) n-leg-ladder  Hubbard
\item ...
\end{itemize} 
Would be nice to discuss this point.}



\subsection{  O(2N)  t-V models  \texttt{tV\_mod.F90}}

This would include the $SU(N)$  $t-V$ models on various lattices.  The defining property of this set of Hamiltonians would be the enlarged O(2N) symmetry.  Again  this   module should support our standard bipartite lattices. 

The parameter space for this model class  reads: 

\begin{lstlisting}[style=fortran,escapechar=\#,breaklines=true]
&VAR_Hubbard               !! Variables for the Hubbard class
ham_T     = 1.d0            ! Hopping parameter
ham_chem  = 0.d0            ! Chemical potential
ham_V     = 4.d0            ! Hubbard interaction
ham_T2    = 1.d0            ! For bilayer systems
ham_V2    = 4.d0            ! For bilayer systems
ham_Tperp = 1.d0            ! For bilayer systems
ham_Vperp = 1.d0            ! For bilayer systems
/

\end{lstlisting}
In the above   \texttt{ham\_T} and \texttt{ham\_T2} and \texttt{ham\_Tperp}   correspond to the hopping in the first and second layers respectively and  \texttt{ham\_Tperp}   is to the interlayer hopping.   The interaction term has an orbital index, 
\texttt{ham\_V}  for the first and  \texttt{ham\_V2}  for the second layers,  and \texttt{ham\_Vperp} for interlayer coupling.  Finally   \texttt{ham\_chem}  corresponds to the chemical potential. Let us define the operator
\begin{eqnarray}
\hat{b}_{\langle (\ve{i},\ve{\delta}), (\ve{j},\ve{\delta}') \rangle} & = &  \sum_{\sigma =1}^{N}    c^{\dagger}_{(\ve{i},\ve{\delta}), \sigma }   e^{\frac{2 \pi i}{\Phi_0} \int_{\ve{i} + \ve{\delta}}^{\ve{j} + \ve{\delta}'}  
	\vec{A}(\ve{l})  d \ve{l}} c^{}_{(\ve{j},\ve{\delta}'),\sigma} 
+ \textrm{H.c.} \,.
\end{eqnarray}
The model is then defined as follows:
\begin{eqnarray}
\hat{H}= & & \sum_{\langle (\ve{i},\ve{\delta}), (\ve{j},\ve{\delta}') \rangle}   T_{(\ve{i},\ve{\delta}), (\ve{j},\ve{\delta}')}    \hat{b}_{\langle (\ve{i},\ve{\delta}), (\ve{j},\ve{\delta}') \rangle}
+ \sum_{\langle (\ve{i},\ve{\delta}), (\ve{j},\ve{\delta}') \rangle}  \frac{V_{(\ve{i},\ve{\delta}), (\ve{j},\ve{\delta}')}}{N} \left(  \hat{b}_{\langle (\ve{i},\ve{\delta}), (\ve{j},\ve{\delta}') \rangle}  \right)^2  \nonumber \\
& & - \mu \sum_{(\ve{i},\ve{\delta})}  \sum_{\sigma =1}^{N} \hat{c}^{\dagger}_{(\vec{i},\ve{\delta}),\sigma } \hat{c}^{\phantom\dagger}_{(\vec{i},\ve{\delta}),\sigma } \,.
\end{eqnarray}
The  generic hopping is taken from Eq.~\ref{generic_hopping.eq}   with appropriate boundary conditions given by Eq.~\ref{generic_boundary.eq}.    $\ve{i}$ runs over the unit cells, $\ve{\delta}$ over the orbitals in each unit cell and $\sigma$  from $1 \cdots N$  and encodes  the SU(N) symmetry.    Note that  $N$ corresponds to \texttt{N\_SUN}  in the code.  The flavor index is set to  unity such that it does not appear in the  Hamiltonian.  $\mu$ corresponds to the chemical potential  and is relevant only for the finite temperature code. 

\subsection{  SU(N) Kondo lattice models  \texttt{Kondo\_mod.F90}}


\subsection{  Models with long range Coulomb interactions  \texttt{LRC\_mod.F90}}

This is the long range Coulomb. See above.    Again we should include the   standard lattices. 

\subsection{Z$_2$ lattice gauge theories coupled to fermion and  Z$_2$ matter    \texttt{Z2\_mod.F90}}
I suggest to work on the  Hamiltonian of Ref.~\ref{Z2.Sec} since this is the most general  model I can think of.   Would be nice to add a Hubbard-$U$ term.  It is actually not so easy to generalize this model to 
arbitrary lattices, so that for the moment, I would concentrate only on the square lattice. 
