% Copyright (c) 2016-2020 The ALF project.
% This is a part of the ALF project documentation.
% The ALF project documentation by the ALF contributors is licensed
% under a Creative Commons Attribution-ShareAlike 4.0 International License.
% For the licensing details of the documentation see license.CCBYSA.

% !TEX root = doc.tex

\section{  Model Classes }

The ALF  library comes with five model classes. a) SU(N) Hubbard models, (b)  O(2N) t-V models, (c)   Kondo models, (d)    Models with long ranged coulomb and (e)  Generic $Z_2$ lattice gauges theories coupled to $Z_2$ matter  and fermions.   Below we will 
detail the functioning of these classes.  


\subsection{ SU(N) Hubbard models   \texttt{Hamiltonian\_Hubbard\_mod.F90}}

The parameter space for this model class  reads: 

\begin{lstlisting}[style=fortran,escapechar=\#,breaklines=true]
&VAR_Hubbard               !! Variables for the Hubbard class
Mz        = .T.             ! If true , sets the M_z-Hubbard model: Nf=2, N_SUN has to be even.  
                            ! HS field couples to the z-component of magnetization
ham_T     = 1.d0            ! Hopping parameter
ham_chem  = 0.d0            ! Chemical potential
ham_U     = 4.d0            ! Hubbard interaction
ham_T2    = 1.d0            ! For bilayer systems
ham_U2    = 4.d0            ! For bilayer systems
ham_Tperp = 1.d0            ! For bilayer systems
/
               
\end{lstlisting}
In the above   \texttt{ham\_T} and \texttt{ham\_T2}   correspond to the hopping in the first and second layers respectively and  \texttt{ham\_Tperp}   is to the interlayer hopping.   The Hubbard $U$ term has an orbital index, 
 \texttt{ham\_U}  for the first and  \texttt{ham\_U2}   for the second layers.  Finally   \texttt{ham\_chem}  corresponds to the chemical potential.     If  the flag \texttt{Mz} is set to \texttt{.False.},   then the code will simulate the  following 
SU(N)   symmetry Hubbard model. 
\begin{eqnarray}
\hat{H}= & & \sum_{(\ve{i},\ve{\delta}), (\ve{j},\ve{\delta}')}  \sum_{\sigma =1}^{N}  T_{(\ve{i},\ve{\delta}), (\ve{j},\ve{\delta}')}    \hat{c}^{\dagger}_{(\ve{i},\ve{\delta}), \sigma }   e^{\frac{2 \pi i}{\Phi_0} \int_{\ve{i} + \ve{\delta}}^{\ve{j} + \ve{\delta}'}  
     \vec{A}(\ve{l})  d \ve{l}} \hat{c}^{}_{(\ve{j},\ve{\delta}'),\sigma} 
+ \sum_{\vec{i}} \sum_{\delta}   \frac{U_{\ve{\delta}} }{N} \left(\sum_{\sigma=1}^{N}  \left[   \hat{c}^{\dagger}_{(\vec{i},\ve{\delta}),\sigma } 
    \hat{c}^{\phantom\dagger}_{(\vec{i}, \ve{\delta}),\sigma }  - 1/2  \right] \right)^2  \nonumber \\
    & & - \mu \sum_{(\ve{i},\ve{\delta})}  \sum_{\sigma =1}^{N} \hat{c}^{\dagger}_{(\vec{i},\ve{\delta}),\sigma } \hat{c}^{\phantom\dagger}_{(\vec{i},\ve{\delta}),\sigma } 
\end{eqnarray}
The  generic hopping is taken fron Eq.~\ref{generic_hopping.eq}   with appropriate boundary conditions given by Eq.~\ref{generic_boundary.eq}.    $\ve{i}$ runs over the unit cells, $\ve{\delta}$ over the orbitals in each unit cell and $\sigma$  from $1 \cdots N$  and encodes  the SU(N) symmetry.    Note that  $N$ corresponds to \texttt{N\_SUN}  in the code.  The flavor index is set to  unity such that it does not appear in the  Hamiltonian.  $\mu$ corresponds to the chemical potential  and is relevant only for the finite temperature code. 

If the variable \texttt{Mz} is set to \texttt{.True.}, then the code    will require  \texttt{N\_SUN}  to be even and will simulate the following Hamiltonian. 

\begin{align}
\hat{H} =& \sum_{(\ve{i},\ve{\delta}), (\ve{j},\ve{\delta}')}  \sum_{\sigma =1}^{N/2}  \sum_{s=1,2} T_{(\ve{i},\ve{\delta}), (\ve{j},\ve{\delta}')}    \hat{c}^{\dagger}_{(\ve{i},\ve{\delta}), \sigma,s }   e^{\frac{2 \pi i}{\Phi_0} \int_{\ve{i} + \ve{\delta}}^{\ve{j} + \ve{\delta}'}  
     \vec{A}(\ve{l})  d \ve{l}} \hat{c}^{}_{(\ve{j},\ve{\delta}'),\sigma,s}     \nonumber   \\
    &- \sum_{\vec{i}} \sum_{\delta}   \frac{U_\delta}{N} \left(\sum_{\sigma=1}^{N/2}  \left[   \hat{c}^{\dagger}_{(\vec{i},\ve{\delta}),\sigma, 2} 
    \hat{c}^{\phantom\dagger}_{(\vec{i}, \ve{\delta}),\sigma,2 }  -  \hat{c}^{\dagger}_{(\vec{i},\ve{\delta}),\sigma, 1} 
    \hat{c}^{\phantom\dagger}_{(\vec{i}, \ve{\delta}),\sigma,1} \right] \right)^2  \nonumber \\
    & - \mu \sum_{(\ve{i},\ve{\delta})}  \sum_{\sigma =1}^{N/2}  \sum_{s=1,2}\hat{c}^{\dagger}_{(\vec{i},\ve{\delta}),\sigma, s} \hat{c}^{\phantom\dagger}_{(\vec{i},\ve{\delta}),\sigma, s}.
\end{align}
In this case, the flavor index \texttt{N\_FL}   takes the value 2. Cleary at $N=2$, both modes  correspond  to the Hubbard model.  For $N$  even and $N > 2$  the models differ.  In particular  in the latter  Hamiltonian the U(N) symmetry is broken down to  U(N/2) $\otimes $U(N/2).  

Since this model class  works for all predefined lattices  (see Fig.~\ref{fig_predefined_lattices})  is includes the SU(N) periodic Anderson model on   the square and Honeycomb lattices.     Finally,  we note that the executable for this class is given by \texttt{Hubbard.out}

{\color{red}  If we do a good job in the previous sections we actually do not need much more explanation for this.   We could also provide Juypyter notebooks to start a set of Hubbard  hamiltonians.  e.g.  
\begin{itemize}
\item   \texttt{Pam\_square.ipynb}            SU(N) Square lattice PAM 
\item  \texttt{Pam\_honeycomb.ipynb}     SU(N) Honeycomb lattice lattice PAM 
\item  \texttt{Bilayer\_Hubbard.ipynb}      SU(N) Hubbard model on bilayers.
\item  \texttt{N\_leg\_ladder\_Hubbard.ipynb}   SU(N) n-leg-ladder  Hubbard
\item ...
\end{itemize} 
Would be nice to discuss this point.}



\subsection{  O(2N)  t-V models  \texttt{tV\_mod.F90}}

This would include the $SU(N)$  $t-V$ models on various lattices.  The defining property of this set of Hamiltonians would be the enlarged O(2N) symmetry.  Again  this   module should support our standard bipartite lattices. 

The parameter space for this model class  reads: 

\begin{lstlisting}[style=fortran,escapechar=\#,breaklines=true]
&VAR_Hubbard               !! Variables for the Hubbard class
ham_T     = 1.d0            ! Hopping parameter
ham_chem  = 0.d0            ! Chemical potential
ham_V     = 4.d0            ! Hubbard interaction
ham_T2    = 1.d0            ! For bilayer systems
ham_V2    = 4.d0            ! For bilayer systems
ham_Tperp = 1.d0            ! For bilayer systems
ham_Vperp = 1.d0            ! For bilayer systems
/

\end{lstlisting}
In the above   \texttt{ham\_T} and \texttt{ham\_T2} and \texttt{ham\_Tperp}   correspond to the hopping in the first and second layers respectively and  \texttt{ham\_Tperp}   is to the interlayer hopping.   The interaction term has an orbital index, 
\texttt{ham\_V}  for the first and  \texttt{ham\_V2}  for the second layers,  and \texttt{ham\_Vperp} for interlayer coupling.  Finally   \texttt{ham\_chem}  corresponds to the chemical potential. Let us define the operator
\begin{equation}
\hat{b}_{\langle (\ve{i},\ve{\delta}), (\ve{j},\ve{\delta}') \rangle} =  \sum_{\sigma =1}^{N}    \hat{c}^{\dagger}_{(\ve{i},\ve{\delta}), \sigma }   e^{\frac{2 \pi i}{\Phi_0} \int_{\ve{i} + \ve{\delta}}^{\ve{j} + \ve{\delta}'}  
	\vec{A}(\ve{l})  d \ve{l}} \hat{c}^{}_{(\ve{j},\ve{\delta}'),\sigma} 
+ \textrm{H.c.}
\end{equation}
The model is then defined as follows:
\begin{align}
\hat{H}= & \sum_{\langle (\ve{i},\ve{\delta}), (\ve{j},\ve{\delta}') \rangle}   T_{(\ve{i},\ve{\delta}), (\ve{j},\ve{\delta}')}    \hat{b}_{\langle (\ve{i},\ve{\delta}), (\ve{j},\ve{\delta}') \rangle}
+ \sum_{\langle (\ve{i},\ve{\delta}), (\ve{j},\ve{\delta}') \rangle}  \frac{V_{(\ve{i},\ve{\delta}), (\ve{j},\ve{\delta}')}}{N} \left(  \hat{b}_{\langle (\ve{i},\ve{\delta}), (\ve{j},\ve{\delta}') \rangle}  \right)^2  \nonumber \\
& - \mu \sum_{(\ve{i},\ve{\delta})}  \sum_{\sigma =1}^{N} \hat{c}^{\dagger}_{(\vec{i},\ve{\delta}),\sigma } \hat{c}^{\phantom\dagger}_{(\vec{i},\ve{\delta}),\sigma } \,.
\end{align}
The  generic hopping is taken from Eq.~\ref{generic_hopping.eq}   with appropriate boundary conditions given by Eq.~\ref{generic_boundary.eq}.    $\ve{i}$ runs over the unit cells, $\ve{\delta}$ over the orbitals in each unit cell and $\sigma$  from $1 \cdots N$  and encodes  the SU(N) symmetry.    Note that  $N$ corresponds to \texttt{N\_SUN}  in the code.  The flavor index is set to  unity such that it does not appear in the  Hamiltonian.  $\mu$ corresponds to the chemical potential  and is relevant only for the finite temperature code. 

\subsection{  SU(N) Kondo lattice models  \texttt{Kondo\_mod.F90}}

% Copyright (c) 2016 2017 The ALF project.
% This is a part of the ALF project documentation.
% The ALF project documentation by the ALF contributors is licensed
% under a Creative Commons Attribution-ShareAlike 4.0 International License.
% For the licensing details of the documentation see license.CCBYSA.
% !TEX root = doc.tex


\subsection{SU(N) Kondo lattice models \texttt{Kondo\_mod.F90}}  \label{sec:kondo}

The Kondo lattice model we consider is an SU(N) generalization of the SU(2) Kondo-model discussed in \cite{Capponi00,Assaad99a}.
Here we follow the work of  Ref.~\cite{Raczkowski20}.
Let $T^{a}$ be the $N^2 -1  $ generators of SU(N) that satisfy the normalization condition: 
\begin{equation}
	\Tr  \left[ T^{a} T^{b} \right]   = \frac{1}{2}\delta_{a,b}.
\label{Normalization_condition.eq}
\end{equation}
For the SU(2) case, $T^{a}$  corresponds to the $T  = \frac{1}{2} \ve{\sigma}$ with $\ve{\sigma}$   a vector of the three Pauli spin matrices, Eq.~\eqref{eq:vecPaulimat}.      The   Hamiltonian is defined on bilayer  square or honeycomb lattices, with  hopping restricted to the  first layer  (i.e  conduction orbitals $\ve{c}^{\dagger}_{i}  )$   and  spins, f-orbitals, on the second layer. 
\begin{multline}
	\hat{H} = - t  \sum_{\langle i,j \rangle}    \sum_{\sigma=1}^{N}  \left(  \hat{c}^{\dagger}_{i,\sigma}  e^{\frac{2\pi i}{\Phi_0}  \int_{i}^{j} \ve{A}\cdot d \ve{l}}\hat{c}^{\phantom\dagger}_{j,\sigma}   + \hc  \right)  - \mu \sum_{i,\sigma} \hat{c}^{\dagger}_{i,\sigma}  \hat{c}^{\phantom\dagger}_{i,\sigma}  \\ 
   + \frac{U_c}{N}  \sum_{i}   \left( \hat{n}^c_i -  \frac{N}{2} \right)^2  
         +  \frac{2 J}{N} \sum_{i, a=1  }^{N^2 -1}  \hat{T}^{a,c}_{i}  \hat{T}^{a,f}_{i}. 
\label{Kondo_SUN_Ham.eq}
\end{multline}
In the above, $i$ is a super-index accounting for the unit cell and orbital,
\begin{equation}
	 \hat{T}^{a,c}_{i}   =   \sum_{\sigma,\sigma'=1}^{N} \hat{c}^{\dagger}_{i,\sigma}T^{a}_{\sigma,\sigma'}  \hat{c}^{\phantom\dagger}_{i,\sigma'}, \; \; 
	  \hat{T}^{a,f}_{i}   = \sum_{\sigma,\sigma'=1}^{N} \hat{f}^{\dagger}_{i,\sigma} T^{a}_{\sigma,\sigma'}  \hat{f}^{\phantom\dagger}_{i,\sigma'},  
	  \;\text{ and }\;   \hat{n}^c_i  = \sum_{\sigma=1}^{N} \hat{c}_{i,\sigma}^{\dagger} \hat{c}_{i,\sigma}^{\phantom\dagger} .
\end{equation}
Finally, the constraint
\begin{equation}
   \sum_{\sigma=1}^{N}  \hat{f}^{\dagger}_{i,\sigma}   \hat{f}^{\phantom\dagger}_{i,\sigma}  \equiv  \hat{n}^{f}_i = \frac{N}{2}
\end{equation}
holds. Some rewriting has to be carried out so as to implement  the model.
First, we  use the  relation:
\begin{equation*}
	\sum_{a} T^{a}_{\alpha,\beta} T^{a}_{\alpha',\beta'} = \frac{1}{2} \left(  \delta_{\alpha,\beta'}  \delta_{\alpha',\beta} - \frac{1}{N} \delta_{\alpha,\beta} \delta_{\alpha', \beta'} \right), 
\end{equation*}
to  show that  in the unconstrained Hilbert space,
\begin{align}
	 \frac{2 J}{N} \sum_{ a=1  }^{N^2 -1}  \hat{T}^{a,c}_{i}  \hat{T}^{a,f}_{i}   = &   - \frac{J}{2N} \sum_{i}  \left( 
                \hat{D}^{\dagger}_{i} \hat{D}^{\phantom\dagger}_{i}   +    \hat{D}^{\phantom\dagger}_{i} \hat{D}^{\dagger}_{i}    \right)    + \frac{J}{N}   \left(   \frac{\hat{n}^{c}_i}{2}  + \frac{\hat{n}^{f}_i}{2} -  \frac{\hat{n}^{c}_i \hat{n}^{f}_i}{N}   \right) \nonumber 
 \end{align}
with
\begin{equation*}
	   \hat{D}^{\dagger}_{i}   =  \sum_{\sigma=1}^{N} \hat{c}^{\dagger}_{i,\sigma}  \hat{f}^{\phantom\dagger}_{i,\sigma}.
\end{equation*}
In the constrained Hilbert space, $\hat{n}^{f}_i = N/2 $, the above gives:
\begin{align}
	 \frac{2 J}{N} \sum_{ a=1  }^{N^2 -1}  \hat{T}^{a,c}_{i}  \hat{T}^{a,f}_{i}   =     -  \frac{J}{4N}    \left[ \left(   \hat{D}^{\dagger}_{i}  + \hat{D}^{\phantom\dagger}_{i}    \right)^{2}  + 
                                                       \left(  i\hat{D}^{\dagger}_{i}  - i  \hat{D}^{\phantom\dagger}_{i}    \right)^2  \right] + \frac{J}{4}.  
 \end{align}
The perfect square form complies with the requirements of ALF.
We still have to impose the constraint.
To do so, we work in the unconstrained Hilbert space and add a Hubbard  $U$-term on  the f-orbitals.
With this addition, the Hamiltonian we simulate reads: 

\begin{multline}
	\hat{H}_\textrm{QMC}     =    - t  \sum_{\langle i,j \rangle}    \sum_{\sigma=1}^{N}  \left(  \hat{c}^{\dagger}_{i,\sigma}  e^{\frac{2\pi i}{\Phi_0}  \int_{i}^{j} \ve{A}\cdot d \ve{l}}\hat{c}^{\phantom\dagger}_{j,\sigma}  + \hc \right)  - \mu \sum_{i,\sigma} \hat{c}^{\dagger}_{i,\sigma}  \hat{c}^{\phantom\dagger}_{i,\sigma} 
	+    \frac{U_c}{N}  \sum_{i}   \left( \hat{n}^c_i -  \frac{N}{2} \right)^2  \\
    -    \frac{J}{4N}    \left[ \left(   \hat{D}^{\dagger}_{i}  + \hat{D}^{\phantom\dagger}_{i}    \right)^{2}  + 
                                                       \left(  i\hat{D}^{\dagger}_{i}  - i  \hat{D}^{\phantom\dagger}_{i}    \right)^2  \right]  
       +    \frac{U_f}{N}  \sum_{i}   \left( \hat{n}^f_i -  \frac{N}{2} \right)^2.
\label{Kondo_SUN_Ham_QMC.eq}
\end{multline}
The key point for the efficiency of the code, is to  see that 
\begin{equation}
	\left[   \hat{H}_\textrm{QMC},  \left( \hat{n}^f_i -  \frac{N}{2} \right)^2  \right]    = 0 
\label{Constraint_KLM.eq}
\end{equation}
such that the  constraint is implemented  efficiently.  In fact, for the finite temperature code  at inverse temperature $\beta$,  the unphysical Hilbert space   is suppressed by a  
factor  $e^{- \beta U_f/N} $. 



\subsubsection*{The SU(2) case} 
The SU(2) case is special and allows for a more efficient implementation than the one described above.  The  key point is that  for the SU(2) case, the  Hubbard term is  related to  the fermion parity,
\begin{equation} 
   \left(   \hat{n}^f_i - 1 \right)^2    = \frac{  (-1)^{\hat{n}^f_i}  +1 }{2}
\end{equation}
such that we can omit the \textit{current}-term  $ \left(  i\hat{D}^{\dagger}_{i}  - i  \hat{D}^{\phantom\dagger}_{i}    \right)^2 $    without violating  Eq.~\eqref{Constraint_KLM.eq}.  
As in Refs.~\cite{Assaad99a,Capponi00,Beach04}, the Hamiltonian that one will simulate reads: 
 \begin{multline}
 \label{eqn:ham_kondo}
 	\hat{\mathcal{H}}  =
	\underbrace{-t \sum_{\langle i,j \rangle,\sigma} \left( \hat{c}_{i,\sigma}^{\dagger} e^{\frac{2\pi i}{\Phi_0}  \int_{i}^{j} \ve{A}\cdot d \ve{l}} \hat{c}_{j,\sigma}^{\phantom\dagger}   + \hc \right) +
	  \frac{U_c}{2}   \sum_{i}   \left( \hat{n}^{c}_{i} -1 \right)^2    }_{\equiv \hat{\mathcal{H}}_{tU_c}}  \\
  - \frac{J}{4} 	\sum_{i} \left( \sum_{\sigma} \hat{c}_{i,\sigma}^{\dagger}  \hat{f}_{i,\sigma}^{\phantom\dagger}  + 
	                                                        \hat{f}_{i,\sigma}^{\dagger}  \hat{c}_{i,\sigma}^{\phantom\dagger}   \right)^{2}   +
        \underbrace{\frac{U_f}{2}   \sum_{i}   \left( \hat{n}^{f}_{i} -1 \right)^2}_{\equiv \hat{\mathcal{H}}_{U_f}}.
 \end{multline}
The  relation to the Kondo lattice model follows  from expanding the square  of the hybridization to obtain: 
 \begin{equation}
 	\hat{\mathcal{H}}  =\hat{\mathcal{H}}_{tU_c}   
	+ J \sum_{\vec{i}}  \left(  \hat{\vec{S}}^{c}_{\vec{i}} \cdot  \hat{\vec{S}}^{f}_{\vec{i}}    +   \hat{\eta}^{z,c}_{\vec{i}} \cdot  \hat{\eta}^{z,f}_{\vec{i}}  
		-  \hat{\eta}^{x,c}_{\vec{i}} \cdot  \hat{\eta}^{x,f}_{\vec{i}}  -  \hat{\eta}^{y,c}_{\vec{i}} \cdot  \hat{\eta}^{y,f}_{\vec{i}} \right) 
	 + \hat{\mathcal{H}}_{U_f},
 \end{equation}
 where the $\eta$-operators  relate to the spin-operators via a particle-hole transformation in one spin sector: 
 \begin{equation} 
 	\hat{\eta}^{\alpha}_{\vec{i}}  = \hat{P}^{-1}  \hat{S}^{\alpha}_{\vec{i}} \hat{P}  	\; \text{ with }  \;   
	\hat{P}^{-1}  \hat{c}^{\phantom\dagger}_{\vec{i},\uparrow} \hat{P}  =   (-1)^{i_x+i_y} \hat{c}^{\dagger}_{\vec{i},\uparrow}  \; \text{ and }  \;   
	\hat{P}^{-1}  \hat{c}^{\phantom\dagger}_{\vec{i},\downarrow} \hat{P}  = \hat{c}^{\phantom\dagger}_{\vec{i},\downarrow} .
 \end{equation}
 Since the $\hat{\eta}^{f}$ and $ \hat{S}^{f}$ operators  do not alter the  parity [$(-1)^{\hat{n}^{f}_{\vec{i}}}$ ] of the $f$-sites, 
 \begin{equation}
 	\left[  \hat{\mathcal{H}}, \hat{\mathcal{H}}_{U_f} \right] = 0.
 \end{equation}
 Thereby, and for positive values of $U$,  doubly occupied  or empty $f$-sites -- corresponding to even parity sites -- are suppressed  by a  Boltzmann factor 
 $e^{-\beta U_f/2} $ in comparison to odd parity sites.  Thus, essentially, choosing $\beta U_f $ adequately allows one to restrict the Hilbert space to odd parity $f$-sites.  
 In this Hilbert space, $\hat{\eta}^{x,f} = \hat{\eta}^{y,f} =  \hat{\eta}^{z,f} =0$  such that the Hamiltonian (\ref{eqn:ham_kondo}) reduces to the Kondo lattice model. 

%An implementation of the Kondo Lattice model on the  Honeycomb lattice with additional z-z frustration considered in Ref.~\cite{SatoT17_1}, 
%\begin{equation}
%\hat{H}_{\text{Spin}} = J^{z}\sum_{\langle \langle i,j \rangle \rangle}\hat{S}_{i}^{z}\hat{S}_{j}^{z},  \hfill  \hat{H}_{\text{Fermion}} = -t\sum_{\langle i,j \rangle,\sigma}\hat{c}_{i\sigma}^\dagger \hat{c}^{\phantom\dagger} _{j\sigma},  \hfill
%\hat{H}_{\text{Kondo}}  =  J^{\text{K}} \sum_{i}    \frac{1}{2} \hat{\pmb{c}}^{\dagger}_{i} \pmb{\sigma}\hat{\pmb{c}}^{\phantom\dagger}_{i} \cdot \hat{{\bm S}}^{\phantom\dagger} _{i} , \hfill
%\end{equation}
%can be found in  the \texttt{Hamiltonian\_Kondo\_Honey\_mod.F90}

\subsubsection*{QMC implementation} 

The name space for this model class  reads: 

\begin{lstlisting}[style=fortran,escapechar=\#,breaklines=true]
&VAR_Kondo                 !! Variables for the Kondo  class
ham_T     = 1.d0            ! Hopping parameter
ham_chem  = 0.d0            ! Chemical potential
ham_Uc    = 0.d0            ! Hubbard interaction  on  c-orbitals Uc
ham_Uf    = 2.d0            ! Hubbard interaction  on  f-orbials  Uf
ham_JK    = 2.d0            ! Kondo Coupling  J
/
\end{lstlisting}

Aside from the usual observables  we have included  the scalar observable \texttt{Constraint\_scal}    that measures 
\begin{equation}
	\left<    \sum_{i}   \left( \hat{n}_i^f - \frac{N}{2} \right)^2 \right>.
\end{equation}
$U_f$ has to be chosen large enough  such that  the above quantity vanishes within statistical uncertainty.  For the square lattice,  Fig.~\ref{Constraint.fig}   plots   the  aforementioned quantity as a function of $U_f$  for the SU(2) model.  As apparent $ \Big<    \sum_{i}   \big( \hat{n}_i^f - N/2 \big)^2 \Big> \propto e^{-\beta U_f/2} $.
\begin{figure}
\center
\includegraphics[width=0.49\textwidth]{Constraint.pdf}
\includegraphics[width=0.49\textwidth]{Spin.pdf}

\caption{Left:  Suppression of charge fluctuations of the f-orbitals as a function of $U_f$.  Right:   When  charge fluctuations  on the f-orbitals vanish, quantities such as the Fourier transform of the $f$ spin-spin  correlations at $\ve{q} = (\pi,\pi) $  converge to their KLM value. Typically,  for the SU(2) case, $\beta U_f > 10 $ suffices to reach convergent results.
The pyALF script used to produce the data of  the plot can be found in \href{https://git.physik.uni-wuerzburg.de/ALF/ALF/-/blob/ALF-2.0/Documentation/Figures/Kondo/Kondo.py}{\texttt{Kondo.py}}  }
        \label{Constraint.fig}
\end{figure}






\subsection{  Models with long range Coulomb interactions  \texttt{LRC\_mod.F90}}

% Copyright (c) 2016 2017 The ALF project.
% This is a part of the ALF project documentation.
% The ALF project documentation by the ALF contributors is licensed
% under a Creative Commons Attribution-ShareAlike 4.0 International License.
% For the licensing details of the documentation see license.CCBYSA.
% !TEX root = Doc.tex
\subsubsection{  Long range Coulomb }

The model we consider here reads: 
\begin{equation}
	\hat{H}  = -t \sum_{\vec{i},\vec{j},\sigma=1}^{N}   \hat{c}^{\dagger}_{\vec{i},\sigma}  T_{\vec{i},\vec{j}} \hat{c}^{}_{\vec{j},\sigma}   +     
	\frac{1} { N } \sum_{\vec{i},\vec{j}}  \left(  \hat{n}_{\vec{i}} -  \frac{N}{2}  \right)  V_{\vec{i},\vec{j}} \left(  \hat{n}_{\vec{j}} -  \frac{N}{2}  \right) \; \; 
	\text{ with }  \hat{n}_{\vec{i}} = \sum_{\sigma=1}^{N}  \hat{c}^{\dagger}_{\vec{i},\sigma}  \hat{c}^{}_{\vec{i},\sigma}
\end{equation}
The interaction is specified in the following way: 
\begin{equation}
	V_{\vec{i}, \vec{j}}   =   U \left\{
	\begin{array}{ll}  
	1          &   \text{ if } \vec{i} - \vec{j}    = 0 \\
	\frac{\alpha   \;   d_{min}}{ |   \vec{i} - \vec{j} | } &     \text{ otherwise }
	\end{array}
\right.
\end{equation}
Here $d_{min}$ is the minimal distance between two orbitals.     The code uses the following  HS decomposition:
\begin{equation}
e^{-\Delta \tau \hat{H}_V }  =  \int \prod_{\vec{i}} d \phi_{\vec{i}}   e^{ - \frac{N \Delta \tau} {4} \phi_{\pmb{i}} V^{-1}_{\pmb{i},\pmb{j}}  \phi_{\pmb{j}} - \sum_{\pmb{i}}  i \Delta \tau \phi_i \left( n_{i} - \frac{N}{2} \right) } 
\end{equation}

The implementation follows Ref.~\cite{Hohenadler14}  but now supports various lattice geometries. 


\subsection{Z$_2$ lattice gauge theories coupled to fermion and  Z$_2$ matter    \texttt{Z2\_mod.F90}}

%I suggest to work on the  Hamiltonian of Ref.~\ref{Z2.Sec} since this is the most general  model I can think of.   Would be nice to add a Hubbard-$U$ term.  It is actually not so easy to generalize this model to 
%arbitrary lattices, so that for the moment, I would concentrate only on the square lattice. 

% Copyright (c) 2016 2017 The ALF project.
% This is a part of the ALF project documentation.
% The ALF project documentation by the ALF contributors is licensed
% under a Creative Commons Attribution-ShareAlike 4.0 International License.
% For the licensing details of the documentation see license.CCBYSA.
% !TEX root = doc.tex
\subsubsection{$Z_2$ gauge theory coupled to $Z_2$ matter.   }
\label{Z2.Sec}
The Hamiltonian we will consider here reads
\begin{align}
	\hat{H} =& -  t_{Z_2} \sum_{\langle \vec{i}, \vec{j} \rangle, \sigma } \hat{Z}_{\langle \vec{i}, \vec{j} \rangle}
	\left(\hat{\Psi}^{\dagger}_{\vec{i},\sigma} \hat{\Psi}^{\phantom{\dagger}}_{\vec{j},\sigma}   + h.c. \right) - \mu \sum_{\vec{i},\sigma} \hat{\Psi}^{\dagger}_{\vec{i},\sigma} \hat{\Psi}^{\phantom{\dagger}}_{\vec{i},\sigma}  
	-g \sum_{\langle \vec{i}, \vec{j} \rangle } \hat{X}_{\langle \vec{i}, \vec{j} \rangle }  +
	  K \sum_{\square} \prod_{\langle \vec{i}, \vec{j} \rangle \in \partial \square} \hat{Z}_{\langle \vec{i}, \vec{j} \rangle}  \nonumber \\
	& + J  \sum_{\langle \vec{i}, \vec{j} \rangle}  \hat{\tau}^z_{\pmb{i}}  \hat{Z}_{\langle \vec{i}, \vec{j} \rangle} \hat{\tau}^z_{\pmb{j}}   
	      -  h \sum_{ \vec{i} } \hat{\tau}^x_{\vec{i}}   - t  \sum_{\langle \vec{i}, \vec{j} \rangle, \sigma }   \hat{\tau}^z_{\pmb{i}}   \hat{\tau}^z_{\pmb{j}}  \left( \hat{\Psi}^{\dagger}_{\vec{i},\sigma} \hat{\Psi}^{\phantom{\dagger}}_{\vec{j},\sigma} 	+ h.c. \right)
\end{align}  
Here the  $\hat{\Psi}^{\dagger}_{\vec{i},\sigma}$  creates an orthogonal fermion with $Z_2$ and  and electric charges.    
The implementation of this Hamiltonian can be found in the file \texttt{Hamiltonian\_Z2\_Matter\_mod.F90}.
 For this Hamiltonian, the the $Z_2$ local conservation law reads: 
\begin{equation}
	\hat{Q}_{\vec{i}} =  (-1)^{\sum_{\sigma} \hat{\Psi}^{\dagger}_{\vec{i},\sigma} \hat{\Psi}^{\phantom{\dagger}}_{\vec{j},\sigma}   } 
	\;  \hat{\tau}^{x}_{\vec{i}}  \; \hat{X}_{\vec{i},\vec{i} +  \vec{a}_x} \hat{X}_{\vec{i},\vec{i} -  \vec{a}_x} \hat{X}_{\vec{i},\vec{i} +  \vec{a}_y} \hat{X}_{\vec{i}}.
\end{equation} 

The Hamiltonian was investigated in Ref.~\cite{Gazit19}.   Here is a todo list. 
\begin{itemize}
\item  Include an attractive $U$-term. This breaks the O(4) symmetry down to SU(2)$\times$SU(2)   and selects the $\hat{Q}_{\vec{i}} =1 $ sector.   Note that a repulsive U should  can also be included and should produce equivalent results under particle-hole symmetry.
\item  Include dynamics, so as to study the dynamics of the OSM to  FL* phase. Note that the FL* phase will ultimately be unstable to the AFM* phase, but the energy scale is expected to be extremely small at small U.  
\item  Include a projective version, with different left and right wave functions. The right imposes translation invariance and  the left  the constraint.  That is,  we choose the right trial wave function to be the ground state of 
\begin{equation}
	\hat{H}_T^{R}  = -  t  \sum_{\langle \vec{i}, \vec{j} \rangle, \sigma } 
	  \left( \hat{\Psi}^{\dagger}_{\vec{i},\sigma} \hat{\Psi}^{\phantom{\dagger}}_{\vec{j},\sigma}    + h.c. \right)  
	  - h \sum_{\vec{i} } \left( \hat{X}_{\vec{i},\vec{i} +  \vec{a}_x }+  \hat{X}_{\vec{i},\vec{i} +  \vec{a}_y}  + \tau^{x}_{\vec{i}} \right)
\end{equation}
and the left one to be the ground state of
\begin{equation}
	\hat{H}_T^{L}  =  - U  \sum_{ \vec{i},  \sigma } 
	    e^{i \vec{Q} \cdot \vec{i} } \hat{\Psi}^{\dagger}_{\vec{i},\sigma} \hat{\Psi}^{\phantom{\dagger}}_{\vec{i},\sigma}  
	  - h \sum_{\vec{i} } \left( \hat{X}_{\vec{i},\vec{i} +  \vec{a}_x }+  \hat{X}_{\vec{i},\vec{i} +  \vec{a}_y}  + \tau^{x}_{\vec{i}} \right)
\end{equation}
with $\vec{Q} = ( \pi,\pi ) $. 
\end{itemize}  

