% Copyright (c) 2016 The ALF project.
% This is a part of the ALF project documentation.
% The ALF project documentation by the ALF contributors is licensed
% under a Creative Commons Attribution-ShareAlike 4.0 International License.
% For the licensing details of the documentation see license.CCBYSA.
% !TEX root = doc.tex

In its present form, the  auxiliary-field QMC code of the ALF project allows us to simulate a large class of non-trivial models, both efficiently and at minimal  programming cost.  ALF 2.0 contains many advanced functionalities, including a projective formulation, various updating schemes, better control of Trotter errors, predefined structures that facilitate reuse, a large class of models, continuous fields and, finally, stochastic analytical continuation code. Also the usability of the code has improved in comparison with ALF 1.0. In particular the \href{https://git.physik.uni-wuerzburg.de/ALF/pyALF}{pyALF} project provides a Python interface to the ALF which substantially facilitates running the code for established models.

There are further capabilities that we would like to see in future versions of ALF. Introducing time-dependent Hamiltonians, for instance, will require some rethinking, but will allow, for example, to access entanglement properties of interacting fermionic systems \cite{Broecker14,Assaad14,Assaad13a,Assaad15}. Moreover, the auxiliary field approach is not the only method to simulate fermionic systems.
It would be desirable to include additional lattice fermion algorithms such as the CT-INT \cite{Rubtsov05,Assaad07}.
Lastly, at the more technical level, improved IO (e.~g.~HDF5 support), post-processing, object oriented programming, as well as increased compatibility with other software projects are all certainly desirable improvements to look forward to. 
