% Copyright (c) 2016 The ALF project.
% This is a part of the ALF project documentation.
% The ALF project documentation by the ALF contributors is licensed
% under a Creative Commons Attribution-ShareAlike 4.0 International License.
% For the licensing details of the documentation see license.CCBYSA.

%-------------------------------------------------------------------------------------
\section{Conclusions and future directions}
%-------------------------------------------------------------------------------------

In its present form, the  ALF-project  allows to simulate a very large class of non-trivial models efficiently and at a minimal  programming cost.  There are many possible extensions which deserve to be considered in future releases.    The Hamiltonians we presently defining are imaginary time independent. This however, can be easily generalized  to time dependent Hamiltonians thus allowing, for example, to access  entanglement properties of interacting fermionic systems \cite{Broecker14,Assaad14,Assaad13a,Assaad15}. Generalizations to include global moves are equally desirable. This is a prerequisite to  play with recent ideas of self-learning algorithms  \cite{Xu16a} so as to  possibly avoid critical slowing down.  At present we are restricted to discrete fields such that  implementations  of  the long range Coulomb repulsion as introduced in \cite{Hohenadler14,Ulybyshev2013,Brower12} is not included in the package.   Extensions  to  continuous fields are certainly possible, but require an efficient upgrading scheme. Finally,  a ground state projective formulation   is equally desirable.
