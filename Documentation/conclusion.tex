% Copyright (c) 2016 The ALF project.
% This is a part of the ALF project documentation.
% The ALF project documentation by the ALF contributors is licensed
% under a Creative Commons Attribution-ShareAlike 4.0 International License.
% For the licensing details of the documentation see license.CCBYSA.
% !TEX root = doc.tex

In its present form, the  auxiliary field QMC code of the ALF project  allows to simulate a large class of non-trivial models, both efficiently and at minimal  programming cost.  ALF 2.0 contains many add ons including a projective formulation, various updating schemes, better control of Trotter errors, predefined structures that facilitate reuse, a large class of models, continuous fields,  and  finally inclusion  of a stochastic analytical continuation code.   
We have equally progressed in usability of the code. In particular the \href{https://git.physik.uni-wuerzburg.de/ALF/pyALF}{pyALF}   project provides a python interface to the ALF. 
Many further features are desirable.  Introducing time dependent Hamiltonians will require  some   rethinking  but will  allow, for example, to access  entanglement properties of interacting fermionic systems \cite{Broecker14,Assaad14,Assaad13a,Assaad15}.  The auxiliary field approach is not the only method to simulate fermonic systems.   In the future it would desirable to include further lattice fermion algorithms such as the CT-INT \cite{Rubtsov05,Assaad07}.   At the more technical level  improvements  in IO  (eg. HDF5), post-processing   object oriented  programming as well as  compatibility with other software projects is certainly desirable. 
