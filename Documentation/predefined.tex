% Copyright (c) 2016-2019 The ALF project.
% This is a part of the ALF project documentation.
% The ALF project documentation by the ALF contributors is licensed
% under a Creative Commons Attribution-ShareAlike 4.0 International License.
% For the licensing details of the documentation see license.CCBYSA.

% !TEX root = doc.tex

ALF includes modules providing predefined structures which the user can combine together or use as templates for defining new structures, namely: 
\begin{itemize}
	\item lattices and unit cells -- \texttt{Predefined\_Latt\_mod.F90}
	\item hopping Hamiltonians -- \texttt{Predefined\_Hop\_mod.F90 }
	\item interaction Hamiltonians -- \texttt{Predefined\_Int\_mod.F90}
	\item observables -- \texttt{Predefined\_Obs\_mod.F90 }
	\item trial wave functions -- \texttt{Predefined\_Trial\_mod.F90 }
\end{itemize}
which are defined using the data structures described in the Sec.~\ref{sec:imp}, as detailed below.


\subsection{Interaction Hamiltonians}

In its most general form, an interaction Hamiltonian expressed in terms of sums of perfect squares can be written, as presented in Section~\ref{sec:intro}, in the form %Eq.~\eqref{eqn:general_ham_v}:

\begin{align*}
\hat{\mathcal{H}}_{V} =  \sum\limits_{k=1}^{M_V}U_{k}
\left\{ \sum\limits_{\sigma=1}^{N_{\mathrm{col}}}
\sum\limits_{s=1}^{N_{\mathrm{fl}}} \left[ \left(
\sum\limits_{x,y}^{N_{\mathrm{dim}}} \hat{c}^{\dagger}_{x \sigma s}V_{xy}^{(k s)}\hat{c}^{\phantom\dagger}_{y \sigma s}\right)  +\alpha_{k s}  \right] \right\}^{2}
\equiv    \sum\limits_{k=1}^{M_V}U_{k}   \left(\hat{V}^{(k)} \right)^2.
\tag{\ref{eqn:general_ham_v}}
\end{align*}
The module \texttt{Predefined\_Int\_mod.F90} implements some of the most common of such interaction Hamiltonians.

\subsubsection{Hubbard interaction}

The SU(N) Hubbard interaction is given by 
\begin{align}
%\label{eqn_hubbard_sun}
\hat{\mathcal{H}}_{V} =
+ \frac{U}{2}\sum\limits_{x}\left[
\sum\limits_{\sigma=1}^{2}
\left(  c^{\dagger}_{x \sigma} c^{\phantom\dagger}_{x \sigma}  -1/2 \right) \right]^{2}\;,
\end{align} 
which defined by a single operator
We can make contact with the general form of the Hamiltonian by setting: 
$N_{\mathrm{fl}} = 1$, $N_{\mathrm{col}} \equiv \texttt{N\_SUN}     =2 $,   $M_T    =    1$,  $T^{(ks)}_{x y}   =  T_{x,y}$,  $M_V   =  N_{\text{unit-cell}} $,  $U_{k}       =   -\frac{U}{2}$, 
$V_{x y}^{(ks)} =  \delta_{x,y} \delta_{x,k}$,  $\alpha_{ks}   = - \frac{1}{2}  $ and $M_I       = 0 $.




\subsection{Template}

\red{Maybe discard this subsection.}\\
	
\red{Maybe merge this with Chapter on Examples and Models/Model Classes.\\}

 \red{TODO} Go through everything one has to defined/set in order to define a new Hamiltonian. \red{TODO}\\

We'd perhaps want to provide a \emph{minimum} Hamiltonian, probably written in pseudo-code, from which one could write their own.\\ \\

\noindent A \emph{complete} list of all the code that would have to be changed for the \emph{most general} case could read as:

\begin{itemize}
	\item Hamiltonian -- one-body, imaginary-time propagator for a given configuration of HS and  Ising fields
	\item Table~\ref{table:hamiltonian}:
	\begin{itemize}
		\item \texttt{Ham\_Set}
		\item \texttt{Ham\_V}
		\item \texttt{S0}
		\item \texttt{Setup\_Ising\_action}
		\item \texttt{Global\_move}
		\item \texttt{Delta\_S0\_global}
		\item \texttt{Global\_move\_tau}
		\item \texttt{Alloc\_obs}
		\item \texttt{Obser}
		\item \texttt{ObserT}
	\end{itemize}
	\item Check Table~\ref{table:operator} (Operator type).
	\item Table~\ref{table:lattice} and \ref{table:unit_cell}:
	\begin{itemize}
		\item \texttt{Lattice\%a1\_p}, \texttt{Lattice\%a2\_p}
		\item \texttt{Lattice\%L1\_p}, \texttt{Lattice\%L2\_p}
		\item \texttt{Lattice\%N}
		\item \texttt{Norb}
		\item \texttt{N\_coord}
		\item \texttt{Orb\_pos(1..Norb,2)}
	\end{itemize}
\end{itemize}

