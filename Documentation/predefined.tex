% Copyright (c) 2016-2020 The ALF project.
% This is a part of the ALF project documentation.
% The ALF project documentation by the ALF contributors is licensed
% under a Creative Commons Attribution-ShareAlike 4.0 International License.
% For the licensing details of the documentation see license.CCBYSA.

% !TEX root = doc.tex

\label{Predefined_chap}
The ALF package includes predefined structures, which the user can combine together or use as templates for defining new ones. Using the data types defined in the Sec.~\ref{sec:imp} the following modules are available:
\begin{itemize}
	\item lattices and unit cells -- \texttt{Predefined\_Latt\_mod.F90}
	\item hopping Hamiltonians -- \texttt{Predefined\_Hop\_mod.F90 }
	\item interaction Hamiltonians -- \texttt{Predefined\_Int\_mod.F90}
	\item observables -- \texttt{Predefined\_Obs\_mod.F90 }
	\item trial wave functions -- \texttt{Predefined\_Trial\_mod.F90 }
\end{itemize}
which we describe in the remaining of this section.



%%%%%%%%%%%%%%%%%%%%%%%%%%%%
% !TEX root = doc.tex
% Copyright (c) 2017-2020 The ALF project.
% This is a part of the ALF project documentation.
% The ALF project documentation by the ALF contributors is licensed
% under a Creative Commons Attribution-ShareAlike 4.0 International License.
% For the licensing details of the documentation see license.CCBYSA.
%
%-----------------------------------------------------------------------------------
\subsection{Predefined lattices} \label{sec:predefined_lattices}
%-----------------------------------------------------------------------------------


The types \texttt{Lattice} and \texttt{Unit\_cell}, described in Section~\ref{sec:latt}, allow us to define arbitrary one- and two-dimensional Bravais lattices. The subroutine \texttt{Predefined\_Latt} provides some of the most common lattices, as described bellow.

The subroutine is called as:
\begin{lstlisting}[style=fortran]
Predefined_Latt(Lattice_type, L1, L2, Ndim, List, Invlist, Latt, Latt_Unit)
\end{lstlisting}
which returns a lattice of size \texttt{L1$\times$L2} of the given \texttt{Lattice\_type}, as detailed in Table~\ref{table:predefined_lattices}.
%
\begin{center}
\begin{table}[h]
	\begin{tabular}{@{} p{0.13\columnwidth}  p{0.1\columnwidth} p{0.09\columnwidth} p{0.58\columnwidth}  @{}}
		\toprule
		Argument                 & Type       & Role   & Description \\
		\midrule
		\texttt{Lattice\_type}   & String     & Input  & lattice configuration, which can take the values:
		%\vspace{-\topsep}
		\begin{itemize}
			\setlength{\itemsep}{0pt} \setlength{\parskip}{0pt} \setlength{\parsep}{0pt}
			\item[-] Square
			\item[-] Honeycomb
			\item[-] Pi\_Flux \vspace{-1.4\topsep}
		\end{itemize} \\
	    %\vspace{-\topsep} \\ \vspace{-\topsep}
		\texttt{L1}, \texttt{L2} & Integer    & Input  & lattice sizes (set \texttt{L2=1} for 1D lattices)\\
		\texttt{Ndim}            & Integer    & Output & total number of orbitals\\
		\texttt{List}            & Integer    & Output & for every site index $\texttt{I} \in [1,\texttt{Ndim}]$, stores the corresponding lattice position, \texttt{List(I,1)}, and the (local) orbital index, \texttt{List(I,2)}\\
		\texttt{Invlist}         & Integer    & Output &  for every $\texttt{lattice\_position} \in [1,\texttt{Latt\%N}]$ and $\texttt{orbital} \in [1,\texttt{Norb}]$ stores the corresponding site index \texttt{I(lattice\_position,orbital)}\\
		\texttt{Latt}            & Lattice    & Output &  sets the lattice\\
		\texttt{Latt\_Unit}      & Unit\_cell & Output & sets the unit cell\\
		\bottomrule
	\end{tabular}
\caption{Arguments of the subroutine \texttt{Predefined\_Latt}.}		\label{table:predefined_lattices}
\end{table}
\end{center}






%%%%%%%%%%%%%%%%%%%%%%%%%%%%

%%%%%%%%%%%%%%%%%%%%%%%%%%%%
% !TEX root = doc.tex
% Copyright (c) 2017 The ALF project.
% This is a part of the ALF project documentation.
% The ALF project documentation by the ALF contributors is licensed
% under a Creative Commons Attribution-ShareAlike 4.0 International License.
% For the licensing details of the documentation see license.CCBYSA.
%
%-----------------------------------------------------------------------------------
\subsection{Generic hopping matrix elements}\label{sec:generic_hopping}
%-----------------------------------------------------------------------------------


Here we compute the hopping matrix element  between two sites of a given lattice  in the presence of twisted boundary conditions and  orbital magnetic field. 
The generic Hopping Hamiltonian will read: 
\begin{equation}
	   \hat{H}_T = \sum_{(i,\delta), (j,\delta'), s, \sigma}    T_{(i,\delta), (j,\delta')}^{(s)}    c^{\dagger}_{(i,\delta),s,\sigma }   e^{\frac{2 \pi i}{\Phi_0} \int_{i + \delta}^{j + \delta'}  \vec{A}(\vec{l})  d \vec{l}} c^{}_{(j,\delta'),s,\sigma }
\end{equation}
with boundary conditions 
\begin{equation}
	c^{\dagger}_{(i + L_i,\delta) ,s,\sigma }   =  e^{- 2 \pi i\frac{\Phi_i}{\Phi_0}} \, e^{\frac{2 \pi i }{\Phi_0} \chi_{L_1} ( i + \delta ) } \, c^{\dagger}_{(i,\delta) ,s,\sigma }.
\end{equation}
The vector potential accounts for an orbital magnetic field that is implemented  in the Landau  gauge:  $\vec{A}(\vec{x})  =  -B(y,0,0) $ with $ \vec{x} = (x,y,z)$. $\Phi_0$ corresponds to the flux  quanta and the scalar function $\chi$ is defined  through as:
\begin{equation}
	\vec{A}( \vec{x} + \vec{L}_{i} )  = \vec{A}( \vec{x} )   +  \vec{\nabla} \chi_{L_{\alpha}}(\vec{x}). 
\end{equation}

 Provided that the bare hopping Hamiltonian, $T$,  is invariant under lattice translations, $\hat{H}_T$ commutes with magnetic translations  that satisfy the  Algebra: 
\begin{equation}
	\hat{T}_{\vec{a}} \hat{T}_{\vec{b}} =  e^{ \frac{2 \pi i}{\Phi_0}   \vec{B} \cdot \left( \vec{a} \times \vec{b} \right) }  \hat{T}_{\vec{b}} \hat{T}_{\vec{a}}. 
\end{equation}
On the  torus, the uniqueness of the wave functions requires that  $\hat{T}_{\vec{L}_1} \hat{T}_{\vec{L}_2}  =   \hat{T}_{\vec{L}_2} \hat{T}_{\vec{L}_1} $ such
that
\begin{equation}
	 \frac{\vec{B} \cdot \left( \vec{a} \times \vec{b}  \right) }{\Phi_0 } = N_{\Phi}   
\end{equation}
with  $N_\Phi $ an integer  The variable \texttt{N\_Phi},   specified in the parameter file,   denotes the number of flux quanta piercing the lattice.    The variables \texttt{Phi\_1}  and   \texttt{Phi\_2} also   in the parameter file denote  the twists  -- in units of the flux quanta  --  along the $\vec{L}_1$ and  $\vec{L}_2$ directions.     There are gauge  equivalent ways to insert the  twist in the boundary conditions. In the above we  have inserted   twist as a boundary condition such  for example setting  \texttt{Phi\_1=0.5}  corresponds to anti-periodic boundary conditions along the $L_1$  axis.   Alternatively we  can  consider the 
Hamiltonian:
\begin{equation}
	   \hat{H}_T = \sum_{(i,\delta), (j,\delta'), s, \sigma}    T_{(i,\delta), (j,\delta')}^{(s)}    \tilde{c}^{\dagger}_{(i,\delta),s,\sigma }   e^{\frac{2 \pi i}{\Phi_0} \int_{i + \delta}^{j + \delta'} \left(  \vec{A}(\vec{l})  + \vec{A}_{\phi} \right)  d \vec{l}} \tilde{c}^{}_{(j,\delta'),s,\sigma }
\end{equation}
with boundary conditions 
\begin{equation}
	\tilde{c}^{\dagger}_{(i + L_i,\delta) ,s,\sigma }   =  e^{\frac{2 \pi i }{\Phi_0} \chi_{L_1} ( i + \delta ) } \, \tilde{c}^{\dagger}_{(i,\delta) ,s,\sigma }.
\end{equation}
Here 
\begin{equation}
	\vec{A}_{\phi} =\frac{  \phi_1  |\vec{a}_1|} { 2 \pi |\vec{L}_1| } \vec{b}_1 +  \frac{  \phi_2  |\vec{a}_2|}{2 \pi  |\vec{L}_2| } \vec{b}_2
\end{equation}
and $\vec{b}_i$  correspond to the reciprocal lattice vectors satisfying  $ \vec{a}_i  \cdot  \vec{b}_j  = 2 \pi \delta_{i,j} $.   The logical variable $\texttt{bulk} $ chooses between these two  gauge equivalent ways  on inserting the twist angle. If \texttt{bulk=\.true\.}    then  we use periodic boundary conditions  --  in the absence of an orbital field -- otherwise  twisted boundaries are used.  

%%%%%%%%%%%%%%%%%%%%%%%%%%%%

%%%%%%%%%%%%%%%%%%%%%%%%%%%%
% !TEX root = doc.tex
% Copyright (c) 2017-2020 The ALF project.
% This is a part of the ALF project documentation.
% The ALF project documentation by the ALF contributors is licensed
% under a Creative Commons Attribution-ShareAlike 4.0 International License.
% For the licensing details of the documentation see license.CCBYSA.
%
%-----------------------------------------------------------------------------------
\subsection{Interaction vertices} \label{sec:interaction_vertices}
%-----------------------------------------------------------------------------------

In its most general form, an interaction Hamiltonian expressed in terms of sums of perfect squares can be written, as presented in Section~\ref{sec:intro}, as a sum of $M_V$ vertices: %Eq.~\eqref{eqn:general_ham_v}:

\begin{align*}
\hat{\mathcal{H}}_{V} &=  \sum\limits_{k=1}^{M_V}U_{k}
\left\{ \sum\limits_{\sigma=1}^{N_{\mathrm{col}}}
\sum\limits_{s=1}^{N_{\mathrm{fl}}} \left[ \left(
\sum\limits_{x,y}^{N_{\mathrm{dim}}} \hat{c}^{\dagger}_{x \sigma s}V_{xy}^{(k s)}\hat{c}^{\phantom\dagger}_{y \sigma s}\right)  +\alpha_{k s}  \right] \right\}^{2}
\equiv    \sum\limits_{k=1}^{M_V}U_{k}   \left(\hat{V}^{(k)} \right)^2 \tag{\ref{eqn:general_ham_v}}\\
&\equiv    \sum\limits_{k=1}^{M_V}\hat{\mathcal{H}}_V^{(k)}.
\end{align*}
The module \texttt{Predefined\_Int\_mod.F90} implements some of the most common of such interaction vertices $\hat{\mathcal{H}}_V^{(k)}$, as detailed in the remaining of this section, where we drop the superscript $(k)$ when unambiguous.

\subsubsection{The $SU(N)$ Hubbard interaction}

The $SU(N)$ Hubbard interaction on a given site $i$ is given by 
\begin{align}
%\label{eqn_hubbard_sun}
\hat{\mathcal{H}}_{V,i} =
+ \frac{U}{N_{\mathrm{col}}}\left[
\sum\limits_{\sigma=1}^{N_{\mathrm{col}}}
\left(  c^{\dagger}_{i \sigma} c^{\phantom\dagger}_{i\sigma}  -1/2 \right) \right]^{2}.
\end{align} 
Assuming that no other term in the Hamiltonian breaks the $SU(N) $  color symmetry,  then this interaction term  conveniently corresponds to  a single 
operator  which is defined in the subroutine \texttt{Predefined\_Int\_U\_SUN}.   
%which corresponds to the general form of Eq.~\eqref{eqn:general_ham_v} by setting: 
%$N_{\mathrm{fl}} = 1$,  $M_V = N_{\text{unit-cell}} $,  $U_{k} =  -\frac{U}{N_{\mathrm{col}}}$,  $V_{x y}^{(ks)} =  \delta_{x,y} \delta_{x,k}$, and $\alpha_{ks} = -\frac{1}{2}$; and which is defined in the subroutine \texttt{Predefined\_Int\_U\_SUN} by a single operator:

\begin{lstlisting}[style=fortran]

Op%P(1)   = i
Op%O(1,1) = cmplx(1.d0,  0.d0, kind(0.D0))
Op%alpha  = cmplx(-0.5d0,0.d0, kind(0.D0))
Op%g      = SQRT(CMPLX(-DTAU*U/(DBLE(N_SUN)), 0.D0, kind(0.D0))) 
Op%type   = 2

\end{lstlisting}

To relate to  Eq.~\eqref{eqn:general_ham_v} we have,   $V_{x y}^{(is)} =  \delta_{x,y} \delta_{x,i}$, $\alpha_{is} = -\frac{1}{2}$ and $U_{k} =  \frac{U}{N_{\mathrm{col}}}$.   Here  the flavor index, $s$,  plays no role. 


\subsubsection{The $M_z$-Hubbard interaction}

The $M_z$-Hubbard interaction is given by 
\begin{align}
%\label{eqn_hubbard_Mz}
\hat{\mathcal{H}}_{V} = - \frac{U}{2}\sum\limits_{i}\left[
c^{\dagger}_{i \uparrow} c^{\phantom\dagger}_{i \uparrow}  -   c^{\dagger}_{i \downarrow} c^{\phantom\dagger}_{i \downarrow}  \right]^{2},
\end{align} 
which corresponds to the general form of Eq.~\eqref{eqn:general_ham_v} by setting: 
$N_{\mathrm{fl}} = 2$, $N_{\mathrm{col}} \equiv \texttt{N\_SUN} =1 $,  $M_V =  N_{\text{unit-cell}} $,  $U_{k} = \frac{U}{2}$, 
$V_{x y}^{(i, s=1)} =  \delta_{x,y} \delta_{x,i}  $,  $V_{x y}^{(i, s=2)} =  - \delta_{x,y} \delta_{x,i}  $, and $\alpha_{is}   = 0  $; and which is defined in the subroutine \texttt{Predefined\_Int\_U\_MZ} by two operators:
\begin{lstlisting}[style=fortran]

Op_up%P(1)   = I
Op_up%O(1,1) = cmplx(1.d0, 0.d0, kind(0.D0))
Op_up%alpha  = cmplx(0.d0, 0.d0, kind(0.D0))
Op_up%g      = SQRT(CMPLX(DTAU*U/2.d0, 0.D0, kind(0.D0))) 
Op_up%type   = 2

Op_do%P(1)   = I
Op_do%O(1,1) = cmplx(1.d0, 0.d0, kind(0.D0))
Op_do%alpha  = cmplx(0.d0, 0.d0, kind(0.D0))
Op_do%g      = -SQRT(CMPLX(DTAU*U/2.d0, 0.D0, kind(0.D0))) 
Op_do%type   = 2

\end{lstlisting}


\subsubsection{The $SU(N)$ $V$ interaction}

The interaction term of the generalized t-V model, given by 
\begin{align}
\hat{\mathcal{H}}_{V,i,j} =
-\frac{V}{N_\mathrm{col}}\left[ \sum_{\sigma=1}^{N_\mathrm{col}}\left( c^{\dagger}_{i \sigma} c^{\phantom\dagger}_{j \sigma} + c^{\dagger}_{j \sigma} c^{\phantom\dagger}_{i \sigma} \right) \right]^2,
\end{align} 
is coded in the subroutine \texttt{Predefined\_Int\_V\_SUN} by a single symmetric operator:
\begin{lstlisting}[style=fortran]

Op%P(1)   = I
Op%P(2)   = J
Op%O(1,2) = cmplx(1.d0 ,0.d0, kind(0.D0)) 
Op%O(2,1) = cmplx(1.d0 ,0.d0, kind(0.D0))
Op%g      = SQRT(CMPLX(DTAU*V/real(N_SUN,kind(0.d0)), 0.D0, kind(0.D0))) 
Op%alpha  = cmplx(0.d0, 0.d0, kind(0.D0))
Op%type   = 2

\end{lstlisting}


\subsubsection{The Fermion-Ising coupling}

The interaction between the Ising and a fermion degree of freedom, given by
\begin{align}
%\label{eqn_hubbard_sun_Ising}
\hat{\mathcal{H}}_{V,i,j} =
\hat{Z}_{i,j} \xi  \sum_{\sigma=1}^{N_\mathrm{col}}\left( c^{\dagger}_{i \sigma} c^{\phantom\dagger}_{j \sigma} + c^{\dagger}_{j \sigma} c^{\phantom\dagger}_{i \sigma} \right),
\end{align} 
where $\xi$ determines the coupling strength, is implemented in the subroutine \texttt{Predefined\_Int\_Ising\_SUN}:
\begin{lstlisting}[style=fortran]

Op%P(1)   = I
Op%P(2)   = J
Op%O(1,2) = cmplx(1.d0 ,0.d0, kind(0.D0)) 
Op%O(2,1) = cmplx(1.d0 ,0.d0, kind(0.D0)) 
Op%g      = cmplx(-dtau*xi,0.D0,kind(0.D0))
Op%alpha  = cmplx(0d0,0.d0, kind(0.D0)) 
Op%type   = 1

\end{lstlisting}



\subsubsection{The long-Range Coulomb repulsion}

The Long-Range Coulomb (LRC) interaction can be written as
\begin{align}
\hat{\mathcal{H}}_{V} =
\frac{1} { N } \sum_{\vec{i},\vec{j}}  \left(  \hat{n}_{\vec{i}} -  \frac{N}{2}  \right)  V_{\vec{i},\vec{j}} \left(  \hat{n}_{\vec{j}} -  \frac{N}{2}  \right), 
\end{align} 
where
\begin{align}
\hat{n}_{\vec{i}} = \sum_{\sigma=1}^{N}  \hat{c}^{\dagger}_{\vec{i},\sigma}  \hat{c}^{}_{\vec{i},\sigma}
\end{align} 
and
\begin{equation}
V_{\vec{i}, \vec{j}}   =   U \left\{
\begin{array}{ll}  
1          &   \text{ if } \vec{i} - \vec{j}    = 0 \\
\frac{\alpha   \;   d_\mathrm{min}}{ |   \vec{i} - \vec{j} | } &     \text{ otherwise }
\end{array}
\right. .
\end{equation}
Here $d_\mathrm{min}$ is the minimal distance between two orbitals.     The code uses the following  HS decomposition:
\begin{equation}
e^{-\Delta \tau \hat{H}_{V,k} }  =  \int \prod_{\vec{i}} d \phi_{\vec{i}}   e^{ - \frac{N \Delta \tau} {4} \phi_{\pmb{i}} V^{-1}_{\pmb{i},\pmb{j}}  \phi_{\pmb{j}} - \sum_{\pmb{i}}  i \Delta \tau \phi_i \left( n_{i} - \frac{N}{2} \right) }.
\end{equation}

The implementation follows Ref.~\cite{Hohenadler14}  but now supports various lattice geometries.    The definition of  the Coulomb repulsion is as follows. 
A general lattice site  \texttt{I,n}   where \texttt{I: 1...Latt\%N} is the unit cell and \texttt{ n = 1 ...Latt\_unit\%NORB}  the orbital  is given by: 
\begin{lstlisting}[style=fortran]
X_p(:) = Latt%list(I,1)*latt%a1_p(:)  + Latt%list(I,2)*latt%a2_p(:) 
+   Latt_unit%Orb_pos_p(no_j,:)
\end{lstlisting}
or in more compact notation $ \vec{i}  + \vec{\delta}_i $.   By definition \texttt{Latt\_unit\%Orb\_pos\_p(1,:)=0}.
The Coulomb repulsion between points   $ \vec{i}  + \vec{\delta}_i $   and $ \vec{j}  + \vec{\delta}_j $   reads: 
\begin{equation}
V(\vec{i}  + \vec{\delta}_i ,  \vec{j}  + \vec{\delta}_j  )  =  \frac{U d_\mathrm{min} \alpha}{  |  \overline{\vec{i} - \vec{j}} + \vec{\delta}_i - \vec{\delta}_j  |}.
\end{equation}
Here  we use periodic boundary conditions such that  $\overline{\vec{i} - \vec{j}}$  is an element of the real space lattice. Note that this is encoded in the array \texttt{Latt\%imj(I,J)}.

The LRC interaction is implemented in the subroutine \texttt{Predefined\_Int\_LRC}:
\begin{lstlisting}[style=fortran]

Op%P(1)   = I
Op%O(1,1) = cmplx(1.d0  ,0.d0, kind(0.D0))
Op%alpha  = cmplx(-0.5d0,0.d0, kind(0.D0))
Op%g      = cmplx(0.d0  ,Dtau, kind(0.D0)) 
Op%type   = 3

\end{lstlisting}


\subsubsection{The $J_z$-$J_z$ interaction}

Another predefined vertex is:
\begin{align}
\hat{\mathcal{H}}_{V,i,j} =
- \frac{|J_z|}{2}  \left( S^{z}_i - \sgn|J_z| S^{z}_j \right)^2 =
J_z  S^{z}_i  S^{z}_j  - \frac{|J_z|}{2} (S^{z}_i)^2 - \frac{|J_z|}{2}(S^{z}_j)^2 
\end{align} 
which, if particle fluctuations are frozen on the $i$ and $j$ sites, then $(S^{z}_i)^2 = 1/4$ and the interactions corresponds to a $J_z$-$J_z$ ferro or antiferro coupling.

The implementation of the interaction in \texttt{Predefined\_Int\_Jz} defines two operators:
\begin{lstlisting}[style=fortran]

Op_up%P(1)   = I
Op_up%P(2)   = J
Op_up%O(1,1) = cmplx(1.d0,              0.d0, kind(0.D0))
Op_up%O(2,2) = cmplx(-Jz/Abs(Jz),       0.d0, kind(0.D0))
Op_up%alpha  = cmplx(0.d0,              0.d0, kind(0.D0))
Op_up%g      = SQRT(CMPLX(DTAU*Jz/8.d0, 0.d0, kind(0.D0))) 
Op_up%type   = 2

Op_do%P(1)   = I
Op_do%P(2)   = J
Op_do%O(1,1) = cmplx(1.d0,               0.d0, kind(0.d0))
Op_do%O(2,2) = cmplx(-Jz/Abs(Jz),        0.d0, kind(0.d0))
Op_do%alpha  = cmplx(0.d0,               0.d0, kind(0.d0))
Op_do%g      = -SQRT(CMPLX(DTAU*Jz/8.d0, 0.d0, kind(0.d0))) 
Op_do%type   = 2

\end{lstlisting}

%%%%%%%%%%%%%%%%%%%%%%%%%%%%

%%%%%%%%%%%%%%%%%%%%%%%%%%%%
% !TEX root = doc.tex
% Copyright (c) 2017-2020 The ALF project.
% This is a part of the ALF project documentation.
% The ALF project documentation by the ALF contributors is licensed
% under a Creative Commons Attribution-ShareAlike 4.0 International License.
% For the licensing details of the documentation see license.CCBYSA.
%
%-----------------------------------------------------------------------------------
\subsection{Predefined observables} \label{sec:predefined_observales}
%-----------------------------------------------------------------------------------

The types \texttt{Obser\_Vec} and \texttt{Obser\_Latt} described in Section~\ref{sec:obs} handle arrays of scalar observables and correlation functions with lattice symmetry respectively.
The module \texttt{Predefined\_Obs} provides a set of standard equal-time and time-displaced observables, as described below.

The predefined measurements methods take as input Green functions \texttt{GR}, \texttt{GT0}, \texttt{G0T}, \texttt{G00}, and \texttt{GTT}, defined in Sec.~\ref{sec:EqualTimeobs} and~\ref{sec:TimeDispObs}, as well as \texttt{N\_SUN}, time slice \texttt{Ntau}, lattice information, and so on -- see Table~\ref{table:predefined_obs}.
%
\begin{table}[h]
	\begin{center}
		\begin{tabular}{@{} p{0.26\columnwidth}  p{0.14\columnwidth} @{\hspace{1.5ex}} p{0.55\columnwidth}  @{}}
			\toprule
			Argument                      & Type                 & Description \\
			\midrule
			\texttt{Latt}                 & \texttt{Lattice}     & Lattice as a variable of type \texttt{Lattice}, see Sec.~\ref{sec:latt}\\
			\texttt{Latt\_Unit}           & \texttt{Unit\_cell}  & Unit cell as a variable of type \texttt{Unit\_cell}, see Sec.~\ref{sec:latt}\\
			\texttt{List(Ndim,2)}         & \texttt{int}         & For every site index $\texttt{I}$, stores the corresponding lattice position, \texttt{List(I,1)}, and the (local) orbital index, \texttt{List(I,2)}\\
			\texttt{NT}                   & \texttt{int}         & Imaginary time $\tau$\\
			\texttt{GR(Ndim,Ndim,N\_FL)}  & \texttt{cmplx}       & Equal-time Green function $\texttt{GR(i,j,s)}  = \langle c^{\phantom{\dagger}}_{i,s} c^{\dagger}_{j,s}  \rangle$\\
			\texttt{GRC(Ndim,Ndim,N\_FL)} & \texttt{cmplx}       & $\texttt{GRC(i,j,s)}  = \langle c^{\dagger}_{i,s} c^{\phantom{\dagger}}_{j,s}  \rangle  =  \delta_{i,j} - \texttt{GR(j,i,s)}$\\
			\texttt{GT0(Ndim,Ndim,N\_FL)} & \texttt{cmplx}       & Time-displaced Green function $\langle \langle \mathcal{T} \hat{c}^{\phantom\dagger}_{i,s}(\tau) \hat{c}^{\dagger}_{j,s}(0) \rangle \rangle$\\
			\texttt{G0T(Ndim,Ndim,N\_FL)} & \texttt{cmplx}       & Time-displaced Green function $\langle \langle \mathcal{T} \hat{c}^{\phantom\dagger}_{i,s}(0) \hat{c}^{\dagger}_{j,s}(\tau) \rangle \rangle $\\
			\texttt{G00(Ndim,Ndim,N\_FL)} & \texttt{cmplx}       & Time-displaced Green function $\langle \langle \mathcal{T} \hat{c}^{\phantom\dagger}_{i,s}(0) \hat{c}^{\dagger}_{j,s}(0) \rangle \rangle $\\
			\texttt{GTT(Ndim,Ndim,N\_FL)} & \texttt{cmplx}       & Time-displaced Green function $\langle \langle \mathcal{T} \hat{c}^{\phantom\dagger}_{i,s}(\tau) \hat{c}^{\dagger}_{j,s}(\tau) \rangle \rangle $\\
			\texttt{N\_SUN}               & \texttt{int}         & Number of fermion colors $N_{\mathrm{col}}$\\
			\texttt{ZS}                   & \texttt{cmplx}       & $\texttt{ZS} = \text{sign}(C)$, see Sec.~\ref{sec:obs}\\
			\texttt{ZP}                   & \texttt{cmplx}       & $\texttt{ZP} = e^{-S(C)}/\Re \left[e^{-S(C)} \right]$, see Sec.~\ref{sec:obs}\\
			\texttt{Obs}                  & \texttt{Obser\_Latt} & \textbf{Output}: one or more measurement result\\
			\bottomrule
		\end{tabular}
		\caption{Arguments taken by the subroutines in the module \texttt{Predefined\_Obs}. Note that a given method makes use of only a subset of this list, as specified in their calls described below.
		   Note also that we use the superindex $i = (\ve{i}, n_{\ve{i}})$  where $\ve{i}$ denotes the unit cell and $n_{\ve{i}}$ the orbital. }		\label{table:predefined_obs}
	\end{center}
\end{table}


\subsubsection{Equal-time $SU(N)$ spin-spin correlations}
\label{SU_N_equal_time.sec}
A measurement of $SU(N)$ spin-spin correlations can be obtained by:
\begin{lstlisting}[style=fortran]
Call Predefined_Obs_eq_SpinSUN_measure(Latt, Latt_unit, List, GR, GRC, N_SUN, ZS, ZP, Obs)
\end{lstlisting}

If \texttt{N\_FL = 1} then  this routine returns
\begin{align}
\texttt{Obs}(\ve{i}-\ve{j},n_{\ve{i}}, n_{\ve{j}})  = \frac{2N}{N^2-1}\sum_{a=1}^{N^2 - 1}  \langle \langle \ve{\hat{c}}^{\dagger}_{\vec{i}, n_{\ve{i}}} T^a \ve{\hat{c}}^{\phantom\dagger}_{\vec{i},n_{\ve{i}}}   \;   \ve{\hat{c}}^{\dagger}_{\vec{j},n_{\ve{j}}} T^a  \ve{\hat{c}}^{\phantom\dagger}_{\vec{j},n_{\ve{j}}}\rangle\rangle_C,
\end{align}
where $T^a$ are the generators of $SU(N)$ satisfying the normalization conditions  $\Tr [ T^a  T^b ]= \delta_{a,b}/2$ , $\Tr [ T^a ] = 0 $,    
$ \ve{\hat{c}}^{\dagger}_{\ve{j},n_{\ve{j}}}   =  \left(   \hat{c}^{\dagger}_{\ve{j}, n_{\ve{j}},1},   \cdots, \hat{c}^{\dagger}_{\ve{j},n_{\ve{j}},N}  \right)    $     is an N-flavored spinor,  
$\ve{j}$   corresponds to the unit-cell index and $n_{\ve{j}}$    labels the orbital.

Using  Wick's theorem,  valid for a given  configuration of   fields,  we  obtain
\begin{multline}
 \texttt{Obs} =   \frac{2N}{N^2-1}\sum_{a=1}^{N^2 - 1}   \sum_{\alpha,\beta,\gamma, \delta = 1}^{N}     T^a_{\alpha,\beta}  T^a_{\gamma,\delta}   \times  \\
  \left( \langle \langle \hat{c}^{\dagger}_{\vec{i},n_{\ve{i}},\alpha} \hat{c}^{\phantom\dagger}_{\vec{i},n_{\ve{i}}, \beta }  \rangle\rangle_C  \langle \langle     \hat{c}^{\dagger}_{\vec{j}, n_{\ve{j}},\gamma}  \hat{c}^{\phantom\dagger}_{\vec{j},n_{\ve{j}},\delta}\rangle\rangle_C        +  
  \langle \langle \hat{c}^{\dagger}_{\vec{i},  n_{\ve{i}},\alpha}   \hat{c}^{\phantom\dagger}_{\vec{j},  n_{\ve{j}},\delta}\rangle\rangle_C  \langle \langle   \hat{c}^{\phantom\dagger}_{\vec{i},   n_{\ve{i}}, \beta }   \hat{c}^{\dagger}_{\vec{j},   n_{\ve{j}}, \gamma} \rangle\rangle_C 
   \right).
% \\
%\sum_{a=1}^{N^2 - 1 }  \left\langle \hat{c}^{\dagger}_i T^a c_i  c^{\dagger}_j T^a  c_j\right\rangle   = \sum_{a} \Tr{T^a T^a} \langle c^{\dagger}_i c_j\rangle \langle c_i c^{\dagger}_j\rangle = \frac{N^2 -1}{2} \langle c^{\dagger}_i c_j\rangle \langle c_i c^{\dagger}_j\rangle.
\end{multline}
For this $SU(N)$ symmetric code, the  Green function  is diagonal  in the spin  index and spin independent: 
\begin{equation}
 \langle \langle \hat{c}^{\dagger}_{\vec{i},  n_{\ve{i}},\alpha} \hat{c}^{}_{\vec{j},  n_{\ve{j}}, \beta }  \rangle\rangle_C  =  \delta_{\alpha,\beta}  \langle \langle \hat{c}^{\dagger}_{\vec{i},  n_{\ve{i}}} \hat{c}^{}_{\vec{j} ,  n_{\ve{j}}}  \rangle\rangle_C.
\end{equation}
Hence, 
\begin{align}
 \texttt{Obs} &= \begin{multlined}[t] \frac{2N}{N^2-1}\sum_{a=1}^{N^2 - 1} \left(   
\left[   \text{Tr} T^{a}\right]^2 \langle \langle \hat{c}^{\dagger}_{\vec{i},  n_{\ve{i}} } \hat{c}^{\phantom\dagger}_{\vec{i} ,  n_{\ve{i}}}  \rangle\rangle_C       \langle \langle \hat{c}^{\dagger}_{\vec{j} ,  n_{\ve{j}}} \hat{c}^{\phantom\dagger}_{\vec{j} ,  n_{\ve{j}}}  \rangle\rangle_C   \right.\\   
 + \left.   \text{Tr}\left[ T^{a}  T^{a} \right]    \langle \langle \hat{c}^{\dagger}_{\vec{i},  n_{\ve{i}}}   \hat{c}^{\phantom\dagger}_{\vec{j},  n_{\ve{j}}}\rangle\rangle_C  \langle \langle   \hat{c}^{\phantom\dagger}_{\vec{i},  n_{\ve{i}}}   \hat{c}^{\dagger}_{\vec{j},  n_{\ve{j}}} \rangle\rangle_C  \right)   \end{multlined}
 \nonumber \\
&=   N       \langle \langle \hat{c}^{\dagger}_{\vec{i},  n_{\ve{i}}}   \hat{c}^{\phantom\dagger}_{\vec{j},  n_{\ve{j}}}\rangle\rangle_C  \langle \langle   \hat{c}^{\phantom\dagger}_{\vec{i},  n_{\ve{i}}}   \hat{c}^{\dagger}_{\vec{j},  n_{\ve{j}}} \rangle\rangle_C .
\end{align}

\subsubsection{Equal-time spin correlations}

A measurement of the equal-time spin correlations can be obtained by:
\begin{lstlisting}[style=fortran]
Call Predefined_Obs_eq_SpinMz_measure(Latt, Latt_unit, List, GR, GRC, N_SUN, ZS, ZP,
                                      ObsZ, ObsXY, ObsXYZ)
\end{lstlisting}
If \texttt{N\_FL=2} and \texttt{N\_SUN=1}, then the routine returns:
\begin{align}
&\texttt{ObsZ}\left(\ve{i}-\ve{j}, n_{\ve{i}},  n_{\ve{j}} \right)  =  \begin{multlined}[t]  4 \langle \langle \ve{\hat{c}}^{\dagger}_{\ve{i}, n_{\ve{i}}} S^z \ve{\hat{c}}^{\phantom\dagger}_{\ve{i}, n_{\ve{i}} }   \;  \ve{\hat{c}}^{\dagger}_{\ve{j}, n_{\ve{j}}} S^z  \ve{\hat{c}}^{\phantom\dagger}_{\ve{j}, n_{\ve{j}}}\rangle \rangle_{C} \\
-   {\color{red} 4 \langle \langle \ve{\hat{c}}^{\dagger}_{\ve{i}, n_{\ve{i}}} S^z \ve{\hat{c}}^{\phantom\dagger}_{\ve{i}, n_{\ve{i}} } \rangle \rangle_{C}  \langle \langle \  \ve{\hat{c}}^{\dagger}_{\ve{j}, n_{\ve{j}}} S^z  \ve{\hat{c}}^{\phantom\dagger}_{\ve{j}, n_{\ve{j}}}\rangle \rangle_{C} },  \end{multlined} \nonumber \\  
&\texttt{ObsXY}\left(\ve{i}-\ve{j}, n_{\ve{i}},  n_{\ve{j}} \right)  =  2 \left( \langle \langle \ve{\hat{c}}^{\dagger}_{\ve{i}, n_{\ve{i}}} S^x \ve{\hat{c}}^{\phantom\dagger}_{\ve{i}, n_{\ve{i}} }   \;  \ve{\hat{c}}^{\dagger}_{\ve{j}, n_{\ve{j}}} S^x  
\ve{\hat{c}}^{\phantom\dagger}_{\ve{j}, n_{\ve{j}}}\rangle \rangle_{C}  +
\langle \langle \ve{\hat{c}}^{\dagger}_{\ve{i}, n_{\ve{i}}} S^y \ve{\hat{c}}^{\phantom\dagger}_{\ve{i}, n_{\ve{i}} }   \;  \ve{c}^{\dagger}_{\ve{j}, n_{\ve{j}}} S^y  \ve{\hat{c}}^{\phantom\dagger}_{\ve{j}, n_{\ve{j}}}\rangle \rangle_{C}  \right), \nonumber \\
&\texttt{ObsXYZ} =  \frac{2\cdot\texttt{ObsXY} + \texttt{ObsZ}}{3}.
\end{align}
Here  $\ve{\hat{c}}^{\dagger}_{\ve{i}, n_{\ve{i}}} =  \left( \hat{c}^{\dagger}_{\ve{i}, n_{\ve{i}},\uparrow},  \hat{c}^{\dagger}_{\ve{i}, n_{\ve{i}},\downarrow} \right) $ is a two  component spinor  and    $ \ve{S}  = \frac{1}{2} \ve{\sigma}$  
with
\begin{equation}
\ve{\sigma}   = \left(
\begin{bmatrix} 
0 & 1 \\
1 & 0 
\end{bmatrix},
\begin{bmatrix} 
0 & -i \\
i & 0 
\end{bmatrix},
\begin{bmatrix} 
1 & 0 \\
0 & -1 
\end{bmatrix}
\right) 
 \end{equation}
 the Pauli spin  matrices. 
 {\color{red}  Add this to the code since you are not assuming SU(2) spin symmetry.}

\subsubsection{Equal-time Green function}

A measurement of the equal-time Green function can be obtained by:
\begin{lstlisting}[style=fortran]
Call Predefined_Obs_eq_Green_measure(Latt, Latt_unit, List, GR, GRC, N_SUN, ZS, ZP, Obs)
\end{lstlisting}

Which returns:
\begin{align}
\texttt{Obs}(\ve{i}-\ve{j}, n_{\ve{i}}, n_{\ve{j}}) = \sum_{\sigma=1}^{N_\text{col}} \sum_{s=1}^{N_\text{fl}} \langle  \hat{c}^{\dagger}_{\ve{i}, n_{\ve{i}},\sigma,s} \hat{c}^{\phantom\dagger}_{\ve{j}, n_{\ve{j}},\sigma,s} \rangle.
\end{align}


\subsubsection{Equal-time density-density correlations}

A measurement of equal-time density-density correlations can be obtained by:
\begin{lstlisting}[style=fortran]
Call Predefined_Obs_eq_Den_measure(Latt, Latt_unit, List, GR, GRC, N_SUN, ZS, ZP, Obs)
\end{lstlisting}
Which returns:
\begin{align}
\texttt{Obs}(\ve{i}-\ve{j}, n_{\ve{i}}, n_{\ve{j}}) = \langle  \langle\hat{ N}_{\ve{i},n_{\ve{i}}}  \hat{N}_{\ve{j},n_{\ve{j}}}\rangle - \langle \hat{N}_{\ve{i},n_{\ve{i}}} \rangle  \langle \hat{N}_{\ve{j},n_{\ve{j}}}\rangle \rangle_C,
\end{align}
where
\begin{align}
\label{Density_op.eq}
\hat{N}_{\ve{i},n_{\ve{i}}}  = \sum_{\sigma=1}^{N_\text{col}} \sum_{s=1}^{N_\text{fl}}  \hat{c}^{\dagger}_{\ve{i},n_{\ve{i}},\sigma,s} \hat{c}^{\phantom\dagger}_{\ve{i}, n_{\ve{i}},\sigma,s}.
\end{align}


\subsubsection{Time-displaced Green function}

A measurement of the time-displaced Green function can be obtained by:
\begin{lstlisting}[style=fortran]
Call Predefined_Obs_tau_Green_measure(Latt, Latt_unit, List, NT, GT0, G0T, G00, GTT, 
                                      N_SUN, ZS, ZP, Obs)
\end{lstlisting}
Which returns:
\begin{align}
\texttt{Obs}(\ve{i} - \ve{j}, \tau, n_{\ve{i}},  n_{\ve{j}} )  = \sum_{\sigma=1}^{N_\text{col}} \sum_{s=1}^{N_\text{fl}} \langle \langle  \hat{c}^{\dagger}_{\ve{i},n_{\ve{i}},  \sigma,s}(\tau) \hat{c}^{\phantom\dagger}_{\ve{j}, n_{\ve{j}},\sigma,s} \rangle \rangle_{C}
\end{align}


\subsubsection{Time-displaced $SU(N)$ spin-spin correlations}


A measurement of time-displaced spin-spin correlations for $SU(N)$ models ($N_\text{fl} = 1$) can be obtained by:
\begin{lstlisting}[style=fortran]
Call Predefined_Obs_tau_SpinSUN_measure(Latt, Latt_unit, List, NT, GT0, G0T, G00, GTT, 
                                        N_SUN, ZS, ZP, Obs)
\end{lstlisting}
\begin{align}
\texttt{Obs}(\ve{i} - \ve{j}, \tau, n_{\ve{i}},  n_{\ve{j}} )= \frac{2N}{N^2-1}\sum_{a=1}^{N^2 - 1}  \langle \ve{\hat{c}}^{\dagger}_{\ve{i}, n_{\ve{i}}}(\tau) T^a \ve{\hat{c}}^{\phantom\dagger}_{\ve{i}, n_{\ve{i}}}(\tau)  \;
    \ve{\hat{c}}^{\dagger}_{\ve{j}, n_{\ve{j}}} T^a \ve{\hat{c}}^{\phantom\dagger}_{\ve{j}, n_{\ve{j}}}    \rangle  \rangle_{C}
\end{align}
where $T^a$ are the generators of $SU(N)$ (see Sec.~\ref{SU_N_equal_time.sec}  for more details).


\subsubsection{Time-displaced spin correlations}

A measurement of time-displaced spin-spin correlations for $M\!z$ models ($N_\text{fl} = 2, N_\text{col} = 1$)  is returned by:
\begin{lstlisting}[style=fortran]
Call Predefined_Obs_tau_SpinMz_measure(Latt, Latt_unit, List, NT, GT0, G0T, G00, GTT,
                                       N_SUN, ZS, ZP, ObsZ, ObsXY, ObsXYZ)
\end{lstlisting}
Which calculates the following observables:
\begin{align}
\texttt{ObsZ}(\ve{i} - \ve{j}, \tau, n_{\ve{i}},  n_{\ve{j}} ) &= \begin{multlined}[t] 4 \langle \langle \ve{\hat{c}}^{\dagger}_{\ve{i}, n_{\ve{i}}}(\tau)  S^z \ve{\hat{c}}^{\phantom\dagger}_{\ve{i}, n_{\ve{i}} }(\tau)   \;  \ve{\hat{c}}^{\dagger}_{\ve{j}, n_{\ve{j}}} S^z  \ve{\hat{c}}^{\phantom\dagger}_{\ve{j}, n_{\ve{j}}}\rangle \rangle_{C} \\ {\color{red}
- 4 \langle \langle \ve{\hat{c}}^{\dagger}_{\ve{i}, n_{\ve{i}}} S^z \ve{\hat{c}}^{\phantom\dagger}_{\ve{i}, n_{\ve{i}} } \rangle \rangle_{C}  \langle \langle \  \ve{\hat{c}}^{\dagger}_{\ve{j}, n_{\ve{j}}} S^z  \ve{\hat{c}}^{\phantom\dagger}_{\ve{j}, n_{\ve{j}}}\rangle \rangle_{C}  } \end{multlined} \nonumber \\  
\texttt{ObsXY} (\ve{i} - \ve{j}, \tau, n_{\ve{i}},  n_{\ve{j}} ) &=\begin{multlined}[t] 
2 \left( \langle \langle \ve{\hat{c}}^{\dagger}_{\ve{i}, n_{\ve{i}}}(\tau) S^x \ve{\hat{c}}^{\phantom\dagger}_{\ve{i}, n_{\ve{i}} } (\tau)  \;  \ve{\hat{c}}^{\dagger}_{\ve{j}, n_{\ve{j}}} S^x  
\ve{\hat{c}}^{\phantom\dagger}_{\ve{j}, n_{\ve{j}}}\rangle \rangle_{C}  \right.\\
+ \left. \langle \langle \ve{\hat{c}}^{\dagger}_{\ve{i}, n_{\ve{i}}}(\tau) S^y \ve{\hat{c}}^{\phantom\dagger}_{\ve{i}, n_{\ve{i}} }(\tau)   \;  \ve{c}^{\dagger}_{\ve{j}, n_{\ve{j}}} S^y  \ve{\hat{c}}^{\phantom\dagger}_{\ve{j}, n_{\ve{j}}}\rangle \rangle_{C}  \right)  \end{multlined}  \nonumber \\
\texttt{ObsXYZ} &= \frac{2\cdot\texttt{ObsXY} + \texttt{ObsZ}}{3}.
\end{align}


\subsubsection{Time-displaced density-density correlations}

A measurement of time-displaced density-density correlations for general $SU(N)$ models is given by:
\begin{lstlisting}[style=fortran]
Call Predefined_Obs_tau_Den_measure(Latt, Latt_unit, List,  NT, GT0, G0T, G00, GTT,
                                    N_SUN, ZS, ZP,  Obs)
\end{lstlisting}
Which returns:
\begin{align}
\texttt{Obs}(\ve{i} - \ve{j}, \tau, n_{\ve{i}},  n_{\ve{j}} ) = \langle  \langle \hat{ N}_{\ve{i},n_{\ve{i}}} (\tau)  \hat{N}_{\ve{j},n_{\ve{j}}}\rangle - \langle \hat{N}_{\ve{i},n_{\ve{i}}} \rangle  \langle \hat{N}_{\ve{j},n_{\ve{j}}}\rangle \rangle_C.
 \end{align}
 The density operator is defined in Eq.~\ref{Density_op.eq}.

% !TEX root = doc.tex
% Copyright (c) 2017-2020 The ALF project.
% This is a part of the ALF project documentation.
% The ALF project documentation by the ALF contributors is licensed
% under a Creative Commons Attribution-ShareAlike 4.0 International License.
% For the licensing details of the documentation see license.CCBYSA.
%
%-----------------------------------------------------------------------------------
\subsubsection{R{\'e}nyi Entropy}
%-----------------------------------------------------------------------------------
The provided module \texttt{entanglement\_mod.F90} allows to compute the $2^{\rm o}$ R\'enyi entropy $S_2$ for a subsystem.
This is obtained by sampling an observable constructed from two independent simulations of the model \cite{Grover13}
\begin{equation}
e^{-S_2} = \sum_{C_1,C_2} P(C_2) P(C_1) {\rm det}\left[G_A(C_1)G_A(C_2)-(\mathds{1} - G_A(C_1))(\mathds{1} - G_A(C_2))\right],
\label{renyi_obs}
\end{equation}
where $G_A(C_i)$, $i=1$, $2$ is the Green's function matrix restricted to the subsystem $A$ in exam, depending on the configuration $C_i$ of the replica $i$.
Due to the formulation, sampling of $S_2$ requires a MPI simulation with at least $2$ tasks.
Only real-space partitions are currently supported.
The degrees of freedom defining the subsystem are identified by the lattice site, flavor, and color indexes.

\begin{lstlisting}[style=fortran]
Call Predefined_Obs_scal_Renyi_Ent(GRC, List, Nsites, N_SUN, ZS, ZP, Obs)
\end{lstlisting}
This subroutine returns an observable \texttt{Obs} such that $\langle\texttt{Obs}\rangle=e^{-S_2}$.
It can be called for different levels of specialization.

In the most general case, \texttt{List(:, N\_FL, N\_SUN)} is a three-dimensional array that contains, for every flavor and color index, the list of lattice sites pertaining to the subsystem. \texttt{Nsites(N\_FL, N\_SUN)} is a bidimensional array that provides the number of lattice sites in the subsystem for every flavor and color index. \texttt{N\_SUN} must be omitted.

For a subsystem whose degrees of freedom, for a given flavor index, have a common value of color indexes, \texttt{Predefined\_Obs\_scal\_Renyi\_Ent} can be called by providing in \texttt{List(:, N\_FL)} a bidimensional array that contains the list of lattice sites for every flavor index. In this case, \texttt{Nsites(N\_FL)} provides the number of sites in the subsystem for any given flavor index, while \texttt{N\_SUN(N\_FL)} contains the number of color indexes for a given flavor index.

Finally, an additional specialization exists for a subsystem whose lattice degrees of freedom are flavor- and color-independent. In this case, \texttt{List(:)} is a one-dimensional array containing the lattice sites of the subsystem. \texttt{Nsites} is the number of sites, and \texttt{N\_SUN} is the number of color indexes belonging to the subsystem.
Accordingly, for every element \texttt{I} of \texttt{List}, the subsystem contains all degrees of freedom with site index \texttt{I}, any flavor index, and \texttt{1} \ldots \texttt{N\_SUN} color index.

\begin{lstlisting}[style=fortran,breaklines=true]
Call Predefined_Obs_scal_Mutual_Inf(GRC, List_A, Nsites_A, List_B, Nsites_B, N_SUN, 
                                    ZS, ZP, Obs )
\end{lstlisting}
This routine computes the observables for the second R\'enyi entropies, which are needed to extract the mutual information between two subsystems $A$ and $B$.
\texttt{List\_A} and \texttt{Nsites\_A} are input parameters describing the subsystem $A$, with the same conventions and specializations described above.
\texttt{List\_B} and \texttt{Nsites\_B} are the corresponding input parameters for the subsystem $B$. \texttt{N\_SUN} is assumed to be identical for $A$ and $B$.
\texttt{Renyi\_A}, \texttt{Renyi\_B}, and \texttt{Renyi\_AB} are the output observables, pertaining to $A$, $B$, and $A\cup B$. The mutual information between $A$ and $B$ is $I_2=-\ln \langle \texttt{Renyi\_A}\rangle -\ln \langle \texttt{Renyi\_B}\rangle +\ln \langle \texttt{Renyi\_AB}\rangle$.
 %temporary, to be merged into predefined_observables
%%%%%%%%%%%%%%%%%%%%%%%%%%%%

%%%%%%%%%%%%%%%%%%%%%%%%%%%%
% Copyright (c) 2016-2020 The ALF project.
% This is a part of the ALF project documentation.
% The ALF project documentation by the ALF contributors is licensed
% under a Creative Commons Attribution-ShareAlike 4.0 International License.
% For the licensing details of the documentation see license.CCBYSA.

% !TEX root = doc.tex

\subsection{Predefined trial wave functions} \label{sec:predefined_trial_wave_function}

When using the projective algorithm (see Sec.~\ref{sec:defT0}), trial wave functions must be specified.
%These are stored in variables of the \texttt{WaveFunction} type.
%, which includes the routine \path{WF_overlap(WF_L, WF_R, Z_norm)} for normalizing the right trial wave function, such that $\langle \Psi_{T,L} | \Psi_{T,R} \rangle = 1$.
The ALF package provides a set of trial wave functions, corresponding to the solution of the non-interacting tight binding Hamiltonian on one of the predefined lattices, namely:
\begin{itemize}
	\item Square
	\item Honeycomb
	\item N-leg ladder
	\item Bilayer square
	\item Bilayer honeycomb
\end{itemize}

The wave functions $|\Psi_{T,L/R}\rangle$=\texttt{WF\_L/R} are returned by the call:

\begin{lstlisting}[style=fortran]
Call Predefined_TrialWaveFunction(Lattice_type, Ndim, List, Invlist, Latt, Latt_unit, N_part,
                             N_FL, WF_L, WF_R)
\end{lstlisting}

Twisted boundary conditions (\texttt{Phi\_X\_vec=0.01}) are implemented so as to generate non-degenerate trial wave functions. 

%%%%%%%%%%%%%%%%%%%%%%%%%%%%


%\subsection{Template}
%	
%\red{Rethink. Maybe merge this with Chapter on Examples and Models/Model Classes.\\}
%
% \red{TODO} Go through everything one has to defined/set in order to define a new Hamiltonian. \red{TODO}\\
%
%We'd perhaps want to provide a \emph{minimum} Hamiltonian, probably written in pseudo-code, from which one could write their own.\\ \\
%
%\noindent A \emph{complete} list of all the code that would have to be changed for the \emph{most general} case could read as:
%
%\begin{itemize}
%	\item Hamiltonian -- one-body, imaginary-time propagator for a given configuration of HS and  Ising fields
%	\item Table~\ref{table:hamiltonian}:
%	\begin{itemize}
%		\item \texttt{Ham\_Set}
%		\item \texttt{Ham\_V}
%		\item \texttt{S0}
%		\item \texttt{Setup\_Ising\_action}
%		\item \texttt{Global\_move}
%		\item \texttt{Delta\_S0\_global}
%		\item \texttt{Global\_move\_tau}
%		\item \texttt{Alloc\_obs}
%		\item \texttt{Obser}
%		\item \texttt{ObserT}
%	\end{itemize}
%	\item Check Table~\ref{table:operator} (Operator type).
%	\item Table~\ref{table:lattice} and \ref{table:unit_cell}:
%	\begin{itemize}
%		\item \texttt{Lattice\%a1\_p}, \texttt{Lattice\%a2\_p}
%		\item \texttt{Lattice\%L1\_p}, \texttt{Lattice\%L2\_p}
%		\item \texttt{Lattice\%N}
%		\item \texttt{Norb}
%		\item \texttt{N\_coord}
%		\item \texttt{Orb\_pos(1..Norb,2)}
%	\end{itemize}
%\end{itemize}

