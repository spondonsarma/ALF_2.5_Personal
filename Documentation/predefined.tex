% Copyright (c) 2016-2019 The ALF project.
% This is a part of the ALF project documentation.
% The ALF project documentation by the ALF contributors is licensed
% under a Creative Commons Attribution-ShareAlike 4.0 International License.
% For the licensing details of the documentation see license.CCBYSA.

% !TEX root = doc.tex

ALF includes modules providing predefined structures which the user can combine together or use as templates for defining new structures, namely: 
\begin{itemize}
	\item lattices and unit cells
	\item hopping Hamiltonians
	\item interaction Hamiltonians
	\item observables
	\item trial wave functions
\end{itemize}
which are defined using the data structures described in the Sec.~\ref{sec:imp}, as detailed below.

[take a look at the first example of next section]


\subsection{Template}

\red{Maybe discard this subsection.}\\
	
\red{Maybe merge this with Chapter on Examples and Models/Model Classes.\\}

 \red{TODO} Go through everything one has to defined/set in order to define a new Hamiltonian. \red{TODO}\\

We'd perhaps want to provide a \emph{minimum} Hamiltonian, probably written in pseudo-code, from which one could write their own.\\ \\

\noindent A \emph{complete} list of all the code that would have to be changed for the \emph{most general} case could read as:

\begin{itemize}
	\item Hamiltonian -- one-body, imaginary-time propagator for a given configuration of HS and  Ising fields
	\item Table~\ref{table:hamiltonian}:
	\begin{itemize}
		\item \texttt{Ham\_Set}
		\item \texttt{Ham\_V}
		\item \texttt{S0}
		\item \texttt{Setup\_Ising\_action}
		\item \texttt{Global\_move}
		\item \texttt{Delta\_S0\_global}
		\item \texttt{Global\_move\_tau}
		\item \texttt{Alloc\_obs}
		\item \texttt{Obser}
		\item \texttt{ObserT}
	\end{itemize}
	\item Check Table~\ref{table:operator} (Operator type).
	\item Table~\ref{table:lattice} and \ref{table:unit_cell}:
	\begin{itemize}
		\item \texttt{Lattice\%a1\_p}, \texttt{Lattice\%a2\_p}
		\item \texttt{Lattice\%L1\_p}, \texttt{Lattice\%L2\_p}
		\item \texttt{Lattice\%N}
		\item \texttt{Norb}
		\item \texttt{N\_coord}
		\item \texttt{Orb\_pos(1..Norb,2)}
	\end{itemize}
\end{itemize}

