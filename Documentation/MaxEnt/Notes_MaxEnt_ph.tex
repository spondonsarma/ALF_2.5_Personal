\documentclass[12pt]{article}
\usepackage{a4}
\usepackage{amsmath}
%%% DFG prefers Arial =>
% substitute Helvetica (called "Arial" by Windows people)
% for the default \sf font
% use with \bfseries instead of \bf and \slshape instead of \it
\usepackage{helvet}
\renewcommand{\familydefault}{\sfdefault}

\setlength{\textheight}{23.8cm}
\setlength{\textwidth}{15.5cm} %previously 14.5
\setlength{\evensidemargin}{0.8cm}
\setlength{\oddsidemargin}{0.8cm}
\setlength{\topmargin}{0.3cm}
\setlength{\parindent}{0.5cm}
\headheight=0.0cm
\headsep=0.0cm

\hfuzz=5pt
\parindent=0pt
\parskip=5pt


\begin{document}
\section{Maximum entropy in the  particle-hole channel}
\subsection{Imaginary time formulation}
	 For particle-hole quantities such as spin-spin or charge-charge correlations, 
the  Kernel reads:
\begin{equation}
	\langle S(q,\tau) S(-q,0) \rangle  = \frac{1}{\pi} 
   \int {\rm d} \omega  \frac{e^{- \tau \omega} }{ 1 - e^{-\beta  \omega} } \chi''(q,\omega).
\end{equation}
This follows directly from the  Lehmann representation: 
\begin{equation}
 \chi''(q,\omega)  = \frac{\pi}{Z} \sum_{n,m} e^{-\beta E_n} |\langle n | S(q) | \rangle m |^2 
\delta ( \omega + E_n - E_m) \left( 1 - e^{-\beta  \omega} \right) 
\end{equation}

In principle that's it. In practice  the setup of the Stochastic MaxEnt is a bit tricky, 
since as input one  needs the sum rule.  
Consider: 
\begin{equation}
	coth(\beta \omega/2) \chi''(q,\omega)
\end{equation}
For this quantitiy, we have the sum rule since: 
\begin{equation}
	\int {\rm d} \omega 	coth(\beta \omega/2) \chi''(q,\omega) = 
  2 \pi \langle S(q,\tau=0) S(-q,0) \rangle
\end{equation}
which is just the first point in the data. 

Hence,
\begin{equation}
	\langle S(q,\tau) S(-q,0) \rangle  =  
       \int {\rm d} \omega  \underbrace{ \frac{1}{\pi} \frac{e^{- \tau \omega} }
            { 1 - e^{-\beta  \omega} } tanh(\beta \omega/2)  }_{K(\tau,\omega)} 
       \underbrace{ coth(\beta \omega/2)   \chi''(q,\omega) }_{A(\omega)} 
\end{equation}
and one extracts with the MaxEnt $A(\omega) $ which one then transforms back to the 
quantitiy one wants.  In general,  the codes will produce the dynamical  structure factor: 
\begin{equation}
 S(q\omega)  = \chi''(q,\omega)/\left( 1 - e^{-\beta  \omega} \right)    
\end{equation}

Note  that 
$\langle S(q,\tau) S(-q,0) \rangle  =  \langle S(q,\beta - \tau) S(-q,0) \rangle   $ so that 
it reads in only  the data for $ \tau = 0,\beta/2 $.  
Also since $A(\omega) $ is a symmetric function the omega range  can be restricted to positive values. 

\subsection{Matsubara frequency  formulation}

Let 
\begin{equation}
  \chi(q,i\Omega_m) = \int_0^{\beta} {\rm d} \tau  e^{i \Omega_m \tau}
	\langle S(q,\tau) S(-q,0) \rangle  = \frac{1}{\pi}
   \int {\rm d} \omega  \frac{\chi''(q,\omega)}{ \omega - i \Omega_m }.
\end{equation}
Using the fact that $\chi''(q,\omega) = -\chi''(q,-\omega)$ one obtains:
\begin{equation}
\begin{gathered}
  \chi(q,i\Omega_m) = 
	\frac{1}{\pi}
   \int_0^{\infty} {\rm d} \omega \left(\frac{1}{ \omega - i \Omega_m } - \frac{1}{ -\omega - i \Omega_m } \right)
         \chi''(q,\omega) \\
    = \frac{2}{\pi} \int_0^{\infty} {\rm d} \omega \frac{\omega^2}{ \omega^2  + \Omega_m^2 } 
  \frac{\chi''(q,\omega)}{\omega} 
   \equiv \int_0^{\infty} {\rm d} \omega K(\omega,i\Omega_m) A(q,\omega)
\end{gathered}
\end{equation}
with
\begin{equation}
   K(\omega,i\Omega_m) = \frac{\omega^2}{ \omega^2  + \Omega_m^2 } 
\end{equation}
and
\begin{equation}
A(q,\omega) =  \frac{2}{\pi}   \frac{\chi''(q,\omega)}{\omega} 
\end{equation}
The above definitions are useful since the image satisfies the sum rule:
\begin{equation}
\int_0^{\infty} {\rm d} \omega A(q,\omega) =  \frac{1}{\pi}  \int_{-\infty}^{\infty} {\rm d} \omega 
   \frac{\chi''(q,\omega)}{\omega}   \equiv \chi(q,i\Omega_m=0)
\end{equation}

\end{document} 


