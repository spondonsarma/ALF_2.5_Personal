% !TEX root = doc.tex
% Copyright (c) 2017-2020 The ALF project.
% This is a part of the ALF project documentation.
% The ALF project documentation by the ALF contributors is licensed
% under a Creative Commons Attribution-ShareAlike 4.0 International License.
% For the licensing details of the documentation see license.CCBYSA.
%
%-----------------------------------------------------------------------------------


%-----------------------------------------------------------------------------------
\subsection{Predefined hopping  matrices}
%-----------------------------------------------------------------------------------

The  module provides hopping   and  checkerboard decompositions, defining a  \path{Hopping_Matrix} (an array of length \path{N_FL} of type  \path{Hopping_Matrix_type}) for each of the following predefined lattices.

\subsubsection*{Square}
The call:
\begin{lstlisting}[style=fortran]
 Call Set_Default_hopping_parameters_square(Hopping_Matrix, T_vec, Chem_vec, Phi_X_vec,  
          Phi_Y_vec, Bulk, N_Phi_vec, N_FL, List, Invlist, Latt, Latt_unit)
\end{lstlisting}
defines  the  \path{Hopping_Matrix} for the square  lattice: 
\begin{equation}
\hat{H}_T  =   \sum_{\ve{i}, \sigma, s}  \left( \left[ \sum_{ \ve{\delta} = \left\{ \ve{a}_1, \ve{a}_2\right\} }    - t^{(s)} \hat{c}^{\dagger}_{\ve{i},s,\sigma}   e^{\frac{2 \pi i}{\Phi_0} \int_{\ve{i}}^{\ve{i}+ \ve{\delta}}  \vec{A}^{(s)}(\vec{l})  d \vec{l}}   \hat{c}^{}_{\ve{i} + \ve{\delta},s,\sigma} +  H.c.   \right]    -  \mu^{(s)} \hat{c}^{\dagger}_{\ve{i},s,\sigma} \hat{c}^{}_{\ve{i},s,\sigma}  \right).
\end{equation}
The vectors  \path{T_vec} and \path{Chem_vec} have  length \texttt{N\_FL} and specify the hopping and the chemical potentials, while the  vectors \path{Phi_X_vec},  \path{Phi_Y_vec} and \path{N_Phi_vec},  also of  length  \texttt{N\_FL},    define the vector potential. 


\subsubsection*{Honeycomb}
The call: 
 \begin{lstlisting}[style=fortran]
 Call Set_Default_hopping_parameters_honeycomb(Hopping_Matrix,T_vec, Chem_vec, Phi_X_vec,
          Phi_Y_vec, Bulk, N_Phi_vec, N_FL, List, Invlist, Latt, Latt_unit)
\end{lstlisting}
defines  the  \path{Hopping_Matrix} for the  honeycomb lattice: 
\begin{equation}
\hat{H}_T  =   \sum_{\ve{i}, \sigma, s}  \left( \left[ \sum_{ \ve{\delta} = \left\{ \ve{\delta}_1, \ve{\delta}_2, \ve{\delta}_3\right\} }    - t^{(s)} \hat{c}^{\dagger}_{\ve{i},s,\sigma}   e^{\frac{2 \pi i}{\Phi_0} \int_{\ve{i}}^{\ve{i}+ \ve{\delta}}  \vec{A}^{(s)}(\vec{l})  d \vec{l}}   \hat{c}^{}_{\ve{i} + \ve{\delta},s,\sigma} +  H.c.  \right]      -  \mu^{(s)} \hat{c}^{\dagger}_{\ve{i},s,\sigma} \hat{c}^{}_{\ve{i},s,\sigma}  \right),
\end{equation}
where the \path{T_vec} and \path{Chem_vec} have  length \texttt{N\_FL} and specify the hopping and the chemical potentials, while the  vectors \path{Phi_X_vec},  \path{Phi_Y_vec} and \path{N_Phi_vec},  also of  length  \texttt{N\_FL}, define the vector potential.  Here $\ve{i}$  runs over  sublattice  A, and $\ve{i} + \ve{\delta}$  over the three nearest neighbors of site $\ve{i}$


\subsubsection*{Square bilayer}
The call:
 \begin{lstlisting}[style=fortran]
Call Set_Default_hopping_parameters_Bilayer_square(Hopping_Matrix, T1_vec, T2_vec, Tperp_vec, 
         Chem_vec, Phi_X_vec, Phi_Y_vec, Bulk, N_Phi_vec, N_FL, List, Invlist, Latt, Latt_unit)
\end{lstlisting}  
defines  the  \path{Hopping_Matrix} for the  bilayer  square  lattice:                 
\begin{align}
\hat{H}_T  =  &   \sum_{\ve{i}, \sigma, s,n } \left(    \left[  \sum_{ \ve{\delta} = \left\{ \ve{a}_1, \ve{a}_2\right\} }  - t_n^{(s)} \hat{c}^{\dagger}_{\ve{i},s,\sigma,n}   e^{\frac{2 \pi i}{\Phi_0} \int_{\ve{i}}^{\ve{i}+ \ve{\delta}}  \vec{A}^{(s)}(\vec{l})  d \vec{l}}   \hat{c}^{}_{\ve{i} + \ve{\delta},s,\sigma,n} +  H.c. \right]       -  \mu^{(s)} \hat{c}^{\dagger}_{\ve{i},s,\sigma,n} \hat{c}^{}_{\ve{i},s,\sigma,n}  \right)  +  \nonumber \\
      &    \sum_{\ve{i}, \sigma, s } -  t_{\perp}^{(s)}  \left( \hat{c}^{\dagger}_{\ve{i},s,\sigma,1} \hat{c}^{}_{\ve{i},s,\sigma,2}    + H.c.  \right), 
\end{align}
where the additional  index  $n$  labels the layers.


\subsubsection*{Square honeycomb}
The call:
 \begin{lstlisting}[style=fortran]
Call Set_Default_hopping_parameters_Bilayer_honeycomb(Hopping_Matrix, T1_vec, T2_vec, Tperp_vec, 
         Chem_vec, Phi_X_vec, Phi_Y_vec, Bulk, N_Phi_vec, N_FL, List, Invlist, Latt, Latt_unit)
\end{lstlisting}  
defines  the  \path{Hopping_Matrix} for the  bilayer  honeycomb  lattice:                 
\begin{align}
\hat{H}_T  =  &   \sum_{\ve{i}, \sigma, s,n } \left(    \left[  \sum_{ \delta = \left\{ \ve{\delta}_1, \ve{\delta}_2, \ve{\delta}_3 \right\} }  - t_n^{(s)} \hat{c}^{\dagger}_{\ve{i},s,\sigma,n}   e^{\frac{2 \pi i}{\Phi_0} \int_{\ve{i}}^{\ve{i}+ \ve{\delta}}  \vec{A}^{(s)}(\vec{l})  d \vec{l}}   \hat{c}^{}_{\ve{i} + \ve{\delta},s,\sigma,n} +  H.c. \right]       -  \mu^{(s)} \hat{c}^{\dagger}_{\ve{i},s,\sigma,n} \hat{c}^{}_{\ve{i},s,\sigma,n}  \right)  +  \nonumber \\
      &    \sum_{\ve{i}, \sigma, s } -  t_{\perp}^{(s)}  \left( \hat{c}^{\dagger}_{\ve{i},s,\sigma,1} \hat{c}^{}_{\ve{i},s,\sigma,2}   +
                   \hat{c}^{\dagger}_{\ve{i} + \ve{\delta}_1,s,\sigma,1} \hat{c}^{}_{\ve{i} + \ve{\delta}_1,s,\sigma,2}  + H.c.  \right).
\end{align}
Here, the additional  index  $n$  labels the layer.  


\subsubsection*{N-leg ladder}
The call:
 \begin{lstlisting}[style=fortran]
Call Set_Default_hopping_parameters_n_lag_ladder(Hopping_Matrix, T_vec, Tperp_vec, Chem_vec,
         Phi_X_vec, Phi_Y_vec, Bulk, N_Phi_vec, N_FL, List, Invlist, Latt, Latt_unit)
\end{lstlisting}  
defines  the  \path{Hopping_Matrix} for the  the  N-leg ladder lattice:                 
\begin{align}
\hat{H}_T  =  & \sum_{\ve{i}, \sigma, s }  \sum_{n=1}^{\texttt{Norb}} \left(      - t^{(s)} \hat{c}^{\dagger}_{\ve{i},s,\sigma,n}   e^{\frac{2 \pi i}{\Phi_0} \int_{\ve{i}}^{\ve{i}+ \ve{a}_1}  \vec{A}^{(s)}(\vec{l})  d \vec{l}}   \hat{c}^{}_{\ve{i} + \ve{a}_1,s,\sigma,n} +  H.c.       -  \mu^{(s)} \hat{c}^{\dagger}_{\ve{i},s,\sigma,n} \hat{c}^{}_{\ve{i},s,\sigma,n}  \right)  +  \nonumber \\
      &  \sum_{\ve{i}, \sigma, s } \sum_{n=1}^{\texttt{Norb}-1}  -  t_{\perp}^{(s)}  \left( 
                   \hat{c}^{\dagger}_{\ve{i} + \ve{\delta}_1,s,\sigma,n}  e^{\frac{2 \pi i}{\Phi_0} \int_{(n-1)\ve{a}_2}^{(n)\ve{a}_2}  \vec{A}^{(s)}(\vec{l})  d \vec{l}}    \hat{c}^{}_{\ve{i} + \ve{\delta}_1,s,\sigma,n+1}  + H.c.  \right). 
\end{align}
Here, the additional  index  $n$  defines  the orbital.  Note that this lattice  has open boundary conditions in the $\vec{a}_2$  direction. 
