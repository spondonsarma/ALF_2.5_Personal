% !TEX root = doc.tex
% Copyright (c) 2017-2020 The ALF project.
% This is a part of the ALF project documentation.
% The ALF project documentation by the ALF contributors is licensed
% under a Creative Commons Attribution-ShareAlike 4.0 International License.
% For the licensing details of the documentation see license.CCBYSA.
%
%-----------------------------------------------------------------------------------
\subsection{ Setting up the hopping matrix }

The module \texttt{Predefined\_Hopping}   provides a set of generic Hopping Hamiltonians.    For each  predefined lattice type 
\texttt{Square}, \texttt{Honeycomb}, \texttt{N\_leg\_ladder},  \texttt{Bilayer\_honeycomb},  \texttt{Bilayer\_square},
    this routine sets  up the nearest  neighbor hopping matrix with twisted boundary conditions in the $\vec{L}_1$ and $\vec{L}_2$ lattice directions. The twist is given by 
\texttt{Phi\_1}  and \texttt{Phi\_2}  respectively.  If the flag  \texttt{bulk=.true.}   then   the twist will be implemented with a vector potential. Otherwise, if  \texttt{bulk=.false.}, the twist will be imposed at the boundary.   The routine also accounts for  the inclusion of a  total number of \texttt{N\_Phi}  flux quanta traversing the lattice.   All above phase factors  can be flavor dependent.  

Lattice specific information for the checkerboard  decomposition is   equally   provided such that for each aforementioned lattice type,  symmetric and asymmetric  checkerboard decompositions  are available.    

\subsubsection{Generic hopping matrix elements}\label{sec:generic_hopping}
%-----------------------------------------------------------------------------------


Here we compute the hopping matrix element  between two sites of a given lattice  in the presence of twisted boundary conditions and  orbital magnetic field. 
The generic Hopping Hamiltonian will read: 
\begin{equation}
	   \hat{H}_T = \sum_{(i,\delta), (j,\delta'), s, \sigma}    T_{(i,\delta), (j,\delta')}^{(s)}    c^{\dagger}_{(i,\delta),s,\sigma }   e^{\frac{2 \pi i}{\Phi_0} \int_{i + \delta}^{j + \delta'}  \vec{A}^{(s)}(\vec{l})  d \vec{l}} c^{}_{(j,\delta'),s,\sigma }
\end{equation}
with boundary conditions 
\begin{equation}
	c^{\dagger}_{(i + L_i,\delta) ,s,\sigma }   =  e^{- 2 \pi i\frac{\Phi_i^{(s)}}{\Phi_0}} \, e^{\frac{2 \pi i }{\Phi_0} \chi^{(s)}_{L_1} ( i + \delta ) } \, c^{\dagger}_{(i,\delta) ,s,\sigma }.
\end{equation}
Both the twist and  vector  potential can have a flavor dependency. For  now onwards we will  omit the flavor index $^{s}$.  

\noindent
\textbf{Phase factors}.  
The vector potential accounts for an orbital magnetic field that is implemented  in the Landau  gauge:  $\vec{A}(\vec{x})  =  -B(y,0,0) $ with $ \vec{x} = (x,y,z)$. $\Phi_0$ corresponds to the flux  quanta and the scalar function $\chi$ is defined  through as:
\begin{equation}
	\vec{A}( \vec{x} + \vec{L}_{i} )  = \vec{A}( \vec{x} )   +  \vec{\nabla} \chi_{L_{\alpha}}(\vec{x}). 
\end{equation}

 Provided that the bare hopping Hamiltonian, $T$,  is invariant under lattice translations, $\hat{H}_T$ commutes with magnetic translations  that satisfy the  Algebra: 
\begin{equation}
	\hat{T}_{\vec{a}} \hat{T}_{\vec{b}} =  e^{ \frac{2 \pi i}{\Phi_0}   \vec{B} \cdot \left( \vec{a} \times \vec{b} \right) }  \hat{T}_{\vec{b}} \hat{T}_{\vec{a}}. 
\end{equation}
On the  torus, the uniqueness of the wave functions requires that  $\hat{T}_{\vec{L}_1} \hat{T}_{\vec{L}_2}  =   \hat{T}_{\vec{L}_2} \hat{T}_{\vec{L}_1} $ such
that
\begin{equation}
	 \frac{\vec{B} \cdot \left( \vec{a} \times \vec{b}  \right) }{\Phi_0 } = N_{\Phi}   
\end{equation}
with  $N_\Phi $ an integer  The variable \texttt{N\_Phi},   specified in the parameter file,   denotes the number of flux quanta piercing the lattice.    The variables \texttt{Phi\_1}  and   \texttt{Phi\_2} also   in the parameter file denote  the twists  -- in units of the flux quanta  --  along the $\vec{L}_1$ and  $\vec{L}_2$ directions.     There are gauge  equivalent ways to insert the  twist in the boundary conditions. In the above we  have inserted   twist as a boundary condition such  for example setting  \texttt{Phi\_1=0.5}  corresponds to anti-periodic boundary conditions along the $L_1$  axis.   Alternatively we  can  consider the 
Hamiltonian:
\begin{equation}
	   \hat{H}_T = \sum_{(i,\delta), (j,\delta'), s, \sigma}    T_{(i,\delta), (j,\delta')}^{(s)}    \tilde{c}^{\dagger}_{(i,\delta),s,\sigma }   e^{\frac{2 \pi i}{\Phi_0} \int_{i + \delta}^{j + \delta'} \left(  \vec{A}(\vec{l})  + \vec{A}_{\phi} \right)  d \vec{l}} \tilde{c}^{}_{(j,\delta'),s,\sigma }
\end{equation}
with boundary conditions 
\begin{equation}
	\tilde{c}^{\dagger}_{(i + L_i,\delta) ,s,\sigma }   =  e^{\frac{2 \pi i }{\Phi_0} \chi_{L_1} ( i + \delta ) } \, \tilde{c}^{\dagger}_{(i,\delta) ,s,\sigma }.
\end{equation}
Here 
\begin{equation}
	\vec{A}_{\phi} =\frac{  \phi_1  |\vec{a}_1|} { 2 \pi |\vec{L}_1| } \vec{b}_1 +  \frac{  \phi_2  |\vec{a}_2|}{2 \pi  |\vec{L}_2| } \vec{b}_2
\end{equation}
and $\vec{b}_i$  correspond to the reciprocal lattice vectors satisfying  $ \vec{a}_i  \cdot  \vec{b}_j  = 2 \pi \delta_{i,j} $.   The logical variable $\texttt{bulk} $ chooses between these two  gauge equivalent ways  on inserting the twist angle. If \texttt{bulk=\.true\.}    then  we use periodic boundary conditions  --  in the absence of an orbital field -- otherwise  twisted boundaries are used.  
The information for the phases,    is encoded in  the type \texttt{Hopping\_matrix\_type} 

\noindent
\textbf{The  Hopping matrix elements}. 
To specify the hopping matrix  we  assume  only translation invariance.  The point group symmetry of the lattice can be broken.   In particular, we set, for each flavor  index:
\begin{equation} 
	 T_{(i,\delta), (i +  n_1\vec{a}_1  + n_2 \vec{a}_2,\delta')}^{(s)}   =   t^{(s)}_{\delta_1,\delta_2,n_1,n_2, s}
\end{equation}
The right  hand side of the above equation is given  the type  \texttt{Hopping\_matrix\_type}.


\noindent
\textbf{The checkerboard decomposition.}   Aside from the hopping phases and hopping matrix elements, the \texttt{Hopping\_matrix\_type} type, contains information  concerning the checkerboard   decomposition.  In Eq.~\ref{Checkerboard.Eq} we wrote the hopping Hamiltonian as:
\begin{equation}
	\hat{\mathcal{H}}_{T}     = \sum_{i=1}^{N_T} \sum_{k \in \mathcal{S}^{T}_i} \hat{T}^{(k)}   
\end{equation}
with the rule that  if $k$ and $k'$  belong to the same set $\mathcal{S}^{T}_i $ then   $ \left[ \hat{T}^{(k)} , \hat{T}^{(k')} \right] = 0 $.  In the checkerboard decomposition, $\hat{T}^{(k)}$   corresponds to  hopping on a bond.   This decomposition is carried out for the aforementioned set of lattices with nearest neighbor  hopping matrix elements.  This information is stored in  \texttt{Hopping\_matrix\_type}. In this data type  \texttt{N\_FAM}  corresponds to the number of sets  ($N_T$ in the above equation). \texttt{L\_FAM(1:N\_FAM}   corresponds to the number of bonds in the set,  and finally, 
 \texttt{LIST\_FAM(1:N\_FAM, 1:max(L\_FAM(:)), 2)}    contains  information concerning the two legs of the bonds.    In the checkerboard decomposition, care has to be taken for local terms: each site  occurs multiple times in the list of bonds.    Since we have postulated translation symmetry,    a one-dimensional array, \texttt{Multiplicity},  of length  given by  the number of orbitals per unit cell suffices to  encode the required information.  
 Finally to be able to generate  the imaginary time step of length $\Delta \tau$  we  have to know   by which fraction of  $\Delta \tau$   we have to propagate each set.  This information is given in  the array  \texttt{Prop\_Fam}.  
 

The  data in the \texttt{Hopping\_matrix\_type} type suffices to uniquely define the  unit step propagation    for the Kinetic energy, for the  any combinations of the  \texttt{Checkerboard} and  \texttt{Symm}  options (see Sec.~\ref{sec:trotter}).   This is carried out in the routine \texttt{Predefined\_Hoppings\_set\_OPT}. 
Finally  \texttt{Hopping\_matrix\_type}     allows to compute the kinetic energy. This  task is carried out in  \texttt{Predefined\_Hoppings\_Compute\_Kin}.


\subsubsection{Usage}

Two steps. 
a)  Set  \texttt{Hopping\_matrix\_type}  with predefined routines for the supported lattices   b) Set \texttt{OP\_T}  with  \texttt{Predefined\_Hoppings\_set\_OPT}.   
{\color{red}FFA:  Jefferson, it would be nice if you would provide more detailed info on this. }

