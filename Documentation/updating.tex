% Copyright (c) 2016 The ALF project.
% This is a part of the ALF project documentation.
% The ALF project documentation by the ALF contributors is licensed
% under a Creative Commons Attribution-ShareAlike 4.0 International License.
% For the licensing details of the documentation see license.CCBYSA.

% !TEX root = Doc.tex
%------------------------------------------------------------
\subsection{Updating schemes}\label{sec:updating}
%------------------------------------------------------------
%
The program allows for different types of updating schemes.    Given a configuration $C$ we propose a new one, $C'$, with probability $T_0(C \rightarrow C')$  and accept it according to   the  Metropolis-Hastings   acceptance-rejection probability, 
\begin{equation}
	P(C \rightarrow C') =  \text{min}  \left( 1, \frac{T_0(C' \rightarrow C) W(C')}{T_0(C \rightarrow C') W(C)} \right),
\end{equation}
so as to guarantee the stationarity condition.  Here, $ W(C) = \left| \Re \left[ e^{-S(C)} \right] \right|   $.

\begin{table}[h]
   \begin{tabular}{@{} l l l @{}}\toprule
        Updating schemes &  Type                  &  Description   \\
         \\\midrule
         \texttt{Propose\_S0}                &   Logical     &  If true, proposes sequential local moves according to the probability $e^{-S_0}$. \\
         \texttt{Sequential\_Sweeps}    &   Logical     &  If true,  the program carries out sequential single spin-flip sweeps  \\ 
         \texttt{Global\_tau\_moves}     &   Logical     &  This option allows to carry out  global moves on a single time slice.  \\
                                                         &                   &  For a given time slice the user can define which part of the operator string is to  \\
	                                                &                    &  be computed sequentially. This is specified by the  variable  \\
	                                                &                    &  \texttt{N\_sequential\_start}  and \texttt{Nt\_sequential\_end}. A number of   \texttt{N\_tau\_Global}\\
	                                                &                    &  user defined global moves on the given time slice  will then be carried out.   \\
        \texttt{Global\_moves} &  Logical        &   If true, allows for global moves in space and time.   A number of  \texttt{N\_Global }    \\ 
        			                   &                      & user defined global moves in space and time  will be carried out at the end  \\ 
			                   &                      & of each sweep. \\
         \texttt{TEMPERING}   &  Compiling  &    Requires MPI and  runs the code in a parallel tempering mode.  \\
         & option & \\
         \\\bottomrule
   \end{tabular}
   \caption{Variables required to control the updating scheme. \label{table:Updating_schemes}}
\end{table}
% 
%------------------------------------------------------------
\subsubsection{Sampling of $e^{-S_0}$}
%------------------------------------------------------------
% 
This section applies for single spin flip updates  in the present of a non-vanishing Ising action $S_0(C)$. 
Consider an Ising spin at space-time $i,\tau$ in the configuration $C$. Flipping this spin will generate the configuration $C'$ and we will propose the move according to 
  \begin{equation}
 T_0(C \rightarrow C')  =  \frac{e^{-S_0(C')}}{ e^{-S_0(C')} + e^{-S_0(C)} }   = 1 - \frac{1}{1 +  e^{-S_0(C')} /e^{-S_0(C)}}
  \end{equation}
 Note that the function $\texttt{S0}$ in the  \texttt{Hamitonian\_example.f90}  module  computes precisely the ratio\\
 ${e^{-S_0(C')} /e^{-S_0(C)}}$ so that  $T_0(C \rightarrow C') $ does not require any further programming. 
 Thereby one will accept  the proposed move with the probability: 
 \begin{equation}
 P(C \rightarrow C') =  \text{min}  \left( 1,  \frac{e^{-S_0(C)}   W(C')}{ e^{-S_0(C')} W(C)} \right).
 \end{equation}
 With Eq.~\ref{eqn:partition_2}  one sees that the bare action $S_0(C)$  determining the  dynamics of the Ising spin  in the absence of coupling to the fermions  does not enter the Metropolis acceptance-rejection step. This sampling scheme is used if the logical variable \texttt{Propose\_S0} is set to \texttt{true}.
% 
%------------------------------------------------------------
\subsubsection{Sequential  single spin flips}
%------------------------------------------------------------
%
If the boolean variable \texttt{Sequential\_Sweeps}  is set to true, then  the program  adopt a single spin flip sequential strategy. It will  visit sequentially each  Hubbard Stratonovich field in the space time operator list and  propose a spin flip. Consider  the Ising spin $s_{i,\tau}$. We will flip it with probability one such that for  this local move  the  proposal matrix is symmetric.  If we are considering the Hubbard-Stratonovich field $l_{i,\tau}$  we will propose with probability $1/3$ one  of the other three  possible fields.   Again, for this local move, the proposal matrix is symmetric.  Hence in both cases we will accept or reject the move according to 
 \begin{equation}
 	P(C \rightarrow C') =  \text{min}  \left( 1, \frac{ W(C')}{W(C)} \right).
 \end{equation}
 It is worth noting that this type of sequential spin flip updating does not satisfy detailed balance but the more fundamental stationarity condition \cite{Sokal89}. 
% 

%------------------------------------------------------------
\subsubsection{Global updates in space}
\label{Global_space.sec}
This option allows  carry out  user defined global moves on a single time slice.  This option is enabled by setting the logical variable  \texttt{Global\_tau\_moves}   to true.  To set the stage  we recall that the propagation over a time step $\Delta \tau$   (see Eq. \ref{Btau.eq}) can be  written as: 
\begin{equation}
	e^{-V_{M_I+M_V}(s_{M_I+M_V,\tau})}  \cdots e^{-V_{1}(s_{1,\tau})}  \prod_{k=1}^{M_T}   e^{-\Delta \tau {\bm T}^{(k)}}  
\end{equation}
where $e^{-V_{n}(s_{n})}$ denotes one element of the  operator list  containing the HS fields.  One can provide  an interval of indices, 
$ \left[ \texttt{Nt\_sequential\_start}, 
\texttt{Nt\_sequential\_end} \right] $,  in which the operators will be updated  sequentially. Setting $ \texttt{Nt\_sequential\_start} =1 $ and 
$ \texttt{Nt\_sequential\_end} = M_I+M_V$  reproduces the  sequential single spin flip strategy of the above section.   One can also define globals moves on this time-slice.  Let  \texttt{N\_tau\_Global}  be the number of global move to be carried out on the given time slice \texttt{ntau}.  Each global move is generated in the routine  \texttt{Global\_move\_tau} provided by the user in the Hamiltonian file.    To  define this move, one specifies the following: 
\begin{itemize}
\item \texttt{Flip\_length}:  An integer  array  stipulating the  number of spin that one  will flip.
\item \texttt{Flip\_list(1:Flip\_length)}:   Integer array containing the  indices of the operators  one will flip.
\item \texttt{Flip\_value(1:Flip\_length)}:  \texttt{Flip\_value(n)} is an  integer containing the new value of the  HS  field for the operator 
\texttt{Flip\_list(1:Flip\_length)}
\item  \texttt{T0\_Proposal\_ratio}:   Real number containing  the quotient
\begin{equation}
	 \frac{T_0(C' \rightarrow C)}{T_0(C \rightarrow C') }  \;.
\end{equation}
where $ C'$  denotes the new configuration  obtained by flipping the spins specified in the \texttt{Flip\_list}  array. 
Since we allow for a stochastic  generation of  the global move, it may very well be that no change is proposed. In this case, \texttt{T0\_Proposal\_ratio}   takes the value 0 upon exit of the routine \texttt{Global\_move\_tau} and  no update will be carried out. 
\item \texttt{S0\_ratio}:   Real number containing  the ration  $e^{-S_0(C')}/e^{-S_0(C)}$. 
\end{itemize}
%------------------------------------------------------------
%------------------------------------------------------------
\subsubsection{Global updates in time and space}
%------------------------------------------------------------
%  
The code equally allows for global updates.  The user will have to provide two other functions in the module \texttt{Hamiltonian\_Examples.f90}.   


The subroutine  \texttt{Global\_move(T0\_Proposal\_ratio,nsigma\_old,size\_clust)}  proposes  a global move. 
The two-dimensional array \texttt{nsigma\_old(M\_V+ M\_I, Ltrot)}  contains  the full  configuration $C$.  On output, the new configuration,   C', determined by the user,  is to be stored in the 
array  \texttt{nsigma(M\_V+ M\_I, Ltrot)}.   The global variable \texttt{nsigma(M\_V+ M\_I, Ltrot)} is declared in the module \texttt{Hamiltonian}.  Equally, on output, the variable 
\texttt{T0\_Proposal\_ratio} contains the proposal ratio 
\begin{equation}
	 \frac{T_0(C' \rightarrow C)}{T_0(C \rightarrow C') }  \;.
\end{equation}
Since we allow for a stochastic  generation of  the global move, it may very well be that no change is proposed. In this case, \texttt{T0\_Proposal\_ratio}   takes the value 0 upon exit, and  
\texttt{nsigma=nsigma\_old}.   
Furthermore, the real-valued \texttt{size\_clust} should give the size of the proposed move (e.g. $\tfrac{\text{Number of flipped spins}}{\text{Total number of spins}}$). This will be used to calculate the average sizes of proposed and of accepted moves which will be printed in the \texttt{info} file. For the simulation it is not necessary to calculate \texttt{size\_clust}, but the information will help the user to estimate the effectiveness of the global update.


To compute the acceptance-rejection ratio,  the user  will equally have to provide the function \\
\texttt{Delta\_S0\_global(Nsigma\_old)} that computes the ratio $e^{-S_0(C')}/e^{-S_0(C)}$. Again the configuration $C'$ is   given by the array \texttt{nsigma(M\_V+ M\_I, Ltrot)}  which is 
a global variable declared in the module \texttt{Hamiltonian}.

Note that global updates are expensive since they require a complete recalculation of the weight. We thereby  allow the user to set a variable \texttt{N\_Global} that allows to  determine how many global updates per sweeps will be carried out. 
% 

%------------------------------------------------------------
\subsubsection{Parallel tempering } 
%------------------------------------------------------------
% 
Exchange Monte Carlo \cite{Hukushima96}  or parallel tempering \cite{Greyer91} is a possible route to overcome sampling issues in parts of the parameter space.
Let $h$ be a parameter which one can vary without  altering the configuration space $ \{C  \}  $ and let us assume that for some values of $h$ one encounters sampling problems.   For example, in the realm of spin glasses, $h$  could correspond to the  inverse temperature.  Here at high temperatures,  phase space is easily sampled   but at low temperatures  simulations get stuck in local minima. For quantum systems, $h$ could   trigger a quantum phase transition where  sampling issues are encountered, for example, in the ordered phase and not in the disordered one.   As its name suggests, parallel tempering  carries out in parallel simulations at consecutive  values of  $h$:  $h_1, h_2, h_3   \cdots h_n$, with  $h_{1} < h_2 < \cdots < h_n$.  One will sample the extended ensemble: \begin{equation}
	P(\left[h_1,C_1\right], \left[h_2,C_2\right], \cdots, \left[h_n,C_n\right] ) =  \frac{W(h_1,C_1) W(h_2,C_2) \cdots   W(h_n,C_n) } {\sum_{C_1, C_2, \cdots, C_n} W( h_1,C_1) W( h_2,C_2) \cdots   W(h_n,C_n)}
\end{equation}
where $W(h,C)$ corresponds   to the weight  for  for a given value of $h$ and configuration C. 
Clearly, one can sample  $P( \left[h_1,C_1\right], \left[h_2,C_2\right], \cdots, \left[h_n,C_n\right])$ by carrying out $n$ independent runs.
However, parallel tempering  includes the following   exchange step:
\begin{equation}
	\left[h_1,C_1\right], \cdots, \left[h_i,C_i\right],\left[h_{i+1},C_{i+1}\right] \cdots, \left[h_n,C_n\right]   \rightarrow 
	\left[h_1,C_1\right], \cdots, \left[h_i,C_{i+1}\right],\left[h_{i+1},C_{i}\right] \cdots, \left[h_n,C_n\right] 
\end{equation}
which, for a symmetric proposal matrix, will  be accepted with probability: 
\begin{equation}
	\text{ min} \left( 1,   \frac{ W(h_i,C_{i+1}) W(h_{i+1},C_{i})}{W(h_i,C_{i}) W(h_{i+1},C_{i+1})} \right).
\end{equation}
 Thereby,  a configuration can meander in parameter space $h$ and  explore regions where ergodicity  is not an issue.     In the context of spin-glasses,  a low temperature  configuration, stuck in a local minima, can heat up, overcome the potential  barrier and then cool down again. 
 
The choice of the   $h_i$'s  is important  to  obtain a good acceptance rate for the exchange step.  With  $W(h,C)  = e^{- S(h,C) }$, the  distribution of the action $S$  reads:
\begin{equation}
	 {\cal P}( h, S ) =   \sum_{C}     P( h,C )   \delta ( S(h,C) -  S ). 
\end{equation} 
Acceptance of the exchange  step requires the distributions  ${ \cal P}( h, S )  $ and       ${ \cal P}( h  + \Delta h , S )  $ to overlap. For 
$\langle S \rangle_{h}  < \langle S \rangle_{h +  \Delta h} $   one can formulate this  requirement as:
\begin{equation}
	\langle S \rangle_{h}  +\langle \Delta S \rangle_{h}   \simeq \langle S \rangle_{h +  \Delta h}  - \langle \Delta S \rangle_{h + \Delta h} ,  \text{    with   }   
\langle \Delta S \rangle_{h}   =  \sqrt{ \langle \left(    S -  \langle S   \rangle_h  	\right)^2 \rangle_h} .
\end{equation}
Assuming  $ \langle \Delta S \rangle_{h + \Delta h}  \simeq \langle \Delta S \rangle_{h} $  and expanding in $\Delta h$ one obtains: 
\begin{equation}
	\Delta h \simeq \frac{ 2  \langle \Delta S \rangle_{h}    }{ \partial \langle S \rangle_{h} / \partial h}.  
\end{equation} 
The above equation becomes transparent  for  classical systems  with $ S(h,C) =  h H(C) $.  In this case, the above equation reads: 
\begin{equation}
	\Delta h       \simeq  2 h \frac{  \sqrt{C} } { C    + h \langle H \rangle_h},  \text{   with  } C = h^2    \langle \left(  H -  \langle H   \rangle_h \right)^2 \rangle_h .
\end{equation} 
Several comments are in order.
\begin{itemize}
\item[i)] Let us identify $h$ with the inverse temperature  such that $C$ corresponds to the specific heat. This quantity is extensive,  as well as the energy, such that $ \Delta h \simeq 1/{\sqrt{N}} $ where $N$ is the system size.
\item[ii)] In the proximity of a phase transition,   the specific heat can diverge such that   care has to be taken in the choices of  $h$.
\item[iii)]  Since the action is formulation dependent, one expects the acceptance of the  exchange move to equally depend  upon the fomulation. 
\end{itemize}
%\mycomment{MB: Do you track the $n-1$ exchange acceptance rates $acc(i,i+1)$ for the $n$ replicas in the code? Could the exchange rates be an efficient way to locate  a phase transition in the parameters space of $h$, without a priori knowing the order parameter? Also for topological phase transitions w/o an order parameter?  }
%\mycomment{FFA:  Yes I do track the individual acceptances and I do see  singularities in the  acceptance at the phase transition. However, owing to comment iii) at would now be very careful at interpreting the results since they are really formulation dependent. }
 The auxiliary field quantum Monte Carlo code in the ALF project  comes with a parallel tempering  compiler option which we will discuss  in section \ref{Parallel.Sec}. 
