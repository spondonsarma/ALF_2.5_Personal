% Copyright (c) 2016-2020 The ALF project.
% This is a part of the ALF project documentation.
% The ALF project documentation by the ALF contributors is licensed
% under a Creative Commons Attribution-ShareAlike 4.0 International License.
% For the licensing details of the documentation see license.CCBYSA.

% !TEX root = doc.tex

\subsection{Predefined trial wave functions} \label{sec:predefined_trial_wave_function}

When using the projective algorithm (see Sec.~\ref{sec:defT0}), trial wave functions must be specified.
%These are stored in variables of the \texttt{WaveFunction} type.
%, which includes the routine \path{WF_overlap(WF_L, WF_R, Z_norm)} for normalizing the right trial wave function, such that $\langle \Psi_{T,L} | \Psi_{T,R} \rangle = 1$.
The ALF package provides a set of predefined trial wave functions $|\Psi_{T,L/R}\rangle$=\texttt{WF\_L/R}, returned by the call:

\begin{lstlisting}[style=fortran]
Call Predefined_TrialWaveFunction(Lattice_type, Ndim, List, Invlist, Latt, Latt_unit, N_part,
                                  N_FL, WF_L, WF_R)
\end{lstlisting}
Twisted boundary conditions (\texttt{Phi\_X\_vec=0.01}) are implemented for some lattices so as to generate a non-degenerate trial wave function. Here the marker "\texttt{\_vec}" indicates the variable may assume different values depending on the flavor (e.g., spin up and down). Currently the predefined trial wave functions are flavor independent.

The predefined trial wave functions correspond to the solution of the non-interacting tight binding Hamiltonian on each of the predefined lattices. These solutions are the ground states of the predefined hopping matrices (Sec.~\ref{sec:predefined_hopping}) with default parameters, for each lattice, as follows.

\subsubsection{Square}

Parameter values for the predefined trial wave function on the square lattice:
\begin{lstlisting}[style=fortran]
Checkerboard  = .false.
Symm          = .false.
Bulk          = .false.
N_Phi_vec     = 0
Phi_X_vec     = 0.01d0
Phi_Y_vec     = 0.d0
Ham_T_vec     = 1.d0
Ham_Chem_vec  = 0.d0
Dtau          = 1.d0
\end{lstlisting}
%Ham_T2_vec    = 0.d0
%Ham_Tperp_vec = 0.d0


\subsubsection{Honeycomb}

\red{[JSEP: I'm not sure how to best describe the definition of the trial wave function for the honeycomb, since it doesn't use the Predefined Hopping module.\\
	  Also, do we want to mention the \texttt{Kekule\_Trial} option?]}\\
Parameter values for the predefined trial wave function on the Honeycomb lattice:
\begin{lstlisting}[style=fortran,escapechar=\#]
Checkerboard  = .false.
Symm          = .false.
Bulk          = .false.
N_Phi_vec     = 0
Phi_X_vec     = 0.d0
Phi_Y_vec     = 0.d0
Ham_T_vec     = 1.d0
Ham_Tperp_vec = 0.d0
Ham_Chem_vec  = 0.d0
Dtau          = 1.d0
\end{lstlisting}
%delta         = 0.01d0
%Kekule_Trial  = .false. #\red{(include?)}#
%Ham_T2_vec    = 0.d0


\subsubsection{N-leg ladder}

Parameter values for the predefined trial wave function on the N-leg ladder lattice:
\begin{lstlisting}[style=fortran]
Checkerboard  = .false.
Symm          = .false.
Bulk          = .false.
N_Phi_vec     = 0
Phi_X_vec     = 0.01d0
Phi_Y_vec     = 0.d0
Ham_T_vec     = 1.d0
Ham_Tperp_vec = 1.d0
Ham_Chem_vec  = 0.d0
Dtau          = 1.d0
\end{lstlisting}
%Ham_T2_vec    = 0.d0


\subsubsection{Bilayer square}

Parameter values for the predefined trial wave function on the bilayer square lattice:
\begin{lstlisting}[style=fortran]
Checkerboard  = .false.
Symm          = .false.
Bulk          = .false.
N_Phi_vec     = 0
Phi_X_vec     = 0.d0
Phi_Y_vec     = 0.d0
Ham_T_vec     = 1.d0
Ham_T2_vec    = 0.d0
Ham_Tperp_vec = 1.d0
Ham_Chem_vec  = 0.d0
Dtau          = 1.d0
\end{lstlisting}


\subsubsection{Bilayer honeycomb}

Parameter values for the predefined trial wave function on the bilayer honeycomb lattice:
\begin{lstlisting}[style=fortran]

Checkerboard  = .false.
Symm          = .false.
Bulk          = .false.
N_Phi_vec     = 0
Phi_X_vec     = 0.d0
Phi_Y_vec     = 0.d0
Ham_T_vec     = 1.d0
Ham_T2_vec    = 0.d0
Ham_Tperp_vec = 1.d0
Ham_Chem_vec  = 0.d0
Dtau          = 1.d0
\end{lstlisting}

