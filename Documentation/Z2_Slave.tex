% Copyright (c) 2016 2017 The ALF project.
% This is a part of the ALF project documentation.
% The ALF project documentation by the ALF contributors is licensed
% under a Creative Commons Attribution-ShareAlike 4.0 International License.
% For the licensing details of the documentation see license.CCBYSA.
% !TEX root = Doc.tex
\newpage
\subsubsection{$Z_2$ slave spin formualtion of the Hubbard model }
In this subsection, we  demonstrate that the code can be used to  simulate the attractive Hubbard model in the  $Z_2$-slave spin formulation \cite{Ruegg10}.    
\begin{equation}
	\hat{H} = -t \sum_{\langle \vec{i}, \vec{j} \rangle, \sigma }\hat{c}^{\dagger}_{\vec{i},\sigma} \hat{c}^{\phantom{\dagger}}_{\vec{j},\sigma}   -  U  \sum_i  
	\left( \hat{n}_{\vec{i}, \uparrow} - 1/2\right)  \left( \hat{n}_{\vec{i}, \downarrow} - 1/2\right)
\end{equation}  
In the $Z_2$ slave spin  representation, the physical fermion, $\hat{c}_{\vec{i},\sigma} $,   is fractionalized into  an Ising spin carrying $Z_2$ charge and a fermion, $\hat{f}_{\vec{i},\sigma} $, carrying $Z_2$ and  global $U(1)$ charge:
\begin{equation}
	\hat{c}^{\dagger}_{\vec{i},\sigma}  = \hat{\tau}^{z}_{\vec{i}} \hat{f}^{\dagger}_{\vec{i},\sigma}.
\end{equation}
To ensure that we remain in the correct Hilbert space, the constraint:
\begin{equation}
	\hat{\tau}^{x}_{i}   - (-1)^{\sum_{\sigma}\hat{f}^{\dagger}_{i,\sigma}  \hat{f}^{\phantom{\dagger}}_{i,\sigma}  }  = 0
\end{equation}
has to be impose locally. Since $\left(  \tau^{x}_{\vec{i}}\right)^2 = 1 $ it is   equivalent to 
 \begin{equation}
 	\hat{Q}_{\vec{i}} = \tau^{x}_{\vec{i}}  (-1)^{\sum_{\sigma}\hat{f}^{\dagger}_{\vec{i},\sigma}  \hat{f}^{\phantom{\dagger}}_{\vec{i},\sigma}  }   = 1.
 \end{equation}
 In the  $Z_2$ slave spin representation the Hubbard model now reads: 
 \begin{equation}
 	 \hat{H}_{Z_2}= -t \sum_{\langle \vec{i}, \vec{j} \rangle, \sigma }  \hat{\tau}^{z}_{\vec{i}}  \hat{\tau}^{z}_{\vec{j}} \hat{f}^{\dagger}_{\vec{i},\sigma} \hat{f}^{\phantom{\dagger}}_{\vec{j},\sigma}   -  \frac{U}{4}  \sum_{\vec{i}}  \hat{\tau}^{x}_{\vec{i}}
 \end{equation}
 and  one will readily see that the constraint will commute with Hamiltonian: 
 \begin{equation}
 	\left[ \hat{H}_{Z_2}, \hat{Q}_{\vec{i}} \right] = 0.
 \end{equation}
 One can foresee that the constraint will be dynamically imposed  and that at  $T=0$ on a finite lattice both models should give the same results.    
 
 To implement  this Hamiltonian in the ALF,  it is convenient to  carry out the variable substitution  
 \begin{equation}
 	\hat{Z}_{\langle \vec{i},\vec{j}\rangle} = \hat{\tau}^{z}_{\vec{i}}  \hat{\tau}^{z}_{\vec{j}} 
 \end{equation}
 such that 
 \begin{equation}
 	\hat{\tau}^{x}_{\vec{i}}  = \hat{X}_{\vec{i},\vec{i} +  \vec{a}_x} \hat{X}_{\vec{i},\vec{i} -  \vec{a}_x} \hat{X}_{\vec{i},\vec{i} +  \vec{a}_y} \hat{X}_{\vec{i},\vec{i} -  \vec{a}_y}.
 \end{equation}
 Since there are twice as many bond variables as site variables, a constraint has to be  imposed on the $\hat{Z}_{\langle \vec{i},\vec{j}\rangle} $ variables. In fact, it is easy to see that the flux per plaquette has to take a unit value: 
 \begin{equation}
 \label{Z_constraint.Eq}
 \hat{Z}_{\vec{i},\vec{i} + \vec{a}_x} \hat{Z}_{\vec{i} +\vec{a}_x,\vec{i} + \vec{a}_x +  \vec{a}_y} 
\hat{Z}_{\vec{i} + \vec{a}_x +  \vec{a}_y ,\vec{i} + \vec{a}_y} \hat{Z}_{ \vec{i} + \vec{a}_y, \vec{i}}   = 1 \; \; \;  \forall  \; \; \;  \vec{i}.
 \end{equation}
 
 
 Within this formulation  the model  takes  the form: 
 \begin{equation}
 	 \hat{H}_{Z_2}= -t \sum_{\langle \vec{i}, \vec{j} \rangle, \sigma }  \hat{Z}_{\langle \vec{i}, \vec{j} \rangle } \hat{f}^{\dagger}_{\vec{i},\sigma} \hat{f}^{\phantom{\dagger}}_{\vec{j},\sigma}   -  \frac{U}{4}  \sum_{\vec{i}}  \hat{X}_{\vec{i},\vec{i} +  \vec{a}_x} \hat{X}_{\vec{i},\vec{i} -  \vec{a}_x} \hat{X}_{\vec{i},\vec{i} +  \vec{a}_y} \hat{X}_{\vec{i},\vec{i} -  \vec{a}_y}.
 \end{equation}
The fermion part has formally the same from  as in the Hubbard model coupled to  a dynamical Ising field  discussed in Sec.~\ref{sec:walk2}.   There are however two important differences.
\begin{itemize}
\item The moves have to respect the constraint of Eq.~\ref{Z_constraint.Eq}. Thereby single spin flip terms are prohibited and the minimal move one can carry out on  a given time slice is the following. We randomly choose a site $\vec{i} $ and  propose a move where: 
$ Z_{\vec{i},\vec{i} +  \vec{a}_x} \rightarrow - Z_{\vec{i},\vec{i} +  \vec{a}_x} $,  $ Z_{\vec{i},\vec{i} -  \vec{a}_x} \rightarrow - Z_{\vec{i},\vec{i} -  \vec{a}_x} $,
$ Z_{\vec{i},\vec{i} +  \vec{a}_y} \rightarrow - Z_{\vec{i},\vec{i} +  \vec{a}_y} $ and $ Z_{\vec{i},\vec{i} -  \vec{a}_y} \rightarrow - Z_{\vec{i},\vec{i} -  \vec{a}_y} $.  One can carry out such moves by using the global move in real space option presented in Secs.~\ref{Global_space.sec} and \ref{sec:input}.
\item The map from  $ \left\{ \tau^{z}_{\vec{i}}  \right\} $ to $ \left\{ Z_{\langle \vec{i}, \vec{j} \rangle } \right\} $  is unique.  The reverse however  is valid only up to  to a global sign.  To pin down this sign  (and thereby   the  relative signs between different time slices)  we  store per time slice the $  Z_{\langle \vec{i},\vec{j} \rangle } $ fields as well as the value of the Ising field  at  a reference site $\tau^{z}_{\vec{i} = 1} $. Within the ALF, this can be done by adding a dummy operator in the \texttt{Op\_V}  list which will carry this degree of freedom.    With this extra degree of freedom we can switch  between the two representations, without loosing any information.   To compute the Ising part of the action it is certainly more transparent to work  with the $ \left\{ \tau^{z}_{\vec{i}}  \right\} $  variables. For the  fermion determinant,  the $ \left\{ Z_{\langle \vec{i}, \vec{j} \rangle } \right\} $   are more convenient. 
 \end{itemize}
 
 We have carried out some test simulations at half-filling and at low temperatures.  The simulation can be found in directory \texttt{Examples/Z2\_Slave} and the  Hamiltonian in \texttt{Hamiltonian\_Z2\_slave\_spins.f90 }.  The simulations  are carried out at half-filling such that particle-hole symmetry leads to 
 \begin{equation}
 \langle \hat{Q}_{\vec{i}}    \rangle_{H_{Z_2}} =0.
 \end{equation} 
However the simulations suggest that 
 \begin{equation}
 \frac{1}{N}\sum_{i,j} \langle \hat{Q}_{\vec{i}}   \hat{Q}_{\vec{j}} \rangle_{H_{Z_2}}  = N 
 \end{equation} 
 at low temperatures  where  $N$  corresponds  to size of the lattice.  Note that this measurement is very noisy,  and suffers from very long autocorrelation times.  Thereby, the constraint is dynamically imposed and   simulations of the attractive Hubbard model with  \texttt{Hamiltonian\_Examples.f90}  should yield identical results. To test this,  we have computed equal time Green functions:
\begin{equation}
\langle  \hat{\tau}^{z}_{\vec{i}} \hat{f}^{\dagger}_{\vec{i},\sigma} \hat{\tau}^{z}_{\vec{j}} \vec{f}^{\phantom{\dagger}}_{\vec{j},\sigma} \rangle_{H_{Z_2}} = 
\langle  \hat{c}^{\dagger}_{\vec{i},\sigma} \hat{c}^{\phantom{\dagger}}_{\vec{j},\sigma} \rangle_{H} 
\end{equation}
as obtained  from  the \texttt{Hamiltonian\_Examples.f90} and    \texttt{Hamiltonian\_Z2\_slave\_spins.f90 } codes.  
A test run for the $8\times 8 $ lattice at $U/t = 4$ and $\beta t = 40$ gives: 
\begin{center}
\begin{tabular}{ l | c | r }
 \hline			
   k   &  $\langle n_k \rangle_{H} $  &  $\langle n_k \rangle_{H_{Z_2}} $ \\
  \hline
   (0,0)                               & 1.93348548    $\pm$    0.00011322  & 1.93336807   $\pm$      0.00080473 \\
   ($\pi/4$,$\pi/4$)             & 1.90120688     $\pm$   0.00014854  & 1.90107164    $\pm$     0.00097029  \\
   ($\pi/2$,$\pi/2$)             & 0.99942957     $\pm$   0.00091377  & 1.00000000    $\pm$     0.00000000\\
   ($3\pi/4$,$3\pi/4$)         &  0.09905425     $\pm$   0.00015940 & 0.09892836    $\pm$     0.00097029 \\
   ($\pi$,$\pi$)                   & 0.06651452     $\pm$   0.00011321  & 0.06663193     $\pm$    0.00080473 \\
  \hline  
\end{tabular}

\end{center} 
\vspace*{0.5cm}
Here a Trotter time step of  $\Delta \tau t = 0.05$ was used  so as to minimize the systematic error   which should be different  between the two codes. 

\newpage