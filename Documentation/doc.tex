% Copyright (c) 2016 The ALF project.
% This is the ALF project documentation.
% The ALF project documentation by the ALF contributors is licensed
% under a Creative Commons Attribution-ShareAlike 4.0 International License.
% For the licensing details of the documentation see license.CCBYSA.


\documentclass[10pt,Arial]{scrartcl}
\usepackage{graphicx}
\usepackage[margin=2.5cm]{geometry}
\usepackage[utf8]{inputenc}
\usepackage{bbm}            
\usepackage[fleqn]{amsmath} 
\usepackage{amssymb}
\usepackage{wasysym}
\usepackage{color}
\usepackage{soul}
\usepackage{xspace}
\usepackage{bm}
\usepackage{subfigure}
\usepackage{tcolorbox}
\usepackage{dsfont}
\usepackage{footnote}
\usepackage{alltt}
\usepackage{xcolor}
\usepackage{lmodern}
\usepackage{listings}
\usepackage{url}
\usepackage{booktabs}
%\usepackage{hyperref} 
\usepackage[breaklinks=true,colorlinks,pdfborder={0 0 0}]{hyperref} 
\hypersetup{linkcolor={blue!80!black},citecolor={blue!80!black},urlcolor={blue!99!black}}
\usepackage{float} 

\usepackage[framemethod=default]{mdframed}
\usepackage{showexpl}

\makeatletter
\renewcommand\paragraph{\@startsection{paragraph}{4}{\z@}%
            {-2.5ex\@plus -1ex \@minus -.25ex}%
            {1.25ex \@plus .25ex}%
            {\normalfont\normalsize\bfseries}}
\makeatother
\setcounter{secnumdepth}{4} 
\setcounter{tocdepth}{4}
\definecolor{light-gray}{gray}{0.95}

\lstdefinestyle{fortran}{
  language=Fortran,
  basicstyle=\ttfamily,
  keywordstyle=\color{red},
  commentstyle=\color{blue},
  morecomment=[l]{!\ }% Comment only with space after !
  breakatwhitespace=false,
  keepspaces=true,
  showstringspaces=false,
  columns=flexible,
  backgroundcolor=\color{light-gray},
  frame=single
}

\lstdefinestyle{fortran_pseudo_code}{
  language=Fortran,
  basicstyle=\ttfamily,
  keywordstyle=\color{red},
  commentstyle=\normalfont\color{black},
  morecomment=[l]{!\ }% Comment only with space after !
  breakatwhitespace=false,
  keepspaces=true,
  showstringspaces=false,
  columns=flexible,
  backgroundcolor=\color{white},
  frame=single,
  escapeinside={\!*}{*)},
}

\lstdefinestyle{bash}{
  language=bash,
  basicstyle=\ttfamily,
  keywordstyle=\color{red},
  commentstyle=\color{blue},
  morecomment=[l]{\#\ }% Comment only with space after #
  breakatwhitespace=false,
  keepspaces=true,
  showstringspaces=false,
  columns=flexible
}

\def\Tr{\mathop{\mathrm{Tr}}}
\def\Trf{\mathop{\mathrm{Tr}_{\mathrm{F}}}}
\makesavenoteenv{tabular}
\makesavenoteenv{table}

% % only for the scrartcl class:
\setkomafont{author}{\large}
\setkomafont{date}{\large}
% \RedeclareSectionCommand[style=section,indent=0pt]{part}
% \renewcommand*\partformat{\thepart\autodot\enskip}

\newcommand{\mycomment}[1]{{\color{red} #1}}
\newcommand{\FAcomment}[1]{{\color{red} #1}}

\renewcommand{\Re}{\operatorname{Re}}
\renewcommand{\Im}{\operatorname{Im}}

\makeindex
\begin{document}
%---------------------------------------------------------------------------------------------------------
\title{The \emph{ALF} (\emph{A}lgorithms for \emph{L}attice \emph{F}ermions) project release 1.2}

\subtitle{Documentation for the  auxiliary-field quantum Monte Carlo code.}
\author{Martin Bercx,  Florian Goth,  Johannes S. Hofmann, Fakher F. Assaad }
%---------------------------------------------------------------------------------------------------------
\maketitle

Copyright \textcopyright ~2016, 2017 The \textit{ALF} Project.\\
This is the ALF Project Documentation by the ALF contributors.
It is licensed under a Creative Commons Attribution-ShareAlike 4.0 International License.
You are free to share and benefit from this documentation as long as this license is preserved
and proper attribution to the authors is given. For details see the ALF project
homepage \url{alf.physik.uni-wuerzburg.de}.
\tableofcontents
\clearpage
\section{Introduction}\label{sec:intro}
% !TEX root = doc.tex
% Please do not remove this.
\section{Introduction}\label{sec:intro}
The auxiliary field quantum Monte Carlo approach has by now proven to be a key algorithm to simulate a variety of  electron systems where correlations effects play a dominant role.  This includes correlation effects in topological band structures, quantum phase transitions between semimetals and insulators, deconfined quantum critical points, topologically ordered phases, heavy fermion systems, nematic and magnetic quantum phase transitions in metals,   superconductivity in the presence of spin orbit coupling. This list of ever growing phenomena is based on  recent  symmetry based insights, which allows one to  find formulations that avoid the so called negative sign problem.   The aim of this project is to introduce a general formulation of the auxiliary methods, 

 \mycomment{placeholder for a general introduction, mentioning the purpose and the powers of the general QMC code.}
This documentation is organized as follows. The general Hamiltonian operator is written down in Sec.~\ref{sec:def}, followed by 
a brief outline  of the quantum Monte Carlo algorithm. 
In Sec.~\ref{sec:imp}, we discuss the implementation of a model, introducing the \texttt{Operator} data structure which is the building block of the Hamiltonian. And we disuss the implementation of the lattice and the observables.
Section ~\ref{sec:io} is about actually running the code. We describe input and output files, the analysis protocoll and the compilation procedure. 
In Sec.~\ref{sec:walk1} and \ref{sec:walk2} two detailed walkthroughs are performed: the $SU(2)$-symmetric Hubbard  on a square lattice (Sec.~\ref{sec:walk1}) and the same model, but additionally coupled to a transverse field Ising model (Sec.~\ref{sec:walk2}).



\section{Auxiliary Field Quantum Monte Carlo: finite temperature }\label{sec:def}
% !TEX root = Doc.tex
\section{Definition of the model Hamiltonian}\label{sec:def}

%\mycomment{Notation: Hats for second quantized operators, bold for matrices. \\
%  Structure:  \\ 1) We first want to define the model.  \\ 
 %                     2)  implementation of the QMC.  \\ 
 %                  3) Data structure  \\ 
%                   4) Practical implementation and some  simple test cases. }

 
The class of solvable models includes  Hamiltonians $\hat{\mathcal{H}}$ that have the following general form:
\begin{eqnarray}
\hat{\mathcal{H}}&=&\hat{\mathcal{H}}_{T}+\hat{\mathcal{H}}_{V} +  \hat{\mathcal{H}}_{I} +   \hat{\mathcal{H}}_{0,I}\;,\mathrm{where}
\label{eqn:general_ham}\\
\hat{\mathcal{H}}_{T}
&=&
\sum\limits_{k=1}^{M_T}
\sum\limits_{s=1}^{N_{\mathrm{fl}}}
\sum\limits_{\sigma=1}^{N_{\mathrm{col}}}
\sum\limits_{x,y}^{N_{\mathrm{dim}}}
\hat{c}^{\dagger}_{x \sigma   s}T_{xy}^{(k s)} \hat{c}^{\phantom\dagger}_{y \sigma s}  \equiv  \sum\limits_{k=1}^{M_T} \hat{T}^{(k)}
\label{eqn:general_ham_t}\\
\hat{\mathcal{H}}_{V}
&=&
-
\sum\limits_{k=1}^{M_V}U_{k}
\left\{
\sum\limits_{s=1}^{N_{\mathrm{fl}}}
\sum\limits_{\sigma=1}^{N_{\mathrm{col}}}
\left[
\left(
\sum\limits_{x,y}^{N_{\mathrm{dim}}}
\hat{c}^{\dagger}_{x \sigma s}V_{xy}^{(k s)}\hat{c}^{\phantom\dagger}_{y \sigma s}
\right)
-\alpha_{k s} 
\right]
\right\}^{2}  \equiv   -
\sum\limits_{k=1}^{M_V}U_{k}   \left(\hat{V}{(k)} \right)^2
\label{eqn:general_ham_v}\\
\hat{\mathcal{H}}_{I}  & = &
\sum\limits_{k=1}^{M_I} \hat{Z}_{k} 
\left\{
\sum\limits_{s=1}^{N_{\mathrm{fl}}}
\sum\limits_{\sigma=1}^{N_{\mathrm{col}}}
\left[
\sum\limits_{x,y}^{N_{\mathrm{dim}}}
\hat{c}^{\dagger}_{x \sigma s} I_{xy}^{(k s)}\hat{c}^{\phantom\dagger}_{y \sigma s}
\right]
\right\} \equiv \sum\limits_{k=1}^{M_I} \hat{Z}_{k}    \hat{I}^{(k)} 
\;.\label{eqn:general_ham_i}
\end{eqnarray}
The indices have the following meaning:
\begin{itemize}
\item The number of fermion \textit{flavors} is set by $N_{\mathrm{fl}}$.  After the Hubbard-Stratonovich transformation, the action will be block diagonal in the flavor index. 
\item The number of fermion \textit{colors} is set by $N_{\mathrm{col}}$.    The Hamiltonian is invariant under  SU($N_{\mathrm{col}}$)  rotations. Note that  in the code $ N_{\mathrm{col}} \equiv N_{sun} $. 
%\mycomment{ Does it set the symmetry group of the fermions, namely 
%the dimension of the special unitary group $SU(N_{sun})$?}
\item The indices $x,y$ label lattice sites where $x,y=1,\cdots, N_{\mathrm{dim}}$. 
$N_{\mathrm{dim}}$ is the total number of spacial vertices: $N_{\mathrm{dim}}=N_{unit\;cell} N_{orbital}$, where $N_{unit\;cell}$ is the number of unit cells of the underlying Bravais lattice and $N_{orbital}$ is the number of (spacial) orbitals per unit cell \mycomment{Check the definition of $N_{orbital}$ in the code.} 
\item Therefore, the  matrices $\bm{T}^{(k s)}$, $\bm{V}^{(ks)}$  and $\bm{I}^{(ks)}$ are  of dimension $N_{\mathrm{dim}}\times N_{\mathrm{dim}}$
\item The number of interaction terms  is labelled by $M_V$   and $M_I$.   $M_T> 1 $ would allow for a checkerboard decomposition. 
\item $\hat{Z}_k$ is an Ising spin operator which corresponds to the Pauli matrix $\hat{\sigma}_{z}$. It couples to a general one-body term. 
\item  $\mathcal{H}_{0,I}$  gives the dynamics of the Ising spins. 
This term has to be specified by the user and is only relevant when the Monte Carlo update probability is computed in the code (see Sec.~\ref{}).
%\mycomment{Be more general here and sreak of correlated blocks?}
\end{itemize}
Note that the matrices  $\bm{T}^{(ks)}$,  $\bm{V}^{(ks)}$ and  $\bm{I}^{(ks)}$ explicitly depend on the flavor index $s$ but not on the color index $\sigma$. 
The color index $\sigma$ only appears in  the  second quantized operators such that the Hamiltonian is manifestly SU($N_{\mathrm{col}}$)    symmetric.  We also require
the matrices $\bm{T}^{(ks)}$,  $\bm{V}^{(ks)}$ and  $\bm{I}^{(ks)}$  to be  hermitian.


\subsection{Formulation of the QMC}  
\subsubsection{The partition function}
The formulation of the  Monte Carlo simulation is based on the following.
\begin{itemize}
\item  We will discretize the imaginary time propagation: $\beta = \Delta \tau L_{\text{Trotter}} $
\item  We will use  the   discrete Hubbard-Stratonovich transformation:
\begin{equation}
\label{HS_squares}
        e^{\Delta \tau  \lambda  \hat{A}^2 } =
        \sum_{ l = \pm 1, \pm 2}  \gamma(l)
e^{ \sqrt{\Delta \tau \lambda }
       \eta(l)  \hat{A} }
                + {\cal O} (\Delta \tau ^4)\;,
\end{equation}
where the fields $\eta$ and $\gamma$ take the values:
\begin{eqnarray}
 \gamma(\pm 1)  = 1 + \sqrt{6}/3, \quad  \eta(\pm 1 ) = \pm \sqrt{2 \left(3 - \sqrt{6} \right)}\;,\\\nonumber
  \gamma(\pm 2) = 1 - \sqrt{6}/3, \quad  \eta(\pm 2 ) = \pm \sqrt{2 \left(3 + \sqrt{6} \right)}\;.
\nonumber
\end{eqnarray}
\item  We will work in  a basis  where  $\hat{Z}_k$ is diagonal: $\hat{Z}_{k}|s_{j}\rangle = s_{k}\delta_{kj}|s_{k}\rangle$, where $s_{k}=\pm 1$.
\item From the above it follows that the  Monte Carlo configuration space $C$  
is given by the combined spaces of Ising spin configurations  and of Hubbard-Stratonovich discrete field configurations:
\begin{equation}
	C = \left\{   s_{i,\tau} ,  l_{j,\tau}  \text{ with }  i=1\cdots M_I,\;  j = 1\cdots M_V,\; \tau=1\cdots L_{\mathrm{Trotter}}  \right\}
\end{equation}
Here, the Ising spins take the values  $s_{i,\tau} = \pm 1$ and  the Hubbard-Stratonovich fields take the values  $l_{j,\tau}  = \pm 2, \pm 1 $.
\end{itemize}
With the above, the partition function of the model (\ref{eqn:general_ham}) can be written as follows.
\begin{eqnarray}
Z &=& \Tr{\left(e^{-\beta \hat{\mathcal{H}} }\right)}\nonumber\\
  &=&   \Tr{  \left[ e^{-\Delta \tau \hat{\mathcal{H}}_{0,I}}   \prod_{k=1}^{M_T}   e^{-\Delta \tau \hat{T}^{(k)}}  
    \prod_{k=1}^{M_V}   e^{  \Delta \tau  U_k \left(  \hat{V}^{(k)} \right)^2}   \prod_{k=1}^{M_I}   e^{  -\Delta \tau  \hat{\sigma}_{k}  \hat{I}^{(k)}} 
   \right]^{L_{\text{Trotter}}}}  \nonumber \\
   &=&
   \sum_{C} \left( \prod_{j=1}^{M_V} \prod_{\tau=1}^{L_{\mathrm{Trotter}}} \gamma_{j,\tau} \right) e^{-S_{0,I} \left( \left\{ s_{i,\tau} \right\}  \right) }\times \nonumber\\
   &\quad&
    \Trf{ \left\{  \prod_{\tau=1}^{L_{\mathrm{Trotter}}} \left[   \prod_{k=1}^{M_T}   e^{-\Delta \tau \hat{T}^{(k)}}  
    \prod_{k=1}^{M_V}   e^{  \sqrt{ \Delta \tau  U_k} \eta_{k,\tau} \hat{V}^{(k)} }   \prod_{k=1}^{M_I}   e^{  -\Delta \tau s_{k,\tau}  \hat{I}^{(k)}}  \right]\right\} }
\end{eqnarray}
In the above,  the trace $\mathrm{Tr} $  runs over the Ising spins as well as over the fermionic degrees of freedom, and $ \mathrm{Tr}_{\mathrm{F}}  $ only over the  fermionc Fock space. 
$S_{0,I} \left( \left\{ s_{i,\tau} \right\}  \right)  $ is the action  corresponding to the Ising Hamiltonian,  and is only dependent on the Ising spins so that  it can be pulled out of the fermionic trace.
At this point,  and  since for a given configuration $C$  we are dealing with a free propagation, we can integrate out the fermions to obtain a determinant: 
\begin{eqnarray}
 &\quad&\Trf{ \left\{  \prod_{\tau=1}^{L_{\mathrm{Trotter}}} \left[   \prod_{k=1}^{M_T}   e^{-\Delta \tau \hat{T}^{(k)}}  
    \prod_{k=1}^{M_V}   e^{  \sqrt{ \Delta \tau  U_k} \eta_{k,\tau} \hat{V}^{(k)} }   \prod_{k=1}^{M_I}   e^{  -\Delta \tau s_{k,\tau}  \hat{I}^{(k)}}  \right] \right\}} = \nonumber\\
&\quad& \quad\prod\limits_{s=1}^{N_{\mathrm{fl}}} \left[  e^{- \sum_{k=1}^{M_V} \sum_{\tau = 1}^{L_{\mathrm{Trotter}}}\sqrt{\Delta \tau U_k}  \alpha_{k,s} \eta_{k,\tau} }
   \right]^{N_{\mathrm{col}}}\times
\nonumber\\
&\quad&\quad   \prod\limits_{s=1}^{N_{\mathrm{fl}}} 
   \left[
    \det\left(  1 + 
     \prod_{\tau=1}^{L_{\mathrm{Trotter}}}   \prod_{k=1}^{M_T}   e^{-\Delta \tau {\bf T}^{(ks)}}  
    \prod_{k=1}^{M_V}   e^{  \sqrt{ \Delta \tau  U_k} \eta_{k,\tau} {\bm V}^{(ks)} }   \prod_{k=1}^{M_I}   e^{  -\Delta \tau s_{k,\tau}  {\bm I}^{(ks)}}  
     \right) \right]^{N_{\mathrm{col}}}\;.
\end{eqnarray}
All in all,   the partition function is given by:
\begin{eqnarray}
    Z &=& \Tr{  \left( e^{-\beta \hat{\mathcal{H}} }\right) }\nonumber\\
    &=&   \sum_{C}   e^{-S_{0,I} \left( \left\{ s_{i,\tau} \right\}  \right) }     \left[ \prod_{k=1}^{M_V} \prod_{\tau=1}^{L_{\mathrm{Trotter}}} \gamma_{k,\tau} \right] 
    e^{- N_{\mathrm{col}}\sum_{s=1}^{N_{\mathrm{fl}}} \sum_{k=1}^{M_V} \sum_{\tau = 1}^{L_{\mathrm{Trotter}}}\sqrt{\Delta \tau U_k}  \alpha_{k,s} \eta_{k,\tau} } 
  \times   \nonumber \\
  &\quad&
      \prod_{s=1}^{N_{\mathrm{fl}}}\left[\det\left(  1 + 
     \prod_{\tau=1}^{L_{\mathrm{Trotter}}}   \prod_{k=1}^{M_T}   e^{-\Delta \tau {\bm T}^{(ks)}}  
    \prod_{k=1}^{M_V}   e^{  \sqrt{ \Delta \tau  U_k} \eta_{k,\tau} {\bm V}^{(ks)} }   \prod_{k=1}^{M_I}   e^{  -\Delta \tau s_{k,\tau}  {\bm I}^{(ks)}}  
     \right) \right]^{N_{\mathrm{col}}}  \nonumber \\ 
     & \equiv&  \sum_{C} e^{-S(C) }\;.
\end{eqnarray}
In the above, one notices that the weight factorizes in  the flavor index. The color index raises the determinant to the power $N_{\mathrm{col}}$. This corresponds to  an explicit $SU(N_{\mathrm{col}})$ symmetry   for each  configuration. This symmetry is manifest in the fact that the single particle  Green functions, again for a given  configuration C are color independent. 

\subsubsection{Observables}
In the auxiliary field QMC approach, the single particle Green function plays a crucial role.  It determines the Monte Carlo dynamics and is used to compute  observables:
\begin{equation}	
\langle \hat{O}  \rangle  = \frac{ \text{Tr}   \left[ e^{- \beta \hat{H}}  \hat{O}   \right] }{ \text{Tr}   \left[ e^{- \beta \hat{H}}  \right] } =   \sum_{C}   P(C) 
   \langle \langle \hat{O}  \rangle \rangle_{(C)} , \text{   with   } 
  P(C)   = \frac{ e^{-S(C)}}{\sum_C e^{-S(C)}}.
\end{equation}
For a given configuration $C$  one can use Wicks theorem to compute $O (C) $   from the knowledge of the single particle Green function: 
\begin{equation}
       G( x,\sigma,s, \tau |    x',\sigma',s', \tau')   =       \langle \langle {\cal T} \hat{c}^{\phantom\dagger}_{x \sigma s} (\tau)  \hat{c}^{\dagger}_{x' \sigma' s'} (\tau') \rangle \rangle_{C}
\end{equation}
and the corresponding equal time quantity reads, 
\begin{equation}
       G( x,\sigma,s, \tau |    x',\sigma',s', \tau)   =       \langle \langle {\cal T} \hat{c}^{\phantom\dagger}_{x \sigma s} (\tau)  \hat{c}^{\dagger}_{x' \sigma' s'} (\tau) \rangle \rangle_{C}
\end{equation}
Since  for a given HS field translation invariance in imaginary time is broken, the Green function has an explicit $\tau$ and $\tau'$ dependence.   On the other hand it is diagonal in the 

 

% Copyright (c) 2016 The ALF project.
% This is a part of the ALF project documentation.
% The ALF project documentation by the ALF contributors is licensed
% under a Creative Commons Attribution-ShareAlike 4.0 International License.
% For the licensing details of the documentation see license.CCBYSA.

% !TEX root = Doc.tex
%------------------------------------------------------------
\subsection{Updating schemes}\label{sec:updating}
%------------------------------------------------------------
%
The program allows for different types of updating schemes.    Given a configuration $C$ we propose a new one, $C'$, with probability $T_0(C \rightarrow C')$  and accept it according to   the  Metropolis-Hastings   acceptance-rejection probability, 
\begin{equation}
	P(C \rightarrow C') =  \text{min}  \left( 1, \frac{T_0(C' \rightarrow C) W(C')}{T_0(C \rightarrow C') W(C)} \right),
\end{equation}
so as to guarantee the stationarity condition.  Here, $ W(C) = \left| \Re \left[ e^{-S(C)} \right] \right|   $.

\begin{table}[h]
   \begin{tabular}{@{} l l l @{}}\toprule
        Variable  &  Type                  &  Description   \\
         \\\midrule
       \texttt{Propose\_S0}   &    Logical       &  If true, proposes local moves according to the probability $e^{-S_0}$ \\
       \texttt{Global\_moves} & Logical       & If true, allows for global moves. \\
        \texttt{N\_Global }       & Integer        &   Number of global moves per sweep of single spin flips. \\
        \texttt{TEMPERING}   & Compiling option &    Requires MPI and  runs the code in a parallel tempering mode. 
         \\\bottomrule
   \end{tabular}
   \caption{   Variables  required to control the updating scheme    \label{table:Updating_schemes}}
\end{table}
% 
%------------------------------------------------------------
\subsubsection{The default: sequential  single spin flips}
%------------------------------------------------------------
%
The default updating scheme is a  sequential single  spin flip algorithm.   Consider   the Ising spin $s_{i,\tau}$, we will flip it with probability one such that for  this local move  the  proposal matrix is symmetric.  If we are considering the Hubbard-Stratonovich field $l_{i,\tau}$  we will propose with probability $1/3$ one  of the other three  possible fields.   Again, for this local move, the proposal matrix is symmetric.  Hence in both cases we will accept or reject the move according to 
 \begin{equation}
 	P(C \rightarrow C') =  \text{min}  \left( 1, \frac{ W(C')}{W(C)} \right)
 \end{equation}
 It is worth noting that this type of sequential spin flip updating does not satisfy detailed balance but the more fundamental stationarity condition. 
% 
%------------------------------------------------------------
\subsubsection{Sampling of $e^{-S_0}$}
%------------------------------------------------------------
% 
Consider an Ising spin at space-time $i,\tau$ and the configuration $C$. Flipping this spin will generate the configuration $C'$ and we will propose the move according to 
  \begin{equation}
 T_0(C \rightarrow C')  =  \frac{e^{-S_0(C')}}{ e^{-S_0(C')} + e^{-S_0(C)} }   = 1 - \frac{1}{1 +  e^{-S_0(C')} /e^{-S_0(C)}}
  \end{equation}
 Note that the function $\texttt{S0}$ in the  \texttt{Hamitonian\_example.f90}  module  computes precisely the ratio\\
 ${e^{-S_0(C')} /e^{-S_0(C)}}$ so that  $T_0(C \rightarrow C') $ does not require any further programming. 
 Thereby one will accept  the proposed move with the probability: 
 \begin{equation}
 P(C \rightarrow C') =  \text{min}  \left( 1,  \frac{e^{-S_0(C)}   W(C')}{ e^{-S_0(C')} W(C)} \right).
 \end{equation}
 With Eq.~\ref{eqn:partition_2}  one sees that the bare action $S_0(C)$  determining the  dynamics of the Ising spin  in the absence of coupling to the fermions  does not enter the Metropolis acceptance-rejection step.  This sampling scheme is used  if the logical variable \texttt{Propose\_S0}   is switched on. 
% 
%------------------------------------------------------------
\subsubsection{Global updates}
%------------------------------------------------------------
%  
The code equally allows for global updates.  The user will have to provide two other functions in the module \texttt{Hamiltonian\_Examples.f90}.   

The subroutine  \texttt{Global\_move(T0\_Proposal\_ratio,nsigma\_old)}  proposes  a global move. 
The two-dimensional array \texttt{nsigma\_old(M\_V+ M\_I, Ltrot)}  contains  the full  configuration $C$.  On output, the new configuration,   C', determined by the user,  is to be stored in the 
array  \texttt{nsigma(M\_V+ M\_I, Ltrot)}.   The global variable \texttt{nsigma(M\_V+ M\_I, Ltrot)} is declared in the module \texttt{Hamiltonian}.  Equally, on output, the variable 
\texttt{T0\_Proposal\_ratio} contains the proposal ratio 
\begin{equation}
	 \frac{T_0(C' \rightarrow C)}{T_0(C \rightarrow C') }  \;.
\end{equation}
Since we allow for a stochastic  generation of  the global move, it may very well be that no change is proposed. In this case, \texttt{T0\_Proposal\_ratio}   takes the value 0 upon exit, and  
\texttt{nsigma=nsigma\_old}.   

To compute the acceptance-rejection ratio,  the user  will equally have to provide the function \\
\texttt{Delta\_S0\_global(Nsigma\_old)} that computes the ratio $e^{-S_0(C')}/e^{-S_0(C)}$. Again the configuration $C'$ is   given by the array \texttt{nsigma(M\_V+ M\_I, Ltrot)}  which is 
a global variable declared in the module \texttt{Hamiltonian}.

Note that global updates are expensive since they require a complete recalculation of the weight. We thereby  allow the user to set a variable \texttt{N\_Global} that allows to  determine how many global updates per sweeps will be carried out. 
% 
%------------------------------------------------------------
\subsubsection{Parallel tempering } 
%------------------------------------------------------------
% 
Exchange Monte Carlo \cite{Hukushima96}  or parallel tempering \cite{Greyer91}   is a possible route to overcome  sampling issues in part of  parameter space.  Let $h$ be a parameter which one can vary without  altering the configuration space $ \{C  \}  $ and let us assume that for some values of $h$ one encounters sampling problems.   For example, in the realm of spin glasses, $h$  could correspond to the  inverse temperature.  Here at high temperatures,  phase space is easily sampled   but at low temperatures  simulations get stuck in local minima. For quantum systems, $h$ could   trigger a quantum phase transition where  sampling issues are encountered, for example, in the ordered phase and not in the disordered one.   As its name suggests, parallel tempering  carries out in parallel simulations at consecutive  values of  $h$:  $h_1, h_2, h_3   \cdots h_n$, with  $h_{1} < h_2 < \cdots < h_n$.  One will sample the extended ensemble: \begin{equation}
	P(\left[h_1,C_1\right], \left[h_2,C_2\right], \cdots, \left[h_n,C_n\right] ) =  \frac{W(h_1,C_1) W(h_2,C_2) \cdots   W(h_n,C_n) } {\sum_{C_1, C_2, \cdots, C_n} W( h_1,C_1) W( h_2,C_2) \cdots   W(h_n,C_n)}
\end{equation}
where $W(h,C)$ corresponds   to the weight  for  for a given value of $h$ and configuration C.     Clearly, one can sample  $P( \left[h_1,C_1\right], \left[h_2,C_2\right], \cdots, \left[h_n,C_n\right])$ by carrying out $n$-independent runs.  However, parallel tempering  includes the following   exchange step:
\begin{equation}
	\left[h_1,C_1\right], \cdots, \left[h_i,C_i\right],\left[h_{i+1},C_{i+1}\right] \cdots, \left[h_n,C_n\right]   \rightarrow 
	\left[h_1,C_1\right], \cdots, \left[h_i,C_{i+1}\right],\left[h_{i+1},C_{i}\right] \cdots, \left[h_n,C_n\right] 
\end{equation}
which, for a symmetric proposal matrix, will  be accepted with probability: 
\begin{equation}
	\text{ min} \left( 1,   \frac{ W(h_i,C_{i+1}) W(h_{i+1},C_{i})}{W(h_i,C_{i}) W(h_{i+1},C_{i+1})} \right).
\end{equation}
 Thereby,  a configuration can meander in parameter space $h$ and  explore regions where ergodicity  is not an issue.     In the context of spin-glasses,  a low temperature  configuration, stuck in a local minima, can heat up, overcome the potential  barrier and then cool down again. 
 
The choice of the   $h_i$'s  is important  to  obtain a good acceptance rate for the exchange step.  With  $W(h,C)  = e^{- S(h,C) }$, the  distribution of the action $S$  reads:
\begin{equation}
	 {\cal P}( h, S ) =   \sum_{C}     P( h,C )   \delta ( S(h,C) -  S ). 
\end{equation} 
Acceptance of the exchange  step requires the distributions  ${ \cal P}( h, S )  $ and       ${ \cal P}( h  + \Delta h , S )  $ to overlap. For 
$\langle S \rangle_{h}  < \langle S \rangle_{h +  \Delta h} $   one can formulate this  requirement as:
\begin{equation}
	\langle S \rangle_{h}  +\langle \Delta S \rangle_{h}   \simeq \langle S \rangle_{h +  \Delta h}  - \langle \Delta S \rangle_{h + \Delta h} ,  \text{    with   }   
\langle \Delta S \rangle_{h}   =  \sqrt{ \langle \left(    S -  \langle S   \rangle_h  	\right)^2 \rangle_h} .
\end{equation}
Assuming  $ \langle \Delta S \rangle_{h + \Delta h}  \simeq \langle \Delta S \rangle_{h} $  and expanding in $\Delta h$ one obtains: 
\begin{equation}
	\Delta h \simeq \frac{ 2  \langle \Delta S \rangle_{h}    }{ \partial \langle S \rangle_{h} / \partial h}.  
\end{equation} 
The above equation becomes transparent  for  classical systems  with $ S(h,C) =  h H(C) $.  In this case, the above equation reads: 
\begin{equation}
	\Delta h       \simeq  2 h \frac{  \sqrt{C} } { C    + h \langle H \rangle_h},  \text{   with  } C = h^2    \langle \left(  H -  \langle H   \rangle_h \right)^2 \rangle_h
\end{equation} 
Several comments are in order. i)    Let us identify $h$ with the inverse temperature  such that $C$ corresponds to the specific heat. This quantity is extensive,  as well as the energy, such that $ \Delta h \simeq 1/{\sqrt{N}} $ where $N$ is the system size.   ii) In the proximity of a phase transition,   the specific heat can diverge such that   care has to be taken in the choices of  $h$.  iii)  Since the action is formulation dependent,   one expects the acceptance of the  exchange move to equally depend  upon the fomulation. 

%\mycomment{MB: Do you track the $n-1$ exchange acceptance rates $acc(i,i+1)$ for the $n$ replicas in the code? Could the exchange rates be an efficient way to locate  a phase transition in the parameters space of $h$, without a priori knowing the order parameter? Also for topological phase transitions w/o an order parameter?  }
%\mycomment{FFA:  Yes I do track the individual acceptances and I do see  singularities in the  acceptance at the phase transition. However, owing to comment iii) at would now be very careful at interpreting the results since they are really formulation dependent. }
 The auxiliary field quantum Monte Carlo code in the ALF project  comes with a parallel tempering  compiler option which we will discuss  in section \ref{Parallel.Sec}. 

% !TEX root = doc.tex
% Copyright (c) 2016 The ALF project.
% This is a part of the ALF project documentation.
% The ALF project documentation by the ALF contributors is licensed
% under a Creative Commons Attribution-ShareAlike 4.0 International License.
% For the licensing details of the documentation see license.CCBYSA.
%
%-----------------------------------------------------------------------------------
\subsection{Stabilization - A Peculiarity of the BSS Algorithm}\label{sec:stable}
%-----------------------------------------------------------------------------------
%
From \eqref{eqn:partition_2} it can be seen that for the calculation of the Monte Carlo weight
and for the observables a long product of matrix exponentials has to be formed.
On top of that we need to be able to extract the single particle Green function  for a given flavor index at say time slice $\tau = 0$.  As  mentioned above, this quantity is given by: 
\begin{equation}
G = \left( 1 + \prod_{ \tau= 1}^{L_{\text{Trotter}}} B_\tau \right)^{-1}.
\end{equation}
To boil this down to more familiar terms from linear algebra we remark that we can recast this problem as the question to the solution of the linear system
\begin{equation}
(1 + \prod_\tau B_\tau) x = b.
\end{equation}
The $B_\tau$ depend on the system size as well as other physical parameters that can be chosen such that a matrix norm of $B_i$ can have any number.
From standard perturbation theory for linear systems it is known that the computed solution $\tilde{x}$ would 
contain a relative error of
\begin{equation}
\frac{|\tilde{x} - x|}{|x|} = \mathcal{O}\left(\epsilon \kappa(1 + \prod_\tau B_\tau)\right).
\end{equation}
Here $\epsilon$ denotes the machine precision which is $2^{-53}$ for IEEE double precision numbers
and $\kappa(M)$ is the condition number of the matrix $M$.
The important fact that makes straight-forward inversion so badly suited  stems from the fact that $  \prod_ \tau B_\tau $ contains exponentially large and small scales as can be seen in \eqref{eqn:partition_2}.  Thereby, as a function of increasing inverse temperature, 
the condition number  will grow exponentially so that the computed solution $\tilde{x}$
will often contain no correct digits at all.
To circumvent this more sophisticated methods have to be employed.   We will first of all assume that  the multiplication of  \texttt{NWRAP}  B matrices   has an acceptable condition number.   Assuming for simplicity that \texttt{NWRAP} is a multiple of  $L_{\text{Trotter}}$, we  can write: 
\begin{equation}
G = \left( 1 + \prod_{ i = 0}^{L_{\text{Trotter}} /\texttt{NWrap} -1}       \underbrace{\prod_{\tau=1}^{\texttt{NWrap}} B_{i  \cdot  \texttt{NWrap}+ \tau} }_{ \equiv {\cal B}_i}\right)^{-1}.
\end{equation}

ALF is by default employing
the strategy of forming a product of QR-decompositions which was proven to be weakly backwards stable in \cite{Bai2011}.
The key idea is to efficiently separate the scales of a matrix from the orthogonal part of a matrix.
This can be achieved using a QR decomposition of the ${\cal B}_i = Q_i \tilde{R_i}$. $Q_i$ is a unitary matrix and hence $\kappa(Q_i) = 1$.
To get a handle on the condition number of $\tilde{R}_i$ we will form the
diagonal matrix $(D_i)_{jj} = |(\tilde{R}_i)_{jj}|$ and rescale $\tilde{R}_i$ accordingly, $\tilde{R}_i = D_i R_i$.
This gives the decomposition
\begin{equation}
{\cal B}_i = Q_i D_i R_i.
\end{equation}
$D_i$ now contains the row norms of the original $\tilde{R}_i$ matrix and hence separates off the total scales of the problem since $\tilde{R}_i$ is now only of modest condition number.  \mycomment{ FFA. You are guessing this.}
This given an initial decomposition of $B_{j-1} = Q_{j-1} D_{j-1} T_{j-1}$ any product 
of ${\cal B}$ matrices is formed in the following two steps:
\begin{enumerate}
\item Form $ M_j = ({\cal B}_j Q_{j-1}) D_{j-1}$. Note the parentheses.
\item Do a QR decomposition of $M_j = Q_j D_j R_j$.
\item Form the updated $R$ matrices $T_j = R_j T_{j-1}$.
\end{enumerate}
%While this provides provides a stable method to calculate the involved matrix product
%it can be pretty expensive. Therefore the user can specify to skip a certain number of 
%QR Decompositions and perform plain multiplications instead. This is specified in the parameters file by the \path{NWrap} parameter.
%\path{NWrap}~=~1 corresponds to always performing QR decompositions whereas larger integers give longer intervals where no QR decomposition will be performed.
The effectiveness of the stabilization \emph{HAS} to be judged for every simulation from the info
file. For most simulations there are two values to look out for:
\begin{itemize}
\item \path{Precision Green}
\item \path{Precision Phase}
\end{itemize}
The Green's function as well as the average phase are usually numbers with a magnitude of $\mathcal{O} (1)$. 
For that reason we recommend that \path{Nwrap} is chosen such that the mean precision is  of the order of $10^{-8}$  or better.  
\mycomment{Think about formulation}

% !TEX root = doc.tex
% Copyright (c) 2016 The ALF project.
% This is a part of the ALF project documentation.
% The ALF project documentation by the ALF contributors is licensed
% under a Creative Commons Attribution-ShareAlike 4.0 International License.
% For the licensing details of the documentation see license.CCBYSA.
%
%------------------------------------------------------------
\subsection{Monte Carlo sampling}\label{sec:sampling}
%------------------------------------------------------------
%
The default updating scheme consists of local moves which change (upon acceptance) only one  entry of $L_{\mathrm{Trotter}}(M_I+M_V)$  fields (see Sec. \ref{sec:updating}). 
To generate  an independent configuration $C$,   one has to visit at least each field  once.  Our unit of \textbf{sweeps} is defined such that each field is visited twice in a sequential propagation from $\tau = 0$ to $\tau = L_{\text{ Trotter}}$  and back.  A single sweep will  generically not  suffice to produce an independent  configuration.
% This is however only the lower bound as there can be a region in the spin space where the fields are correlated and it requires a larger or even global move to significantly change the configuration to an independent one. One might imagine a ferromagnet due to spontaneous symmetry breaking. All spins are parallel aligned and, let' say, point upwards. The configuration of only down spins is equally justified, but rotating one to the other requires a global operation. Flipping the spins individually one after another generates intermediate states of relative high energy which corresponds to a low probability in the QMC algorithm.
In fact, the auto-correlation time, $T_\mathrm{auto}$, characterizes the required time scale to generate an independent configuration or values $\langle\langle\hat{O}\rangle\rangle_C$ for the Observable $O$.

This has several consequences for the Monte Carlo simulation:
\begin{itemize}
	\item First of all, we start from a randomly chosen field configuration such that one has to invest at least one $T_\mathrm{auto}$ to generate relevant configurations before reliable measurements are possible. This phase of the simulation is known as the warm-up. In order to keep the code as flexible as possible (different simulations might have different auto-correlation times), measurements are taken from the very beginning. Instead we provide the parameter \path{n_skip} for the analysis to ignore the first \path{n_skip} bins.
	\item Secondly, our implementation averages over a given amount of measurements   set by the variable \texttt{NSWEEPS}  before storing the results, known as one bin, on the disk.  A bin corresponds to \texttt{NSWEEPS}  sweeps. The  error analysis requires statistically  independent bins to generate reliable confidence estimates. If bins are to small (averaged over a period shorter then $T_\mathrm{auto}$), the error bars are then typically underestimated. Most of the time, the auto-correlation time is unknown before the simulation is started, sometime, the compute cluster does not allow single runs long enough to generate appropriately sized bins. Therefore we provide the \path{N_rebin} parameter that specifies how many bins are combined into a new bin during the error analysis. In general, one should check, that a further increase of the bin size does not change the error estimate.  (For an explicit example, the reader is referred to the Appendix of Ref.~\cite{Assaad02}.)

The \path{N_rebin} variable can be used to control a second issue. The distribution of the Monte Carlo estimates $\langle\langle\hat{O}\rangle\rangle_C$ are unknown. The result in the form $(\mathrm{best}\pm \mathrm{error})$ assumes a Gaussian distribution. Luckily, every original distribution with a finite variance turns into a Gaussian one, once it is folded often enough (central limit theorem). Due to the internal averaging (folding) within one bin, many observables are already quite Gaussian. Otherwise one can increase \path{N_rebin} further, even if the bins are already independent~\cite{Bercx17}.
	\item The third issue concerns time displaced correlation functions. Even if the configurations are independent, the fields within the configuration are still correlated. Hence, the data for $S_{\alpha,\beta}(\vec{k},\tau)$ (see Sec.~\ref{sec:obs}; Eqn.~\ref{eqn:s}) and $S_{\alpha,\beta}(\vec{k},\tau+\Delta\tau)$ are also correlated. Setting the switch \path{N_Cov = 1} triggers the calculation of the covariance matrix in addition to the usual error analysis. The covariance is defined by
	\begin{equation}
		Cov_{\tau \tau'}=\frac{1}{N_{Bins}}\left\langle\left(S_{\alpha,\beta}(\vec{k},\tau)-\langle S_{\alpha,\beta}(\vec{k},\tau)\rangle\right)\left(S_{\alpha,\beta}(\vec{k},\tau')-\langle S_{\alpha,\beta}(\vec{k},\tau')\rangle\right)\right\rangle\,.
	\end{equation}
An example where this information is necessary in in the  calculation of mass gaps extracted by fitting the  tail  of the time-displaced correlation function.  Omitting  the covariance matrix will  underestimate the  error.
\end{itemize}

% !TEX root = doc.tex
% Copyright (c) 2017 The ALF project.
% This is a part of the ALF project documentation.
% The ALF project documentation by the ALF contributors is licensed
% under a Creative Commons Attribution-ShareAlike 4.0 International License.
% For the licensing details of the documentation see license.CCBYSA.
%
%------------------------------------------------------------
\subsection{Pseudo code description}\label{sec:pseudocode}
%------------------------------------------------------------
%
The following pseudo code describes the main structure of the quantum Monte Carlo program (see \path{Prog/main.f90}):

\begin{mdframed}[frametitle={Implementation of the auxiliary-field QMC method:}]
{\setlength{\parindent}{0pt}
Set the Hamiltonian and the lattice:\\
\textbf{call} ham\_set\\
Read in an auxiliary-field configuration or generate it randomly:\\
\textbf{call} confin\\

This loop fills the storage, needed for the first actual Monte Carlo sweep:\\
\textbf{do} \texttt{ntau} from  \texttt{ltrot} to \texttt{1}\\
\hspace*{2em} Compute propagation matrices and store them at the stabilization points:\\
\hspace*{2em} \textbf{call} wrapul\\
\textbf{enddo}\\
 
Loop over bins. The bin defines the unit of Monte Carlo time:\\
\textbf{do} \texttt{nbc} from  \texttt{1} to \texttt{nbin} \\

   Loop over sweeps. Each sweep updates twice (sweeping upward and downward in time)\\
   the whole space-time lattice of auxiliary fields:\\
   \textbf{do} \texttt{nsw} from  \texttt{1} to \texttt{nsweep}  \\
   
      Upward sweep:\\
      \textbf{do} \texttt{ntau} from \texttt{1} to \texttt{ltrot}\\
      
         Propagate the Green function from time \texttt{ntau -1} to \texttt{ntau},  \\
         and compute new estimate (using sequential update scheme) of Green at \texttt{ntau}: \\
         \textbf{call} wrapgrup\\
         
         Stabilization: \\     
         \textbf{if} \texttt{ntau} is equal to a stabilization point)\\
            Compute propagation matrix from previous stabilization point to \texttt{ntau}: \\
            \textbf{call} wrapur\\
            Read from storage: propagation from \texttt{ltrot} to \texttt{ntau}\\
            Write to storage : the just computed propagation \\
                        
            Recalculate Green function at time \texttt{ntau} in a stable way:\\
            \textbf{call} cgr\\
            
            Check the precision between propagated and recalculated Green function\\
         \textbf{endif}\\
        
         Measure the equal time observables, \\
         \textbf{if} \texttt{ntau} is in the intervall \texttt{[LOBS\_ST, LOBS\_EN]}:\\
         \textbf{call} obser\\
      \textbf{enddo}\\
      
      Downward sweep:\\
      \textbf{do} \texttt{ntau} from \texttt{ltrot} to \texttt{1}\\
         Repeat the above steps (update, propagation, stabilization, measurements) \\
         for the downward direction in imaginary time\\
      \textbf{enddo}\\
      
   \textbf{enddo} (loop over sweeps)\\
    
   Calculate averages of the measurements of the previous bin and write to disk\\
   Write auxiliary-field configuration to disk\\
   
\textbf{enddo} (loop over bins)     
   

}
\end{mdframed}
% 
% \noindent\fbox{%
%     \parbox{\textwidth}{%
%   ! Set the Hamiltonian and the lattice:\\
% call ham_set\\
% 
% ! Read in an auxiliary-field configuration or generate it randomly:\\
% call confin\\
% 
% ! This loop fills the storage, needed for the first actual Monte Carlo sweep:\\
% do ntau = ltrot, 1, -1 \\
%    ! Compute propagation matrices and store them at the stabilization points:\\
%    call wrapul \\
% enddo\\
% 
% ! Loop over bins. The bin defines the unit of Monte Carlo time:\\
% do nbc = 1, nbin \\
% 
%    ! Loop over sweeps. Each sweep updates twice (sweeping upward and downward in time)\\
%    ! the whole space-time lattice of auxiliary fields:\\
%    do nsw = 1, nsweep \\
%    
%       ! Upward sweep:\\
%       do ntau = 1, ltrot\\
%       
%          ! Propagate the Green function from time ntau -1 to ntau,  \\
%          ! and compute new estimate (using sequential update scheme) of Green at ntau: \\
%          call wrapgrup\\
%          
%          ! Stabilization: \\     
%          if (ntau == stabilization point)\\
%             ! Compute propagation matrix from previous stabilization point to ntau: \\
%             call wrapur\\
%             ! Read from storage: propagation from ltrot to ntau\\
%             ! Write to storage : the just computed propagation \\
%                         
%             ! Recalculate Green function at time ntau in a stable way:\\
%             call cgr\\
%             
%             ! Check the precision between propagated and recalculated Green function\\
%          endif\\
%         
%          ! Measure the equal time observables, \\
%          ! if ntau is in the measuring range [LOBS_ST, LOBS_EN]:\\
%          call obser\\
%       enddo\\
%       
%       ! Downward sweep:\\
%       do ntau = ltrot, 1, -1\\
%          ! Repeat the above steps (update, propagation, stabilization, measurements) \\
%          ! for the downward direction in imaginary time\\
%       enddo\\
%       
%    enddo ! Loop over sweeps\\
%     
%    ! Calculate averages of the measurements of the previous bin and write to disk\\
%    ! Write auxiliary-field configuration to disk\\
%    
% enddo ! Loop over bins     
%    
%    
%     }%
% }

\lstset{style=fortran_pseudo_code}
\begin{lstlisting}

! Set the Hamiltonian and the lattice:
call ham_set

! Read in an auxiliary-field configuration or generate it randomly:
call confin

! This loop fills the storage, needed for the first actual Monte Carlo sweep:
do ntau = ltrot, 1, -1 
   ! Compute propagation matrices and store them at the stabilization points:
   call wrapul 
enddo

! Loop over bins. The bin defines the unit of Monte Carlo time:
do nbc = 1, nbin 

   ! Loop over sweeps. Each sweep updates twice (sweeping upward and downward in time)
   ! the whole space-time lattice of auxiliary fields:
   do nsw = 1, nsweep 
   
      ! Upward sweep:
      do ntau = 1, ltrot
      
         ! Propagate the Green function from time ntau -1 to ntau, 
         ! and compute new estimate (using sequential update scheme) of Green at ntau: 
         call wrapgrup
         
         ! Stabilization:      
         if (ntau == stabilization point)
            ! Compute propagation matrix from previous stabilization point to ntau: 
            call wrapur
            ! Read from storage: propagation from ltrot to ntau
            ! Write to storage : the just computed propagation 
                        
            ! Recalculate Green function at time ntau in a stable way:
            call cgr
            
            ! Check the precision between propagated and recalculated Green function
         endif
        
         ! Measure the equal time observables, 
         ! if ntau is in the measuring range [LOBS_ST, LOBS_EN]:
         call obser
      enddo
      
      ! Downward sweep:
      do ntau = ltrot, 1, -1
         ! Repeat the above steps (update, propagation, stabilization, measurements) 
         ! for the downward direction in imaginary time
      enddo
      
   enddo ! Loop over sweeps
    
   ! Calculate averages of the measurements of the previous bin and write to disk
   ! Write auxiliary-field configuration to disk
   
enddo ! Loop over bins        

\end{lstlisting}

\section{Auxiliary Field Quantum Monte Carlo: projective algorithm }\label{sec:defT0}
% Copyright (c) 2016-2019 The ALF project.
% This is a part of the ALF project documentation.
% The ALF project documentation by the ALF contributors is licensed
% under a Creative Commons Attribution-ShareAlike 4.0 International License.
% For the licensing details of the documentation see license.CCBYSA.

% !TEX root = doc.tex

The projective  approach is the method of choice if  one is interested in ground state properties.   For a given trial wave functions  $| \Psi_{T,L/R} \rangle  $  that are  not orthogonal to the ground state,  $| \Psi_0 \rangle  $,   
($  \langle \Psi_{T,L/R}  | \Psi_T \rangle  \neq 0  $), the ground state expectation value of an observable  $\hat{O} $ is given by: 
\begin{equation}
	 \frac{ \langle \Psi_0 | \hat{O} | \Psi_0 \rangle }{ \langle \Psi_0 | \Psi_0 \rangle}   = \lim_{\theta \rightarrow \infty}  
	 \frac{ \langle \Psi_{T,L} | e^{-\theta \hat{H}}  e^{-(\beta - \tau)\hat{H}  }\hat{O} e^{- \tau  \hat{H} }   e^{-\theta \hat{H}} | \Psi_{T,R} \rangle } 
	        { \langle \Psi_{T,L} | e^{-(2 \theta + \beta) \hat{H}  } | \Psi_{T,R} \rangle } 
\end{equation}
The simulations are carried out at large  but finite values of  $\Theta$ so as to guarantee convergence to the ground  state within the statistical uncertainly.   $\beta$ denotes an imaginary time range where  observables 
(time displaced and equal time) can be measured.  


\subsection{The choice and specification of the trial wave function}
\section{Data Structures and Input/Output}\label{sec:imp}
% !TEX root = Doc.tex
\section{Implementation of the model} \label{sec:imp}
In the code, the module \texttt{Hamiltonian} defines the model Hamiltonian, the lattice under consideration and the desired observables (Table~\ref{table:hamiltonian}).
The respective file name is \texttt{Hamiltonian\_\textit{<Model Name>}.f90}: for example, \texttt{Hamiltonian\_Hub.f90} defines the plain Hubbard model on the two-dimensional square lattice. To implement a user-defined model, therefore only the module \texttt{Hamil}-\texttt{tonian} has to be set up. Accordingly, this documentation focusses almost entirely  on this module and the subprograms it includes. 
The remaining parts of the code may be treated as as a black box.  

To specify the Hamiltonian, one needs  an  \texttt{Operator},  \texttt{Lattice}   and  observable types. These three data structures will be described in the following. 

%
\begin{table}[h]
   \begin{tabular}{l l l}
    Name of &  &  \\
    subprogram & Description & Section \\\hline
    \texttt{Ham\_Set}  & Reads in model and lattice parameters from the file \texttt{parameters}. \\
                       & And it sets the Hamiltonian by calling \texttt{Ham\_latt}, \texttt{Ham\_hop}, and \texttt{Ham\_V}. & \\
    \texttt{Ham\_hop}  & Sets the hopping term  $\hat{\mathcal{H}}_{T}$ by calling \texttt{Op\_make} and \texttt{Op\_set}. & \ref{sec:op}, \ref{sec:specific}\\
    \texttt{Ham\_V}    & Sets the interaction terms  $\hat{\mathcal{H}}_{V}$ and $\hat{\mathcal{H}}_{I}$ 
                         by calling \texttt{Op\_make} and \texttt{Op\_set}.& \ref{sec:op}, \ref{sec:specific}\\  
    \texttt{Ham\_Latt} & Sets the lattice by calling \texttt{Make\_Lattice}.& \ref{sec:latt}\\
    \texttt{S0}        & A function which returns an update ratio for the Ising term $\hat{\mathcal{H}}_{I,0}$. 
    & \ref{sec:s0} \\
    \texttt{Alloc\_obs} & Asigns memory storage to the observables & \\
    \texttt{Init\_obs}  & Initializes the observables to zero. & \\
    \texttt{Obser}      & Computes the scalar observables and equal-time correlation functions. & \ref{sec:obs} \\
    \texttt{ObserT}     & Computes time-displaced correlation functions. & \ref{sec:obs}\\
    \texttt{Pr\_obs}    & Writes the observables to the disk by calling \texttt{Print\_bin}.   
    
   \end{tabular}
   \caption{   Overview of the subprograms of the  module \texttt{Hamiltonian} to define the Hamiltonian, the lattice and the observables.
    \label{table:hamiltonian}}
\end{table}
%

\subsection{The \texttt{Operator} type}\label{sec:op}
The fundamental data structure in the code is the derived data type \texttt{Operator}. 
This type is used to define the Hamiltonian (\ref{eqn:general_ham}).
In general, the matrices ${\bf T}^{(ks)}$, ${\bf V}^{(ks)}$ and ${\bf I}^{(ks)}$ are sparse Hermitian matrices.
Consider the  matrix   ${\bm X}$ of dimension  $N_{\mathrm{dim}} \times N_{\mathrm{dim}}$, as an representative of each of the above three matrices .  Let us  denote  with  $ \left\{z_{1},\cdots,  z_{N}  \right\}$  a subset  of $N$ indices,  
for which
\begin{equation}
X_{x,y}  =
\left\{\begin{matrix}  X_{x,y}  &  \text{ if }   x,  y  \in \left\{ z_1, \cdots z_N \right\}\\ 
                                  0         &  \text{ otherwise } 
      \end{matrix}\right.
\end{equation}
 We define the $N \times N_{\mathrm{dim}}$ matrices $\mathbf{P}$  as
\begin{equation}
P_{i,x}=\delta_{z_{i},x}\;,
\end{equation}
where $i \in [1,\cdots, N ]$ and $ x  \in [1,\cdots, N_{\mathrm{dim}}]$. The matrix  $\bm{P}$ picks out the non-vanishing entries of $\bm{X}$, 
which are contained in the rank-$N$  matrix $\bm{O}$.  Thereby: 
\begin{equation}
\bm{X} =\bm{P}^{T} \bm{O} \bm{P}\;,
\end{equation}
such that:
\begin{equation}
X_{x,y} = \sum\limits_{i,j}^{N}  P_{i,x}  O_{i,j} P_{j,y}=\sum\limits_{i,j}^{N} \delta_{z_{i},x}  O_{ij} \delta_{z_{j},y} \;.
\end{equation}
Since  the  $\bm{P}$ matrices have only one non-vanishing entry per column,  they can be stored as a vector $\vec{P}$:
\begin{equation}
     P_i = z_i.
\end{equation}  
There are  many useful  identities which emerge from this  structure. For example: 
\begin{equation}
	e^{\bm{X}} =  e^{\bm{P}^{T} \bm{O} \bm{P}}   = \sum_{n=0}^{\infty}  \frac{\left( \bm{P}^{T} \bm{O} \bm{P} \right)^n}{n!} =  \bm{P}^{T} e^{ \bm{O} } \bm{P}
\end{equation}
since 
\begin{equation} 
	 \bm{P} \bm{P}^{T}= 1_{N\times N}.
\end{equation}

In the code, we define a structure called \texttt{Operator} to capture the above. 
This type \texttt{Operator} bundles several components that are needed to define and use an operator matrix in the program.  

\subsubsection{Specification of the model}\label{sec:specific}
%
\begin{table}[h]
   \begin{tabular}{l l}
    Name of variable in the code & Description \\\hline
    \hl{\texttt{Op\_X\%N}}            &  effective dimension $N$ \\
    \hl{\texttt{Op\_X\%O}}            &  matrix  $\mathbf{O}$  of dimension $N \times N$\\
    \hl{\texttt{Op\_X\%P}}            &  projection matrix $\mathbf{P}$  encoded as a vector of dimension $N$.\\
    \hl{\texttt{Op\_X\%g}}            &  coupling strength $g$ \\  
    \hl{\texttt{Op\_X\%alpha}}      &  constant $\alpha$ \\
    \hl{\texttt{Op\_X\%type}}        &  integer parameter to set the type of 
                                             HS transformation\\
                                &  (1 = Ising, 2 = Discrete HS, for perfect square)  \\ 
    \texttt{Op\_X\%U}            &  matrix containing the eigenvectors of $\mathbf{O}$  \\
    \texttt{Op\_X\%E}            &  eigenvalues of $\mathbf{O}$ \\
    \texttt{Op\_X\%N\_non\_zero} &  number of non-vanishing eigenvalues of $\mathbf{O}$ 
   \end{tabular}
   \caption{Components of the \texttt{Operator}  type. 
   In the left column, the letter \texttt{X} is a placeholder for the letters \texttt{T} and \texttt{V}, 
   indicating hopping and interaction operators, respectively.
   The highlighted variables have to be specified by the user.
  %  One will have to specify $N$, $O$, $P$, $g$, $\alpha$ and the type.  The other variables will be automatically generated in the routine \texttt{Op\_Set}.  
    \label{table:operator}}
\end{table}
%
In order to specify the  Hamiltonian (\ref{eqn:general_ham}), we will  need several arrays of  structure variables \texttt{Operator}. Its components are listed in Table~\ref{table:operator}.  
Since the implementation exploits the $SU(N_{\mathrm{col}})$ invariance of the Hamiltonian, we have dropped the color index $\sigma$ in the following.
\begin{itemize}
\item Hopping Hamiltonian (\ref{eqn:general_ham_t}): 
In this case $\bm{X}=\bm{T}^{(k,s)}$. The corresponding array of structure variables \texttt{Op\_T} is  \texttt{Op\_T(M$_T$,N$_{fl}$)} . 
Precisely, a single variable  \texttt{Op\_T}  describes the operator matrix:
\begin{equation}
            \left( \sum_{x,y}^{N_{\mathrm{dim}}} \hat{c}^{\dagger}_x T_{xy}^{(ks)} \hat{c}^{\phantom{\dagger}}_{y}  \right)  \;,
\end{equation} 
where $k=[1, M_{T}]$ and $s=[1, N_{\mathrm{fl}}]$.
We have $g=-\Delta \tau$, $\alpha = 0$, and the type variable $\texttt{Op\_T\%type}$  is irrelevant. 



\item Interaction Hamiltonian (\ref{eqn:general_ham_v}):
If the interaction is of perfect-square type, we set  ${\bm X}  = \bm{V}^{(k,s)}$ 
and  define the corresponding structure variables \texttt{Op\_V}  using the array \texttt{Op\_V(M\_V,N\_{fl})}.
A single variable  \texttt{Op\_V}  describes the operator matrix:
\begin{equation}
             \left[ \left( \sum_{x,y}^{N_{\mathrm{dim}}} \hat{c}^{\dagger}_x V_{x,y}^{(ks)} \hat{c}^{\phantom{\dagger}}_{y}  \right) - \alpha_{ks} \right]  \;,
\end{equation} 
where $k=[1, M_{V}]$ and $s=[1, N_{\mathrm{fl}}]$. For the perfect-square interaction, $\alpha = \alpha_{ks}$ and $g = \sqrt{\Delta \tau  U_k}$. 
The discrete Hubbard-Stratonovich decomposition is selected by setting the type variable to $\texttt{Op\_V\%type}=2$.

\item Ising interaction Hamiltonian (\ref{eqn:general_ham_i}):
In this case, $\bm{X}  = \bm{I}^{(k,s)} $ and we define the array\\ \texttt{Op\_V(M\_I,N\_{fl})}.  
A single variable  \texttt{Op\_V} then  describes the operator matrix:
\begin{equation}
            \left( \sum_{x,y}^{N_{\mathrm{dim}}} \hat{c}^{\dagger}_x I_{xy}^{(ks)} \hat{c}^{\phantom{\dagger}}_{y}  \right)  \;,
\end{equation} 
where $k=[1, M_{I}]$ and $s=[1, N_{\mathrm{fl}}]$.
The Ising interaction is specified by setting the type variable  $\texttt{Op\_V\%type=1}$, $\alpha = 0$ and $g = -\Delta \tau$.  

\item In case of a full interaction [perfect-square term (\ref{eqn:general_ham_v}) and Ising term (\ref{eqn:general_ham_i})], we  define  the corresponding doubled array \texttt{Op\_V(M$_V$+M$_I$,N$_{fl}$) } and set the variables separately for both ranges of the array according to the above.  

\end{itemize}
  %      There is another array   which defines the full interaction,  Ising as well as perfect square terms. For this  we define  the array \texttt{Op\_V(M$_V$+M$_I$,N$_{fl}$) }). In this context the variable \texttt{Op\_V\%type} specifies the interaction: Ising or  a perfect square.  If the interaction is of Ising type, then  $\bm{V}  = \bm{I}^{(k,s)} $, $\alpha = 0$ and $g = -\Delta \tau$.  
%   If the interaction is a perfect square type, then  $\bm{V}  = \bm{V}^{(k,s)} $, $\alpha = \alpha_{k,s}$ and $g = \sqrt{\Delta \tau  U_k}$.  

%The variable $\texttt{Op\_V\%type}  $  in the operator structure  is required to specify  the following. If the operator  correspond to an interaction part of the Hamiltonian  then for 
%$\texttt{Op\_V\%type} =1 $   the operator referes to an Ising  operator $ \bm{I}^{k,s}$ and for  $\texttt{Op\_V\%type} =2 $  to $\bm{V}^{ks} $
%\begin{itemize}
%\item the projector ${\bm P}$, encoded as the vector $\vec{P}$,
%\item the matrix ${\bm O}$ of dimension $N \times N$  
%\item the effective dimension $N$,
%\item and a couple of auxiliary matrices and scalars.
%\end{itemize}
%The precise definition of the Operator type reads:




\subsection{The Lattice tpye}\label{sec:latt}

We have a lattice module  which  generate   one and two dimensional dimensional Bravais lattices.   Note that the  orbital structure of each unit cell, has to be specified by the user  in the  Hamiltonian module. 
 The user has to specify unit vectors $\vec{a}_1$ and $\vec{a}_2$ as well as   the size of the  lattice. The size is  characterized by  two vectors $\vec{L}_1$ and $\vec{L}_2$   and  the lattice is placed on a torus: 
\begin{equation}
	\hat{c}_{\vec{i} + \vec{L}_1 }  = \hat{c}_{\vec{i} + \vec{L}_2 }  = \hat{c}_{\vec{i}}
\end{equation}
The call 
\texttt{ Call Make\_Lattice( L1, L2, a1,  a2, Latt )} will generate the lattice   \texttt{Latt} of type \texttt{Lattice}.   Note that  the structure of the unit cell has to be provided by the user.    The reciprocal lattice vectors are defined by: 
\begin{equation}
\label{Latt.G.eq}
	\vec{a}_i  \cdot \vec{g}_i = 2 \pi \delta_{i,j}, 
\end{equation}
and the Brillouin zone corresponds to the Wigner Seitz cell of the lattice. 
With $\vec{k} = \sum_{i} \alpha_i  \vec{g}_i $, the  k-space quantization follows from: 
\begin{equation}
\begin{bmatrix}
	\vec{L}_1 \cdot \vec{g}_1  &  \vec{L}_1 \cdot \vec{g}_2  \\
	\vec{L}_2  \cdot \vec{g_1} & \vec{L}_2 \cdot  \vec{g}_2  
\end{bmatrix}
\begin{bmatrix}
   \alpha_1 \\
   \alpha_2
\end{bmatrix}
=
2 \pi 
\begin{bmatrix}
   n \\
   m
\end{bmatrix}
\end{equation}
such that 
\begin{eqnarray}
\label{k.quant.eq}
     \vec{k} =  n \vec{b}_1  + m \vec{b}_2 \text{  with  }   & &   \vec{b}_1 = \frac{2 \pi}{ (\vec{L}_1 \cdot \vec{g}_1)  (\vec{L}_2 \cdot  \vec{g}_2 )  - (\vec{L}_1 \cdot \vec{g}_2) (\vec{L}_2  \cdot \vec{g_1} ) }   \left[  (\vec{L}_2 \cdot  \vec{g}_2) \vec{g}_1 -   (\vec{L}_2  \cdot \vec{g_1} ) \vec{g}_2 \right] \text{   and  } \nonumber \\ 
        & & \vec{b}_2 = \frac{2 \pi}{ (\vec{L}_1 \cdot \vec{g}_1)  (\vec{L}_2 \cdot  \vec{g}_2 )  - (\vec{L}_1 \cdot \vec{g}_2) (\vec{L}_2  \cdot \vec{g_1} ) }   
           \left[  (\vec{L}_1 \cdot  \vec{g}_1) \vec{g}_2 -   (\vec{L}_1  \cdot \vec{g_2} ) \vec{g}_1 \right] 
\end{eqnarray}
\mycomment{Check that the above algebra is correct!  Just checked. Seems to be OK.}
\mycomment{

}
% 
%
\begin{table}[h]
   \begin{tabular}{l l l}
    Name of variable  & Type & Description \\\hline
     \hl{\texttt{Latt\%a1\_p}, \texttt{Latt\%a2\_p}}   & Real     & Unit vectors $\vec{a}_1$,  $\vec{a}_2$ \\ 
     \hl{\texttt{Latt\%L1\_p}, \texttt{Latt\%L2\_p}}   & Real     & Vectors $\vec{L}_1$, $\vec{L}_2$ that define the topology of the  lattice. \\
     									  &              &  Tilted lattices are  thereby possible to implement.  \\
    \texttt{Latt\%N}                                                 &   Integer &  Number of lattice points, $N_{unit\,cell}$   \\
    \texttt{Latt\%list}                                               & Integer &  maps each lattice point $i=1,\cdots, N_{unit\,cell}$ to a real space vector\\ 
                                                                             &   &  denoting the position of the unit cell: \\
                                                                             &   & $\vec{R}_i$ = \texttt{list(i,1)} $\vec{a}_1$ +  \texttt{list(i,2)} $\vec{a}_2$  $  \equiv i_1  \vec{a}_1 + i_2  \vec{a}_2 $ \\
    \texttt{Latt\%invlist}                                        &  Integer &   \texttt{Invlist}$(i_1,i_2) = i $ \\
    \texttt{Latt\%nnlist}                                         &  Integer &   $j = \texttt{nnlist} (i, n_1, n_2) $,  $n_1, n_2 \in [-1,1] $ \\
                                                                           &              &    $\vec{R}_j = \vec{R}_i + n_1 \vec{a}_1  + n_2 \vec{a}_2 $ \\
   \texttt{Latt\%imj}                                             &   Integer  &  $ \vec{R}_{imj(i,j)}  =  \vec{R}_i -  \vec{R}_j$.        $imj, i, j \in  1,\cdots, N_{unit\,cell}$\\
    \texttt{Latt\%BZ1\_p}, \texttt{Latt\%BZ2\_p}  &   Real     & Reciprocal space vectors $\vec{g}_i$   (See Eq.~\ref{Latt.G.eq})\\
    \texttt{Latt\%b1\_p}, \texttt{Latt\%b1\_p}       &   Real     &  k-quantization (See Eq.~\ref{k.quant.eq}) \\
    \texttt{Latt\%listk}                                           &  Integer &  maps each reciprocal lattice point $k=1,\cdots, N_{unit\,cell}$\\
                                                                          &    & to a reciprocal space vector\\
                                                                          &     & $\vec{k}_k= \texttt{listk(k,1)} \vec{b}_1 +  \texttt{listk(k,2)} \vec{b}_2  \equiv k_1  \vec{b}_1 +   k_2  \vec{b}_2 $\\
    \texttt{Latt\%invlistk}                                     &    Integer    &   \texttt{Invlistk}$(k_1,k_2) = k $ \\
   \texttt{Latt\%b1\_perp\_p},  \\ 
   \texttt{Latt\%b2\_perp\_p}                             &    Real         &  Orthonormal vectors to $\vec{b}_i$.  For internal use. 
   \end{tabular}
   \caption{Components of the \texttt{Lattice} type for two-dimensional lattices using as example the default lattice name \texttt{Latt}.
   The highlighted variables have to be specified by the user.  Other components of the Lattice will be generated  when calling: \texttt{ Call Make\_Lattice( L1, L2, a1,  a2, Latt )}.  
    \label{table:lattice}}
\end{table}
%
The \texttt{Lattice}  module equally handles  the Fourier transformation.  For example  the  subroutine  \texttt{Fourier\_R\_to\_K}   carries out the  transformation: 
\begin{equation}
	S(\vec{k}, :,:,:) =  \frac{1}{N_{unit \,cell}}  \sum_{\vec{i},\vec{j} \;\text{\mycomment{change to $\vec{i}-\vec{j}$}}}   e^{-i \vec{k} \cdot \left( \vec{i}-\vec{j} \right)} S(\vec{i}  - \vec{j}, :,:,:)
\end{equation}

and  \texttt{Fourier\_K\_to\_R}  the  inverse Fourier transform 
 \begin{equation}
	S(\vec{r}, :,:,:) =  \frac{1}{N_{unit \,cell}}  \sum_{\vec{k} \in BZ }   e^{ i \vec{k} \cdot \vec{r}} S(\vec{k}, :,:,:).
\end{equation}
In the above,   the unspecified dimensions of   structure factor can refer  to imaginary time,  and orbital indices. 


\subsection{The Observable type}\label{sec:obs}

Our definition  of the model includes observables. We have defined two observable types: \texttt{Obser\_vec}  for a array of scalar observables
such as the energy and  \texttt{Obser\_Latt}   for correlation functions that have the lattice symmetry. In the latter case, translation symmetry can be used to provide improved estimators and to reduce the size of the I/O.   In general, the user will define bins, each bins having a given amount of sweeps. Within a sweep we run sequentially trough the HS and Ising fields from   time slice 1 to $L_{\text{Trotter}}$ and back.  The results of each bin is written  in a file  and analyzed at the end of the run.     

\subsubsection{Scalar observables}
This data type  is described in Table  \ref{table:Obser_vec} and  is useful to compute an array of  scalar observables.   Consider  a variable \texttt{Obs} of type  \texttt{Obser\_vec}.  At the beginning of each bin,  a call to  \texttt{Obser\_Vec\_Init} in the module \texttt{observables\_mod.f90}  will  set   \texttt{Obs\%N=0},   \texttt{Obs\%Phase =0}  and  \texttt{Obs\%Obs\_vec(:)=0}.  Each time the main  program calls the routine \texttt{Obser}  in the  \texttt{Hamiltonian} module,  the counter \texttt{Obs\%N}   is incremented by unity,   the sign  (see Eq.~\ref{Sign.eq}) is cumulated in the  variable \texttt{Obs\%phase},  and the desired  the observables (multiplied by the sign and   $\frac{e^{-S(C)}} {\Re \left[e^{-S(C)} \right]}$, see Sec.~\ref{Observables.General})  are cumulated in the vector \texttt{Obs\%Obs\_vec}.  
\begin{table}[h]
   \begin{tabular}{l ll }
    Name of variable  &  Type      &  Description \\\hline
    \texttt{Obs\%N}                       &  Integer        &   Number of measurements  \\
    \texttt{Obs\%Phase}               &  Complex     &    Cumulated sign (See Eq.~\ref{Sign.eq})  \\
    \texttt{Obs\%Obs\_vec(:)}        & Complex      &    Cumulated vector of observables. 
           $ \langle \langle \hat{O}(:) \rangle \rangle_{C}\frac{e^{-S(C)}} {\Re \left[e^{-S(C)} \right]} \text{ sign }(C) $ \\
     \texttt{Obs\%File\_Vec}           &  Character    &    Filename  in which the bins are written  
   \end{tabular}
   \caption{Components of the \texttt{Obser\_vec}  type.  The table lists the data included in a variable  \texttt{Obs}  of type \texttt{Obser\_vec}.  
\mycomment{I think you do not  use the vector character of the member \texttt{Obser\_vec\%Obs}. For the observables like energy you have created an array of type variables \texttt{Obser\_vec} 
but within the type variable, the vector is of size $1$. And each scalar observable gets its own type variable \texttt{Obs}. This is a detail but it puzzled me first. Do we have an example where the vector would be larger than $1$?}
      \label{table:Obser_vec}}
\end{table}
At the end of the bin, a call to  \texttt{Print\_bin\_Vec}   in  module \texttt{observables\_mod.f90}  will  append the result of the bin in the file  \texttt{File\_Vec}\_scal.  Note that this subroutine will automatically append the suffix  \_scal 
to the the filename \texttt{File\_Vec}.    This suffix  is important to allow automatic analysis of the data at the end of the run. 

\subsubsection{ Equal time and time displaced correlation functions}

\begin{table}[h]
   \begin{tabular}{l ll }
    Name of variable  &  Type      &  Description \\\hline
    \texttt{Obs\%N}                       &  Integer        &   Number of measurements  \\
    \texttt{Obs\%Phase}               &  Complex     &    Cumulated sign (See Eq.~\ref{Sign.eq})  \\
    \texttt{Obs\%Obs\_latt($\vec{i}-\vec{j},\tau,\alpha,\beta$)}        & Complex      &    Cumulated   correlation function  $ \langle \langle \hat{O}_{\vec{i},\alpha} (\tau) \hat{O}_{\vec{j},\beta} \rangle \rangle_{C} \; \frac{e^{-S(C)}} {\Re \left[e^{-S(C)} \right]}  \text{sign}(C) $ \\
     \texttt{Obs\%Obs\_latt0($\alpha$)}        & Complex      &    Cumulated    $ \langle \langle \hat{O}_{\vec{i},\alpha} \rangle \rangle_{C}\frac{e^{-S(C)}} {\Re \left[e^{-S(C)} \right]}  \text{ sign }(C) $ \\
     \texttt{Obs\%File\_Latt}           &  Character    &    Filename  in which the bins are written  
   \end{tabular}
   \caption{Components of the \texttt{Obser\_latt}  type.  The table lists the data included in a variable  \texttt{Obs}  of type \texttt{Obser\_latt}  
      \label{table:Obser_vec}}
\end{table}

This data type is useful so as to deal with  imaginary time displaced as well as equal time correlation functions of the form: 
\begin{equation}
	S_{\alpha,\beta}(\vec{k},\tau) =   \frac{1}{N_{unit\, cell }} \sum_{\vec{i},\vec{j}}  e^{- \vec{k} \cdot \left( \vec{i}-\vec{j}\right) } \left( \langle \hat{O}_{\vec{i},\alpha} (\tau) \hat{O}_{\vec{j},\beta} \rangle  - 
	  \langle \hat{O}_{\vec{i},\alpha} \rangle \langle   \hat{O}_{\vec{i},\beta}  \rangle \right).
\end{equation}
Here,  translation symmetry of the Bravais lattice is explicitly taken into account. Note that this symmetry is broken  for a given  configuration $C$ but is restored by the Monte Carlo sampling. 
\mycomment{MB tries to fit this in here:}
The correlation function splits in a correlated part $S_{\alpha,\beta}^{\mathrm{(corr)}}(\vec{k},\tau)$ and an background part $S_{\alpha,\beta}^{\mathrm{(back)}}(\vec{k},\tau)$:
\begin{eqnarray}
  S_{\alpha,\beta}^{\mathrm{(corr)}}(\vec{k},\tau)
  &=&
   \frac{1}{N_{unit\, cell }} \sum_{\vec{i},\vec{j}}  e^{- i\vec{k} \cdot \left( \vec{i}-\vec{j}\right) }  \langle \hat{O}_{\vec{i},\alpha} (\tau) \hat{O}_{\vec{j},\beta} \rangle\;,\\
   S_{\alpha,\beta}^{\mathrm{(back)}}(\vec{k},\tau)
  &=&
   \frac{1}{N_{unit\, cell }} \sum_{\vec{i},\vec{j}}  e^{- i\vec{k} \cdot \left( \vec{i}-\vec{j}\right) }  \langle \hat{O}_{\vec{i},\alpha} (\tau)\rangle \langle \hat{O}_{\vec{j},\beta} \rangle\\\nonumber
  &=& 
   \frac{1}{N_{unit\, cell }} \sum_{\vec{i}}  e^{- i\vec{k} \cdot \vec{i} }  \langle \hat{O}_{\vec{i},\alpha}\rangle
   \sum_{\vec{j}}    e^{i\vec{k} \cdot  \vec{j} }    \langle \hat{O}_{\vec{j},\beta} \rangle\;,
\end{eqnarray}
where we used translation invariance in imaginary-time to drop the $\tau$ dependency in the last line. 

\mycomment{In the code, the phase factors $e^{\vec{k}\cdot \vec{i}}$ are not 
included in the output for the background. 
The output is simply $\frac{1}{N_{unit\,cell}}\sum\limits_{\vec{i}}\langle \hat{O}_{\vec{i},\alpha}\rangle$.
Is the motivation to first say $\langle \hat{O}_{\vec{i},\alpha} (\tau)\rangle=\langle \hat{O}_{\alpha}\rangle$,
use $\frac{1}{N}\sum\limits_{\vec{i}}e^{i \vec{k}\cdot\vec{i}} = \delta(\vec{i})$ 
and the use the improved estimator $\langle \hat{O}_{\alpha} \rangle=\frac{1}{N}\sum\limits_{\vec{i}}\langle \hat{O}_{\vec{i},\alpha}\rangle$?}

Consider a variable  \texttt{Obs} of type  \texttt{Obser\_latt}. At the beginning of each bin a call to  \texttt{Obser\_Latt\_Init} in the module \texttt{observables\_mod.f90}  will  initialize  the elements of \texttt{Obs} to zero.    Each time the main program calls the   \texttt{Obser} or  \texttt{ObserT} routines one  cumulates $ \langle \langle \hat{O}_{\vec{i},\alpha} (\tau) \hat{O}_{\vec{j},\beta} \rangle \rangle_{C} \; \frac{e^{-S(C)}} {\Re \left[e^{-S(C)} \right]}  \text{sign}(C) $    in  \texttt{Obs\%Obs\_latt($\vec{i}-\vec{j},\tau,\alpha,\beta$)}   
and $ \langle \langle \hat{O}_{\vec{i},\alpha}= \rangle \rangle_{C}\frac{e^{-S(C)}} {\Re \left[e^{-S(C)} \right]}  \text{ sign }(C) $  in \texttt{Obs\%Obs\_latt0($\alpha$)}.   At the end of each bin, a call to \texttt{Print\_bin\_Latt} in the module  \texttt{observables\_mod.f90}   will append the result of the bin in the specified  file \texttt{Obs\%File\_Latt}.   Note that the routine  \texttt{Print\_bin\_Latt}  carries out the Fourier transformation and prints the results in k-space. We have adopted the following name convention.  For    equal time observables , that is  the second  dimension  of the array  \texttt{Obs\%Obs\_latt($\vec{i}-\vec{j},\tau,\alpha,\beta$)}    is equal to unity,  the routine \texttt{Print\_bin\_Latt}  attaches the suffix \_eq to \texttt{Obs\%File\_Latt}.  For  time displaced correlation functions we use the suffix \_tau. 

% We have three types of observables. 
% \begin{itemize}
% \item Scalar observables such as the energy
% \item Equal time correlation functions.  Let $\hat{O}_{\vec{i},\alpha} $ be a local observable,  with $\vec{i}$ labelling the unit cell and $\alpha$ labelling the orbital or bone emanating 
% from the unit cell.   The program will compute: 
% \begin{equation}
% 	S_{\alpha,\beta}(\vec{k}) = \frac{1}{N_{unit \;  cells}} \sum_{\vec{i},\vec{j}} e^{i \vec{k}\cdot (\vec{i} -  \vec{j} ) } \left( \langle \hat{O}_{\vec{i},\alpha}  \hat{O}_{\vec{j},\alpha} \rangle  - 
% 	  \langle \hat{O}_{\vec{i},\beta} \rangle \langle   \hat{O}_{\vec{i},\beta}  \rangle \right) 
% \end{equation}
% \item  Time displaced correlation functions. This has a very similar structure than above but now with an additional time index.
% \begin{equation}
% 	S_{\alpha,\beta}(\vec{k},\tau) = \frac{1}{N_{unit \;  cells}} \sum_{\vec{i},\vec{j}} e^{i \vec{k}\cdot (\vec{i} -  \vec{j} ) } \left( \langle \hat{O}_{\vec{i},\alpha} (\tau) \hat{O}_{\vec{j},\alpha} \rangle  - 
% 	  \langle \hat{O}_{\vec{i},\beta} \rangle \langle   \hat{O}_{\vec{i},\beta}  \rangle \right) 
% \end{equation}
% \end{itemize}

%\mycomment{mention bins, sweeps}
%\mycomment{We have to add some  more details.}
%\subsubsection{Scalar observables}
%Several scalar observables are measured and accumulated in the array \texttt{Obs\_scal} during the simulation (see table \ref{table:obs}).
%%
%\begin{table}[h]
%   \begin{tabular}{l l l}
%    Name of variable in the code & Definition & Description \\\hline
%\texttt{Obs\_scal(1)} & 
%$\rho=\sum\limits_{k=1}^{M_T}
%\sum\limits_{s=1}^{N_{\mathrm{fl}}}
%\sum\limits_{\sigma=1}^{N_{\mathrm{col}}}
%\sum\limits_{x}^{N_{\mathrm{dim}}}
%\langle \hat{c}^{\dagger}_{x \sigma   s} \hat{c}^{\phantom\dagger}_{x \sigma s}   \rangle$ &
%electronic density\\
%\texttt{Obs\_scal(2)} & 
%$E_{\mathrm{kin}}=\sum\limits_{k=1}^{M_T}
%\sum\limits_{s=1}^{N_{\mathrm{fl}}}
%\sum\limits_{\sigma=1}^{N_{\mathrm{col}}}
%\sum\limits_{x,y}^{N_{\mathrm{dim}}}
%\langle \hat{c}^{\dagger}_{x \sigma   s} T_{xy}^{(k s)} \hat{c}^{\phantom\dagger}_{y \sigma s}   \rangle$ &
%kinetic energy\\
%\texttt{Obs\_scal(3)} & 
%$E_{\mathrm{pot}}=\sum\limits_{x,y}^{N_{\mathrm{dim}}}
%\prod\limits_{s=1}^{N_{\mathrm{fl}}}
%\langle \hat{c}^{\dagger}_{x \sigma   s} \hat{c}^{\phantom\dagger}_{x \sigma s}  
%\rangle$ &
%potential energy \mycomment{need input here} \\
%\texttt{Obs\_scal(4)} & 
%$E_{\mathrm{tot}}=E_{\mathrm{kin}}+E_{\mathrm{pot}}$ &
%total energy\\
%\texttt{Obs\_scal(5)} & 
%$\langle \mathrm{phase} \rangle$ &
%phase of MC update probability
%   \end{tabular}
%   \caption{Scalar observables that are stored in the array \texttt{Obs\_scal}.
%       \label{table:obs}}
%\end{table}
%%
%
%
%
%\subsubsection{Equal-time correlation functions}
%
%Let $\hat{O}_{\vec{i},\alpha} $ be a local observable,  with $\vec{i}$ labelling the unit cell and $\alpha$ labelling the orbital or bone emanating 
%from the unit cell.   The program will compute: 
%\begin{equation}
%	S_{\alpha,\beta}(\vec{k}) = \frac{1}{N_{unit \;  cells}} \sum_{\vec{i},\vec{j}} e^{i \vec{k}\cdot (\vec{i} -  \vec{j} ) } \left( \langle \hat{O}_{\vec{i},\alpha}  \hat{O}_{\vec{j},\alpha} \rangle  - 
%	  \langle \hat{O}_{\vec{i},\beta} \rangle \langle   \hat{O}_{\vec{i},\beta}  \rangle \right) 
%\end{equation}
%\mycomment{Should it not be
%}
%
%\subsubsection{Time-displaced correlation functions}
%
%This has a very similar structure than above but now with an additional time index.
%\begin{equation}
%	S_{\alpha,\beta}(\vec{k},\tau) = \frac{1}{N_{unit \;  cells}} \sum_{\vec{i},\vec{j}} e^{i \vec{k}\cdot (\vec{i} -  \vec{j} ) } \left( \langle \hat{O}_{\vec{i},\alpha} (\tau) \hat{O}_{\vec{j},\alpha} \rangle  - 
%	  \langle \hat{O}_{\vec{i},\beta} \rangle \langle   \hat{O}_{\vec{i},\beta}  \rangle \right) 
%\end{equation}
%
%To set the  interaction part, we therefore have to specify the following:
%\begin{itemize}
%\item the matrix elements $\left[O_{V}^{(k)}\right]_{ij}$
%\item the set $[z_{1}^{(k)},\cdots  z_{N_{eff}^{(k)}}^{(k)}]$ 
%\item the interaction strenghts $U_{k}$
%\item the numbers  $\alpha_{k}$.
%\end{itemize}
%\mycomment{Be more specific here what really has to specified in the actual code.}%
%The same logic also applies to the implementation of the hopping interaction \mycomment{be more specific}.






%\begin{itemize}
%\item in the coupling $g$ in the \texttt{Operator} structure (see Sec.~\ref{}).
%\item as normalization constant in the definition of observables (see Sec.~\ref{})
%\item as exponent in the calculation of the phase factor and the Monte Carlo update ratio.
%\end{itemize}
%\subsection{Structure of the hopping matrix  ${\bf T}$ and the interaction matrices ${\bf V}^{(k)}$}


%\subsection{The Hubbard-Stratonovich decomposition} 
%Consider a single-particle (in other words bilinear) operator $O_{i}$.
%One obtains an approximation to the evolution operator by the following series expansion \cite{AssaadBook08}
%\begin{equation}
%\label{eqn_2_HS}
%e^{-\Delta\tau O^{2}_{i} } = \sum\limits_{s=\pm1,\pm2} \gamma(s) e^{i \sqrt{\Delta\tau}\eta(s)O_{i}} + \mathcal{O}(\Delta\tau^{4})\;,
%\end{equation}
%with 
%
%\begin{eqnarray}
%\gamma(\pm 1) = (1+\sqrt{6}/3)/4\;,\;\gamma(\pm 2) = (1-\sqrt{6}/3)/4\;,\nonumber\\
%\eta(\pm 1) =\pm \sqrt{2(3-\sqrt{6})}\;,\;\eta(\pm 2) =\pm \sqrt{2(3+\sqrt{6})}\;.
%\end{eqnarray}
%
%Eq.~(\ref{eqn_2_HS}) can be easily proven by expanding its right hand side  to eighth order in $O_{i}$. 
%The transformation introduces therefore two Ising fields $s$ per lattice site $i$, taking the values $\pm 1$ and $\pm 2$.
%\mycomment{same label as the flavor index}

% Copyright (c) 2016 The ALF project.
% This is a part of the ALF project documentation.
% The ALF project documentation by the ALF contributors is licensed
% under a Creative Commons Attribution-ShareAlike 4.0 International License.
% For the licensing details of the documentation see license.CCBYSA.

% !TEX root = doc.tex
%------------------------------------------------------------
\subsection{File structure}\label{sec:files}
%------------------------------------------------------------
%
\begin{table}[h]
	\begin{tabular}{@{} l l @{}}\toprule
   	Directory & Description \\\midrule
   	\path{Prog/} & Main program and subroutines.  \\
   	\path{Libraries/} & Collection of mathematical routines. \\  
  	\path{Analysis/} & Routines for error analysis. \\
  	\path{Scripts_and_Parameters_files/}   & Helper scripts and the \path{Start/} directory, which contains the files \\ 
  	                                      & required to start a run. \\
  	\path{Documentation/} & This documentation.\\
  	\path{testsuite/} & A suite for automatic testing various parts of the code.\\\bottomrule
  	\hline
	\end{tabular}
   	\caption{Overview of the directories included in the ALF package.\label{table:files}}
\end{table}
%

The code package, summarized in Table~\ref{table:files}, consists of the program directories \path{Prog/}, \path{Libraries/}, and \path{Analysis/}, as well as the directory \path{Scripts_and_Parameters_files/}, which contains supporting scripts and, in its subdirectory \path{Start}, the input files necessary for a run, described in the Sec.~\ref{sec:input}. Additionally, a suite of tests for individual parts of the code (subroutines, functions, operations, etc.) is available at the directory \path{testsuite} -- the tests can be run by executing the following sequence of commands (the script \path{configureHPC.sh} sets environment variables and is described in Sec.~\ref{sec:running}.):
\begin{lstlisting}[style=bash,morekeywords={make,cmake,ctest}]

source configureHPC.sh Devel serial
gfortran -v
make lib
make ana
make Examples
cd testsuite
cmake -E make_directory tests
cd tests
cmake -G "Unix Makefiles" -DCMAKE_Fortran_FLAGS_RELEASE=${F90OPTFLAGS} \
      -DCMAKE_BUILD_TYPE=RELEASE ..
cmake --build . --target all --config Release
ctest -VV -O log.txt
\end{lstlisting}
which will output test results and total success rate.
%The example simulations corresponding to the walkthroughs of Sec.~\ref{sec:walk1} - \ref{sec:walk2} are included in \path{Examples/}.

%------------------------------------------------------------
\subsubsection{Input files}\label{sec:input}
%------------------------------------------------------------
%
\begin{table}[h]
   \begin{tabular}{@{} l l @{}}\toprule
   File & Description \\\midrule
  \path{parameters} &  Sets the parameters for lattice, model, QMC process, and the error analysis.\\
  \path{seeds} & List of integer numbers to initialize the random number generator and \\
   & to start a simulation from scratch.
   %\\
 %  \path{confin_<thread number>} & Input files for the HS and Ising configuration, used to continue a simulation.
  \\\bottomrule
   \end{tabular}
   \caption{Overview of the input files required for a simulation, which can be found in the subdirectory \texttt{Scripts\_and\_Parameters\_files/Start/}. \label{table:input}}
\end{table}
%
The input files are listed in Table~\ref{table:input}. 
The parameter file \path{Start/parameters} has the following form --
using as an example  the $SU(2)$-symmetric Hubbard model on a square lattice (see Sec.~\ref{sec:walk1} for a detailed walkthrough):
%
\begin{lstlisting}[style=fortran]

!===============================================================================
!  Variables for the Hubb program
!-------------------------------------------------------------------------------
&VAR_lattice
L1 = 4                    ! Length in direction a_1
L2 = 4                    ! Length in direction a_2
Lattice_type = "Square"	  ! a_1 = (1,0), a_2=(0,1), Norb=1, N_coord=2
!Lattice_type ="Honeycomb"! a_1 = (1,0), a_2 =(1/2,sqrt(3)/2), Norb=2, N_coord=3
Model = "Hubbard_SU2"     ! Sets Nf=1, N_sun=2. HS field couples to the density
!Model = "Hubbard_Mz"     ! Sets Nf=2, N_sun=1. HS field couples to the 
                          ! z-component of magnetization.  
!Model="Hubbard_SU2_Ising"! Sets Nf_1, N_sun=2 and runs only for the square lattice
                          ! Hubbard model coupled to transverse Ising field
/

&VAR_Hubbard              ! Variables for the Hubbard model
ham_T   = 1.d0            ! Hopping parameter
ham_chem= 0.d0            ! chemical potential
ham_U   = 4.d0            ! Hubbard interaction
Beta    = 10.d0           ! inverse temperature
dtau    = 0.1d0           ! Thereby Ltrot=Beta/dtau
/

&VAR_Ising                ! Model parameters for the Ising code
Ham_xi = 1.d0             ! Only needed if Model="Hubbard_SU2_Ising"
Ham_J  = 0.2d0
Ham_h  = 2.d0
/

&VAR_QMC                  ! Variables for the QMC run
Nwrap   = 10              ! Stabilization. Green functions will be computed from 
                          ! scratch after each time interval Nwrap*Dtau
NSweep  = 10              ! Number of sweeps
NBin    = 10              ! Number of bins
Ltau    = 1               ! 1 for calculation of time displaced Green functions;
                          ! 0 otherwise
LOBS_ST = 1               ! Start measurements at time slice LOBS_ST
LOBS_EN = 100             ! End   measurements at time slice LOBS_EN
CPU_MAX = 0.1             ! Code will stop after CPU_MAX hours. 
                          ! If not specified, code will stop after Nbin bins.
/

&VAR_errors               ! Variables for analysis programs
n_skip  = 1               ! Number of bins that will be skipped. 
N_rebin = 1               ! Rebinning  
N_Cov   = 0               ! If set to 1 covariance will be computed
                          ! for non-equal-time correlation functions.                   
/            
\end{lstlisting}
%

The program allows for a number of different  updating schemes.  If no other variables are specified in the \texttt{VAR\_QMC} name space, then the program will run in its default mode, namely the sequential single spin-flip mode.   The additional, optional variables in   \texttt{VAR\_QMC}   include the following: 
\begin{lstlisting}[style=fortran]

&VAR_QMC                 ! Variables for the QMC run 
Propose_S0      = .true. ! Proposes single spin flip moves with probability exp(-S0) 
Global_moves    = .true. ! Allows for global moves in space and time 
N_Global        = 1      ! Number of global moves  per sweep 
Global_tau_moves= .true. ! Allows for global moves on a single time slice.  
N_Global_tau    = 10     ! Number of global moves that will be carried out on a 
                         ! single time slice
Nt_sequential_start = 1  ! One can combine sequential and global moves on 
                         ! a time slice.  
Nt_sequential_end =      ! The program will carry our sequential local moves in the
                         ! range [Nt_sequential_start, Nt_sequential_end] and then
                         ! N_Global_tau global moves
/   
\end{lstlisting}
Note that if \texttt{Nt\_sequential\_start}  and \texttt{Nt\_sequential\_end}  are not specified and that the variable \texttt{Global\_tau\_moves}  is set to true, then  the program will  carry out only global moves, by setting  \\  \texttt{Nt\_sequential\_start=1}  and \texttt{Nt\_sequential\_end=0}. 

If the program is compiled with the parallel tempering flag, then the additional name space \texttt{VAR\_TEMP} has to be included in the parameter file.
\begin{lstlisting}[style=fortran,escapechar=\%]

&VAR_TEMP                      ! Variables for parallel tempering
N_exchange_steps      = 6      ! Number of exchange moves %[see Eq.~\eqref{eq:exchangestep}]%
N_Tempering_frequency = 10     ! The frequency in units of sweeps at which the
                               ! exchange moves will be carried 
mpi_per_parameter_set = 2      ! Number of mpi-processes per parameter set
Tempering_calc_det    = .true. ! Specifies whether the fermion weight has to be taken
                               ! into account while tempering. The default is .true.,
                               ! and it can be set to .false. if the parameters that
                               ! get varied only enter the Ising action S_0
/
\end{lstlisting}

Additionally, in order for the maximum entropy code, described in Sec.~\ref{sec:maxent}, to be used, the namelist \texttt{VAR\_Max\_Stoch} should also be defined:
\begin{lstlisting}[style=fortran]

&VAR_Max_Stoch               ! Variables for Stochastic Maximum entropy
Ngamma     = 400             ! # of Dirac delta-functions for parametrization
Om_st      = 0               ! Frequency range lower bound
Om_en      = 8               ! Frequency range upper bound
NDis       = 2000            ! # of boxes for histogram
Nbins      = 250             ! # of bins for Monte Carlo
Nsweeps    = 70              ! # of sweeps per bin
NWarm      = 20              ! The Nwarm first bins will be ommitted
N_alpha    = 14              ! # of tempertures
alpha_st   = 1.d0            ! smallest inverse temperature
R          = 1.2d0           ! increment for inverse temperature (see above) 
Channel    = "P"             ! T0       : Zero temperature
                             ! P        : Finite temperarure particle 
                             ! PH       : Finite temperarure particle-hole
                             ! PP       : Finite temperarure particle-particle 
Checkpoint = .false.         !.true.    : dump files will be produced so as to be able
                             !            to restart the simulation
                             !.false.   : dump files will not be produced 
Tolerance  = 0.1d0           ! Data points for which the relative error exceeds the
                             ! tolerance threshold will be omitted.
/
\end{lstlisting}


%------------------------------------------------------------
\subsubsection{Output: Observables} \label{sec:output_obs}
%------------------------------------------------------------
%
\begin{table}[h]
   \begin{tabular}{@{} l l @{}}\toprule
   File & Description \\\midrule
   \path{info} & After completion of the simulation, this file documents the parameters of\\
   & the model, as well as the QMC run and simulation metrics (precision,\\
   & acceptance rate, wallclock time).\\
   \path{X_scal} & Results of equal-time measurements of scalar observables. \\
   & The placeholder \path{X} stands for the observables \path{Kin}, \path{Pot}, \path{Part}, and \path{Ener}. \\
   \path{Y_eq, Y_tau} & Results of equal-time and time-displaced measurements of correlation\\
   & functions. The placeholder \path{Y} stands for \path{Green}, \path{SpinZ}, \path{SpinXY}, and \path{Den}. \\   
   \path{confout_<thread number>} & Output files (one per MPI instance) for the HS and Ising configuration. \\\bottomrule
   \end{tabular}
   \caption{Overview of the standard output files. See Sec.~\ref{sec:obs} for the definitions of observables and correlation functions. \label{table:output}}
\end{table}
%
The standard output files are listed in Table~\ref{table:output}. 
The output of the measured data is organized in bins. One bin corresponds to the arithmetic average 
over a fixed number of individual measurements which depends 
on the chosen measurement interval \path{[LOBS_ST,LOBS_EN]} on the imaginary-time axis and on the number \path{NSweep} of Monte Carlo sweeps. If the user runs an MPI parallelized version of the code, the average also extends over the number of MPI threads. The formatting of a single bin's output depends on the observable type, \path{Obs_vec} or \path{Obs_Latt}:
\begin{itemize}
\item Observables of type \path{Obs_vec}:
For each additional bin, a single new line is added to the output file.
In case of an observable with \path{N_size} components, the formatting is 
\begin{verbatim}
N_size + 1    <measured value, 1> ... <measured value, N_size>    <measured sign>
\end{verbatim}
The counter variable \path{N_size+1} refers to the number of measurements per line, including the phase measurement. 
This format is required by the error analysis routine (see Sec.~\ref{sec:analysis}). 
Scalar observables like kinetic energy, potential energy, total energy and particle number are treated as a vector 
of size \path{N_size=1}.

\item Observables of type \path{Obs_Latt}:
For each additional bin, a new data block is added to the output file. 
The block consists of the expectation values [Eq.~(\ref{eqn:o})] contributing to the background part [Eq.~(\ref{eqn:s_back})] of the correlation function,
and the correlated part [Eq.~(\ref{eqn:s_corr})] of the correlation function.
For imaginary-time displaced correlation functions, the formatting of the block is given by:
\begin{alltt}
<measured sign>  <N_orbital>  <N_unit_cell>  <N_time_slices>  <dtau>
do alpha = 1, N_orbital
    \(\langle\hat{O}\sb{\alpha}\rangle \)
enddo
do i = 1, N_unit_cell
   <reciprocal lattice vector k(i)>
   do tau = 1, N_time_slices
      do alpha = 1, N_orbital
         do beta = 1, N_orbital
            \(\langle{S}\sb{\alpha,\beta}\sp{(\mathrm{corr})}(k(i),\tau)\rangle\)
         enddo
      enddo
   enddo
enddo
\end{alltt}
The same block structure is used for equal-time correlation functions, except for the entries  \path{<N_time_slices>} and \path{<dtau>}, which are then omitted.
Using this structure for the bins as input, the full correlation function $S_{\alpha,\beta}(\vec{k},\tau)$ [Eq.~(\ref{eqn:s})] is then calculated by calling the error analysis routine (see Sec.~\ref{sec:analysis}).
\end{itemize}

%
%------------------------------------------------------------
\subsubsection{Output: Precision} \label{sec:output_prec}
%------------------------------------------------------------
%


\red{[THIS SECTION MAYBE BELONGS INTO THE "RUNNING"...]}

The finite-temperature, auxiliary-field QMC algorithm is known to be numerically unstable, as discussed in Sec.~\ref{sec:stable}.
The numerical instabilities arise from the imaginary-time propagation, which invariably leads to exponentially small and exponentially large scales.
As shown in Ref.~\cite{Assaad08_rev}, scales can be omitted in the ground state algorithm -- thus rendering it very stable --  but have to be taken into account in the  finite-temperature code.

Numerical stabilization of the code is a delicate procedure that has been pioneered in Ref.~\cite{White89}  for the finite-temperature algorithm and in Refs.~\cite{Sugiyama86,Sorella89} for the zero-temperature projective algorithm.
It is important to be aware of the fragility of the numerical stabilization and that there is no guarantee that it will work for a given model. It is therefore crucial to always check the file \texttt{info}, which, apart from runtime data, contains important information concerning the stability of the code, in particular \texttt{Precision Green}.
If the numerical stabilization fails, one possible measure is to reduce the value of the parameter \texttt{Nwrap} in the parameter file, which will however also impact performance -- see Sec.~\ref{sec:optimize} for further optimization tips.

For performing the stabilization of the involved matrix multiplications we rely on routines from LAPACK. Notice that results are very likely to change
%significantly
depending on the specific implementation of the library used\footnote{The linked library should implement at least the LAPACK-3.4.0 interface.}.
In order to deal with this possibility, we offer a simple baseline which can be used as a quick check as tho whether results depend on the library used for linear algebra routines. Namely, we have included QR-decomposition related routines of the LAPACK-3.6.1 reference implementation from \url{http://www.netlib.org/lapack/}, which you can use by 
%including the switch \texttt{-DQRREF} into the \texttt{STABCONFIGURATION} string in the 
running the script \path{configureHPC.sh}, (described in Sec.~\ref{sec:running}), with the flag \texttt{STAB1} and recompiling ALF\footnote{This flag may trigger compiling issues, in particular, the Intel ifort compiler version 10.1 fails for all optimization levels.}.

In order to provide further flexibility, we offer various stabilization schemes that can be selected through the appropriate flags when running \texttt{configureHPC.sh}: \red{[MAKE A TABLE INSTEAD?]} \texttt{STAB1}, for using the reference stabilization scheme;
\texttt{STAB2}, which sets a stabilization scheme based on the QR decomposition, but not using the LAPACK reference implementation and with additional normalizations;
\texttt{STAB3}, for the newest and fastest stabilization, which separates large and small scales -- it generally works well, but there are models for which it fails;
and \texttt{LOG}, for using log storage for internal scales.

Typical values for the numerical precision can be found in the examples of Sec.~\ref{sec:ex} (see Sec.~\ref{sec:prec_charge} and \ref{sec:prec_spin}).

%------------------------------------------------------------
\subsection{Scripts}\label{sec:scripts}
%------------------------------------------------------------
%

\red{[IMPROVE or eliminate?]}

\begin{table}[h]
   \begin{tabular}{@{} l l l @{}}\toprule
   Script & Description & Section\\\midrule
   \path{Start/out_to_in.sh} & Copies the output field configurations to the respective input files. & \ref{sec:running} \\
   \path{Start/analysis.sh} & Starts the error analysis. & \ref{sec:analysis}\\\bottomrule
   \end{tabular}
   \caption{Overview of the bash script files. 
      \label{table:scripts}}
\end{table}
%

% Copyright (c) 2016 The ALF project.
% This is a part of the ALF project documentation.
% The ALF project documentation by the ALF contributors is licensed
% under a Creative Commons Attribution-ShareAlike 4.0 International License.
% For the licensing details of the documentation see license.CCBYSA.

% !TEX root = doc.tex

%-------------------------------------------------------------------------------------
\subsection{ Analysis programs }\label{sec:analysis}
%-------------------------------------------------------------------------------------
%
\begin{table}[h]
  \begin{tabular}{@{} l l @{}}\toprule
   Program & Description \\\midrule
   \texttt{cov\_scal.F90}  &  In combination with the script \texttt{analysis.sh}, the bin files with suffix \texttt{\_scal} are read in, \\
                           & and  the corresponding files with suffix \texttt{\_scalJ} are produced. They  contain the  result \\
                           & of the Jackknife rebinning analysis  (see Sec.~\ref{sec:sampling}).  \\
   \texttt{cov\_eq.F90}    &  In combination with the script \texttt{analysis.sh}, the bin files with suffix \texttt{\_eq} are read in, \\
                           & and the corresponding files with suffix  \texttt{\_eqJR}  and  \texttt{\_eqJK}  are produced. They  correspond \\
                           & to correlation functions in real and Fourier space, respectively.  \\
   \texttt{cov\_tau.F90}   &  In combination with the script \texttt{analysis.sh}, the bin files  \texttt{X\_tau} are read in, \\
                           & and the directories  \texttt{X\_kx\_ky} are produced  for all \texttt{kx} and \texttt{ky} greater or equal to zero. \\
                           & Here \texttt{X}  is a place holder from \texttt{Green}, \texttt{SpinXY}, etc   as specified in \texttt{ Alloc\_obs(Ltau)} \\
                           & (See section \ref{Alloc_obs_sec}). Each directory contains  a  file    \texttt{g\_kx\_ky}  containing the  \\
                           & time displaced correlation function traced over the  orbitals.  It also contains the  \\
                           & covariance matrix if \texttt{N\_cov} is set to unity in the parameter file  (see Sec.~\ref{sec:input}). \\
                           & Equally, a directory  \texttt{X\_R0}  for the local  time displaced  correlation function is generated.  \\                         
   \texttt{cov\_tau\_ph.F90}            & At compilation time  the file \texttt{cov\_tau\_ph.F90} is generated, and  should be used to compute \\ 
                           & particle-hole  imaginary time correlation functions such as Spin and Charge.   Here we use  \\
                           &  the fact that these  correlation functions  are symmetric around $\tau = \beta/2$ so that we \\
                           &  can define an improved estimator by averaging over $\tau$ and $\beta - \tau$.  
                                  \\\bottomrule
   \end{tabular}
   \caption{ Overview of analysis programs that are called within the script \texttt{analysis.sh}. \label{table:analysis_programs}}
\end{table}
%
Here we briefly   discuss the analysis programs which read in bins and carry out the error analysis. (See Sec.~\ref{sec:sampling}  for a more detailed discussion.)
Error analysis   is based  on the central limit theorem,  which requires bins to be statistically independent, and also the existence of a well-defined variance  for the observable under consideration. 
The former will be the case if bins are  longer than the autocorrelation time.  The latter has to be checked by the user.  In the parameter file listed in Sec.~\ref{sec:input}, the user  can specify how many initial bins should be omitted (variable \texttt{n\_skip}). 
This  number should be comparable to the autocorrelation time.     
The  rebinning  variable \texttt{N\_rebin} will merge \texttt{N\_rebin}  bins into a single new bin. 
If the autocorrelation time  is smaller than the effective bin size, the error should become independent of the bin size and thereby of the variable \texttt{N\_rebin}.  
Our analysis is based on the Jackknife resampling\cite{efron1981}.
As listed in Table  \ref{table:analysis_programs}  we provide three analysis programs to account for the three observable types. The programs can be found in the directory \texttt{Analysis}  and   are executed by running the  bash shell script 
\texttt{analysis.sh}.
%
\begin{table}[h]
   \begin{tabular}{@{} l l @{}}\toprule
   File & Description \\\midrule
   \texttt{parameters}  &  Contains also variables for the error analysis:\\
   & \texttt{n\_skip}, \texttt{N\_rebin} and \texttt{N\_Cov} (see Sec.~\ref{sec:input}) \\
   \texttt{X\_scal}, \texttt{Y\_eq}, \texttt{Y\_tau} & Monte Carlo bins (see Table \ref{table:output}) \\\bottomrule
    \end{tabular}
   \caption{Standard input files for the error analysis. \label{table:analysis_input}}
\end{table}
%
\begin{table}[h]
   \begin{tabular}{@{} l l l @{}}\toprule
   File & Description \\\midrule
   \texttt{X\_scalJ} & Jackknife mean and error of \texttt{X}, where  \texttt{X} stands for \texttt{Kin, Pot, Part}, and \texttt{Ener}.\\
   \texttt{Y\_eqJR} and \texttt{Y\_eqJK} & Jackknife mean and error of \texttt{Y}, where \texttt{Y} stands for \texttt{Green, SpinZ, SpinXY}, and \texttt{Den}.\\
   & The suffixes \texttt{R} and \texttt{K} refer to real and reciprocal space, respectively.\\
   \texttt{Y\_R0/g\_R0} & Time-resolved and spatially local Jackknife mean and error of \texttt{Y},\\
   & where \texttt{Y} stands for \texttt{Green, SpinZ, SpinXY}, and \texttt{Den}.\\
   \texttt{Y\_kx\_ky/g\_kx\_ky} & Time resolved and $\vec{k}$-dependent Jackknife mean and error of \texttt{Y},\\
   & where \texttt{Y} stands for \texttt{Green, SpinZ, SpinXY}, and \texttt{Den}.\\\bottomrule
    \end{tabular}
   \caption{ Standard output files of the error analysis. \label{table:analysis_output}}
\end{table}
%
In the following, we describe the formatting of the output files mentioned in Table \ref{table:analysis_output}.
\begin{itemize}
\item For the scalar quantities \texttt{X}, the output files  \texttt{X\_scalJ} have the following formatting:
\begin{alltt}
Effective number of bins, and bins:           <N_bin - n_skip>          <N_bin>

OBS :    1      <mean(X)>      <error(X)>

OBS :    2      <mean(sign)>   <error(sign)>
\end{alltt}

\item For the equal-time correlation functions \texttt{Y}, the formatting of the output files \texttt{Y\_eqJR} and \texttt{Y\_eqJK} follows this structure:
\begin{alltt}
do i = 1, N_unit_cell
   <k_x(i)>   <k_y(i)>
   do alpha = 1, N_orbital
   do beta  = 1, N_orbital
      alpha   beta   Re<mean(Y)>   Re<error(Y)>   Im<mean(Y)>   Im<error(Y)>
   enddo
   enddo
enddo
\end{alltt}
where \texttt{Re} and \texttt{Im} refer to the real and imaginary part, respectively.

\item The imaginary-time displaced correlation functions \texttt{Y} are written to the output files \texttt{Y\_R0/g\_R0}, when measured locally in space, 
and to the output files \texttt{Y\_kx\_ky/g\_kx\_ky} when they are measured $\vec{k}$-resolved.    The first line of the  file prints the number of time slices, 
 the number of bins and the inverse temperature. 
Both output files have the following formatting:
\begin{alltt}
do i = 0, Ltau
   tau(i)   <mean( Tr[Y] )>   <error( Tr[Y])>
enddo
\end{alltt}
where \texttt{Tr} corresponds to the trace over the orbital degrees of freedom.   For particle-hole quantities at finite temperature,  $\tau$ runs from 
$0$ to $\beta/2$.   In all other cases it runs from $0$ to $\beta$. 


\end{itemize}

% Copyright (c) 2016 2017 The ALF project.
% This is a part of the ALF project documentation.
% The ALF project documentation by the ALF contributors is licensed
% under a Creative Commons Attribution-ShareAlike 4.0 International License.
% For the licensing details of the documentation see license.CCBYSA.

% !TEX root = Doc.tex
%
%-------------------------------------------------------------------------------------
\subsection{Running the code}\label{sec:running}
%-------------------------------------------------------------------------------------
%
In this section we describe the steps how to compile and run the code, as well as how to perform the error analysis of the data.
%
%-------------------------------------------------------------------------------------
\subsubsection{Compilation}
\label{sec:compilation}
%-------------------------------------------------------------------------------------
%
The environment variables and the directives to compile the code are set in the following makefile \texttt{Makefile}:
\lstset{style=bash}
\begin{lstlisting}

# -DMPI selects MPI.
# -DTEMPERING selects tempering mode.  MPI has to be switched on.
# -DSTAB1   Alternative stabilization, using the singular value decomposition.
# -DSTAB2   Alternative stabilization, lapack QR with  manual pivoting.
#           Packed form of QR factorization is not used.
# -DSTAB3   Alternative stabilization, using QR  with pivoting.
#           Internally, scales larger and smaller one are distinguished.
# -DLOG     Alternative stabilization, using QR  with pivoting.
#           Internally, scales are stored on log axsis to allow larger beta and
#           larger and smaller have to be distinguished.
# (no flag) Default  stabilization, using lapack QR with pivoting. 
#           Packed form of QR factorization  is used. 
# -DQRREF   Enables reference lapack implementation of QR decomposition.
# Recommendation: just use the -DMPI flag if you want to run in parallel or 
#                 leave it empty for serial jobs.  
#                 The default stabilization, no flag, is generically the best. 
#                 Consider using -DLOG if you run into overflows
PROGRAMCONFIGURATION = -DMPI 
PROGRAMCONFIGURATION = 
f90 = gfortran
export f90
F90OPTFLAGS = -O3 -Wconversion  -fcheck=all
F90OPTFLAGS = -O3
export F90OPTFLAGS
F90USEFULFLAGS = -cpp -std=f2003
F90USEFULFLAGS = -cpp
export F90USEFULFLAGS
FL = -c ${F90OPTFLAGS} ${PROGRAMCONFIGURATION}
export FL
DIR = ${CURDIR}
export DIR
Libs = ${DIR}/Libraries/
export Libs
LIB_BLAS_LAPACK = -llapack -lblas
export LIB_BLAS_LAPACK

all: lib ana program

lib:
	cd Libraries && $(MAKE)
ana:
	cd Analysis && $(MAKE)
program:
	cd Prog && $(MAKE)


clean: cleanall
cleanall: cleanprog cleanlib cleanana
cleanprog:
	cd Prog && $(MAKE) clean
cleanlib:
	cd Libraries && $(MAKE) clean
cleanana:
	cd Analysis && $(MAKE) clean
help:
	@echo "The following are some of the valid targets of this Makefile"
	@echo "all, program, lib, ana, clean, cleanall, cleanprog, cleanlib,
	       cleanana"

\end{lstlisting}
In the above, the GNU Fortan compiler \texttt{gfortran} is set.\footnote{A known issue with the alternative Intel Fortran compiler \texttt{ifort} is the handling of automatic, temporary arrays 
which \texttt{ifort} allocates on the stack. For large system sizes and/or low temperatures this may lead to 
a runtime error. One solution is to demand allocation of arrays above a certain size on the heap instead of the stack. 
This is accomplished by the \texttt{ifort} compiler flag \texttt{-heap-arrays [n]} where \texttt{[n]} is the minimal size (in kilobytes, for example \texttt{n=1024}) of arrays 
that are allocated on the heap.}
We provide a set of options for compilation of the QMC code. The present options are \texttt{-DMPI}, \texttt{-DQRREF}, \texttt{-DSTAB1}, and \texttt{-DSTAB2}. 
They can be included in the string variable \texttt{PROGRAMCONFIGURATION} by the user, as shown above.
The program can be compiled and ran either in single-thread mode (default) or 
in multi-threading mode (define \texttt{-DMPI}) using the MPI standard for parallelization. The remaining three compiler options select a particular stabilization scheme for the matrix multiplications (see Sec.~\ref{sec:output_prec}).
To compile the libraries, the analysis routines and the QMC program at once, just execute the single command:
\begin{verbatim}
make
\end{verbatim}
To clean up all directories and remove the object files and executables, execute the command \texttt{make clean}. As can be seen in the above makefile, there exist also rules to compile/clean up the library, the analysis routines and the QMC program separately.  

%
%-------------------------------------------------------------------------------------
\subsubsection{Starting a simulation}
%-------------------------------------------------------------------------------------
%
To start a simulation from scratch, the following files have to be present: \texttt{parameters} and \texttt{seeds}. 
To run a single-thread simulation, for example by using the parameters of one of the  Hubbard models described in Sec.~\ref{sec:ex}, issue the command
\begin{verbatim}
./Prog/Examples.out
\end{verbatim}
To restart the code using an existing simulation as a starting point, first run the script \texttt{out\_to\_in.sh} to set 
the input configuration files.
%
%-------------------------------------------------------------------------------------
\subsubsection{Error analysis}
%-------------------------------------------------------------------------------------
%
Note that the error analysis script requires the presence of the environment variable \path{DIR} which defines the path to the error analysis programs.
So before starting the error analysis, one has to make this variable available which is done by the script \path{setenv.sh}. The command is
\begin{verbatim}
source ./setenv.sh
\end{verbatim}
To perform an error analysis based on the Jackknife resampling method (Sec.~\ref{sec:jack})  of the Monte Carlo bins for all observables run the script \texttt{analysis.sh} 
(see Sec.~\ref{sec:analysis}). In case that the parameter \path{N_auto} is set to a finite value the script will also trigger the computation of autocorrelation functions (Sec.~\ref{sec:autocorr}).


% Copyright (c) 2016-2019 The ALF project.
% This is a part of the ALF project documentation.
% The ALF project documentation by the ALF contributors is licensed
% under a Creative Commons Attribution-ShareAlike 4.0 International License.
% For the licensing details of the documentation see license.CCBYSA.

% !TEX root = doc.tex

\section{Maximum entropy }

\subsection{General setup}
Generically, the maximum entropy code computes the  image  $A(\omega) $ for a given  data  set $g(\tau) $  and kernel $K(\tau,\omega) $:
\begin{equation}
g(\tau) =  \int_{\omega_\text{start}}^{\omega_\text{end}} d {\omega} K(\tau,\omega) A(\omega).
\end{equation} 
The  ALF-package includes a standard implementation of the stochastic MaxEnt as formulated in the article of K. Beach Ref.~\cite{Beach04a}. Here we will comment on the workflow.  The module 
\texttt{Libraries/Modules/\allowbreak{}maxent\_stoch.f90} contains a general implementation and the wrapper is in \texttt{Analysis/Max\_SAC.f90}. 

The stochastic MaxEnt is essentially a parallel tempering Monte Carlo simulation.    For a discrete set of $\tau_i$ points,  $i \in 1 \cdots n $ the energy reads
\begin{equation}
  \chi^{2}(A) =  \sum_{i,j=1}^{n}   \left[ g(\tau_i)  –    \overline{g(\tau_i)} \right] C^{-1}(\tau_i,\tau_j) \left[    g(\tau_j)  –  \overline{g(\tau_j)} \right] 
\end{equation} with $ \overline{g(i)} =\int d{\omega} K(\tau_{i},\omega)  A(\omega)$ and  $C$ the covariance matrix. 
The set  of inverse temperatures  we will consider  in the parallel tempering reads:
$ \alpha_m = \alpha_{st}  R^{m}, \; \; m = 1 \cdots N_{\alpha} $.   The phase space corresponds to all possible spectral functions with given sum rule and required positivity.  Finally,  the partition function reads
$Z =  \int{DA} e^{-\alpha \chi^{2}(A)}$.  

In the code, the spectral function is parametrized  by a  set of Dirac $\delta$ functions: 
\begin{equation}
      A(\omega)  = \sum_{i=1}^{N_{\gamma}} a_{i} \delta \left( \omega - \omega_i \right).
\end{equation}
To produce a histogram of  $ A(\omega) $ we divide  the frequency range in \texttt{Ndis} intervals. 
The Green function is read from the file \texttt{g\_dat}  corresponding to the  output of the the  \texttt{cov\_tau.f90} analysis program.  
Below, we summarize the  parameters   in   name-lists  \texttt{VAR\_Max\_Stoch }  and  \texttt{VAR\_errors }   in the  \texttt{parameters}  file   required  to run the maxent code .  
\lstset{style=fortran}
\begin{lstlisting} 

&VAR_Max_Stoch               ! Variables for Stochastic Maximum entropy
Ngamma                       ! # of  Dirac functions for parametrization
Om_st                        ! Frequency range lower bound
Om_en                        ! Frequency range upper bound
NDis                         ! # of boxes for histogram
Nbins                        ! # of bins for Monte Carlo
Nsweeps                      ! # of sweeps per bin
NWarm                        ! The Nwarm first bins will be ommitted
N_alpha                      ! # of tempertures
alpha_st                     ! smallest inverse temperature
R                            ! increment for inverse temperature (see above) 
Channel                      ! T0       : Zero temperature
                             ! P        : Finite temperarure particle 
                             ! PH       : Finite temperarure particle-hole
                             ! PP       : Finite temperarure particle-particle 
Checkpoint                   !.true.    : dump files will be produced so  
                             !            as to be able to restart the simulation
                             !.false.   : dump files will not be produced 
Tolerance                    !Data points for which the relative error
                             !exceeds the tolerance threshold will be omitted.
/

&VAR_erros                   ! Variables for the error analyis
....                         !
N_cov                        ! =1  Covariance will be taken into account
                             ! =0  Covariance will not be taken into account 
/
\end{lstlisting}
Note that the variable  \texttt{N\_cov}  req

\noindent
\textbf{Output files} \\
The code produces  the following output files.
\begin{itemize}
\item The files  \texttt{Aom\_n}  correspond to the average spectral function at inverse  temperature  $ \alpha_n $. This corresponds to
$  \langle A_n(\omega) \rangle =   \frac{1}{Z}   \int DA(\omega)    e^{-\alpha_n \chi^{2}(A)  } A(\omega). $
The file contains three colums  $ \omega, \;  \langle A_n(\omega) \rangle , \;  \Delta \langle A_n(\omega) \rangle $.

\item The files \texttt{Aom\_ps\_n}   contain the average image over  the  inverse   temperatures  $ \alpha_n $ to $ \alpha_{N_\gamma} $  see Ref.~\cite{Beach04a} for more details.   
 The first three columns have the same meaning as for the files \texttt{Aom\_n}

\item The file \texttt{Green} contains the Green function. The three columns correspond to $ \omega, \;   \text{ Re} G(\omega), \;  \text{  Im} G(\omega)  $.  This is obtained from the spectral function through:
\begin{equation}
 G(\omega) =  -\frac{1}{\pi} \int d \Omega   \frac{A(\Omega)}{\omega – \Omega + i \delta}
 \end{equation}
where  $ \delta =  \Delta \omega$ with $ \Delta \omega = (\omega_\text{end} -  \omega_\text{start})/\text{Ndis}$ and the image corresponds to that of the file \texttt{Aom\_ps\_m} with $ m = N_{\alpha} -10 $. 
The first column of the  \texttt{Green}  file is a place holder for post-processing. The last three columns   correspond to $\omega, \text{Re} G(\omega) ,   - \text{Im} G(\omega)/\pi $. 

\item  One of the most important files is the file  \texttt{energies}. It contains there columns:  $ \alpha_n, \langle \chi^2 \rangle, \Delta \langle \chi^2 \rangle $.

\item   \texttt{best\_fit}  gives the values of $a_i$ and $\omega_i$   (recall that $ A(\omega)  = \sum_{i=1}^{N_{\gamma}} a_{i} \delta \left( \omega - \omega_i \right)$) corresponding to the last configuration of the  lowest temperature run.

\item  The File \texttt{data\_out}  is a crosscheck. It plots   $ \tau,  g(\tau),  \Delta g(\tau), \int d \omega  K(\tau, \omega) A(\omega) $ where the image  corresponds to the best fit (i.e. the lowest temperature). 
This file will give you  a feeling on how good the fit actually is.  Note that  \texttt{data\_out} contains only the data points that have  passed the tolerance test. 


\item There are two \texttt{dump} files which are generated. Since  the MaxEnt is a  Monte Carlo code, one  would like to be able to continue a simulation to improve. The data in the dump files will allow you to pursue the simulation without loosing the first run(s).   These files are  only generated if the variable  \texttt{checkpoint} is set to true. 
 \end{itemize}

The essential question is: which image should one use. There is no real answer to this question in the context of the stochastic MaxEnt. The only rule of thumb is to consider temperatures for which the \( \chi^2 \) is  comparable to the number of data points.


\subsection{Single particle quantities}
For the single-particle Green function, 

\begin{equation} 
	\langle \hat{c}^{\phantom\dagger}_{k} (\tau)  \hat{c}^{\dagger}_{k} (0)   \rangle   = \int d \omega  K_p(\tau,\omega)   A_p(k, \omega) 
\end{equation}
with 
\begin{equation}
K_{p}(\tau,\omega) =    \frac{1}{\pi} \frac{e^{-\tau \omega} }  {  1 + e^{-\beta\omega} }
\end{equation}
and in the Lehmann representation, 
 \begin{equation}
   A_p(k, \omega) = \frac{ \pi}{Z} \sum_{n,m} e^{-\beta E_n } \left( 1 + e^{-\beta \omega}\right) | \langle n | c_n | m  \rangle |^{2} \delta \left( E_m - E_n - \omega \right)  
\end{equation}  
Here $ \left( \hat{H} - \mu \hat{N} \right) | n \rangle = E_n | n \rangle  $.

Note that  $ A_p(k, \omega)  = - \text{Im} G^{\text{ret}} (k, \omega) $ with 
\begin{equation}
	G^{\text{ret}} (k, \omega)  = -i \int d t \Theta(t)  e^{i \omega t} \langle \left\{ \hat{c}^{\phantom\dagger}_{k} (t), \hat{c}^{\dagger}_{k} (0) \right\} \rangle
\end{equation}
Finally the sum rule reads:
\begin{equation}
	\int d \omega  A_p(k, \omega)  = \pi \langle  \left\{ \hat{c}^{\phantom\dagger}_{k} , \hat{c}^{\dagger}_{k}  \right\}   \rangle = \pi 
\end{equation}
Using the \texttt{Max\_Sac.f90}  with \texttt{Channel="P"}   will  load the above Kernel in the MaxEnt library.  Note that in this case the back  transformation is set to unity.  
Note that since for each  configuration of fields,  $ \langle  \langle \hat{c}^{\phantom\dagger}_{k} (\tau=0)  \hat{c}^{\dagger}_{k} (0)   \rangle  \rangle_{C} +   
\langle \langle \hat{c}^{\phantom\dagger}_{k} (\tau=\beta)  \hat{c}^{\dagger}_{k} (0)   \rangle \rangle_{C} = 
\langle \langle \left\{ \hat{c}^{\phantom\dagger}_{k},   \hat{c}^{\dagger}_{k}    \right\} \rangle \rangle_{C}   = 1$.  Hence if both  the $\tau=0$ anf $\tau=\beta$ data points are included, the covariance matrix will have a zero eigenvalue and the $\chi^{2}$. measure is not defined. Hence for the particle channel, the program omits the $\tau=\beta$ data point.     One should also not that there are special  particle-hole symmetric  cases where the $\tau=0$ data point shows no  fluctuations. In this case, the 
code equally omits the $\tau=0$ data point. 
\subsection{Particle-hole quantities }

\noindent
\textbf{Imaginary time formulation.}
 For particle-hole quantities such as spin-spin or charge-charge correlations, 
the  Kernel reads:
\begin{equation}
	\langle \hat{S}(q,\tau) \hat{S}(-q,0) \rangle  = \frac{1}{\pi} 
   \int {\text d} \omega  \frac{e^{- \tau \omega} }{ 1 - e^{-\beta  \omega} } \chi''(q,\omega).
\end{equation}
This follows directly from the  Lehmann representation: 
\begin{equation}
 \chi''(q,\omega)  = \frac{\pi}{Z} \sum_{n,m} e^{-\beta E_n} |\langle n | \hat{S}(q) | \rangle m |^2 
\delta ( \omega + E_n - E_m) \left( 1 - e^{-\beta  \omega} \right) 
\end{equation}
Since the linear response to a Hermitian perturbation  is real, $\chi''(q,\omega)  = - \chi''(-q,-\omega)$.  Hence for systems with inversion symmetry -- that 
we will consider here -- $\langle \hat{S}(q,\tau) \hat{S}(-q,0) \rangle $ is a symmetric function around $\beta= \tau/2$.  The analysis  file \texttt{cov\_tau\_ph.f90} produced at compilation
time will use this  to define an improved estimator. 

The  Stochastic MaxEnt requires a sum rule, such that   the Kernel and image have to be adequately redefined. 
Consider: 
\begin{equation}
	\text{coth}(\beta \omega/2) \chi''(q,\omega)
\end{equation}
For this quantitiy, we have the sum rule since: 
\begin{equation}
	\int {\text d} \omega 	\text{coth}(\beta \omega/2) \chi''(q,\omega) = 
  2 \pi \langle \hat{S}(q,\tau=0) \hat{S}(-q,0) \rangle
\end{equation}
which is just the first point in the data. 

Hence,
\begin{equation}
	\langle \hat{S}(q,\tau) \hat{S}(-q,0) \rangle  =  
       \int {\text d} \omega  \underbrace{ \frac{1}{\pi} \frac{e^{- \tau \omega} }
            { 1 - e^{-\beta  \omega} } \text{tanh}(\beta \omega/2)  }_{K_{pp}(\tau,\omega)} 
       \underbrace{ \text{coth}(\beta \omega/2)   \chi''(q,\omega) }_{A(\omega)} 
\label{Kpp.eq}
\end{equation}
and one  computes $A(\omega)$. Note that since $\chi'' $ is an odd function of $\omega$  one restricts the integration range  positive values of $\omega$. 
Hence: 
\begin{equation}
	\langle \hat{S}(q,\tau) \hat{S}(-q,0) \rangle  =  
       \int_{0}^{\infty}  {\text d} \omega \underbrace{\left( K(\tau,\omega)  + K(\tau,-\omega) \right)}_{K_{ph}(\tau,\omega)}  A(\omega).
\end{equation}
In the code, $\omega_\text{start}$ is set to zero by default and the Kernel $K_{ph}$ is used in the code and is defined in the  routine \texttt{XKER\_ph}. 
In general,  one would like to produce the  dynamical structure factor that relates to the susceptibility according to
\begin{equation}
 S(q,\omega)  = \chi''(q,\omega)/\left( 1 - e^{-\beta  \omega} \right). 
\end{equation}

In the code the routine \texttt{BACK\_TRANS\_ph}   transforms the image $A$ to the desired quantity.
\begin{equation}
	S(q,\omega) = \frac{A(\omega)}{1 + e^{-\beta \omega} }  
\end{equation}

\noindent
\textbf{Matsubara frequency formulation.}
The ALF  library uses  imaginary time. It is however possible to formulate the MaxEnt in  Matsubara frequencies.
Consider:
\begin{equation}
  \chi(q,i\Omega_m) = \int_0^{\beta} {\text d} \tau  e^{i \Omega_m \tau}
	\langle S(q,\tau) S(-q,0) \rangle  = \frac{1}{\pi}
   \int {\text d} \omega  \frac{\chi''(q,\omega)}{ \omega - i \Omega_m }.
\end{equation}
Using the fact that $\chi''(q,\omega) = -\chi''(-q,-\omega) = -\chi''(q,-\omega)$ one obtains:
\begin{equation}
\begin{gathered}
  \chi(q,i\Omega_m) = 
	\frac{1}{\pi}
   \int_0^{\infty} {\text d} \omega \left(\frac{1}{ \omega - i \Omega_m } - \frac{1}{ -\omega - i \Omega_m } \right)
         \chi''(q,\omega) \\
    = \frac{2}{\pi} \int_0^{\infty} {\text d} \omega \frac{\omega^2}{ \omega^2  + \Omega_m^2 } 
  \frac{\chi''(q,\omega)}{\omega} 
   \equiv \int_0^{\infty} {\text d} \omega K(\omega,i\Omega_m) A(q,\omega)
\end{gathered}
\end{equation}
with
\begin{equation}
   K(\omega,i\Omega_m) = \frac{\omega^2}{ \omega^2  + \Omega_m^2 } 
\end{equation}
and
\begin{equation}
A(q,\omega) =  \frac{2}{\pi}   \frac{\chi''(q,\omega)}{\omega} 
\end{equation}
The above definitions are useful since the image satisfies the sum rule:
\begin{equation}
\int_0^{\infty} {\text d} \omega A(q,\omega) =  \frac{1}{\pi}  \int_{-\infty}^{\infty} {\text d} \omega 
   \frac{\chi''(q,\omega)}{\omega}   \equiv \chi(q,i\Omega_m=0)
\end{equation}


\subsection{Particle-Particle quantities}

Similarly to the particle-hole channel  the particle-particle channel is also a bosonic correlation function. Here however we do not assume that the 
imaginary time data is symmetric around   the $\tau = \beta/2$ point.  We use the Kernel $K_{pp}$ define in Eq.~\ref{Kpp.eq}  and consider the whole frequency range. 
The back transformation  yields
\begin{equation}
 \frac{\chi''(\omega)} {\omega}   = \frac{\text{tanh} \left( \beta \omega/2 \right) }{ \omega }   A(\omega) 
\end{equation}



\subsection{Zero temperature, projective code}

 In the zero temperature limit,  the spectral function associated to an operator $\hat{O} $    reads:
 \begin{equation}
 	  A_o(\omega)    = \pi  \sum_{n}    | \langle n  | \hat{O} | 0 \rangle |^2 \delta( E_n - E_0 - \omega) 
 \end{equation}
 such that 
 \begin{equation}
 	\langle 0 | \hat{O}^{\dagger}(\tau) \hat{O}^{}(0) | 0 \rangle =  \int d  \omega  K_0(\tau,\omega) A_0(\omega) 
 \end{equation}
 with 
 \begin{equation}
 	K_0(\tau,\omega)  = \frac{1}{\pi}e^{-\tau \omega}.
 \end{equation}
 The zeroth moment of the spectral function reads, 
 \begin{equation}
  \int d \omega A_o(\omega) = \pi \langle 0 | \hat{O}^{\dagger}(0) \hat{O}^{}(0) | 0 \rangle, 
 \end{equation}
 and hence corresponds to the first data point. 
 In the zero-temperature limit one does not distinguish between  particle, particle-hole, or particle-particle channels.
 Using the \texttt{Max\_Sac.f90}  with \texttt{Channel="T0"}   will  load the above Kernel in the MaxEnt library. In this case the back  transformation is set to unity. 
 The code will also cut-off the tail of the  imaginary time correlation function  if the relative error is greater that the variable \texttt{Tolerance}. 

\section{Examples}\label{sec:ex}
% !TEX root = Doc.tex
\section{Walkthrough: the $SU(2)$-Hubbard model on a square lattice}
In this section, we describe the subroutine \texttt{Hamiltonian\_Hub.f90} which is an implementation of the Hubbard model on the square lattice. 
The $SU(2)$-symmetric Hubbard model is given by
\begin{equation}
\label{eqn_hubbard_sun}
\mathcal{H}=
\sum\limits_{\sigma=1}^{2} 
\sum\limits_{x,y } 
  c^{\dagger}_{x \sigma} T_{x,y}c^{\phantom\dagger}_{y \sigma} 
+ \frac{U}{2}\sum\limits_{x}\left[
\sum\limits_{\sigma=1}^{2}
\left(  c^{\dagger}_{x \sigma} c^{\phantom\dagger}_{x \sigma}  -1/2 \right) \right]^{2}\;.
\end{equation}

\subsection{The lattice  and basic parameters}
Here we set $\vec{a}_1 =  (1,0) $ and $\vec{a}_2 =  (0,1) $  and for an $L_1 \times L_2$  lattice  $\vec{L}_1 = L_1 \vec{a}_1$ and  $\vec{L}_2 = L_2 \vec{a}_2$.     
With this choice   the call to  \texttt{ Call Make\_Lattice( L1, L2, a1,  a2, Latt )} will generate the lattice   \texttt{Latt} of type \texttt{Lattice} such that  $N_{dim}   =N_{unit\;cell} \equiv Latt\%N$. 

In order to bring the general Hamiltonian (\ref{eqn_general_ham2}) to this form, we set
\begin{eqnarray}
N_{fl}         &=&  1 \nonumber\\
N_{col} \equiv N_{SUn}       &=&  2 \nonumber\\
M_T      & = &   1 \nonumber \\
T^{(ks)}_{x y}        &=&    T_{x,y}  \nonumber\\
M_V              &=&  N_{dim} \nonumber\\
U_{k}          &=&   -\frac{U}{2} \nonumber\\
V_{x y}^{(ks)} &=&  \delta_{x,y} \nonumber\\
\alpha_{ks}     &=&  \frac{1}{2} \nonumber \\
M_I              &=&  0
\end{eqnarray}
Since $N_{fl}=1$ for $SU(N)$-symmetric Hubbard models, we will drop the flavor index $\sigma$ in the following.  

\subsection{Hopping term}
The hopping matrix is implemented as follows. 
We allocate an array of dimension $1\times 1$, called \texttt{Op\_T}. It therefore contains only a single \texttt{Operator} structure.
We set the effective dimension for the hopping term: $N=N_{dim}$. 
And we allocate and initialize this structure by a single call to the subroutine \texttt{Op\_make}: 
\begin{verbatim}
call Op_make(Op_T(1,1),Ndim)
\end{verbatim}

Since the effective dimension is identical to the total dimension, it follows trivially, that ${\bm P}_{T}=\mathds{1}$ and ${\bm O}_{T}={\bm T}$. 


\mycomment{Note that although a checkerboard decomposition is not yet used for the Hubbard model, in principle it can be implemented.}

\subsection{Interaction term}
To implement this interaction, we allocate an array of \texttt{Operator} structures. The array is called  \texttt{Op\_V} and has dimensions $N_{dim}\times N_{fl}=N_{dim}\times 1$. 
We set the effective dimension for the interaction term: $N_{eff}=1$. 
And we allocate and initialize this array of structures by repeatedly calling the subroutine \texttt{Op\_make}: 
\begin{verbatim}
N_dim = Latt%N
N_fl = 1
N_eff = 1

do nf = 1, N_FL
do i  = 1, Latt%N
call Op_make(Op_V(i,nf),N_eff)
enddo
enddo
\end{verbatim}
For each lattice site $i$, the projection matrices ${\bm P}_{V}^{(i)}$ are of dimension $1\times N_{dim} $ and have one non-vanishing entry: $(P_{V}^{(i)})_{1j}=\delta_{ij}$. 
The effective matrices are scalars in this example: ${\bm O}_{V}^{(i)}=1$.\\


\begin{table}[h]
   \begin{tabular}{l l}
    Name of variable in the code & Description \\\hline
    \texttt{Ndim}    & Spacial dimension of the lattice (total number of sites) \\
    \texttt{Latt\%N} & Number of unit cells of the underlying Bravais lattice  \\
    \texttt{Op\_T}   & Array of structure variables that bundles all variables\\
                     & needed to define the hopping operator.\\
    \texttt{Op\_V}   & Array of structure variables that bundles all variables\\
                     & needed to define the two-particle interaction operator.\\ 
    \texttt{N\_sun}  & Number of fermion colors \mycomment{spin states of the $SU(N_{sun})$-symmetric fermions}\\
    \texttt{N\_fl}   & Number of fermion flavors\\
   \end{tabular}
   \caption{Common variables that are set in the Hamiltonian, operator and lattice modules of the code. 
   \mycomment{!!! We have a missmatch in the labelling: $N_{col}=\texttt{N\_sun}$ !!!}
   \label{tab:definitions}}
\end{table}

\subsection{Definition of the square lattice}
This is set in the subroutine \texttt{Ham\_latt}.
The square lattice is already implemented. In principle, one can specify other lattice geometries and use them by specifying the keyword \texttt{Lattice\_type} in the parameter file.



\subsection{Observables for the Hubbard model}


To do next:
\begin{itemize}
\item dicuss the measurements: what observables exit and how do I add a new one?
\item  discuss the implementation of the lattice.
\item discuss the Hubbard-Stratonovich decompositions (this is related to the coupling in the operator structure), discuss also the spin-symmetry-breaking HS-decomposition for the Hubbard model.
\end{itemize}


\section{Miscellaneous}\label{sec:misc}
% Copyright (c) 2016 The ALF project.
% This is a part of the ALF project documentation.
% The ALF project documentation by the ALF contributors is licensed
% under a Creative Commons Attribution-ShareAlike 4.0 International License.
% For the licensing details of the documentation see license.CCBYSA.

% !TEX root = Doc.tex
%-------------------------------------------------------------------------------------
\subsection{Other models}
%-------------------------------------------------------------------------------------
\label{sec:other_models}

The aim of this section is to briefly mention  a small  selection of  other models that can be studied using the QMC code of the ALF project.  
  
%-------------------------------------------------------------------------------------
\subsubsection{Kondo lattice model}
%-------------------------------------------------------------------------------------

Simulating the Kondo lattice with the QMC code of the ALF project    requires rewriting of the model along the lines of Refs.~\cite{Assaad99a,Capponi00,Beach04}.  
Adopting the notation of these articles,   the Hamiltonian that one will simulate reads: 
 \begin{equation}\label{eqn:ham_kondo}
 	\hat{\mathcal{H}}  = 
	\underbrace{-t \sum_{\langle  \vec{i},\vec{j} \rangle,\sigma} \left( \hat{c}_{\vec{i},\sigma}^{\dagger}  \hat{c}_{\vec{j},\sigma}^{\phantom\dagger}   + \text{H.c.} \right) }_{\equiv \hat{\mathcal{H}}_t} - \frac{J}{4} 
	\sum_{\vec{i}} \left( \sum_{\sigma} \hat{c}_{\vec{i},\sigma}^{\dagger}  \hat{f}_{\vec{i},\sigma}^{\phantom\dagger}  + 
	                                                        \hat{f}_{\vec{i},\sigma}^{\dagger}  \hat{c}_{\vec{i},\sigma}^{\phantom\dagger}   \right)^{2}   +
        \underbrace{\frac{U}{2}   \sum_{\vec{i}}   \left( \hat{n}^{f}_{\vec{i}} -1 \right)^2}_{\equiv \hat{\mathcal{H}}_U}.
 \end{equation}
This form is included in the general Hamiltonian (\ref{eqn:general_ham})  such that the above Hamiltonian can  be implemented in our program package.  
The  relation to the Kondo lattice model follows  from expanding the square  of the hybridization to obtain: 
 \begin{equation}
 	\hat{\mathcal{H}}  =\hat{\mathcal{H}}_t   
	+ J \sum_{\vec{i}}  \left(  \hat{\vec{S}}^{c}_{\vec{i}} \cdot  \hat{\vec{S}}^{f}_{\vec{i}}    +   \hat{\eta}^{z,c}_{\vec{i}} \cdot  \hat{\eta}^{z,f}_{\vec{i}}  
		-  \hat{\eta}^{x,c}_{\vec{i}} \cdot  \hat{\eta}^{x,f}_{\vec{i}}  -  \hat{\eta}^{y,c}_{\vec{i}} \cdot  \hat{\eta}^{y,f}_{\vec{i}} \right) 
	 + \hat{\mathcal{H}}_U.
 \end{equation}
 where the $\eta$-operators  relate to the spin-operators via a particle-hole transformation in one spin sector: 
 \begin{equation} 
 	\hat{\eta}^{\alpha}_{\vec{i}}  = \hat{P}^{-1}  \hat{S}^{\alpha}_{\vec{i}} \hat{P}  	\; \text{ with }  \;   
	\hat{P}^{-1}  \hat{c}^{\phantom\dagger}_{\vec{i},\uparrow} \hat{P}  =   (-1)^{i_x+i_y} \hat{c}^{\dagger}_{\vec{i},\uparrow}  \; \text{ and }  \;   
	\hat{P}^{-1}  \hat{c}^{\phantom\dagger}_{\vec{i},\downarrow} \hat{P}  = \hat{c}^{\phantom\dagger}_{\vec{i},\downarrow} 
 \end{equation}
 Since the $\hat{\eta}^{f} $- and $ \hat{S}^{f} $-operators  do not alter the  parity [$(-1)^{\hat{n}^{f}_{\vec{i}}}$ ] of the $f$-sites, 
 \begin{equation}
 	\left[  \hat{\mathcal{H}}, \hat{\mathcal{H}}_U \right] = 0.
 \end{equation}
 Thereby,  and for positive values of $U$ ,  doubly occupied  or empty $f$-sites -- corresponding to even parity sites -- are suppressed  by a  Boltzmann factor 
 $e^{-\beta U/2} $ in comparison to odd parity sites.   Choosing $\beta U $ adequately essentially allows to  restrict the Hilbert space to  odd parity $f$-sites.  
 In this Hilbert space $\hat{\eta}^{x,f} = \hat{\eta}^{y,f} =  \hat{\eta}^{z,f} =0$  such that the Hamiltonian (\ref{eqn:ham_kondo}) reduces to the Kondo lattice model. 

%-------------------------------------------------------------------------------------
\subsubsection{SU(N) Hubbard-Heisenberg models}
%-------------------------------------------------------------------------------------

SU(2N) Hubbard-Heisenberg \cite{Assaad04,Lang13} models can be written as:
\begin{equation}
 \hat{\mathcal{H}}  =  
 \underbrace{ - t \sum_{ \langle \vec{i},\vec{j} \rangle }    \left(  \vec{\hat{c}}^{\dagger}_{\vec{i}}  \vec{\hat{c}}^{\phantom{\dagger}}_{\vec{j}} + \text{H.c.} \right) }_{\equiv \hat{\mathcal{H}}_t} \; \; 
\underbrace{ -\frac{J}{2 N}  \sum_{ \langle \vec{i},\vec{j} \rangle  } \left(
           \hat{D}^{\dagger}_{ \vec{i},\vec{j} }\hat{D}^{\phantom\dagger}_{ \vec{i},\vec{j}}  +
            \hat{D}^{\phantom\dagger}_{ \vec{i},\vec{j} } \hat{D}^{\dagger}_{ \vec{i},\vec{j} }  \right) }_{\equiv\hat{\mathcal{H}}_J}
            + 
 \underbrace{\frac{U}{N}  \sum_{\vec{i}} \left(
             \vec{\hat{c}}^{\dagger}_{\vec{i}}  \vec{\hat{c}}^{\phantom\dagger}_{\vec{i}} -  {\frac{N}{2} } \right)^2}_{\equiv \hat{\mathcal{H}}_U}
\end{equation}
Here,
$ \vec{\hat{c}}^{\dagger}_{\vec{i}} =
(\hat{c}^{\dagger}_{\vec{i},1},  \hat{c}^{\dagger}_{\vec{i},2}, \cdots, \hat{c}^{\dagger}_{\vec{i}, N } ) $  is an
$N$-flavored spinor, and $ \hat{D}_{ \vec{i},\vec{j}} = \vec{\hat{c}}^{\dagger}_{\vec{i}}
\vec{\hat{c}}_{\vec{j}}  $.
To use the QMC code of the ALF project  to simulate this model, one will rewrite  the $J$-term as a sum of perfect squares, 
\begin{equation}
        \hat{\mathcal{H}}_J =  -\frac{J}{4 N}  \sum_{  \langle \vec{i}, \vec{j} \rangle }
        \left(\hat{D}^{\dagger}_{  \langle \vec{i}, \vec{j} \rangle  } +  \hat{D}_{  \langle \vec{i}, \vec{j} \rangle }  \right)^2  -
        \left(\hat{D}^{\dagger}_{   \langle \vec{i}, \vec{j} \rangle } -  \hat{D}_{  \langle \vec{i}, \vec{j} \rangle}  \right)^2,
\end{equation}
so to manifestly bring it into the form of the general Hamiltonian(\ref{eqn:general_ham}). 
It is amusing to note that setting the hopping $t=0$,    charge fluctuations  will be suppressed by the  Boltzmann factor $e^{-\beta U /N \left(  \vec{\hat{c}}^{\dagger}_{\vec{i}}  \vec{\hat{c}}^{\phantom\dagger}_{\vec{i}} -  {\frac{N}{2} } \right)^2 } $ 
\mycomment{MB: corrected minus sign in the exponent}
since in this case  $ \left[   \hat{\mathcal{H}}, \hat{\mathcal{H}}_U \right]  = 0 $.
\mycomment{MB: I suggest to use only the J-term here, $ \left[   \hat{\mathcal{H}}_{J}, \hat{\mathcal{H}}_U \right]  = 0 $.}
This provides a route to use the auxiliary field QMC algorithm  to simulate -- free of the sign problem -- SU(2N) Heisenberg models in the self-adjoint antisymmetric representation  \footnote{ This corresponds to a Young tableau with single column and $N/2$ rows.}  
For odd values of $N$ recent progress  in our understanding of the  origins of the sign problem \cite{Wei16}  allows us to simulate  a set of non-trivial Hamiltonians \cite{Li15,Assaad16},  without encountering the sign problem.

%-------------------------------------------------------------------------------------
\subsubsection{Hubbard  model in the canonical ensemble}
%-------------------------------------------------------------------------------------

To simulate the Hubbard model in the canonical ensemble  one can add the constraint: 
\begin{equation}
	\hat{\mathcal{H}}   = \hat{\mathcal{H}_{tU}}     + \underbrace{\lambda \left( \hat{N} -  N \right)^{2}}_{\equiv \hat{H}_\lambda }
\end{equation}
In the limit $\lambda \rightarrow \infty $, the uniform charge fluctuations,
\begin{equation} 
      S ( \vec{q} = 0)   =  \sum_{\vec{r}}   \left[ \langle \hat{n}_{\vec{r}}  \hat{n}_{\vec{0}} \rangle  - \langle \hat{n}_{\vec{r}}\rangle \langle  \hat{n}_{\vec{0}} \rangle  \right]
\end{equation} 
are suppressed and the grand-canonical simulation maps onto a canonical one.  
To implement this in the QMC code of the ALF project,  we have adopted the following strategy.   Since  $ \left( \hat{N} -  N \right)^{2}  $ effectively corresponds to a long-range interaction one may   face the issue that the acceptance rate of a single  HS flip becomes very small on large lattices. To circumvent this problem we have  used the following decomposition: 
\begin{equation}
	e^{-\beta \hat{H}}  =   \prod_{\tau = 1}^{L_{\text{Trotter}}} \left[  e^{-\Delta \tau \hat{H}_t} e^{-\Delta \tau \hat{H}_U}  
	\underbrace{e^{-\frac{\Delta \tau}{n_{\lambda}} \hat{H}_{\lambda} } \cdots e^{-\frac{\Delta \tau}{n_{\lambda}} \hat{H}_{\lambda} } }_{n_\lambda \text{-times } }\right].
\end{equation}
Thereby, we need $n_\lambda $ fields per time slice  to impose the constraint.  Since for each field the coupling  constant is suppressed by a factor $n_{\lambda}$, we can monitor the acceptance. 
An implementation of this program can be found in  \path{Prog/Hamiltonian_Hub_Canonical.f90}  and a test run in the directory \path{Examples/Hubbard_Mz_Square_Can}

% Copyright (c) 2016 2017 The ALF project.
% This is a part of the ALF project documentation.
% The ALF project documentation by the ALF contributors is licensed
% under a Creative Commons Attribution-ShareAlike 4.0 International License.
% For the licensing details of the documentation see license.CCBYSA.
% !TEX root = Doc.tex
\newpage
\subsubsection{$Z_2$ slave spin formualtion of the Hubbard model }
In this subsection, we  demonstrate that the code can be used to  simulate the attractive Hubbard model in the  $Z_2$-slave spin formulation \cite{Ruegg10}.    
\begin{equation}
	\hat{H} = -t \sum_{\langle \vec{i}, \vec{j} \rangle, \sigma }\hat{c}^{\dagger}_{\vec{i},\sigma} \hat{c}^{\phantom{\dagger}}_{\vec{j},\sigma}   -  U  \sum_i  
	\left( \hat{n}_{\vec{i}, \uparrow} - 1/2\right)  \left( \hat{n}_{\vec{i}, \downarrow} - 1/2\right)
\end{equation}  
In the $Z_2$ slave spin  representation, the physical fermion, $\hat{c}_{\vec{i},\sigma} $,   is fractionalized into  an Ising spin carrying $Z_2$ charge and a fermion, $\hat{f}_{\vec{i},\sigma} $, carrying $Z_2$ and  global $U(1)$ charge:
\begin{equation}
	\hat{c}^{\dagger}_{\vec{i},\sigma}  = \hat{\tau}^{z}_{\vec{i}} \hat{f}^{\dagger}_{\vec{i},\sigma}.
\end{equation}
To ensure that we remain in the correct Hilbert space, the constraint:
\begin{equation}
	\hat{\tau}^{x}_{i}   - (-1)^{\sum_{\sigma}\hat{f}^{\dagger}_{i,\sigma}  \hat{f}^{\phantom{\dagger}}_{i,\sigma}  }  = 0
\end{equation}
has to be impose locally. Since $\left(  \tau^{x}_{\vec{i}}\right)^2 = 1 $ it is   equivalent to 
 \begin{equation}
 	\hat{Q}_{\vec{i}} = \tau^{x}_{\vec{i}}  (-1)^{\sum_{\sigma}\hat{f}^{\dagger}_{\vec{i},\sigma}  \hat{f}^{\phantom{\dagger}}_{\vec{i},\sigma}  }   = 1.
 \end{equation}
 In the  $Z_2$ slave spin representation the Hubbard model now reads: 
 \begin{equation}
 	 \hat{H}_{Z_2}= -t \sum_{\langle \vec{i}, \vec{j} \rangle, \sigma }  \hat{\tau}^{z}_{\vec{i}}  \hat{\tau}^{z}_{\vec{j}} \hat{f}^{\dagger}_{\vec{i},\sigma} \hat{f}^{\phantom{\dagger}}_{\vec{j},\sigma}   -  \frac{U}{4}  \sum_{\vec{i}}  \hat{\tau}^{x}_{\vec{i}}
 \end{equation}
 and  one will readily see that the constraint will commute with Hamiltonian: 
 \begin{equation}
 	\left[ \hat{H}_{Z_2}, \hat{Q}_{\vec{i}} \right] = 0.
 \end{equation}
 One can foresee that the constraint will be dynamically imposed  and that at  $T=0$ on a finite lattice both models should give the same results.    
 
 To implement  this Hamiltonian in the ALF,  it is convenient to  carry out the variable substitution  
 \begin{equation}
 	\hat{Z}_{\langle \vec{i},\vec{j}\rangle} = \hat{\tau}^{z}_{\vec{i}}  \hat{\tau}^{z}_{\vec{j}} 
 \end{equation}
 such that 
 \begin{equation}
 	\hat{\tau}^{x}_{\vec{i}}  = \hat{X}_{\vec{i},\vec{i} +  \vec{a}_x} \hat{X}_{\vec{i},\vec{i} -  \vec{a}_x} \hat{X}_{\vec{i},\vec{i} +  \vec{a}_y} \hat{X}_{\vec{i},\vec{i} -  \vec{a}_y}.
 \end{equation}
 Since there are twice as many bond variables as site variables, a constraint has to be  imposed on the $\hat{Z}_{\langle \vec{i},\vec{j}\rangle} $ variables. In fact, it is easy to see that the flux per plaquette has to take a unit value: 
 \begin{equation}
 \label{Z_constraint.Eq}
 \hat{Z}_{\vec{i},\vec{i} + \vec{a}_x} \hat{Z}_{\vec{i} +\vec{a}_x,\vec{i} + \vec{a}_x +  \vec{a}_y} 
\hat{Z}_{\vec{i} + \vec{a}_x +  \vec{a}_y ,\vec{i} + \vec{a}_y} \hat{Z}_{ \vec{i} + \vec{a}_y, \vec{i}}   = 1 \; \; \;  \forall  \; \; \;  \vec{i}.
 \end{equation}
 
 
 Within this formulation  the model  takes  the form: 
 \begin{equation}
 	 \hat{H}_{Z_2}= -t \sum_{\langle \vec{i}, \vec{j} \rangle, \sigma }  \hat{Z}_{\langle \vec{i}, \vec{j} \rangle } \hat{f}^{\dagger}_{\vec{i},\sigma} \hat{f}^{\phantom{\dagger}}_{\vec{j},\sigma}   -  \frac{U}{4}  \sum_{\vec{i}}  \hat{X}_{\vec{i},\vec{i} +  \vec{a}_x} \hat{X}_{\vec{i},\vec{i} -  \vec{a}_x} \hat{X}_{\vec{i},\vec{i} +  \vec{a}_y} \hat{X}_{\vec{i},\vec{i} -  \vec{a}_y}.
 \end{equation}
The fermion part has formally the same from  as in the Hubbard model coupled to  a dynamical Ising field  discussed in Sec.~\ref{sec:walk2}.   There are however two important differences.
\begin{itemize}
\item The moves have to respect the constraint of Eq.~\ref{Z_constraint.Eq}. Thereby single spin flip terms are prohibited and the minimal move one can carry out on  a given time slice is the following. We randomly choose a site $\vec{i} $ and  propose a move where: 
$ Z_{\vec{i},\vec{i} +  \vec{a}_x} \rightarrow - Z_{\vec{i},\vec{i} +  \vec{a}_x} $,  $ Z_{\vec{i},\vec{i} -  \vec{a}_x} \rightarrow - Z_{\vec{i},\vec{i} -  \vec{a}_x} $,
$ Z_{\vec{i},\vec{i} +  \vec{a}_y} \rightarrow - Z_{\vec{i},\vec{i} +  \vec{a}_y} $ and $ Z_{\vec{i},\vec{i} -  \vec{a}_y} \rightarrow - Z_{\vec{i},\vec{i} -  \vec{a}_y} $.  One can carry out such moves by using the global move in real space option presented in Secs.~\ref{Global_space.sec} and \ref{sec:input}.
\item The map from  $ \left\{ \tau^{z}_{\vec{i}}  \right\} $ to $ \left\{ Z_{\langle \vec{i}, \vec{j} \rangle } \right\} $  is unique.  The reverse however  is valid only up to  to a global sign.  To pin down this sign  (and thereby   the  relative signs between different time slices)  we  store per time slice the $  Z_{\langle \vec{i},\vec{j} \rangle } $ fields as well as the value of the Ising field  at  a reference site $\tau^{z}_{\vec{i} = 1} $. Within the ALF, this can be done by adding a dummy operator in the \texttt{Op\_V}  list which will carry this degree of freedom.    With this extra degree of freedom we can switch  between the two representations, without loosing any information.   To compute the Ising part of the action it is certainly more transparent to work  with the $ \left\{ \tau^{z}_{\vec{i}}  \right\} $  variables. For the  fermion determinant,  the $ \left\{ Z_{\langle \vec{i}, \vec{j} \rangle } \right\} $   are more convenient. 
 \end{itemize}
 
 We have carried out some test simulations at half-filling and at low temperatures.  The simulation can be found in directory \texttt{Examples/Z2\_Slave} and the  Hamiltonian in \texttt{Hamiltonian\_Z2\_slave\_spins.f90 }.  The simulations  are carried out at half-filling such that particle-hole symmetry leads to 
 \begin{equation}
 \langle \hat{Q}_{\vec{i}}    \rangle_{H_{Z_2}} =0.
 \end{equation} 
However the simulations suggest that 
 \begin{equation}
 \frac{1}{N}\sum_{i,j} \langle \hat{Q}_{\vec{i}}   \hat{Q}_{\vec{j}} \rangle_{H_{Z_2}}  = N 
 \end{equation} 
 at low temperatures  where  $N$  corresponds  to size of the lattice.  Note that this measurement is very noisy,  and suffers from very long autocorrelation times.  Thereby, the constraint is dynamically imposed and   simulations of the attractive Hubbard model with  \texttt{Hamiltonian\_Examples.f90}  should yield identical results. To test this,  we have computed equal time Green functions:
\begin{equation}
\langle  \hat{\tau}^{z}_{\vec{i}} \hat{f}^{\dagger}_{\vec{i},\sigma} \hat{\tau}^{z}_{\vec{j}} \vec{f}^{\phantom{\dagger}}_{\vec{j},\sigma} \rangle_{H_{Z_2}} = 
\langle  \hat{c}^{\dagger}_{\vec{i},\sigma} \hat{c}^{\phantom{\dagger}}_{\vec{j},\sigma} \rangle_{H} 
\end{equation}
as obtained  from  the \texttt{Hamiltonian\_Examples.f90} and    \texttt{Hamiltonian\_Z2\_slave\_spins.f90 } codes.  
A test run for the $8\times 8 $ lattice at $U/t = 4$ and $\beta t = 40$ gives: 
\begin{center}
\begin{tabular}{ l | c | r }
 \hline			
   k   &  $\langle n_k \rangle_{H} $  &  $\langle n_k \rangle_{H_{Z_2}} $ \\
  \hline
   (0,0)                               & 1.93348548    $\pm$    0.00011322  & 1.93336807   $\pm$      0.00080473 \\
   ($\pi/4$,$\pi/4$)             & 1.90120688     $\pm$   0.00014854  & 1.90107164    $\pm$     0.00097029  \\
   ($\pi/2$,$\pi/2$)             & 0.99942957     $\pm$   0.00091377  & 1.00000000    $\pm$     0.00000000\\
   ($3\pi/4$,$3\pi/4$)         &  0.09905425     $\pm$   0.00015940 & 0.09892836    $\pm$     0.00097029 \\
   ($\pi$,$\pi$)                   & 0.06651452     $\pm$   0.00011321  & 0.06663193     $\pm$    0.00080473 \\
  \hline  
\end{tabular}

\end{center} 
\vspace*{0.5cm}
Here a Trotter time step of  $\Delta \tau t = 0.05$ was used  so as to minimize the systematic error   which should be different  between the two codes. 

\newpage
% Copyright (c) 2016 2017 The ALF project.
% This is a part of the ALF project documentation.
% The ALF project documentation by the ALF contributors is licensed
% under a Creative Commons Attribution-ShareAlike 4.0 International License.
% For the licensing details of the documentation see license.CCBYSA.
% !TEX root = doc.tex
\subsubsection{$Z_2$ gauge theory coupled to $Z_2$ matter.   }
\label{Z2.Sec}
The Hamiltonian we will consider here reads
\begin{align}
	\hat{H} =& -  t_{Z_2} \sum_{\langle \vec{i}, \vec{j} \rangle, \sigma } \hat{Z}_{\langle \vec{i}, \vec{j} \rangle}
	\left(\hat{\Psi}^{\dagger}_{\vec{i},\sigma} \hat{\Psi}^{\phantom{\dagger}}_{\vec{j},\sigma}   + h.c. \right) - \mu \sum_{\vec{i},\sigma} \hat{\Psi}^{\dagger}_{\vec{i},\sigma} \hat{\Psi}^{\phantom{\dagger}}_{\vec{i},\sigma}  
	-g \sum_{\langle \vec{i}, \vec{j} \rangle } \hat{X}_{\langle \vec{i}, \vec{j} \rangle }  +
	  K \sum_{\square} \prod_{\langle \vec{i}, \vec{j} \rangle \in \partial \square} \hat{Z}_{\langle \vec{i}, \vec{j} \rangle}  \nonumber \\
	& + J  \sum_{\langle \vec{i}, \vec{j} \rangle}  \hat{\tau}^z_{\pmb{i}}  \hat{Z}_{\langle \vec{i}, \vec{j} \rangle} \hat{\tau}^z_{\pmb{j}}   
	      -  h \sum_{ \vec{i} } \hat{\tau}^x_{\vec{i}}   - t  \sum_{\langle \vec{i}, \vec{j} \rangle, \sigma }   \hat{\tau}^z_{\pmb{i}}   \hat{\tau}^z_{\pmb{j}}  \left( \hat{\Psi}^{\dagger}_{\vec{i},\sigma} \hat{\Psi}^{\phantom{\dagger}}_{\vec{j},\sigma} 	+ h.c. \right)
\end{align}  
Here the  $\hat{\Psi}^{\dagger}_{\vec{i},\sigma}$  creates an orthogonal fermion with $Z_2$ and  and electric charges.    
The implementation of this Hamiltonian can be found in the file \texttt{Hamiltonian\_Z2\_Matter\_mod.F90}.
 For this Hamiltonian, the the $Z_2$ local conservation law reads: 
\begin{equation}
	\hat{Q}_{\vec{i}} =  (-1)^{\sum_{\sigma} \hat{\Psi}^{\dagger}_{\vec{i},\sigma} \hat{\Psi}^{\phantom{\dagger}}_{\vec{j},\sigma}   } 
	\;  \hat{\tau}^{x}_{\vec{i}}  \; \hat{X}_{\vec{i},\vec{i} +  \vec{a}_x} \hat{X}_{\vec{i},\vec{i} -  \vec{a}_x} \hat{X}_{\vec{i},\vec{i} +  \vec{a}_y} \hat{X}_{\vec{i}}.
\end{equation} 

The Hamiltonian was investigated in Ref.~\cite{Gazit19}.   Here is a todo list. 
\begin{itemize}
\item  Include an attractive $U$-term. This breaks the O(4) symmetry down to SU(2)$\times$SU(2)   and selects the $\hat{Q}_{\vec{i}} =1 $ sector.   Note that a repulsive U should  can also be included and should produce equivalent results under particle-hole symmetry.
\item  Include dynamics, so as to study the dynamics of the OSM to  FL* phase. Note that the FL* phase will ultimately be unstable to the AFM* phase, but the energy scale is expected to be extremely small at small U.  
\item  Include a projective version, with different left and right wave functions. The right imposes translation invariance and  the left  the constraint.  That is,  we choose the right trial wave function to be the ground state of 
\begin{equation}
	\hat{H}_T^{R}  = -  t  \sum_{\langle \vec{i}, \vec{j} \rangle, \sigma } 
	  \left( \hat{\Psi}^{\dagger}_{\vec{i},\sigma} \hat{\Psi}^{\phantom{\dagger}}_{\vec{j},\sigma}    + h.c. \right)  
	  - h \sum_{\vec{i} } \left( \hat{X}_{\vec{i},\vec{i} +  \vec{a}_x }+  \hat{X}_{\vec{i},\vec{i} +  \vec{a}_y}  + \tau^{x}_{\vec{i}} \right)
\end{equation}
and the left one to be the ground state of
\begin{equation}
	\hat{H}_T^{L}  =  - U  \sum_{ \vec{i},  \sigma } 
	    e^{i \vec{Q} \cdot \vec{i} } \hat{\Psi}^{\dagger}_{\vec{i},\sigma} \hat{\Psi}^{\phantom{\dagger}}_{\vec{i},\sigma}  
	  - h \sum_{\vec{i} } \left( \hat{X}_{\vec{i},\vec{i} +  \vec{a}_x }+  \hat{X}_{\vec{i},\vec{i} +  \vec{a}_y}  + \tau^{x}_{\vec{i}} \right)
\end{equation}
with $\vec{Q} = ( \pi,\pi ) $. 
\end{itemize}  
% Copyright (c) 2016-2019 The ALF project.
% This is a part of the ALF project documentation.
% The ALF project documentation by the ALF contributors is licensed
% under a Creative Commons Attribution-ShareAlike 4.0 International License.
% For the licensing details of the documentation see license.CCBYSA.
% !TEX root = Doc.tex
\subsubsection{ Generalized t-V model }

The example code \texttt{Hamiltonian\_Examples.f90}   contains  an implementation of the SU(N)  t-V model, given by: 
\begin{equation}
	  \hat{H}_{tV} =   
	   -  t  \sum_{\langle \vec{i}, \vec{j} \rangle}   \sum_{\sigma =1}^{N} 
	  \left( \hat{c}^{\dagger}_{\vec{i},\sigma} \hat{c}^{\phantom{\dagger}}_{\vec{j},\sigma}    + h.c. \right)    -\frac{V}{N}  \sum_{\langle \vec{i}, \vec{j} \rangle} 
	   \left( \sum_{\sigma =1}^{N} 
	  \left( \hat{c}^{\dagger}_{\vec{i},\sigma} \hat{c}^{\phantom{\dagger}}_{\vec{j},\sigma}    + h.c. \right)  \right)^2. 
\end{equation}
This model  posses a higher O(2N) symmetry and shows no sign problem  for all values of N.   The model  is implemented on the $\pi$-flux and  square lattices. Tests are included in the test suite git \texttt{Testsuite\_General\_QMCT\_code}.



% Copyright (c) 2016 2017 The ALF project.
% This is a part of the ALF project documentation.
% The ALF project documentation by the ALF contributors is licensed
% under a Creative Commons Attribution-ShareAlike 4.0 International License.
% For the licensing details of the documentation see license.CCBYSA.
% !TEX root = Doc.tex
\subsubsection{  Long range Coulomb }

The model we consider here reads: 
\begin{equation}
	\hat{H}  = -t \sum_{\vec{i},\vec{j},\sigma=1}^{N}   \hat{c}^{\dagger}_{\vec{i},\sigma}  T_{\vec{i},\vec{j}} \hat{c}^{}_{\vec{j},\sigma}   +     
	\frac{1} { N } \sum_{\vec{i},\vec{j}}  \left(  \hat{n}_{\vec{i}} -  \frac{N}{2}  \right)  V_{\vec{i},\vec{j}} \left(  \hat{n}_{\vec{j}} -  \frac{N}{2}  \right) \; \; 
	\text{ with }  \hat{n}_{\vec{i}} = \sum_{\sigma=1}^{N}  \hat{c}^{\dagger}_{\vec{i},\sigma}  \hat{c}^{}_{\vec{i},\sigma}
\end{equation}
The interaction is specified in the following way: 
\begin{equation}
	V_{\vec{i}, \vec{j}}   =   U \left\{
	\begin{array}{ll}  
	1          &   \text{ if } \vec{i} - \vec{j}    = 0 \\
	\frac{\alpha   \;   d_{min}}{ |   \vec{i} - \vec{j} | } &     \text{ otherwise }
	\end{array}
\right.
\end{equation}
Here $d_{min}$ is the minimal distance between two orbitals.     The code uses the following  HS decomposition:
\begin{equation}
e^{-\Delta \tau \hat{H}_V }  =  \int \prod_{\vec{i}} d \phi_{\vec{i}}   e^{ - \frac{N \Delta \tau} {4} \phi_{\pmb{i}} V^{-1}_{\pmb{i},\pmb{j}}  \phi_{\pmb{j}} - \sum_{\pmb{i}}  i \Delta \tau \phi_i \left( n_{i} - \frac{N}{2} \right) } 
\end{equation}

The implementation follows Ref.~\cite{Hohenadler14}  but now supports various lattice geometries. 

% Copyright (c) 2016-2019 The ALF project.
% This is a part of the ALF project documentation.
% The ALF project documentation by the ALF contributors is licensed
% under a Creative Commons Attribution-ShareAlike 4.0 International License.
% For the licensing details of the documentation see license.CCBYSA.
% !TEX root = Doc.tex
\subsection{  Model Classes }

I suggest the following for the organization of the models.    Five independent Hamiltonian   files would suffice.


\subsubsection{Hubbard models   \texttt{tU\_mod.F90}}

\begin{equation}
    \sum_{\vec{i},\vec{j},\sigma=1}^{N}  \hat{c}^{\dagger}_{\vec{i},\sigma } T_{\vec{i},\vec{j}} \hat{c}^{\phantom\dagger}_{\vec{i},\sigma }     +  \frac{U}{N} \sum_{\vec{i}} \left(\sum_{\sigma=1}^{N}  \left[   \hat{c}^{\dagger}_{\vec{i},\sigma } 
    \hat{c}^{\phantom\dagger}_{\vec{i},\sigma }  - 1/2  \right] \right)^2 
\end{equation}
This Hamiltonian would
\begin{itemize} 
\item support   square,  honeycomb,  $\pi$-flux.  It would be very nice to have bilayer versions of these lattices as well. 
\item Breakdown of the $U(2N)$ symmetry to $U(N) \times U(N)$ so as to accommodate magnetic fields and pinning fields.  
\end{itemize}


\subsubsection{Hubbard models   \texttt{tV\_mod.F90}}

This would include the $SU(N)$  $t-V$ models on various lattices.  The defining property of this set of Hamiltonians would be the enlarged O(2N) symmetry.  Again  this   module should support our standard bipartitie lattices. 


\subsubsection{Hubbard models   \texttt{LRC\_mod.F90}}

This is the long range Coulomb. See above.    Again we should include the   standard lattices. 

\subsubsection{Hubbard models   \texttt{Z2\_mod.F90}}
I suggest to work on the  Hamiltonian of Ref.~\ref{Z2.Sec} since this is the most general  model I can think of.   Would be nice to add a Hubbard-$U$ term.  It is actually not so easy to generalize this model to 
arbitrary lattices, so that for the moment, I would concentrate only on the square lattice. 

\subsubsection{Hubbard models   \texttt{Kondo\_mod.F90}}
Kondo lattice model on various lattices.  I still have to think about the best way of doing things here. 
% Copyright (c) 2016 The ALF project.
% This is a part of the ALF project documentation.
% The ALF project documentation by the ALF contributors is licensed
% under a Creative Commons Attribution-ShareAlike 4.0 International License.
% For the licensing details of the documentation see license.CCBYSA.
% !TEX root = Doc.tex
%-------------------------------------------------------------------------------------
\subsection{Performance, memory requirements and parallelization}
%-------------------------------------------------------------------------------------


As mentioned in the  introduction, the auxiliary field QMC algorithm scales linearly in inverse temperature $\beta$ and cubic in the volume $N_{\text{dim}}$. Using fast updates,  a single spin flip  requires $(N_{\text{dim}})^2$ operations to update the Green function upon acceptance.  As there are $L_{\text{Trotter}}\times N_{\text{dim}}$ spins to be visited, the total computational cost for one sweep is of the order of $\beta (N_{\text{dim}})^3$. This operation  dominates the performance, see Fig.~\ref{fig_scaling_size}. A profiling analysis of our code shows that 80-90\% of the CPU time is spend in ZGEMM calls of the BLAS library provided in the MKL package by Intel. Consequently, the single-core performance is next to optimal.

\begin{figure}[h]
	\begin{center}
		\includegraphics[scale=.8]{Figures/Size_scaling_ALF_2.pdf}
	\end{center}
	\caption{\label{fig_scaling_size}Volume scaling behavior of the auxiliary field QMC code of the ALF project on SuperMUC (phase 2/Haswell nodes) at the LRZ in Munich. The number of sites $N_{\text{dim}}$ corresponds to the system volume.
	The plot confirms that the leading scaling order is due to matrix multiplications such that the runtime is dominated by calls to ZGEMM. }
\end{figure}

For the implementation which scales linearly in $\beta$, one has to store $L_{\text{Trotter}}/\texttt{NWrap}$ intermediate propagation matrices of dimension $N\times N$. For large lattices and/or low temperatures this dominates the total memory requirements that can exceed 2~GB memory for a sequential version.

At the heart of Monte Carlo schemes lies a random walk through the given configuration space. This is easily parallalized via MPI by associating one random walker to each MPI task. For each task, we start from a random configuration and have to invest the autocorrelation time $T_\mathrm{auto}$ to produce an equilibrated configuration.
Additionally we can also profit from an OpenMP parallelized version of the BLAS/LAPACK library for an additional speedup, which also effects equilibration overhead $N_\text{MPI}\times T_\text{auto} / N_\text{OMP}$, where $N_{\text{MPI}}$ is the number of cores and $N_{\text{OMP}}$ the number of OpenMP threads.
For a given number of independent measurements  $N_\text{meas}$, we  therefore need a wall-clock time given by
\begin{equation}\label{eqn:scaling}
T  =  \frac{T_\text{auto}}{N_\text{OMP}} \left( 1   +    \frac{N_\text{meas}}{N_\text{MPI}}  \right) \,.
\end{equation}
As we typically have $ N_\text{meas}/N_\text{MPI} \gg 1 $, 
the speedup is expected to be almost perfect, in accordance with
the performance test results for the auxiliary field
QMC code  on SuperMUC (see Fig.~\ref{fig_scaling} (left)).

For many problem sizes, 2~GB memory per MPI task (random walker) suffices such that we typically start as many MPI tasks as there are physical cores per node. Due to the large amount of CPU time spent in MKL routines, we do not profit from the hyper-threading option. For large systems, the memory requirement increases and this is tackled by increasing the amount of OpenMP threads to decrease the stress on the memory system and to simultaneously reduce the equilibration overhead (see Fig.~\ref{fig_scaling} (right)). For the displayed speedup, it was crucial to pin the MPI tasks as well as the OpenMP threads in a pattern which keeps the threads as compact as possible to profit from a shared cache. This also explains the drop in efficiency from 14 to 28 threads where the OpenMP threads are spread over both sockets. 

We store the field configurations of the random walker as checkpoints, such that a long simulation can be easily split into several short simulations. This procedure allows us to take advantage of chained jobs using the dependency chains provided by the batch system.

\begin{figure}[H]
	\begin{center}
		\includegraphics[scale=0.6]{Figures/MPI_scaling_ALF_2.pdf}
		\includegraphics[scale=0.6]{Figures/OMP_scaling_ALF_2.pdf}
	\end{center}
	\caption{\label{fig_scaling} MPI (left) and OpenMP (right) scaling behavior of the auxiliary field QMC code of the ALF project on SuperMUC (phase 2/Haswell nodes) at the LRZ in Munich.
		The MPI performance data was normalized to 28 cores and was obtained using a problem size of $N_{\text{dim}}=400$. This is a medium to small system size that is the least favorable in terms of MPI synchronization effects.
		The OpenMP performance data was obtained using a problem size of $N_{\text{dim}}=1296$. Employing 2 and 4 OpenMP threads introduces some synchronization/management overhead such that the per-core performance is slightly reduced, compared to the single thread efficiency. Further increasing the amount of threads to 7 and 14 keeps the efficiency constant. The drop in performance of the 28 thread configuration is due to the architecture as the threads are now spread over both sockets of the node. To obtain the above results, it was crucial to pin the processes in a fashion that keeps the OpenMP threads as compact as possible.}
\end{figure}

%Next to the entire computational time is spent in BLAS routines such that the performance of the code will depend on the particular  implementation of this library. 
%We have found that the code performs well, and that  an efficient  OpenMP  version of the library  can be obtained merely by   loading the corresponding BLAS and LAPACK routines. 
%\mycomment{MB: Do we want to say more about OpenMP here, i.e. that it can be useful when warm-up time is a problem (and getting many CPUs is not). 
%In all other cases, the MPI parallelization is always better than the trivial OpenMP parallelization of library algos.}

\section{Conclusions and Future Directions}\label{sec:con}
% Copyright (c) 2016 The ALF project.
% This is a part of the ALF project documentation.
% The ALF project documentation by the ALF contributors is licensed
% under a Creative Commons Attribution-ShareAlike 4.0 International License.
% For the licensing details of the documentation see license.CCBYSA.
% !TEX root = doc.tex

In its present form, the  auxiliary-field QMC code of the ALF project allows us to simulate a large class of non-trivial models, both efficiently and at minimal  programming cost.  ALF 2.0 contains many advanced functionalities, including a projective formulation, various updating schemes, better control of Trotter errors, predefined structures that facilitate reuse, a large class of models, continuous fields and, finally, stochastic analytical continuation code. Also the usability of the code has improved in comparison with ALF 1.0. In particular the \href{https://git.physik.uni-wuerzburg.de/ALF/pyALF}{pyALF} project provides a Python interface to the ALF which substantially facilitates running the code for established models.  This ease of use renders ALF 2.0 a  powerful  tool to for benchmarking new algorithms. 

There are further capabilities that we would like to see in future versions of ALF. Introducing time-dependent Hamiltonians, for instance, will require some rethinking, but will allow, for example, to access entanglement properties of interacting fermionic systems \cite{Broecker14,Assaad13a,Assaad15}. Moreover, the auxiliary field approach is not the only method to simulate fermionic systems.
It would be desirable to include additional lattice fermion algorithms such as the CT-INT \cite{Rubtsov05,Assaad07}.
Lastly, at the more technical level, improved IO (e.g., HDF5 support), post-processing, object oriented programming, as well as increased compatibility with other software projects are all certainly improvements to look forward to. 

\addcontentsline{toc}{section}{Acknowledgments}
% Copyright (c) 2016 The ALF project.
% This is a part of the ALF project documentation.
% The ALF project documentation by the ALF contributors is licensed
% under a Creative Commons Attribution-ShareAlike 4.0 International License.
% For the licensing details of the documentation see license.CCBYSA.
% !TEX root = Doc.tex
%-------------------------------------------------------------------------------------
\section*{Acknowledgments} 
%-------------------------------------------------------------------------------------

We are very grateful to  S.~Beyl, M.~Hohenadler,  F.~Parisen Toldin,  M.~Raczkowski, T.~Sato, J.~Schwab, Z.~Wang, and M.~Weber  for constant support during the development of this project. 
\mycomment{We equally thank G.~Hager, M.~Wittmann, and G.~Wellein for useful discussions and support.}
FFA would also like to thank T.~Lang   and Z.~Y.~Meng for  developments of the auxiliary field code as well as T.~Grover. 
MB thanks the Bavarian Competence Network for Technical and Scientific High Performance Computing (KONWIHR) for financial support. FG  and JH thank the SFB-1170 for  financial support under projects Z03 and C01.  FFA thanks the DFG-funded FOR1807 and FOR1346 for partial financial support.
Part of the optimization of the code was carried out during  the  Porting and Tuning Workshop 2016 offered by the Forschungszentrum J\"ulich.
Calculations  to extensively test this package were carried out both on  SuperMUC at the  Leibniz Supercomputing Centre and on  JURECA  \cite{Jureca16} at the J\"ulich Supercomputing Centre.  We thank both institutions for generous allocation of computing time.
 %The authors gratefully acknowledge the computing time granted by the John von Neumann Institute for Computing (NIC) and provided on the supercomputer JURECA \cite{Jureca16} at Jülich Supercomputing Centre (JSC). The authors gratefully acknowledge the Gauss Centre for Supercomputing e.V. (www.gauss-centre.eu) for funding this project by providing computing time on the GCS Supercomputer SuperMUC at the Leibniz Supercomputing Centre (LRZ, www.lrz.de).
\addcontentsline{toc}{section}{References}
\bibliographystyle{./prXsty} 
\bibliography{./fassaad,./doc}
\addcontentsline{toc}{section}{License}
% Copyright (c) 2016 The ALF project.
% This is a part of the ALF project documentation.
% The ALF project documentation by the ALF contributors is licensed
% under a Creative Commons Attribution-ShareAlike 4.0 International License.
% For the licensing details of the documentation see license.CCBYSA.

% !TEX root = Doc.tex
%-------------------------------------------------------------------------------------
\section*{License}
%-------------------------------------------------------------------------------------
When we were discussing how to make the ALF code generally available to the world we quickly
agreed that it should be open source so that people can benefit and we have the the hope that the code 
proves useful to others and makes some contribution to the scientific community.
Nevertheless we are all scientists and we have to make ends meet in our careers. Therefore we felt that the scientific practice 
of giving a citation back if one has benefitted from another person's work is something that we felt we can reciprocally hope for, from our users.
To facilitate the communication with our users we have set up our project's homepage \url{alf.physik.uni-wuerzburg.de}
and we hope that it gives us the tools to create a small but vibrant community around the code and provides a suitable
entrypoint for future contributors.
The homepage is also the place where the original source files can be found.
With the coming public release it was necessary to add copyright headers to our source files and to think about the licensing
of our software and therefore the question was on the table of how to make those ideas part of our licensing scheme.
We felt that the Creative Commons licenses are a good way to share our documentation and it is also
accepted well with publishers. Therefore this documentation is licensed to you under a CC-BY-SA license.
This means you can share it and redistribute it as long as you cite the original source and
license your changes under the same license. The details are in the file license.CCBYSA that you shou have received with this documentation.
The source code itself is licensed under a GPL license to keep the source as well as any future work in the community.
To express our desire for a proper attribution we decided to make this a visible part of the license.
To that end we have exercised the rights of section 7 of GPL version 3 and have amended
the license terms with an additional paragraph that expresses our wish that if an author has benfitted from this code
that he/she should consider giving back a citation as specified on \url{alf.physik.uni-wuerzburg.de}.
This is not something that is meant to restrict your freedom of use, but something that we strongly expect to be good scientific conduct.
The original GPL license can be found in the file license.GPL and the additional terms can be found in license.additional.
In favour to our users, \textit{ALF} contains part of the lapack implementation version 3.6.1 from \url{http://www.netlib.org/lapack}.
Lapack is licensed under the modified BSD license whose full text can be found in license.BSD.\\
With that being said, we hope that ALF will prove to you to be a suitable and highly performant tool that enables
you to perform Monte Carlo studies of solid state models of unprecedented complexity.\\
\\
The ALF project's contributors.\\
                        
%-------------------------------------------------------------------------------------
\subsection*{COPYRIGHT}
%-------------------------------------------------------------------------------------

Copyright \textcopyright ~2016, The \textit{ALF} Project.\\
The ALF Project Documentation 
is licensed under a Creative Commons Attribution-ShareAlike 4.0 International License.
You are free to share and benefit from this documentation as long as this license is preserved
and proper attribution to the authors is given. For details see the ALF project
homepage \url{alf.physik.uni-wuerzburg.de} and the file \texttt{license.CCBYSA}.


\end{document}
