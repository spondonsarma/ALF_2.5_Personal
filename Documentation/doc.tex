% !TeX document-id = {2fc16d90-fe07-4ec8-91d5-5acebdd2ee85}
% !BIB TS-program = bibtex

% Copyright (c) 2016-2022 The ALF project.
% This is the ALF project documentation.
% The ALF project documentation by the ALF contributors is licensed
% under a Creative Commons Attribution-ShareAlike 4.0 International License.
% For the licensing details of the documentation see license.CCBYSA.

\documentclass[submission, PhysCodeb]{SciPost}
\pdfoutput=1
\usepackage{lineno}
\usepackage{graphicx}
\graphicspath{{./Figures/}{./Figures/Kondo/}{./Figures/MaxEnt/}{./Figures/MaxEnt/}{./Figures/Projector/}{./Figures/PAM/}{./Figures/Dtau/}{./Figures/Dtau_1/}}
\usepackage[T1]{fontenc}
\usepackage{lmodern}

\usepackage{bbm}
\newcommand{\bmmax}{1}
\newcommand{\hmmax}{1}
\usepackage{amsmath,mathtools} % multilined
%%\usepackage{amssymb}
\usepackage{wasysym}
\usepackage{color}
\usepackage{xcolor}
\usepackage{tcolorbox}
\usepackage{soul}
\usepackage{multirow}
\usepackage{xspace}
\usepackage{bm}
\usepackage{dsfont}
\usepackage[htt]{hyphenat}
\usepackage{footnote}
\usepackage{cite}
\usepackage{alltt}
\usepackage{listings}
\usepackage{enumitem}
\usepackage{algorithm}
\usepackage{algcompatible}
\usepackage{algpseudocode}
\usepackage{url}
\usepackage{booktabs}
\hypersetup{breaklinks=true, colorlinks, pdfborder={0 0 0}, linkcolor={red!50!black}, citecolor={blue!50!black}, urlcolor={blue!80!black}}
\usepackage{float} 
\usepackage{placeins}
\usepackage{makeidx}
\usepackage[xindy]{imakeidx}
\usepackage[framemethod=default]{mdframed}
\usepackage{showexpl}
\usepackage{etoolbox}
\usepackage{calc}
\usepackage{dirtree}
\usepackage[bitstream-charter]{mathdesign}
\urlstyle{sf}

% Fix \cal and \mathcal characters look (so it's not the same as \mathscr)
\DeclareSymbolFont{usualmathcal}{OMS}{cmsy}{m}{n}
\DeclareSymbolFontAlphabet{\mathcal}{usualmathcal}


% Prevent all line breaks in inline equations.
\binoppenalty=10000
\relpenalty=10000

\makeatletter
\renewcommand\paragraph{\@startsection{paragraph}{4}{\z@}%
            {-2.5ex\@plus -1ex \@minus -.25ex}%
            {1.25ex \@plus .25ex}%
            {\normalfont\normalsize\bfseries}}
\makeatother
\makeatletter
\newcommand{\vast}{\bBigg@{3}}
\newcommand{\Vast}{\bBigg@{4}}
\newcommand{\VAST}{\bBigg@{5}}
\makeatother

\setcounter{secnumdepth}{4} 
\setcounter{tocdepth}{4}
\definecolor{light-gray}{gray}{0.95}
\definecolor{light-gray2}{gray}{0.88}
\definecolor{comment-color}{rgb}{0.8,0.1,0.1}
\definecolor{keyword-color}{rgb}{0.3,0.3,1}
\sethlcolor{light-gray2}
\newcommand*{\red}{\textcolor{red}}
\newcommand*{\blue}{\textcolor{blue}}


\lstdefinestyle{fortran}{
  language={[03]Fortran},
  basicstyle=\ttfamily\footnotesize,
  keywordstyle=\color{keyword-color},
  commentstyle=\color{comment-color},
  morecomment=[l]{!\ },% Otherwise it's a comment only with space after !
  breakatwhitespace=false,
  keepspaces=true,
  showstringspaces=false,
  columns=fullflexible,
  backgroundcolor=\color{light-gray},
  frame=single,
  xleftmargin=3.4pt,
  xrightmargin=3.4pt
}

\lstdefinestyle{fortran_pseudo_code}{
  language={[03]Fortran},
  basicstyle=\ttfamily\small,
  keywordstyle=\color{red},
  commentstyle=\normalfont\color{black},
  morecomment=[l]{!\ }% Comment only with space after !
  breakatwhitespace=false,
  keepspaces=true,
  showstringspaces=false,
  columns=flexible,
  backgroundcolor=\color{white},
  frame=single,
  xleftmargin=3.4pt,
  xrightmargin=3.4pt,
  escapeinside={\!*}{*)},
}

\lstdefinestyle{bash}{
  language=bash,
  basicstyle=\ttfamily,
  keywordstyle=\color{keyword-color},
  commentstyle=\color{comment-color},
  morecomment=[l]{\#\ }% Comment only with space after #
  breakatwhitespace=false,
  keepspaces=true,
  showstringspaces=false,
  columns=flexible,
  xleftmargin=3.4pt,
  xrightmargin=3.4pt,
}

\newcommand{\ALFver}{2.5}           %{dev.} version of ALF this document refers to
\newcommand{\ALFbranch}{ALF-2.5}    %{ALF-2.2}{master} corresponding ALF branch
\newcommand{\pyALFver}{2.5}         %{dev.} version of pyALF that works with \ALFver
\newcommand{\pyALFbranch}{ALF-2.5}  %{ALF-2.2}{master} corresponding pyALF branch
\newcommand{\tutALFver}{2.5}        %{dev} version of the tutorial - normally should be the same as \ALFver (ex.: 2.2, dev [NO PERIOD AFTER "DEV", SINCE IT'S ALSO USED IN PATHS])

\def\Tr{\mathop{\mathrm{Tr}}}
\def\Trf{\mathop{\mathrm{Tr}_{\mathrm{F}}}}
\def\d{\mathrm{d}}
\def\hc{\mathrm{H.c.}}
\DeclareMathOperator{\sgn}{sgn}
\makesavenoteenv{tabular}
\makesavenoteenv{table}

\renewcommand{\Re}{\operatorname{Re}}
\renewcommand{\Im}{\operatorname{Im}}
\renewcommand{\vec}[1]{\boldsymbol{#1}}
\newcommand{\ve}[1]{\boldsymbol{#1}}
\newcommand{\tl}{t_{l}}
\newcommand{\ts}{\underline{t}}
\newcommand{\Ztwo}{\mathbf{Z}_{2}}

%------------------------ algpseudocode additions -----------------------------------

\algrenewcommand\algorithmiccomment[1]{\hfill \textcolor{comment-color}{\small  $\triangleright$ \textit{#1}}}
\algrenewcommand\alglinenumber[1]{\scriptsize #1:}

%\algnewcommand{\IIf}[1]{\State\algorithmicif\ #1\ \algorithmicthen} % https://tex.stackexchange.com/a/184228/105662
%\algnewcommand{\IElse}[1]{\State\algorithmicelse\ #1 }
%\algnewcommand{\EndIIf}{\unskip\ \algorithmicend\ \algorithmicif}

\makeatletter % Based on  https://tex.stackexchange.com/a/169608/105662
\newlength{\trianglerightwidth}
\settowidth{\trianglerightwidth}{$\triangleright$~}
\algnewcommand{\LineComment}[1]{\Statex \hskip\ALG@thistlm%
	\parbox[t]{\dimexpr\linewidth-\ALG@thistlm}{\hangindent=\trianglerightwidth \hangafter=1 \strut\textcolor{comment-color}{\small $\triangleright$ \textit{#1}}\strut}}
% If comment line is the first of a block:
\algnewcommand{\FirstLineComment}[1]{\Statex \hskip\ALG@thistlm%
	\parbox[t]{\dimexpr\linewidth-\ALG@thistlm}{\hangindent=\dimexpr\algorithmicindent+\trianglerightwidth \hangafter=1 \strut\makebox[\algorithmicindent][l]{}\textcolor{comment-color}{\small $\triangleright$ \textit{#1}}\strut}}
\makeatother

\makeatletter % https://tex.stackexchange.com/a/428081/105662
\newenvironment{breakablealgorithm}
{% \begin{breakablealgorithm}
	\begin{center}
		\refstepcounter{algorithm}% New algorithm
		\hrule height.8pt depth0pt \kern2pt% \@fs@pre for \@fs@ruled
		\renewcommand{\caption}[2][\relax]{% Make a new \caption
			{\raggedright\textbf{\ALG@name~\thealgorithm} ##2\par}%
			\ifx\relax##1\relax % #1 is \relax
			\addcontentsline{loa}{algorithm}{\protect\numberline{\thealgorithm}##2}%
			\else % #1 is not \relax
			\addcontentsline{loa}{algorithm}{\protect\numberline{\thealgorithm}##1}%
			\fi
			\kern2pt\hrule\kern2pt
		}
	}{% \end{breakablealgorithm}
		\kern2pt\hrule\relax% \@fs@post for \@fs@ruled
	\end{center}
}
\makeatother

% vertical rules for blocks: None of the solutions found work with all of the: page breaks, redefined comments and added vertical spaces. The solution at  https://tex.stackexchange.com/a/351363/105662  and the 1st solution at  https://tex.stackexchange.com/a/147751/105662  appear to break the least, but vertical rules are full of gaps at those special spots.

%------------------------------------------------------------------------------------


\makeindex

\begin{document}
	
% Title
\begin{center}{\Large \textbf{
The \emph{ALF} (\emph{A}lgorithms for \emph{L}attice \emph{F}ermions)\\project release \ALFver\\
{\normalsize Documentation for the  auxiliary-field quantum Monte Carlo code}
}}\end{center}

% Author list
\begin{center}
The \textbf{ALF Collaboration\textsuperscript{$\star$}}:
F.~F.~Assaad\textsuperscript{1,4},
M.~Bercx\textsuperscript{1},
F.~Goth\textsuperscript{1},
A.~G\"otz\textsuperscript{1},
J.~S.~Hofmann\textsuperscript{2},
E.~Huffman\textsuperscript{3},
Z.~Liu\textsuperscript{1},
F.~Parisen~Toldin\textsuperscript{1},
J.~S.~E.~Portela\textsuperscript{1},
J.~Schwab\textsuperscript{1}
\end{center}

% Affiliation
\begin{center}
{\bf 1} Institut f\"ur Theoretische Physik und Astrophysik, Universit\"at W\"urzburg,\\
97074 W\"urzburg, Germany\\
{\bf 2} Department of Condensed Matter Physics, Weizmann Institute of Science,\\
Rehovot, 76100, Israel\\
{\bf 3} Perimeter Institute for Theoretical Physics,\\
Waterloo, Ontario N2L 2Y5, Canada\\
{\bf 4} W\"urzburg-Dresden Cluster of Excellence ct.qmat, \\ Am Hubland, 97074 W\"urzburg, Germany

${}^\star$\href{mailto:alf@physik.uni-wuerzburg.de}{\small \sf alf@physik.uni-wuerzburg.de}
\end{center}

\begin{center}
\today
\end{center}

% For convenience during refereeing: line numbers
%\linenumbers

\section*{Abstract}
{\bf
The \emph{Algorithms for Lattice Fermions} package provides a general code for the finite-temperature and projective auxiliary-field quantum Monte Carlo algorithm. The code is engineered to be able to simulate any model that can be written in terms of sums of single-body operators, of squares of single-body operators and single-body operators coupled to a bosonic field with given dynamics.   The package includes five pre-defined model classes:  SU(N) Kondo, SU(N) Hubbard, SU(N) t-V and SU(N) models with long range Coulomb repulsion on honeycomb, square and N-leg lattices, as well as $\Ztwo$  unconstrained lattice gauge theories coupled to fermionic and $\Ztwo$ matter. An implementation of the stochastic  Maximum Entropy  method is also provided. 
One can download the code from our Git instance at \url{https://git.physik.uni-wuerzburg.de/ALF/ALF/-/tree/\ALFbranch} and sign in to file issues.} \\

% We provide predefined types that allow the user to specify the model, the Bravais lattice as well as equal-time and time-displaced observables. The code supports an MPI implementation. Examples such as the Hubbard model on the honeycomb lattice and the Hubbard model on the square lattice coupled to a transverse Ising field are provided and discussed in the documentation. We furthermore discuss how to use the package to implement the Kondo lattice model and the SU(N)-Hubbard-Heisenberg model. }\\

%\noindent Copyright \textcopyright ~2016--2022 The \textit{ALF} Project.\\
%This is the ALF Project Documentation. % by the ALF contributors.
%It is licensed under a Creative Commons Attribution-ShareAlike 4.0 International License.
%You are free to share and benefit from this documentation as long as this license is preserved
%and proper attribution to the authors is given. For details see the ALF project
%website \url{https://alf.physik.uni-wuerzburg.de}.

\vspace{10pt}
\noindent\rule{\textwidth}{1pt}
\tableofcontents\thispagestyle{fancy}
\noindent\rule{\textwidth}{1pt}
\vspace{10pt}

%\clearpage
\section{Introduction}\label{sec:intro}
% !TEX root = doc.tex
% Please do not remove this.
\section{Introduction}\label{sec:intro}
The auxiliary field quantum Monte Carlo approach has by now proven to be a key algorithm to simulate a variety of  electron systems where correlations effects play a dominant role.  This includes correlation effects in topological band structures, quantum phase transitions between semimetals and insulators, deconfined quantum critical points, topologically ordered phases, heavy fermion systems, nematic and magnetic quantum phase transitions in metals,   superconductivity in the presence of spin orbit coupling. This list of ever growing phenomena is based on  recent  symmetry based insights, which allows one to  find formulations that avoid the so called negative sign problem.   The aim of this project is to introduce a general formulation of the auxiliary methods, 

 \mycomment{placeholder for a general introduction, mentioning the purpose and the powers of the general QMC code.}
This documentation is organized as follows. The general Hamiltonian operator is written down in Sec.~\ref{sec:def}, followed by 
a brief outline  of the quantum Monte Carlo algorithm. 
In Sec.~\ref{sec:imp}, we discuss the implementation of a model, introducing the \texttt{Operator} data structure which is the building block of the Hamiltonian. And we disuss the implementation of the lattice and the observables.
Section ~\ref{sec:io} is about actually running the code. We describe input and output files, the analysis protocoll and the compilation procedure. 
In Sec.~\ref{sec:walk1} and \ref{sec:walk2} two detailed walkthroughs are performed: the $SU(2)$-symmetric Hubbard  on a square lattice (Sec.~\ref{sec:walk1}) and the same model, but additionally coupled to a transverse field Ising model (Sec.~\ref{sec:walk2}).



\section{Auxiliary Field Quantum Monte Carlo: finite temperature}\label{sec:def}
% !TEX root = Doc.tex
\section{Definition of the model Hamiltonian}\label{sec:def}

%\mycomment{Notation: Hats for second quantized operators, bold for matrices. \\
%  Structure:  \\ 1) We first want to define the model.  \\ 
 %                     2)  implementation of the QMC.  \\ 
 %                  3) Data structure  \\ 
%                   4) Practical implementation and some  simple test cases. }

 
The class of solvable models includes  Hamiltonians $\hat{\mathcal{H}}$ that have the following general form:
\begin{eqnarray}
\hat{\mathcal{H}}&=&\hat{\mathcal{H}}_{T}+\hat{\mathcal{H}}_{V} +  \hat{\mathcal{H}}_{I} +   \hat{\mathcal{H}}_{0,I}\;,\mathrm{where}
\label{eqn:general_ham}\\
\hat{\mathcal{H}}_{T}
&=&
\sum\limits_{k=1}^{M_T}
\sum\limits_{s=1}^{N_{\mathrm{fl}}}
\sum\limits_{\sigma=1}^{N_{\mathrm{col}}}
\sum\limits_{x,y}^{N_{\mathrm{dim}}}
\hat{c}^{\dagger}_{x \sigma   s}T_{xy}^{(k s)} \hat{c}^{\phantom\dagger}_{y \sigma s}  \equiv  \sum\limits_{k=1}^{M_T} \hat{T}^{(k)}
\label{eqn:general_ham_t}\\
\hat{\mathcal{H}}_{V}
&=&
-
\sum\limits_{k=1}^{M_V}U_{k}
\left\{
\sum\limits_{s=1}^{N_{\mathrm{fl}}}
\sum\limits_{\sigma=1}^{N_{\mathrm{col}}}
\left[
\left(
\sum\limits_{x,y}^{N_{\mathrm{dim}}}
\hat{c}^{\dagger}_{x \sigma s}V_{xy}^{(k s)}\hat{c}^{\phantom\dagger}_{y \sigma s}
\right)
-\alpha_{k s} 
\right]
\right\}^{2}  \equiv   -
\sum\limits_{k=1}^{M_V}U_{k}   \left(\hat{V}{(k)} \right)^2
\label{eqn:general_ham_v}\\
\hat{\mathcal{H}}_{I}  & = &
\sum\limits_{k=1}^{M_I} \hat{Z}_{k} 
\left\{
\sum\limits_{s=1}^{N_{\mathrm{fl}}}
\sum\limits_{\sigma=1}^{N_{\mathrm{col}}}
\left[
\sum\limits_{x,y}^{N_{\mathrm{dim}}}
\hat{c}^{\dagger}_{x \sigma s} I_{xy}^{(k s)}\hat{c}^{\phantom\dagger}_{y \sigma s}
\right]
\right\} \equiv \sum\limits_{k=1}^{M_I} \hat{Z}_{k}    \hat{I}^{(k)} 
\;.\label{eqn:general_ham_i}
\end{eqnarray}
The indices have the following meaning:
\begin{itemize}
\item The number of fermion \textit{flavors} is set by $N_{\mathrm{fl}}$.  After the Hubbard-Stratonovich transformation, the action will be block diagonal in the flavor index. 
\item The number of fermion \textit{colors} is set by $N_{\mathrm{col}}$.    The Hamiltonian is invariant under  SU($N_{\mathrm{col}}$)  rotations. Note that  in the code $ N_{\mathrm{col}} \equiv N_{sun} $. 
%\mycomment{ Does it set the symmetry group of the fermions, namely 
%the dimension of the special unitary group $SU(N_{sun})$?}
\item The indices $x,y$ label lattice sites where $x,y=1,\cdots, N_{\mathrm{dim}}$. 
$N_{\mathrm{dim}}$ is the total number of spacial vertices: $N_{\mathrm{dim}}=N_{unit\;cell} N_{orbital}$, where $N_{unit\;cell}$ is the number of unit cells of the underlying Bravais lattice and $N_{orbital}$ is the number of (spacial) orbitals per unit cell \mycomment{Check the definition of $N_{orbital}$ in the code.} 
\item Therefore, the  matrices $\bm{T}^{(k s)}$, $\bm{V}^{(ks)}$  and $\bm{I}^{(ks)}$ are  of dimension $N_{\mathrm{dim}}\times N_{\mathrm{dim}}$
\item The number of interaction terms  is labelled by $M_V$   and $M_I$.   $M_T> 1 $ would allow for a checkerboard decomposition. 
\item $\hat{Z}_k$ is an Ising spin operator which corresponds to the Pauli matrix $\hat{\sigma}_{z}$. It couples to a general one-body term. 
\item  $\mathcal{H}_{0,I}$  gives the dynamics of the Ising spins. 
This term has to be specified by the user and is only relevant when the Monte Carlo update probability is computed in the code (see Sec.~\ref{}).
%\mycomment{Be more general here and sreak of correlated blocks?}
\end{itemize}
Note that the matrices  $\bm{T}^{(ks)}$,  $\bm{V}^{(ks)}$ and  $\bm{I}^{(ks)}$ explicitly depend on the flavor index $s$ but not on the color index $\sigma$. 
The color index $\sigma$ only appears in  the  second quantized operators such that the Hamiltonian is manifestly SU($N_{\mathrm{col}}$)    symmetric.  We also require
the matrices $\bm{T}^{(ks)}$,  $\bm{V}^{(ks)}$ and  $\bm{I}^{(ks)}$  to be  hermitian.


\subsection{Formulation of the QMC}  
\subsubsection{The partition function}
The formulation of the  Monte Carlo simulation is based on the following.
\begin{itemize}
\item  We will discretize the imaginary time propagation: $\beta = \Delta \tau L_{\text{Trotter}} $
\item  We will use  the   discrete Hubbard-Stratonovich transformation:
\begin{equation}
\label{HS_squares}
        e^{\Delta \tau  \lambda  \hat{A}^2 } =
        \sum_{ l = \pm 1, \pm 2}  \gamma(l)
e^{ \sqrt{\Delta \tau \lambda }
       \eta(l)  \hat{A} }
                + {\cal O} (\Delta \tau ^4)\;,
\end{equation}
where the fields $\eta$ and $\gamma$ take the values:
\begin{eqnarray}
 \gamma(\pm 1)  = 1 + \sqrt{6}/3, \quad  \eta(\pm 1 ) = \pm \sqrt{2 \left(3 - \sqrt{6} \right)}\;,\\\nonumber
  \gamma(\pm 2) = 1 - \sqrt{6}/3, \quad  \eta(\pm 2 ) = \pm \sqrt{2 \left(3 + \sqrt{6} \right)}\;.
\nonumber
\end{eqnarray}
\item  We will work in  a basis  where  $\hat{Z}_k$ is diagonal: $\hat{Z}_{k}|s_{j}\rangle = s_{k}\delta_{kj}|s_{k}\rangle$, where $s_{k}=\pm 1$.
\item From the above it follows that the  Monte Carlo configuration space $C$  
is given by the combined spaces of Ising spin configurations  and of Hubbard-Stratonovich discrete field configurations:
\begin{equation}
	C = \left\{   s_{i,\tau} ,  l_{j,\tau}  \text{ with }  i=1\cdots M_I,\;  j = 1\cdots M_V,\; \tau=1\cdots L_{\mathrm{Trotter}}  \right\}
\end{equation}
Here, the Ising spins take the values  $s_{i,\tau} = \pm 1$ and  the Hubbard-Stratonovich fields take the values  $l_{j,\tau}  = \pm 2, \pm 1 $.
\end{itemize}
With the above, the partition function of the model (\ref{eqn:general_ham}) can be written as follows.
\begin{eqnarray}
Z &=& \Tr{\left(e^{-\beta \hat{\mathcal{H}} }\right)}\nonumber\\
  &=&   \Tr{  \left[ e^{-\Delta \tau \hat{\mathcal{H}}_{0,I}}   \prod_{k=1}^{M_T}   e^{-\Delta \tau \hat{T}^{(k)}}  
    \prod_{k=1}^{M_V}   e^{  \Delta \tau  U_k \left(  \hat{V}^{(k)} \right)^2}   \prod_{k=1}^{M_I}   e^{  -\Delta \tau  \hat{\sigma}_{k}  \hat{I}^{(k)}} 
   \right]^{L_{\text{Trotter}}}}  \nonumber \\
   &=&
   \sum_{C} \left( \prod_{j=1}^{M_V} \prod_{\tau=1}^{L_{\mathrm{Trotter}}} \gamma_{j,\tau} \right) e^{-S_{0,I} \left( \left\{ s_{i,\tau} \right\}  \right) }\times \nonumber\\
   &\quad&
    \Trf{ \left\{  \prod_{\tau=1}^{L_{\mathrm{Trotter}}} \left[   \prod_{k=1}^{M_T}   e^{-\Delta \tau \hat{T}^{(k)}}  
    \prod_{k=1}^{M_V}   e^{  \sqrt{ \Delta \tau  U_k} \eta_{k,\tau} \hat{V}^{(k)} }   \prod_{k=1}^{M_I}   e^{  -\Delta \tau s_{k,\tau}  \hat{I}^{(k)}}  \right]\right\} }
\end{eqnarray}
In the above,  the trace $\mathrm{Tr} $  runs over the Ising spins as well as over the fermionic degrees of freedom, and $ \mathrm{Tr}_{\mathrm{F}}  $ only over the  fermionc Fock space. 
$S_{0,I} \left( \left\{ s_{i,\tau} \right\}  \right)  $ is the action  corresponding to the Ising Hamiltonian,  and is only dependent on the Ising spins so that  it can be pulled out of the fermionic trace.
At this point,  and  since for a given configuration $C$  we are dealing with a free propagation, we can integrate out the fermions to obtain a determinant: 
\begin{eqnarray}
 &\quad&\Trf{ \left\{  \prod_{\tau=1}^{L_{\mathrm{Trotter}}} \left[   \prod_{k=1}^{M_T}   e^{-\Delta \tau \hat{T}^{(k)}}  
    \prod_{k=1}^{M_V}   e^{  \sqrt{ \Delta \tau  U_k} \eta_{k,\tau} \hat{V}^{(k)} }   \prod_{k=1}^{M_I}   e^{  -\Delta \tau s_{k,\tau}  \hat{I}^{(k)}}  \right] \right\}} = \nonumber\\
&\quad& \quad\prod\limits_{s=1}^{N_{\mathrm{fl}}} \left[  e^{- \sum_{k=1}^{M_V} \sum_{\tau = 1}^{L_{\mathrm{Trotter}}}\sqrt{\Delta \tau U_k}  \alpha_{k,s} \eta_{k,\tau} }
   \right]^{N_{\mathrm{col}}}\times
\nonumber\\
&\quad&\quad   \prod\limits_{s=1}^{N_{\mathrm{fl}}} 
   \left[
    \det\left(  1 + 
     \prod_{\tau=1}^{L_{\mathrm{Trotter}}}   \prod_{k=1}^{M_T}   e^{-\Delta \tau {\bf T}^{(ks)}}  
    \prod_{k=1}^{M_V}   e^{  \sqrt{ \Delta \tau  U_k} \eta_{k,\tau} {\bm V}^{(ks)} }   \prod_{k=1}^{M_I}   e^{  -\Delta \tau s_{k,\tau}  {\bm I}^{(ks)}}  
     \right) \right]^{N_{\mathrm{col}}}\;.
\end{eqnarray}
All in all,   the partition function is given by:
\begin{eqnarray}
    Z &=& \Tr{  \left( e^{-\beta \hat{\mathcal{H}} }\right) }\nonumber\\
    &=&   \sum_{C}   e^{-S_{0,I} \left( \left\{ s_{i,\tau} \right\}  \right) }     \left[ \prod_{k=1}^{M_V} \prod_{\tau=1}^{L_{\mathrm{Trotter}}} \gamma_{k,\tau} \right] 
    e^{- N_{\mathrm{col}}\sum_{s=1}^{N_{\mathrm{fl}}} \sum_{k=1}^{M_V} \sum_{\tau = 1}^{L_{\mathrm{Trotter}}}\sqrt{\Delta \tau U_k}  \alpha_{k,s} \eta_{k,\tau} } 
  \times   \nonumber \\
  &\quad&
      \prod_{s=1}^{N_{\mathrm{fl}}}\left[\det\left(  1 + 
     \prod_{\tau=1}^{L_{\mathrm{Trotter}}}   \prod_{k=1}^{M_T}   e^{-\Delta \tau {\bm T}^{(ks)}}  
    \prod_{k=1}^{M_V}   e^{  \sqrt{ \Delta \tau  U_k} \eta_{k,\tau} {\bm V}^{(ks)} }   \prod_{k=1}^{M_I}   e^{  -\Delta \tau s_{k,\tau}  {\bm I}^{(ks)}}  
     \right) \right]^{N_{\mathrm{col}}}  \nonumber \\ 
     & \equiv&  \sum_{C} e^{-S(C) }\;.
\end{eqnarray}
In the above, one notices that the weight factorizes in  the flavor index. The color index raises the determinant to the power $N_{\mathrm{col}}$. This corresponds to  an explicit $SU(N_{\mathrm{col}})$ symmetry   for each  configuration. This symmetry is manifest in the fact that the single particle  Green functions, again for a given  configuration C are color independent. 

\subsubsection{Observables}
In the auxiliary field QMC approach, the single particle Green function plays a crucial role.  It determines the Monte Carlo dynamics and is used to compute  observables:
\begin{equation}	
\langle \hat{O}  \rangle  = \frac{ \text{Tr}   \left[ e^{- \beta \hat{H}}  \hat{O}   \right] }{ \text{Tr}   \left[ e^{- \beta \hat{H}}  \right] } =   \sum_{C}   P(C) 
   \langle \langle \hat{O}  \rangle \rangle_{(C)} , \text{   with   } 
  P(C)   = \frac{ e^{-S(C)}}{\sum_C e^{-S(C)}}.
\end{equation}
For a given configuration $C$  one can use Wicks theorem to compute $O (C) $   from the knowledge of the single particle Green function: 
\begin{equation}
       G( x,\sigma,s, \tau |    x',\sigma',s', \tau')   =       \langle \langle {\cal T} \hat{c}^{\phantom\dagger}_{x \sigma s} (\tau)  \hat{c}^{\dagger}_{x' \sigma' s'} (\tau') \rangle \rangle_{C}
\end{equation}
and the corresponding equal time quantity reads, 
\begin{equation}
       G( x,\sigma,s, \tau |    x',\sigma',s', \tau)   =       \langle \langle {\cal T} \hat{c}^{\phantom\dagger}_{x \sigma s} (\tau)  \hat{c}^{\dagger}_{x' \sigma' s'} (\tau) \rangle \rangle_{C}
\end{equation}
Since  for a given HS field translation invariance in imaginary time is broken, the Green function has an explicit $\tau$ and $\tau'$ dependence.   On the other hand it is diagonal in the 

 

% Copyright (c) 2016 The ALF project.
% This is a part of the ALF project documentation.
% The ALF project documentation by the ALF contributors is licensed
% under a Creative Commons Attribution-ShareAlike 4.0 International License.
% For the licensing details of the documentation see license.CCBYSA.

% !TEX root = Doc.tex
%------------------------------------------------------------
\subsection{Updating schemes}\label{sec:updating}
%------------------------------------------------------------
%
The program allows for different types of updating schemes.    Given a configuration $C$ we propose a new one, $C'$, with probability $T_0(C \rightarrow C')$  and accept it according to   the  Metropolis-Hastings   acceptance-rejection probability, 
\begin{equation}
	P(C \rightarrow C') =  \text{min}  \left( 1, \frac{T_0(C' \rightarrow C) W(C')}{T_0(C \rightarrow C') W(C)} \right),
\end{equation}
so as to guarantee the stationarity condition.  Here, $ W(C) = \left| \Re \left[ e^{-S(C)} \right] \right|   $.

\begin{table}[h]
   \begin{tabular}{@{} l l l @{}}\toprule
        Variable  &  Type                  &  Description   \\
         \\\midrule
       \texttt{Propose\_S0}   &    Logical       &  If true, proposes local moves according to the probability $e^{-S_0}$ \\
       \texttt{Global\_moves} & Logical       & If true, allows for global moves. \\
        \texttt{N\_Global }       & Integer        &   Number of global moves per sweep of single spin flips. \\
        \texttt{TEMPERING}   & Compiling option &    Requires MPI and  runs the code in a parallel tempering mode. 
         \\\bottomrule
   \end{tabular}
   \caption{   Variables  required to control the updating scheme    \label{table:Updating_schemes}}
\end{table}
% 
%------------------------------------------------------------
\subsubsection{The default: sequential  single spin flips}
%------------------------------------------------------------
%
The default updating scheme is a  sequential single  spin flip algorithm.   Consider   the Ising spin $s_{i,\tau}$, we will flip it with probability one such that for  this local move  the  proposal matrix is symmetric.  If we are considering the Hubbard-Stratonovich field $l_{i,\tau}$  we will propose with probability $1/3$ one  of the other three  possible fields.   Again, for this local move, the proposal matrix is symmetric.  Hence in both cases we will accept or reject the move according to 
 \begin{equation}
 	P(C \rightarrow C') =  \text{min}  \left( 1, \frac{ W(C')}{W(C)} \right)
 \end{equation}
 It is worth noting that this type of sequential spin flip updating does not satisfy detailed balance but the more fundamental stationarity condition. 
% 
%------------------------------------------------------------
\subsubsection{Sampling of $e^{-S_0}$}
%------------------------------------------------------------
% 
Consider an Ising spin at space-time $i,\tau$ and the configuration $C$. Flipping this spin will generate the configuration $C'$ and we will propose the move according to 
  \begin{equation}
 T_0(C \rightarrow C')  =  \frac{e^{-S_0(C')}}{ e^{-S_0(C')} + e^{-S_0(C)} }   = 1 - \frac{1}{1 +  e^{-S_0(C')} /e^{-S_0(C)}}
  \end{equation}
 Note that the function $\texttt{S0}$ in the  \texttt{Hamitonian\_example.f90}  module  computes precisely the ratio\\
 ${e^{-S_0(C')} /e^{-S_0(C)}}$ so that  $T_0(C \rightarrow C') $ does not require any further programming. 
 Thereby one will accept  the proposed move with the probability: 
 \begin{equation}
 P(C \rightarrow C') =  \text{min}  \left( 1,  \frac{e^{-S_0(C)}   W(C')}{ e^{-S_0(C')} W(C)} \right).
 \end{equation}
 With Eq.~\ref{eqn:partition_2}  one sees that the bare action $S_0(C)$  determining the  dynamics of the Ising spin  in the absence of coupling to the fermions  does not enter the Metropolis acceptance-rejection step.  This sampling scheme is used  if the logical variable \texttt{Propose\_S0}   is switched on. 
% 
%------------------------------------------------------------
\subsubsection{Global updates}
%------------------------------------------------------------
%  
The code equally allows for global updates.  The user will have to provide two other functions in the module \texttt{Hamiltonian\_Examples.f90}.   

The subroutine  \texttt{Global\_move(T0\_Proposal\_ratio,nsigma\_old)}  proposes  a global move. 
The two-dimensional array \texttt{nsigma\_old(M\_V+ M\_I, Ltrot)}  contains  the full  configuration $C$.  On output, the new configuration,   C', determined by the user,  is to be stored in the 
array  \texttt{nsigma(M\_V+ M\_I, Ltrot)}.   The global variable \texttt{nsigma(M\_V+ M\_I, Ltrot)} is declared in the module \texttt{Hamiltonian}.  Equally, on output, the variable 
\texttt{T0\_Proposal\_ratio} contains the proposal ratio 
\begin{equation}
	 \frac{T_0(C' \rightarrow C)}{T_0(C \rightarrow C') }  \;.
\end{equation}
Since we allow for a stochastic  generation of  the global move, it may very well be that no change is proposed. In this case, \texttt{T0\_Proposal\_ratio}   takes the value 0 upon exit, and  
\texttt{nsigma=nsigma\_old}.   

To compute the acceptance-rejection ratio,  the user  will equally have to provide the function \\
\texttt{Delta\_S0\_global(Nsigma\_old)} that computes the ratio $e^{-S_0(C')}/e^{-S_0(C)}$. Again the configuration $C'$ is   given by the array \texttt{nsigma(M\_V+ M\_I, Ltrot)}  which is 
a global variable declared in the module \texttt{Hamiltonian}.

Note that global updates are expensive since they require a complete recalculation of the weight. We thereby  allow the user to set a variable \texttt{N\_Global} that allows to  determine how many global updates per sweeps will be carried out. 
% 
%------------------------------------------------------------
\subsubsection{Parallel tempering } 
%------------------------------------------------------------
% 
Exchange Monte Carlo \cite{Hukushima96}  or parallel tempering \cite{Greyer91}   is a possible route to overcome  sampling issues in part of  parameter space.  Let $h$ be a parameter which one can vary without  altering the configuration space $ \{C  \}  $ and let us assume that for some values of $h$ one encounters sampling problems.   For example, in the realm of spin glasses, $h$  could correspond to the  inverse temperature.  Here at high temperatures,  phase space is easily sampled   but at low temperatures  simulations get stuck in local minima. For quantum systems, $h$ could   trigger a quantum phase transition where  sampling issues are encountered, for example, in the ordered phase and not in the disordered one.   As its name suggests, parallel tempering  carries out in parallel simulations at consecutive  values of  $h$:  $h_1, h_2, h_3   \cdots h_n$, with  $h_{1} < h_2 < \cdots < h_n$.  One will sample the extended ensemble: \begin{equation}
	P(\left[h_1,C_1\right], \left[h_2,C_2\right], \cdots, \left[h_n,C_n\right] ) =  \frac{W(h_1,C_1) W(h_2,C_2) \cdots   W(h_n,C_n) } {\sum_{C_1, C_2, \cdots, C_n} W( h_1,C_1) W( h_2,C_2) \cdots   W(h_n,C_n)}
\end{equation}
where $W(h,C)$ corresponds   to the weight  for  for a given value of $h$ and configuration C.     Clearly, one can sample  $P( \left[h_1,C_1\right], \left[h_2,C_2\right], \cdots, \left[h_n,C_n\right])$ by carrying out $n$-independent runs.  However, parallel tempering  includes the following   exchange step:
\begin{equation}
	\left[h_1,C_1\right], \cdots, \left[h_i,C_i\right],\left[h_{i+1},C_{i+1}\right] \cdots, \left[h_n,C_n\right]   \rightarrow 
	\left[h_1,C_1\right], \cdots, \left[h_i,C_{i+1}\right],\left[h_{i+1},C_{i}\right] \cdots, \left[h_n,C_n\right] 
\end{equation}
which, for a symmetric proposal matrix, will  be accepted with probability: 
\begin{equation}
	\text{ min} \left( 1,   \frac{ W(h_i,C_{i+1}) W(h_{i+1},C_{i})}{W(h_i,C_{i}) W(h_{i+1},C_{i+1})} \right).
\end{equation}
 Thereby,  a configuration can meander in parameter space $h$ and  explore regions where ergodicity  is not an issue.     In the context of spin-glasses,  a low temperature  configuration, stuck in a local minima, can heat up, overcome the potential  barrier and then cool down again. 
 
The choice of the   $h_i$'s  is important  to  obtain a good acceptance rate for the exchange step.  With  $W(h,C)  = e^{- S(h,C) }$, the  distribution of the action $S$  reads:
\begin{equation}
	 {\cal P}( h, S ) =   \sum_{C}     P( h,C )   \delta ( S(h,C) -  S ). 
\end{equation} 
Acceptance of the exchange  step requires the distributions  ${ \cal P}( h, S )  $ and       ${ \cal P}( h  + \Delta h , S )  $ to overlap. For 
$\langle S \rangle_{h}  < \langle S \rangle_{h +  \Delta h} $   one can formulate this  requirement as:
\begin{equation}
	\langle S \rangle_{h}  +\langle \Delta S \rangle_{h}   \simeq \langle S \rangle_{h +  \Delta h}  - \langle \Delta S \rangle_{h + \Delta h} ,  \text{    with   }   
\langle \Delta S \rangle_{h}   =  \sqrt{ \langle \left(    S -  \langle S   \rangle_h  	\right)^2 \rangle_h} .
\end{equation}
Assuming  $ \langle \Delta S \rangle_{h + \Delta h}  \simeq \langle \Delta S \rangle_{h} $  and expanding in $\Delta h$ one obtains: 
\begin{equation}
	\Delta h \simeq \frac{ 2  \langle \Delta S \rangle_{h}    }{ \partial \langle S \rangle_{h} / \partial h}.  
\end{equation} 
The above equation becomes transparent  for  classical systems  with $ S(h,C) =  h H(C) $.  In this case, the above equation reads: 
\begin{equation}
	\Delta h       \simeq  2 h \frac{  \sqrt{C} } { C    + h \langle H \rangle_h},  \text{   with  } C = h^2    \langle \left(  H -  \langle H   \rangle_h \right)^2 \rangle_h
\end{equation} 
Several comments are in order. i)    Let us identify $h$ with the inverse temperature  such that $C$ corresponds to the specific heat. This quantity is extensive,  as well as the energy, such that $ \Delta h \simeq 1/{\sqrt{N}} $ where $N$ is the system size.   ii) In the proximity of a phase transition,   the specific heat can diverge such that   care has to be taken in the choices of  $h$.  iii)  Since the action is formulation dependent,   one expects the acceptance of the  exchange move to equally depend  upon the fomulation. 

%\mycomment{MB: Do you track the $n-1$ exchange acceptance rates $acc(i,i+1)$ for the $n$ replicas in the code? Could the exchange rates be an efficient way to locate  a phase transition in the parameters space of $h$, without a priori knowing the order parameter? Also for topological phase transitions w/o an order parameter?  }
%\mycomment{FFA:  Yes I do track the individual acceptances and I do see  singularities in the  acceptance at the phase transition. However, owing to comment iii) at would now be very careful at interpreting the results since they are really formulation dependent. }
 The auxiliary field quantum Monte Carlo code in the ALF project  comes with a parallel tempering  compiler option which we will discuss  in section \ref{Parallel.Sec}. 

% Copyright (c) 2016, 2017 The ALF project.
% This is a part of the ALF project documentation.
% The ALF project documentation by the ALF contributors is licensed
% under a Creative Commons Attribution-ShareAlike 4.0 International License.
% For the licensing details of the documentation see license.CCBYSA.

% !TEX root = doc.tex

%------------------------------------------------------------
\subsubsection{Langevin dynamics}
%------------------------------------------------------------

Neglecting the  systematic error origination from the finite imaginary time step, the general from of the action reads:
\begin{equation}
   Z  =   \sum_{C}   e^{-S_{0,I} \left( \left\{ s_{i,\tau} \right\}  \right) }     \left( \prod_{k=1}^{M_V} \prod_{\tau=1}^{L_{\mathrm{Trotter}}} \gamma_{k,\tau} \right)
    e^{ N_{\mathrm{col}}\sum\limits_{s=1}^{N_{\mathrm{fl}}} \sum\limits_{k=1}^{M_V} \sum\limits_{\tau = 1}^{L_{\mathrm{Trotter}}}\sqrt{-\Delta \tau U_k}  \alpha_{k,s} \eta_{k,\tau} }  e^{ - S^{F}(C) }   \equiv \sum_{C} e^{-S(C)}
\end{equation}
where the fermionic part is  given by
\begin{equation}
e^{ - S^{F}(C) }  =  \prod_{s=1}^{N_{\mathrm{fl}}}\left[\det\left(  \mathds{1} + 
     \prod_{\tau=1}^{L_{\mathrm{Trotter}}}   
    \prod_{k=1}^{M_V}   e^{  \sqrt{ -\Delta \tau  U_k} \eta_{k,\tau} {\bm V}^{(ks)} }   \prod_{k=1}^{M_I}   e^{  -\Delta \tau s_{k,\tau}  {\bm I}^{(ks)}}  
     \prod_{k=1}^{M_T}   e^{-\Delta \tau {\bm T}^{(ks)}} 
     \right) \right]^{N_{\mathrm{col}}} .
\end{equation} 
To formulate both the Langevin dynamics as well as the Hybrid Monte Carlo, we will need  to estimate the fermion forces: 
\begin{equation}
	\frac { \partial S^F(C)}{\partial s_{k,\tau} }.
\end{equation}
The routine  \texttt{Langevin\_update} in the module \texttt{Langevin\_update\_mod.F90}   computes  thees forces  for a  general model and passes  them into a user defined routine in the Hamiltonian file  \texttt{ Ham\_Langevin\_update}  where the Langevin update is carried out.    Clearly, in this context 
$ s_{k,\tau}$  is a real valued quantity. 

Langevin dynamics  corresponds to a  stochastic differential equation   for the fields  $\pmb{s} \equiv \left\{s_{k,\tau} \right\}$. They acquire a Langevin time $\tl$  and satisfy the stochastic differential equation
\begin{equation}
   \ve{s}(\tl +  \delta \tl)    =    \ve{s}(\tl)    - \frac{\partial S(C) }{\partial    \ve{s}(\tl) }    \delta \tl     +\sqrt{2 \delta \tl } \ve{\eta}.
\end{equation}
Here, $  \ve{\eta}  $   are  independent Gaussian  stochastic variables  satisfying:
\begin{equation}
        \langle  \eta_{k,\tau} \rangle_{\eta}  = 0   \text{  and  }    \langle  \eta_{k,\tau}  \eta_{k',\tau'} \rangle_{\eta}  = \delta_{k,k'} \delta_{\tau,\tau'}
\end{equation}
We refer the reader to  Ref.~\cite{Gardiner}   for a more in depth introduction to stochastic differential equations.
To see that the above  indeed produced the desired probability distribution in the long Langevin time limit, we can transform the Langevin equation  to the corresponding Fokker-Plank one.  Let
$P(\ve{s}, \tl) $ be the distribution of fields at Langevin time $\tl$. Then,
\begin{equation}
        P(\ve{s}, \tl  + \delta \tl )    = \int D\ve{s}'  P  (\ve{s}', \tl  )    \langle    \delta \left(  \ve{s} - \left[ \ve{s}'   - \frac{\partial S(\ve{s}' )}{\partial    \ve{s}' }   \delta \tl     +\sqrt{2 \delta \tl } \pmb{\eta}  \right]    \right) \rangle_{\eta}
\end{equation}
where $\delta$ corresponds to the $L_{trot} M_I $  dimensional Dirac $\delta$-function.   Taylor expanding  up to order $\delta \tl$  and averaging over the stochastic variable yields:
\begin{eqnarray}
P(\ve{s}, \tl  + \delta \tl ) & &     = \int D\ve{s'}  P  (\ve{s}', \tl  )   \times  \\  & &  \left(   \delta \left(  \ve{s}' - \ve{s}   \right)
- \frac{\partial S(\ve{s'}) }{\partial    \ve{s'} }   \frac{\partial  }{\partial    \ve{s'} } \delta \left(  \ve{s}' - \ve{s} \right)  \delta \tl   +
   \frac{\partial  }{\partial    \ve{s'} }   \frac{\partial  }{\partial    \ve{s}' }  \delta \left(  \ve{s}' - \ve{s}\right)    \delta \tl
\right)  \nonumber   \\
  &&   + {\cal O}  \left(  \delta \tl^2 \right) \nonumber.
\end{eqnarray}
Partial integration  and taking the limit of infinitesimal time steps   gives the Fokker-Plank equation
\begin{equation}
         \frac{\partial  }{\partial   \tl}  P( \ve{s}, \tl)  =  \frac{\partial  }{\partial    \ve{s} }  \left( P( \ve{s}, \tl)  \frac{\partial S(\ve{s}) }{\partial     \ve{s} }   +
          \frac{\partial P(\ve{s},\tl) }{\partial     \ve{s} }
         \right)
\end{equation}
The stationary,  $ \frac{\partial  }{\partial   \tl}  P( \ve{s}, \tl) =0$,  normalizable,  solution to the above equation corresponds to the desired probability distribution:
\begin{equation}
          P(\ve{s}) =  \frac{ e^{ - S(\ve{s}) } }   {   \int D \ve{s}  e^{ - S(\ve{s}) } }.
\end{equation}
As mentioned above, Langevin dynamics will work well  provided that  the forces show  no  singularities.     The great advantage of such an updating scheme is that there is no rejection and  that all  fields are updated at each step.  The following points that highlight potential issues with Langevin dynamics are in order.
\begin{itemize}
\item   Langevin dynamics will be carried out at a finite  Langevin time step and thereby we have introduced a further source of systematic error.
\item   The factor $\sqrt{2 \delta \tl} $   multiplying the stochastic variable makes the  noise dominant  on short time scales.  On these times scales  Langevin dynamics essentially  corresponds to a random walk. This has the advantage that one can circumvent potential barriers, but  may render the updating scheme less  efficient than the hybrid molecular  dynamics approach.
\end{itemize}
In the code, we have adopted a  variable time step strategy. This is important since   the determinant has zeros and the forces  are unbounded.    The user   provides an upper bound to the force, \texttt{Max\_Force},  and  if the maximal force in a configuration \texttt{Max\_Force\_Conf}, is large than \texttt{Max\_Force} the time step  is rescaled as 
\begin{equation}
     \tilde{\delta \tl}   =  \frac{ \texttt{Max\_Force} *  \delta \tl }{\texttt{Max\_Force\_Conf}}.
\end{equation}
With the adaptive time  step,  averages are computed as: 
\begin{equation}
   \langle \hat{O} \rangle = \frac{ \sum_n (\delta \tl )_n \langle \langle \hat{O} \rangle   \rangle_{(C_n)}} {\sum_n (\delta \tl )_n } 
\end{equation}


We have tested the code for a 6-site Hubbard chain at half-filling  at $U/t = 4$,  $\beta t = 4$    and with periodic boundary conditions.   One can show that for this choice of boundary conditions the   forces are not bounded 
and to make sure that the program does not   crash we have  set \texttt{Max\_Force = 1.5}  


 The discrete  variable code gives
\begin{equation}
 \langle  \hat{H} \rangle  =  -3.468429   \pm     0.000726.
\end{equation} 
The Langevin code at $ \delta \tl = 0.001$  yields 
\begin{equation}
 \langle  \hat{H} \rangle  =  -3.456968   \pm   0.009886 
\end{equation} 
and at $ \delta \tl = 0.01$ 
\begin{equation}
 \langle  \hat{H} \rangle  = -3.495365    \pm  0.007281 
\end{equation} 
 At $ \delta \tl = 0.001$   the maximal force that occurred during the run was 
$ 112$, whereas at $ \delta \tl = 0.01$ it grew to $524$.    In both cases the average force was given by $0.45$.   For larger values of  $ \delta \tl $ the maximal force grows and the fluctuations on the energy become  larger.  
( $ \langle  \hat{H} \rangle  =  -3.718439    \pm   0.206469 $  at $ \delta \tl = 0.02$. For this parameter set  the maximal force we encountered during the run was of $1658$.)

Controlling Langevin dynamics when the action has logarithmic divergences is a challenge, and it is not clear  that the results will be satisfying.  For our specific problem we can solve this issue by considering open boundary conditions. Following an argument put forward in \cite{Assaad07}, we can show, using world lines, that the determinant is always positive.   In this case the  action does not  have logarithmic divergences and the Langevin dynamics works beautifully well, see Fig.~\ref{Langevin.fig}. 

\begin{figure}[H]
        \begin{center}
                \includegraphics[scale=0.9]{Figures/Langevin.pdf}
            \end{center}
        \caption{\label{Langevin.fig}   Total energy for the 6-sites Hubbard chain at $U/t=4$, $\beta t = 4$ and with open boundary conditions.   Here one can show that the determinant is always positive such that  no   singularities occur in the action, and consequently the Langevin dynamics works very well.  The data point at $\delta \tl =0$ stems from running the  discrete  field code with coupling  of the field to the z-component of the magnetization.  The extrapolated value of the energy reads  $-2.87104   \pm 0.001144$ and the reference result from the discrete code is $  -2.871124   \pm  0.000389 $.  Throughout the runs the maximal force was always less than the threshold of 1.5.   }
\end{figure}



% !TEX root = doc.tex
% Copyright (c) 2016 The ALF project.
% This is a part of the ALF project documentation.
% The ALF project documentation by the ALF contributors is licensed
% under a Creative Commons Attribution-ShareAlike 4.0 International License.
% For the licensing details of the documentation see license.CCBYSA.
%
%-----------------------------------------------------------------------------------
\subsection{Stabilization - A Peculiarity of the BSS Algorithm}\label{sec:stable}
%-----------------------------------------------------------------------------------
%
From \eqref{eqn:partition_2} it can be seen that for the calculation of the Monte Carlo weight
and for the observables a long product of matrix exponentials has to be formed.
On top of that we need to be able to extract the single particle Green function  for a given flavor index at say time slice $\tau = 0$.  As  mentioned above, this quantity is given by: 
\begin{equation}
G = \left( 1 + \prod_{ \tau= 1}^{L_{\text{Trotter}}} B_\tau \right)^{-1}.
\end{equation}
To boil this down to more familiar terms from linear algebra we remark that we can recast this problem as the question to the solution of the linear system
\begin{equation}
(1 + \prod_\tau B_\tau) x = b.
\end{equation}
The $B_\tau$ depend on the system size as well as other physical parameters that can be chosen such that a matrix norm of $B_i$ can have any number.
From standard perturbation theory for linear systems it is known that the computed solution $\tilde{x}$ would 
contain a relative error of
\begin{equation}
\frac{|\tilde{x} - x|}{|x|} = \mathcal{O}\left(\epsilon \kappa(1 + \prod_\tau B_\tau)\right).
\end{equation}
Here $\epsilon$ denotes the machine precision which is $2^{-53}$ for IEEE double precision numbers
and $\kappa(M)$ is the condition number of the matrix $M$.
The important fact that makes straight-forward inversion so badly suited  stems from the fact that $  \prod_ \tau B_\tau $ contains exponentially large and small scales as can be seen in \eqref{eqn:partition_2}.  Thereby, as a function of increasing inverse temperature, 
the condition number  will grow exponentially so that the computed solution $\tilde{x}$
will often contain no correct digits at all.
To circumvent this more sophisticated methods have to be employed.   We will first of all assume that  the multiplication of  \texttt{NWRAP}  B matrices   has an acceptable condition number.   Assuming for simplicity that \texttt{NWRAP} is a multiple of  $L_{\text{Trotter}}$, we  can write: 
\begin{equation}
G = \left( 1 + \prod_{ i = 0}^{L_{\text{Trotter}} /\texttt{NWrap} -1}       \underbrace{\prod_{\tau=1}^{\texttt{NWrap}} B_{i  \cdot  \texttt{NWrap}+ \tau} }_{ \equiv {\cal B}_i}\right)^{-1}.
\end{equation}

ALF is by default employing
the strategy of forming a product of QR-decompositions which was proven to be weakly backwards stable in \cite{Bai2011}.
The key idea is to efficiently separate the scales of a matrix from the orthogonal part of a matrix.
This can be achieved using a QR decomposition of the ${\cal B}_i = Q_i \tilde{R_i}$. $Q_i$ is a unitary matrix and hence $\kappa(Q_i) = 1$.
To get a handle on the condition number of $\tilde{R}_i$ we will form the
diagonal matrix $(D_i)_{jj} = |(\tilde{R}_i)_{jj}|$ and rescale $\tilde{R}_i$ accordingly, $\tilde{R}_i = D_i R_i$.
This gives the decomposition
\begin{equation}
{\cal B}_i = Q_i D_i R_i.
\end{equation}
$D_i$ now contains the row norms of the original $\tilde{R}_i$ matrix and hence separates off the total scales of the problem since $\tilde{R}_i$ is now only of modest condition number.  \mycomment{ FFA. You are guessing this.}
This given an initial decomposition of $B_{j-1} = Q_{j-1} D_{j-1} T_{j-1}$ any product 
of ${\cal B}$ matrices is formed in the following two steps:
\begin{enumerate}
\item Form $ M_j = ({\cal B}_j Q_{j-1}) D_{j-1}$. Note the parentheses.
\item Do a QR decomposition of $M_j = Q_j D_j R_j$.
\item Form the updated $R$ matrices $T_j = R_j T_{j-1}$.
\end{enumerate}
%While this provides provides a stable method to calculate the involved matrix product
%it can be pretty expensive. Therefore the user can specify to skip a certain number of 
%QR Decompositions and perform plain multiplications instead. This is specified in the parameters file by the \path{NWrap} parameter.
%\path{NWrap}~=~1 corresponds to always performing QR decompositions whereas larger integers give longer intervals where no QR decomposition will be performed.
The effectiveness of the stabilization \emph{HAS} to be judged for every simulation from the info
file. For most simulations there are two values to look out for:
\begin{itemize}
\item \path{Precision Green}
\item \path{Precision Phase}
\end{itemize}
The Green's function as well as the average phase are usually numbers with a magnitude of $\mathcal{O} (1)$. 
For that reason we recommend that \path{Nwrap} is chosen such that the mean precision is  of the order of $10^{-8}$  or better.  
\mycomment{Think about formulation}

\section{Auxiliary Field Quantum Monte Carlo: projective algorithm}\label{sec:defT0}
% Copyright (c) 2016-2019 The ALF project.
% This is a part of the ALF project documentation.
% The ALF project documentation by the ALF contributors is licensed
% under a Creative Commons Attribution-ShareAlike 4.0 International License.
% For the licensing details of the documentation see license.CCBYSA.

% !TEX root = doc.tex

The projective  approach is the method of choice if  one is interested in ground state properties.   For a given trial wave functions  $| \Psi_{T,L/R} \rangle  $  that are  not orthogonal to the ground state,  $| \Psi_0 \rangle  $,   
($  \langle \Psi_{T,L/R}  | \Psi_T \rangle  \neq 0  $), the ground state expectation value of an observable  $\hat{O} $ is given by: 
\begin{equation}
	 \frac{ \langle \Psi_0 | \hat{O} | \Psi_0 \rangle }{ \langle \Psi_0 | \Psi_0 \rangle}   = \lim_{\theta \rightarrow \infty}  
	 \frac{ \langle \Psi_{T,L} | e^{-\theta \hat{H}}  e^{-(\beta - \tau)\hat{H}  }\hat{O} e^{- \tau  \hat{H} }   e^{-\theta \hat{H}} | \Psi_{T,R} \rangle } 
	        { \langle \Psi_{T,L} | e^{-(2 \theta + \beta) \hat{H}  } | \Psi_{T,R} \rangle } 
\end{equation}
The simulations are carried out at large  but finite values of  $\Theta$ so as to guarantee convergence to the ground  state within the statistical uncertainly.   $\beta$ denotes an imaginary time range where  observables 
(time displaced and equal time) can be measured.  


\subsection{The choice and specification of the trial wave function}
\section{Monte Carlo sampling}\label{sec:sampling}
% !TEX root = doc.tex
% Copyright (c) 2016 The ALF project.
% This is a part of the ALF project documentation.
% The ALF project documentation by the ALF contributors is licensed
% under a Creative Commons Attribution-ShareAlike 4.0 International License.
% For the licensing details of the documentation see license.CCBYSA.
%
%------------------------------------------------------------
\subsection{Monte Carlo sampling}\label{sec:sampling}
%------------------------------------------------------------
%
The default updating scheme consists of local moves which change (upon acceptance) only one  entry of $L_{\mathrm{Trotter}}(M_I+M_V)$  fields (see Sec. \ref{sec:updating}). 
To generate  an independent configuration $C$,   one has to visit at least each field  once.  Our unit of \textbf{sweeps} is defined such that each field is visited twice in a sequential propagation from $\tau = 0$ to $\tau = L_{\text{ Trotter}}$  and back.  A single sweep will  generically not  suffice to produce an independent  configuration.
% This is however only the lower bound as there can be a region in the spin space where the fields are correlated and it requires a larger or even global move to significantly change the configuration to an independent one. One might imagine a ferromagnet due to spontaneous symmetry breaking. All spins are parallel aligned and, let' say, point upwards. The configuration of only down spins is equally justified, but rotating one to the other requires a global operation. Flipping the spins individually one after another generates intermediate states of relative high energy which corresponds to a low probability in the QMC algorithm.
In fact, the auto-correlation time, $T_\mathrm{auto}$, characterizes the required time scale to generate an independent configuration or values $\langle\langle\hat{O}\rangle\rangle_C$ for the Observable $O$.

This has several consequences for the Monte Carlo simulation:
\begin{itemize}
	\item First of all, we start from a randomly chosen field configuration such that one has to invest at least one $T_\mathrm{auto}$ to generate relevant configurations before reliable measurements are possible. This phase of the simulation is known as the warm-up. In order to keep the code as flexible as possible (different simulations might have different auto-correlation times), measurements are taken from the very beginning. Instead we provide the parameter \path{n_skip} for the analysis to ignore the first \path{n_skip} bins.
	\item Secondly, our implementation averages over a given amount of measurements   set by the variable \texttt{NSWEEPS}  before storing the results, known as one bin, on the disk.  A bin corresponds to \texttt{NSWEEPS}  sweeps. The  error analysis requires statistically  independent bins to generate reliable confidence estimates. If bins are to small (averaged over a period shorter then $T_\mathrm{auto}$), the error bars are then typically underestimated. Most of the time, the auto-correlation time is unknown before the simulation is started, sometime, the compute cluster does not allow single runs long enough to generate appropriately sized bins. Therefore we provide the \path{N_rebin} parameter that specifies how many bins are combined into a new bin during the error analysis. In general, one should check, that a further increase of the bin size does not change the error estimate.  (For an explicit example, the reader is referred to the Appendix of Ref.~\cite{Assaad02}.)

The \path{N_rebin} variable can be used to control a second issue. The distribution of the Monte Carlo estimates $\langle\langle\hat{O}\rangle\rangle_C$ are unknown. The result in the form $(\mathrm{best}\pm \mathrm{error})$ assumes a Gaussian distribution. Luckily, every original distribution with a finite variance turns into a Gaussian one, once it is folded often enough (central limit theorem). Due to the internal averaging (folding) within one bin, many observables are already quite Gaussian. Otherwise one can increase \path{N_rebin} further, even if the bins are already independent~\cite{Bercx17}.
	\item The third issue concerns time displaced correlation functions. Even if the configurations are independent, the fields within the configuration are still correlated. Hence, the data for $S_{\alpha,\beta}(\vec{k},\tau)$ (see Sec.~\ref{sec:obs}; Eqn.~\ref{eqn:s}) and $S_{\alpha,\beta}(\vec{k},\tau+\Delta\tau)$ are also correlated. Setting the switch \path{N_Cov = 1} triggers the calculation of the covariance matrix in addition to the usual error analysis. The covariance is defined by
	\begin{equation}
		Cov_{\tau \tau'}=\frac{1}{N_{Bins}}\left\langle\left(S_{\alpha,\beta}(\vec{k},\tau)-\langle S_{\alpha,\beta}(\vec{k},\tau)\rangle\right)\left(S_{\alpha,\beta}(\vec{k},\tau')-\langle S_{\alpha,\beta}(\vec{k},\tau')\rangle\right)\right\rangle\,.
	\end{equation}
An example where this information is necessary in in the  calculation of mass gaps extracted by fitting the  tail  of the time-displaced correlation function.  Omitting  the covariance matrix will  underestimate the  error.
\end{itemize}

\section{Data Structures and Input/Output}\label{sec:imp}
% !TEX root = Doc.tex
\section{Implementation of the model} \label{sec:imp}
In the code, the module \texttt{Hamiltonian} defines the model Hamiltonian, the lattice under consideration and the desired observables (Table~\ref{table:hamiltonian}).
The respective file name is \texttt{Hamiltonian\_\textit{<Model Name>}.f90}: for example, \texttt{Hamiltonian\_Hub.f90} defines the plain Hubbard model on the two-dimensional square lattice. To implement a user-defined model, therefore only the module \texttt{Hamil}-\texttt{tonian} has to be set up. Accordingly, this documentation focusses almost entirely  on this module and the subprograms it includes. 
The remaining parts of the code may be treated as as a black box.  

To specify the Hamiltonian, one needs  an  \texttt{Operator},  \texttt{Lattice}   and  observable types. These three data structures will be described in the following. 

%
\begin{table}[h]
   \begin{tabular}{l l l}
    Name of &  &  \\
    subprogram & Description & Section \\\hline
    \texttt{Ham\_Set}  & Reads in model and lattice parameters from the file \texttt{parameters}. \\
                       & And it sets the Hamiltonian by calling \texttt{Ham\_latt}, \texttt{Ham\_hop}, and \texttt{Ham\_V}. & \\
    \texttt{Ham\_hop}  & Sets the hopping term  $\hat{\mathcal{H}}_{T}$ by calling \texttt{Op\_make} and \texttt{Op\_set}. & \ref{sec:op}, \ref{sec:specific}\\
    \texttt{Ham\_V}    & Sets the interaction terms  $\hat{\mathcal{H}}_{V}$ and $\hat{\mathcal{H}}_{I}$ 
                         by calling \texttt{Op\_make} and \texttt{Op\_set}.& \ref{sec:op}, \ref{sec:specific}\\  
    \texttt{Ham\_Latt} & Sets the lattice by calling \texttt{Make\_Lattice}.& \ref{sec:latt}\\
    \texttt{S0}        & A function which returns an update ratio for the Ising term $\hat{\mathcal{H}}_{I,0}$. 
    & \ref{sec:s0} \\
    \texttt{Alloc\_obs} & Asigns memory storage to the observables & \\
    \texttt{Init\_obs}  & Initializes the observables to zero. & \\
    \texttt{Obser}      & Computes the scalar observables and equal-time correlation functions. & \ref{sec:obs} \\
    \texttt{ObserT}     & Computes time-displaced correlation functions. & \ref{sec:obs}\\
    \texttt{Pr\_obs}    & Writes the observables to the disk by calling \texttt{Print\_bin}.   
    
   \end{tabular}
   \caption{   Overview of the subprograms of the  module \texttt{Hamiltonian} to define the Hamiltonian, the lattice and the observables.
    \label{table:hamiltonian}}
\end{table}
%

\subsection{The \texttt{Operator} type}\label{sec:op}
The fundamental data structure in the code is the derived data type \texttt{Operator}. 
This type is used to define the Hamiltonian (\ref{eqn:general_ham}).
In general, the matrices ${\bf T}^{(ks)}$, ${\bf V}^{(ks)}$ and ${\bf I}^{(ks)}$ are sparse Hermitian matrices.
Consider the  matrix   ${\bm X}$ of dimension  $N_{\mathrm{dim}} \times N_{\mathrm{dim}}$, as an representative of each of the above three matrices .  Let us  denote  with  $ \left\{z_{1},\cdots,  z_{N}  \right\}$  a subset  of $N$ indices,  
for which
\begin{equation}
X_{x,y}  =
\left\{\begin{matrix}  X_{x,y}  &  \text{ if }   x,  y  \in \left\{ z_1, \cdots z_N \right\}\\ 
                                  0         &  \text{ otherwise } 
      \end{matrix}\right.
\end{equation}
 We define the $N \times N_{\mathrm{dim}}$ matrices $\mathbf{P}$  as
\begin{equation}
P_{i,x}=\delta_{z_{i},x}\;,
\end{equation}
where $i \in [1,\cdots, N ]$ and $ x  \in [1,\cdots, N_{\mathrm{dim}}]$. The matrix  $\bm{P}$ picks out the non-vanishing entries of $\bm{X}$, 
which are contained in the rank-$N$  matrix $\bm{O}$.  Thereby: 
\begin{equation}
\bm{X} =\bm{P}^{T} \bm{O} \bm{P}\;,
\end{equation}
such that:
\begin{equation}
X_{x,y} = \sum\limits_{i,j}^{N}  P_{i,x}  O_{i,j} P_{j,y}=\sum\limits_{i,j}^{N} \delta_{z_{i},x}  O_{ij} \delta_{z_{j},y} \;.
\end{equation}
Since  the  $\bm{P}$ matrices have only one non-vanishing entry per column,  they can be stored as a vector $\vec{P}$:
\begin{equation}
     P_i = z_i.
\end{equation}  
There are  many useful  identities which emerge from this  structure. For example: 
\begin{equation}
	e^{\bm{X}} =  e^{\bm{P}^{T} \bm{O} \bm{P}}   = \sum_{n=0}^{\infty}  \frac{\left( \bm{P}^{T} \bm{O} \bm{P} \right)^n}{n!} =  \bm{P}^{T} e^{ \bm{O} } \bm{P}
\end{equation}
since 
\begin{equation} 
	 \bm{P} \bm{P}^{T}= 1_{N\times N}.
\end{equation}

In the code, we define a structure called \texttt{Operator} to capture the above. 
This type \texttt{Operator} bundles several components that are needed to define and use an operator matrix in the program.  

\subsubsection{Specification of the model}\label{sec:specific}
%
\begin{table}[h]
   \begin{tabular}{l l}
    Name of variable in the code & Description \\\hline
    \hl{\texttt{Op\_X\%N}}            &  effective dimension $N$ \\
    \hl{\texttt{Op\_X\%O}}            &  matrix  $\mathbf{O}$  of dimension $N \times N$\\
    \hl{\texttt{Op\_X\%P}}            &  projection matrix $\mathbf{P}$  encoded as a vector of dimension $N$.\\
    \hl{\texttt{Op\_X\%g}}            &  coupling strength $g$ \\  
    \hl{\texttt{Op\_X\%alpha}}      &  constant $\alpha$ \\
    \hl{\texttt{Op\_X\%type}}        &  integer parameter to set the type of 
                                             HS transformation\\
                                &  (1 = Ising, 2 = Discrete HS, for perfect square)  \\ 
    \texttt{Op\_X\%U}            &  matrix containing the eigenvectors of $\mathbf{O}$  \\
    \texttt{Op\_X\%E}            &  eigenvalues of $\mathbf{O}$ \\
    \texttt{Op\_X\%N\_non\_zero} &  number of non-vanishing eigenvalues of $\mathbf{O}$ 
   \end{tabular}
   \caption{Components of the \texttt{Operator}  type. 
   In the left column, the letter \texttt{X} is a placeholder for the letters \texttt{T} and \texttt{V}, 
   indicating hopping and interaction operators, respectively.
   The highlighted variables have to be specified by the user.
  %  One will have to specify $N$, $O$, $P$, $g$, $\alpha$ and the type.  The other variables will be automatically generated in the routine \texttt{Op\_Set}.  
    \label{table:operator}}
\end{table}
%
In order to specify the  Hamiltonian (\ref{eqn:general_ham}), we will  need several arrays of  structure variables \texttt{Operator}. Its components are listed in Table~\ref{table:operator}.  
Since the implementation exploits the $SU(N_{\mathrm{col}})$ invariance of the Hamiltonian, we have dropped the color index $\sigma$ in the following.
\begin{itemize}
\item Hopping Hamiltonian (\ref{eqn:general_ham_t}): 
In this case $\bm{X}=\bm{T}^{(k,s)}$. The corresponding array of structure variables \texttt{Op\_T} is  \texttt{Op\_T(M$_T$,N$_{fl}$)} . 
Precisely, a single variable  \texttt{Op\_T}  describes the operator matrix:
\begin{equation}
            \left( \sum_{x,y}^{N_{\mathrm{dim}}} \hat{c}^{\dagger}_x T_{xy}^{(ks)} \hat{c}^{\phantom{\dagger}}_{y}  \right)  \;,
\end{equation} 
where $k=[1, M_{T}]$ and $s=[1, N_{\mathrm{fl}}]$.
We have $g=-\Delta \tau$, $\alpha = 0$, and the type variable $\texttt{Op\_T\%type}$  is irrelevant. 



\item Interaction Hamiltonian (\ref{eqn:general_ham_v}):
If the interaction is of perfect-square type, we set  ${\bm X}  = \bm{V}^{(k,s)}$ 
and  define the corresponding structure variables \texttt{Op\_V}  using the array \texttt{Op\_V(M\_V,N\_{fl})}.
A single variable  \texttt{Op\_V}  describes the operator matrix:
\begin{equation}
             \left[ \left( \sum_{x,y}^{N_{\mathrm{dim}}} \hat{c}^{\dagger}_x V_{x,y}^{(ks)} \hat{c}^{\phantom{\dagger}}_{y}  \right) - \alpha_{ks} \right]  \;,
\end{equation} 
where $k=[1, M_{V}]$ and $s=[1, N_{\mathrm{fl}}]$. For the perfect-square interaction, $\alpha = \alpha_{ks}$ and $g = \sqrt{\Delta \tau  U_k}$. 
The discrete Hubbard-Stratonovich decomposition is selected by setting the type variable to $\texttt{Op\_V\%type}=2$.

\item Ising interaction Hamiltonian (\ref{eqn:general_ham_i}):
In this case, $\bm{X}  = \bm{I}^{(k,s)} $ and we define the array\\ \texttt{Op\_V(M\_I,N\_{fl})}.  
A single variable  \texttt{Op\_V} then  describes the operator matrix:
\begin{equation}
            \left( \sum_{x,y}^{N_{\mathrm{dim}}} \hat{c}^{\dagger}_x I_{xy}^{(ks)} \hat{c}^{\phantom{\dagger}}_{y}  \right)  \;,
\end{equation} 
where $k=[1, M_{I}]$ and $s=[1, N_{\mathrm{fl}}]$.
The Ising interaction is specified by setting the type variable  $\texttt{Op\_V\%type=1}$, $\alpha = 0$ and $g = -\Delta \tau$.  

\item In case of a full interaction [perfect-square term (\ref{eqn:general_ham_v}) and Ising term (\ref{eqn:general_ham_i})], we  define  the corresponding doubled array \texttt{Op\_V(M$_V$+M$_I$,N$_{fl}$) } and set the variables separately for both ranges of the array according to the above.  

\end{itemize}
  %      There is another array   which defines the full interaction,  Ising as well as perfect square terms. For this  we define  the array \texttt{Op\_V(M$_V$+M$_I$,N$_{fl}$) }). In this context the variable \texttt{Op\_V\%type} specifies the interaction: Ising or  a perfect square.  If the interaction is of Ising type, then  $\bm{V}  = \bm{I}^{(k,s)} $, $\alpha = 0$ and $g = -\Delta \tau$.  
%   If the interaction is a perfect square type, then  $\bm{V}  = \bm{V}^{(k,s)} $, $\alpha = \alpha_{k,s}$ and $g = \sqrt{\Delta \tau  U_k}$.  

%The variable $\texttt{Op\_V\%type}  $  in the operator structure  is required to specify  the following. If the operator  correspond to an interaction part of the Hamiltonian  then for 
%$\texttt{Op\_V\%type} =1 $   the operator referes to an Ising  operator $ \bm{I}^{k,s}$ and for  $\texttt{Op\_V\%type} =2 $  to $\bm{V}^{ks} $
%\begin{itemize}
%\item the projector ${\bm P}$, encoded as the vector $\vec{P}$,
%\item the matrix ${\bm O}$ of dimension $N \times N$  
%\item the effective dimension $N$,
%\item and a couple of auxiliary matrices and scalars.
%\end{itemize}
%The precise definition of the Operator type reads:




\subsection{The Lattice tpye}\label{sec:latt}

We have a lattice module  which  generate   one and two dimensional dimensional Bravais lattices.   Note that the  orbital structure of each unit cell, has to be specified by the user  in the  Hamiltonian module. 
 The user has to specify unit vectors $\vec{a}_1$ and $\vec{a}_2$ as well as   the size of the  lattice. The size is  characterized by  two vectors $\vec{L}_1$ and $\vec{L}_2$   and  the lattice is placed on a torus: 
\begin{equation}
	\hat{c}_{\vec{i} + \vec{L}_1 }  = \hat{c}_{\vec{i} + \vec{L}_2 }  = \hat{c}_{\vec{i}}
\end{equation}
The call 
\texttt{ Call Make\_Lattice( L1, L2, a1,  a2, Latt )} will generate the lattice   \texttt{Latt} of type \texttt{Lattice}.   Note that  the structure of the unit cell has to be provided by the user.    The reciprocal lattice vectors are defined by: 
\begin{equation}
\label{Latt.G.eq}
	\vec{a}_i  \cdot \vec{g}_i = 2 \pi \delta_{i,j}, 
\end{equation}
and the Brillouin zone corresponds to the Wigner Seitz cell of the lattice. 
With $\vec{k} = \sum_{i} \alpha_i  \vec{g}_i $, the  k-space quantization follows from: 
\begin{equation}
\begin{bmatrix}
	\vec{L}_1 \cdot \vec{g}_1  &  \vec{L}_1 \cdot \vec{g}_2  \\
	\vec{L}_2  \cdot \vec{g_1} & \vec{L}_2 \cdot  \vec{g}_2  
\end{bmatrix}
\begin{bmatrix}
   \alpha_1 \\
   \alpha_2
\end{bmatrix}
=
2 \pi 
\begin{bmatrix}
   n \\
   m
\end{bmatrix}
\end{equation}
such that 
\begin{eqnarray}
\label{k.quant.eq}
     \vec{k} =  n \vec{b}_1  + m \vec{b}_2 \text{  with  }   & &   \vec{b}_1 = \frac{2 \pi}{ (\vec{L}_1 \cdot \vec{g}_1)  (\vec{L}_2 \cdot  \vec{g}_2 )  - (\vec{L}_1 \cdot \vec{g}_2) (\vec{L}_2  \cdot \vec{g_1} ) }   \left[  (\vec{L}_2 \cdot  \vec{g}_2) \vec{g}_1 -   (\vec{L}_2  \cdot \vec{g_1} ) \vec{g}_2 \right] \text{   and  } \nonumber \\ 
        & & \vec{b}_2 = \frac{2 \pi}{ (\vec{L}_1 \cdot \vec{g}_1)  (\vec{L}_2 \cdot  \vec{g}_2 )  - (\vec{L}_1 \cdot \vec{g}_2) (\vec{L}_2  \cdot \vec{g_1} ) }   
           \left[  (\vec{L}_1 \cdot  \vec{g}_1) \vec{g}_2 -   (\vec{L}_1  \cdot \vec{g_2} ) \vec{g}_1 \right] 
\end{eqnarray}
\mycomment{Check that the above algebra is correct!  Just checked. Seems to be OK.}
\mycomment{

}
% 
%
\begin{table}[h]
   \begin{tabular}{l l l}
    Name of variable  & Type & Description \\\hline
     \hl{\texttt{Latt\%a1\_p}, \texttt{Latt\%a2\_p}}   & Real     & Unit vectors $\vec{a}_1$,  $\vec{a}_2$ \\ 
     \hl{\texttt{Latt\%L1\_p}, \texttt{Latt\%L2\_p}}   & Real     & Vectors $\vec{L}_1$, $\vec{L}_2$ that define the topology of the  lattice. \\
     									  &              &  Tilted lattices are  thereby possible to implement.  \\
    \texttt{Latt\%N}                                                 &   Integer &  Number of lattice points, $N_{unit\,cell}$   \\
    \texttt{Latt\%list}                                               & Integer &  maps each lattice point $i=1,\cdots, N_{unit\,cell}$ to a real space vector\\ 
                                                                             &   &  denoting the position of the unit cell: \\
                                                                             &   & $\vec{R}_i$ = \texttt{list(i,1)} $\vec{a}_1$ +  \texttt{list(i,2)} $\vec{a}_2$  $  \equiv i_1  \vec{a}_1 + i_2  \vec{a}_2 $ \\
    \texttt{Latt\%invlist}                                        &  Integer &   \texttt{Invlist}$(i_1,i_2) = i $ \\
    \texttt{Latt\%nnlist}                                         &  Integer &   $j = \texttt{nnlist} (i, n_1, n_2) $,  $n_1, n_2 \in [-1,1] $ \\
                                                                           &              &    $\vec{R}_j = \vec{R}_i + n_1 \vec{a}_1  + n_2 \vec{a}_2 $ \\
   \texttt{Latt\%imj}                                             &   Integer  &  $ \vec{R}_{imj(i,j)}  =  \vec{R}_i -  \vec{R}_j$.        $imj, i, j \in  1,\cdots, N_{unit\,cell}$\\
    \texttt{Latt\%BZ1\_p}, \texttt{Latt\%BZ2\_p}  &   Real     & Reciprocal space vectors $\vec{g}_i$   (See Eq.~\ref{Latt.G.eq})\\
    \texttt{Latt\%b1\_p}, \texttt{Latt\%b1\_p}       &   Real     &  k-quantization (See Eq.~\ref{k.quant.eq}) \\
    \texttt{Latt\%listk}                                           &  Integer &  maps each reciprocal lattice point $k=1,\cdots, N_{unit\,cell}$\\
                                                                          &    & to a reciprocal space vector\\
                                                                          &     & $\vec{k}_k= \texttt{listk(k,1)} \vec{b}_1 +  \texttt{listk(k,2)} \vec{b}_2  \equiv k_1  \vec{b}_1 +   k_2  \vec{b}_2 $\\
    \texttt{Latt\%invlistk}                                     &    Integer    &   \texttt{Invlistk}$(k_1,k_2) = k $ \\
   \texttt{Latt\%b1\_perp\_p},  \\ 
   \texttt{Latt\%b2\_perp\_p}                             &    Real         &  Orthonormal vectors to $\vec{b}_i$.  For internal use. 
   \end{tabular}
   \caption{Components of the \texttt{Lattice} type for two-dimensional lattices using as example the default lattice name \texttt{Latt}.
   The highlighted variables have to be specified by the user.  Other components of the Lattice will be generated  when calling: \texttt{ Call Make\_Lattice( L1, L2, a1,  a2, Latt )}.  
    \label{table:lattice}}
\end{table}
%
The \texttt{Lattice}  module equally handles  the Fourier transformation.  For example  the  subroutine  \texttt{Fourier\_R\_to\_K}   carries out the  transformation: 
\begin{equation}
	S(\vec{k}, :,:,:) =  \frac{1}{N_{unit \,cell}}  \sum_{\vec{i},\vec{j} \;\text{\mycomment{change to $\vec{i}-\vec{j}$}}}   e^{-i \vec{k} \cdot \left( \vec{i}-\vec{j} \right)} S(\vec{i}  - \vec{j}, :,:,:)
\end{equation}

and  \texttt{Fourier\_K\_to\_R}  the  inverse Fourier transform 
 \begin{equation}
	S(\vec{r}, :,:,:) =  \frac{1}{N_{unit \,cell}}  \sum_{\vec{k} \in BZ }   e^{ i \vec{k} \cdot \vec{r}} S(\vec{k}, :,:,:).
\end{equation}
In the above,   the unspecified dimensions of   structure factor can refer  to imaginary time,  and orbital indices. 


\subsection{The Observable type}\label{sec:obs}

Our definition  of the model includes observables. We have defined two observable types: \texttt{Obser\_vec}  for a array of scalar observables
such as the energy and  \texttt{Obser\_Latt}   for correlation functions that have the lattice symmetry. In the latter case, translation symmetry can be used to provide improved estimators and to reduce the size of the I/O.   In general, the user will define bins, each bins having a given amount of sweeps. Within a sweep we run sequentially trough the HS and Ising fields from   time slice 1 to $L_{\text{Trotter}}$ and back.  The results of each bin is written  in a file  and analyzed at the end of the run.     

\subsubsection{Scalar observables}
This data type  is described in Table  \ref{table:Obser_vec} and  is useful to compute an array of  scalar observables.   Consider  a variable \texttt{Obs} of type  \texttt{Obser\_vec}.  At the beginning of each bin,  a call to  \texttt{Obser\_Vec\_Init} in the module \texttt{observables\_mod.f90}  will  set   \texttt{Obs\%N=0},   \texttt{Obs\%Phase =0}  and  \texttt{Obs\%Obs\_vec(:)=0}.  Each time the main  program calls the routine \texttt{Obser}  in the  \texttt{Hamiltonian} module,  the counter \texttt{Obs\%N}   is incremented by unity,   the sign  (see Eq.~\ref{Sign.eq}) is cumulated in the  variable \texttt{Obs\%phase},  and the desired  the observables (multiplied by the sign and   $\frac{e^{-S(C)}} {\Re \left[e^{-S(C)} \right]}$, see Sec.~\ref{Observables.General})  are cumulated in the vector \texttt{Obs\%Obs\_vec}.  
\begin{table}[h]
   \begin{tabular}{l ll }
    Name of variable  &  Type      &  Description \\\hline
    \texttt{Obs\%N}                       &  Integer        &   Number of measurements  \\
    \texttt{Obs\%Phase}               &  Complex     &    Cumulated sign (See Eq.~\ref{Sign.eq})  \\
    \texttt{Obs\%Obs\_vec(:)}        & Complex      &    Cumulated vector of observables. 
           $ \langle \langle \hat{O}(:) \rangle \rangle_{C}\frac{e^{-S(C)}} {\Re \left[e^{-S(C)} \right]} \text{ sign }(C) $ \\
     \texttt{Obs\%File\_Vec}           &  Character    &    Filename  in which the bins are written  
   \end{tabular}
   \caption{Components of the \texttt{Obser\_vec}  type.  The table lists the data included in a variable  \texttt{Obs}  of type \texttt{Obser\_vec}.  
\mycomment{I think you do not  use the vector character of the member \texttt{Obser\_vec\%Obs}. For the observables like energy you have created an array of type variables \texttt{Obser\_vec} 
but within the type variable, the vector is of size $1$. And each scalar observable gets its own type variable \texttt{Obs}. This is a detail but it puzzled me first. Do we have an example where the vector would be larger than $1$?}
      \label{table:Obser_vec}}
\end{table}
At the end of the bin, a call to  \texttt{Print\_bin\_Vec}   in  module \texttt{observables\_mod.f90}  will  append the result of the bin in the file  \texttt{File\_Vec}\_scal.  Note that this subroutine will automatically append the suffix  \_scal 
to the the filename \texttt{File\_Vec}.    This suffix  is important to allow automatic analysis of the data at the end of the run. 

\subsubsection{ Equal time and time displaced correlation functions}

\begin{table}[h]
   \begin{tabular}{l ll }
    Name of variable  &  Type      &  Description \\\hline
    \texttt{Obs\%N}                       &  Integer        &   Number of measurements  \\
    \texttt{Obs\%Phase}               &  Complex     &    Cumulated sign (See Eq.~\ref{Sign.eq})  \\
    \texttt{Obs\%Obs\_latt($\vec{i}-\vec{j},\tau,\alpha,\beta$)}        & Complex      &    Cumulated   correlation function  $ \langle \langle \hat{O}_{\vec{i},\alpha} (\tau) \hat{O}_{\vec{j},\beta} \rangle \rangle_{C} \; \frac{e^{-S(C)}} {\Re \left[e^{-S(C)} \right]}  \text{sign}(C) $ \\
     \texttt{Obs\%Obs\_latt0($\alpha$)}        & Complex      &    Cumulated    $ \langle \langle \hat{O}_{\vec{i},\alpha} \rangle \rangle_{C}\frac{e^{-S(C)}} {\Re \left[e^{-S(C)} \right]}  \text{ sign }(C) $ \\
     \texttt{Obs\%File\_Latt}           &  Character    &    Filename  in which the bins are written  
   \end{tabular}
   \caption{Components of the \texttt{Obser\_latt}  type.  The table lists the data included in a variable  \texttt{Obs}  of type \texttt{Obser\_latt}  
      \label{table:Obser_vec}}
\end{table}

This data type is useful so as to deal with  imaginary time displaced as well as equal time correlation functions of the form: 
\begin{equation}
	S_{\alpha,\beta}(\vec{k},\tau) =   \frac{1}{N_{unit\, cell }} \sum_{\vec{i},\vec{j}}  e^{- \vec{k} \cdot \left( \vec{i}-\vec{j}\right) } \left( \langle \hat{O}_{\vec{i},\alpha} (\tau) \hat{O}_{\vec{j},\beta} \rangle  - 
	  \langle \hat{O}_{\vec{i},\alpha} \rangle \langle   \hat{O}_{\vec{i},\beta}  \rangle \right).
\end{equation}
Here,  translation symmetry of the Bravais lattice is explicitly taken into account. Note that this symmetry is broken  for a given  configuration $C$ but is restored by the Monte Carlo sampling. 
\mycomment{MB tries to fit this in here:}
The correlation function splits in a correlated part $S_{\alpha,\beta}^{\mathrm{(corr)}}(\vec{k},\tau)$ and an background part $S_{\alpha,\beta}^{\mathrm{(back)}}(\vec{k},\tau)$:
\begin{eqnarray}
  S_{\alpha,\beta}^{\mathrm{(corr)}}(\vec{k},\tau)
  &=&
   \frac{1}{N_{unit\, cell }} \sum_{\vec{i},\vec{j}}  e^{- i\vec{k} \cdot \left( \vec{i}-\vec{j}\right) }  \langle \hat{O}_{\vec{i},\alpha} (\tau) \hat{O}_{\vec{j},\beta} \rangle\;,\\
   S_{\alpha,\beta}^{\mathrm{(back)}}(\vec{k},\tau)
  &=&
   \frac{1}{N_{unit\, cell }} \sum_{\vec{i},\vec{j}}  e^{- i\vec{k} \cdot \left( \vec{i}-\vec{j}\right) }  \langle \hat{O}_{\vec{i},\alpha} (\tau)\rangle \langle \hat{O}_{\vec{j},\beta} \rangle\\\nonumber
  &=& 
   \frac{1}{N_{unit\, cell }} \sum_{\vec{i}}  e^{- i\vec{k} \cdot \vec{i} }  \langle \hat{O}_{\vec{i},\alpha}\rangle
   \sum_{\vec{j}}    e^{i\vec{k} \cdot  \vec{j} }    \langle \hat{O}_{\vec{j},\beta} \rangle\;,
\end{eqnarray}
where we used translation invariance in imaginary-time to drop the $\tau$ dependency in the last line. 

\mycomment{In the code, the phase factors $e^{\vec{k}\cdot \vec{i}}$ are not 
included in the output for the background. 
The output is simply $\frac{1}{N_{unit\,cell}}\sum\limits_{\vec{i}}\langle \hat{O}_{\vec{i},\alpha}\rangle$.
Is the motivation to first say $\langle \hat{O}_{\vec{i},\alpha} (\tau)\rangle=\langle \hat{O}_{\alpha}\rangle$,
use $\frac{1}{N}\sum\limits_{\vec{i}}e^{i \vec{k}\cdot\vec{i}} = \delta(\vec{i})$ 
and the use the improved estimator $\langle \hat{O}_{\alpha} \rangle=\frac{1}{N}\sum\limits_{\vec{i}}\langle \hat{O}_{\vec{i},\alpha}\rangle$?}

Consider a variable  \texttt{Obs} of type  \texttt{Obser\_latt}. At the beginning of each bin a call to  \texttt{Obser\_Latt\_Init} in the module \texttt{observables\_mod.f90}  will  initialize  the elements of \texttt{Obs} to zero.    Each time the main program calls the   \texttt{Obser} or  \texttt{ObserT} routines one  cumulates $ \langle \langle \hat{O}_{\vec{i},\alpha} (\tau) \hat{O}_{\vec{j},\beta} \rangle \rangle_{C} \; \frac{e^{-S(C)}} {\Re \left[e^{-S(C)} \right]}  \text{sign}(C) $    in  \texttt{Obs\%Obs\_latt($\vec{i}-\vec{j},\tau,\alpha,\beta$)}   
and $ \langle \langle \hat{O}_{\vec{i},\alpha}= \rangle \rangle_{C}\frac{e^{-S(C)}} {\Re \left[e^{-S(C)} \right]}  \text{ sign }(C) $  in \texttt{Obs\%Obs\_latt0($\alpha$)}.   At the end of each bin, a call to \texttt{Print\_bin\_Latt} in the module  \texttt{observables\_mod.f90}   will append the result of the bin in the specified  file \texttt{Obs\%File\_Latt}.   Note that the routine  \texttt{Print\_bin\_Latt}  carries out the Fourier transformation and prints the results in k-space. We have adopted the following name convention.  For    equal time observables , that is  the second  dimension  of the array  \texttt{Obs\%Obs\_latt($\vec{i}-\vec{j},\tau,\alpha,\beta$)}    is equal to unity,  the routine \texttt{Print\_bin\_Latt}  attaches the suffix \_eq to \texttt{Obs\%File\_Latt}.  For  time displaced correlation functions we use the suffix \_tau. 

% We have three types of observables. 
% \begin{itemize}
% \item Scalar observables such as the energy
% \item Equal time correlation functions.  Let $\hat{O}_{\vec{i},\alpha} $ be a local observable,  with $\vec{i}$ labelling the unit cell and $\alpha$ labelling the orbital or bone emanating 
% from the unit cell.   The program will compute: 
% \begin{equation}
% 	S_{\alpha,\beta}(\vec{k}) = \frac{1}{N_{unit \;  cells}} \sum_{\vec{i},\vec{j}} e^{i \vec{k}\cdot (\vec{i} -  \vec{j} ) } \left( \langle \hat{O}_{\vec{i},\alpha}  \hat{O}_{\vec{j},\alpha} \rangle  - 
% 	  \langle \hat{O}_{\vec{i},\beta} \rangle \langle   \hat{O}_{\vec{i},\beta}  \rangle \right) 
% \end{equation}
% \item  Time displaced correlation functions. This has a very similar structure than above but now with an additional time index.
% \begin{equation}
% 	S_{\alpha,\beta}(\vec{k},\tau) = \frac{1}{N_{unit \;  cells}} \sum_{\vec{i},\vec{j}} e^{i \vec{k}\cdot (\vec{i} -  \vec{j} ) } \left( \langle \hat{O}_{\vec{i},\alpha} (\tau) \hat{O}_{\vec{j},\alpha} \rangle  - 
% 	  \langle \hat{O}_{\vec{i},\beta} \rangle \langle   \hat{O}_{\vec{i},\beta}  \rangle \right) 
% \end{equation}
% \end{itemize}

%\mycomment{mention bins, sweeps}
%\mycomment{We have to add some  more details.}
%\subsubsection{Scalar observables}
%Several scalar observables are measured and accumulated in the array \texttt{Obs\_scal} during the simulation (see table \ref{table:obs}).
%%
%\begin{table}[h]
%   \begin{tabular}{l l l}
%    Name of variable in the code & Definition & Description \\\hline
%\texttt{Obs\_scal(1)} & 
%$\rho=\sum\limits_{k=1}^{M_T}
%\sum\limits_{s=1}^{N_{\mathrm{fl}}}
%\sum\limits_{\sigma=1}^{N_{\mathrm{col}}}
%\sum\limits_{x}^{N_{\mathrm{dim}}}
%\langle \hat{c}^{\dagger}_{x \sigma   s} \hat{c}^{\phantom\dagger}_{x \sigma s}   \rangle$ &
%electronic density\\
%\texttt{Obs\_scal(2)} & 
%$E_{\mathrm{kin}}=\sum\limits_{k=1}^{M_T}
%\sum\limits_{s=1}^{N_{\mathrm{fl}}}
%\sum\limits_{\sigma=1}^{N_{\mathrm{col}}}
%\sum\limits_{x,y}^{N_{\mathrm{dim}}}
%\langle \hat{c}^{\dagger}_{x \sigma   s} T_{xy}^{(k s)} \hat{c}^{\phantom\dagger}_{y \sigma s}   \rangle$ &
%kinetic energy\\
%\texttt{Obs\_scal(3)} & 
%$E_{\mathrm{pot}}=\sum\limits_{x,y}^{N_{\mathrm{dim}}}
%\prod\limits_{s=1}^{N_{\mathrm{fl}}}
%\langle \hat{c}^{\dagger}_{x \sigma   s} \hat{c}^{\phantom\dagger}_{x \sigma s}  
%\rangle$ &
%potential energy \mycomment{need input here} \\
%\texttt{Obs\_scal(4)} & 
%$E_{\mathrm{tot}}=E_{\mathrm{kin}}+E_{\mathrm{pot}}$ &
%total energy\\
%\texttt{Obs\_scal(5)} & 
%$\langle \mathrm{phase} \rangle$ &
%phase of MC update probability
%   \end{tabular}
%   \caption{Scalar observables that are stored in the array \texttt{Obs\_scal}.
%       \label{table:obs}}
%\end{table}
%%
%
%
%
%\subsubsection{Equal-time correlation functions}
%
%Let $\hat{O}_{\vec{i},\alpha} $ be a local observable,  with $\vec{i}$ labelling the unit cell and $\alpha$ labelling the orbital or bone emanating 
%from the unit cell.   The program will compute: 
%\begin{equation}
%	S_{\alpha,\beta}(\vec{k}) = \frac{1}{N_{unit \;  cells}} \sum_{\vec{i},\vec{j}} e^{i \vec{k}\cdot (\vec{i} -  \vec{j} ) } \left( \langle \hat{O}_{\vec{i},\alpha}  \hat{O}_{\vec{j},\alpha} \rangle  - 
%	  \langle \hat{O}_{\vec{i},\beta} \rangle \langle   \hat{O}_{\vec{i},\beta}  \rangle \right) 
%\end{equation}
%\mycomment{Should it not be
%}
%
%\subsubsection{Time-displaced correlation functions}
%
%This has a very similar structure than above but now with an additional time index.
%\begin{equation}
%	S_{\alpha,\beta}(\vec{k},\tau) = \frac{1}{N_{unit \;  cells}} \sum_{\vec{i},\vec{j}} e^{i \vec{k}\cdot (\vec{i} -  \vec{j} ) } \left( \langle \hat{O}_{\vec{i},\alpha} (\tau) \hat{O}_{\vec{j},\alpha} \rangle  - 
%	  \langle \hat{O}_{\vec{i},\beta} \rangle \langle   \hat{O}_{\vec{i},\beta}  \rangle \right) 
%\end{equation}
%
%To set the  interaction part, we therefore have to specify the following:
%\begin{itemize}
%\item the matrix elements $\left[O_{V}^{(k)}\right]_{ij}$
%\item the set $[z_{1}^{(k)},\cdots  z_{N_{eff}^{(k)}}^{(k)}]$ 
%\item the interaction strenghts $U_{k}$
%\item the numbers  $\alpha_{k}$.
%\end{itemize}
%\mycomment{Be more specific here what really has to specified in the actual code.}%
%The same logic also applies to the implementation of the hopping interaction \mycomment{be more specific}.






%\begin{itemize}
%\item in the coupling $g$ in the \texttt{Operator} structure (see Sec.~\ref{}).
%\item as normalization constant in the definition of observables (see Sec.~\ref{})
%\item as exponent in the calculation of the phase factor and the Monte Carlo update ratio.
%\end{itemize}
%\subsection{Structure of the hopping matrix  ${\bf T}$ and the interaction matrices ${\bf V}^{(k)}$}


%\subsection{The Hubbard-Stratonovich decomposition} 
%Consider a single-particle (in other words bilinear) operator $O_{i}$.
%One obtains an approximation to the evolution operator by the following series expansion \cite{AssaadBook08}
%\begin{equation}
%\label{eqn_2_HS}
%e^{-\Delta\tau O^{2}_{i} } = \sum\limits_{s=\pm1,\pm2} \gamma(s) e^{i \sqrt{\Delta\tau}\eta(s)O_{i}} + \mathcal{O}(\Delta\tau^{4})\;,
%\end{equation}
%with 
%
%\begin{eqnarray}
%\gamma(\pm 1) = (1+\sqrt{6}/3)/4\;,\;\gamma(\pm 2) = (1-\sqrt{6}/3)/4\;,\nonumber\\
%\eta(\pm 1) =\pm \sqrt{2(3-\sqrt{6})}\;,\;\eta(\pm 2) =\pm \sqrt{2(3+\sqrt{6})}\;.
%\end{eqnarray}
%
%Eq.~(\ref{eqn_2_HS}) can be easily proven by expanding its right hand side  to eighth order in $O_{i}$. 
%The transformation introduces therefore two Ising fields $s$ per lattice site $i$, taking the values $\pm 1$ and $\pm 2$.
%\mycomment{same label as the flavor index}

% Copyright (c) 2016 The ALF project.
% This is a part of the ALF project documentation.
% The ALF project documentation by the ALF contributors is licensed
% under a Creative Commons Attribution-ShareAlike 4.0 International License.
% For the licensing details of the documentation see license.CCBYSA.

% !TEX root = doc.tex
%------------------------------------------------------------
\subsection{File structure}\label{sec:files}
%------------------------------------------------------------
%
\begin{table}[h]
	\begin{tabular}{@{} l l @{}}\toprule
   	Directory & Description \\\midrule
   	\path{Prog/} & Main program and subroutines.  \\
   	\path{Libraries/} & Collection of mathematical routines. \\  
  	\path{Analysis/} & Routines for error analysis. \\
  	\path{Scripts_and_Parameters_files/}   & Helper scripts and the \path{Start/} directory, which contains the files \\ 
  	                                      & required to start a run. \\
  	\path{Documentation/} & This documentation.\\
  	\path{testsuite/} & A suite for automatic testing various parts of the code.\\\bottomrule
  	\hline
	\end{tabular}
   	\caption{Overview of the directories included in the ALF package.\label{table:files}}
\end{table}
%

The code package, summarized in Table~\ref{table:files}, consists of the program directories \path{Prog/}, \path{Libraries/}, and \path{Analysis/}, as well as the directory \path{Scripts_and_Parameters_files/}, which contains supporting scripts and, in its subdirectory \path{Start}, the input files necessary for a run, described in the Sec.~\ref{sec:input}. Additionally, a suite of tests for individual parts of the code (subroutines, functions, operations, etc.) is available at the directory \path{testsuite} -- the tests can be run by executing the following sequence of commands (the script \path{configureHPC.sh} sets environment variables and is described in Sec.~\ref{sec:running}.):
\begin{lstlisting}[style=bash,morekeywords={make,cmake,ctest}]

source configureHPC.sh Devel serial
gfortran -v
make lib
make ana
make Examples
cd testsuite
cmake -E make_directory tests
cd tests
cmake -G "Unix Makefiles" -DCMAKE_Fortran_FLAGS_RELEASE=${F90OPTFLAGS} \
      -DCMAKE_BUILD_TYPE=RELEASE ..
cmake --build . --target all --config Release
ctest -VV -O log.txt
\end{lstlisting}
which will output test results and total success rate.
%The example simulations corresponding to the walkthroughs of Sec.~\ref{sec:walk1} - \ref{sec:walk2} are included in \path{Examples/}.

%------------------------------------------------------------
\subsubsection{Input files}\label{sec:input}
%------------------------------------------------------------
%
\begin{table}[h]
   \begin{tabular}{@{} l l @{}}\toprule
   File & Description \\\midrule
  \path{parameters} &  Sets the parameters for lattice, model, QMC process, and the error analysis.\\
  \path{seeds} & List of integer numbers to initialize the random number generator and \\
   & to start a simulation from scratch.
   %\\
 %  \path{confin_<thread number>} & Input files for the HS and Ising configuration, used to continue a simulation.
  \\\bottomrule
   \end{tabular}
   \caption{Overview of the input files required for a simulation, which can be found in the subdirectory \texttt{Scripts\_and\_Parameters\_files/Start/}. \label{table:input}}
\end{table}
%
The input files are listed in Table~\ref{table:input}. 
The parameter file \path{Start/parameters} has the following form --
using as an example  the $SU(2)$-symmetric Hubbard model on a square lattice (see Sec.~\ref{sec:walk1} for a detailed walkthrough):
%
\begin{lstlisting}[style=fortran]

!===============================================================================
!  Variables for the Hubb program
!-------------------------------------------------------------------------------
&VAR_lattice
L1 = 4                    ! Length in direction a_1
L2 = 4                    ! Length in direction a_2
Lattice_type = "Square"	  ! a_1 = (1,0), a_2=(0,1), Norb=1, N_coord=2
!Lattice_type ="Honeycomb"! a_1 = (1,0), a_2 =(1/2,sqrt(3)/2), Norb=2, N_coord=3
Model = "Hubbard_SU2"     ! Sets Nf=1, N_sun=2. HS field couples to the density
!Model = "Hubbard_Mz"     ! Sets Nf=2, N_sun=1. HS field couples to the 
                          ! z-component of magnetization.  
!Model="Hubbard_SU2_Ising"! Sets Nf_1, N_sun=2 and runs only for the square lattice
                          ! Hubbard model coupled to transverse Ising field
/

&VAR_Hubbard              ! Variables for the Hubbard model
ham_T   = 1.d0            ! Hopping parameter
ham_chem= 0.d0            ! chemical potential
ham_U   = 4.d0            ! Hubbard interaction
Beta    = 10.d0           ! inverse temperature
dtau    = 0.1d0           ! Thereby Ltrot=Beta/dtau
/

&VAR_Ising                ! Model parameters for the Ising code
Ham_xi = 1.d0             ! Only needed if Model="Hubbard_SU2_Ising"
Ham_J  = 0.2d0
Ham_h  = 2.d0
/

&VAR_QMC                  ! Variables for the QMC run
Nwrap   = 10              ! Stabilization. Green functions will be computed from 
                          ! scratch after each time interval Nwrap*Dtau
NSweep  = 10              ! Number of sweeps
NBin    = 10              ! Number of bins
Ltau    = 1               ! 1 for calculation of time displaced Green functions;
                          ! 0 otherwise
LOBS_ST = 1               ! Start measurements at time slice LOBS_ST
LOBS_EN = 100             ! End   measurements at time slice LOBS_EN
CPU_MAX = 0.1             ! Code will stop after CPU_MAX hours. 
                          ! If not specified, code will stop after Nbin bins.
/

&VAR_errors               ! Variables for analysis programs
n_skip  = 1               ! Number of bins that will be skipped. 
N_rebin = 1               ! Rebinning  
N_Cov   = 0               ! If set to 1 covariance will be computed
                          ! for non-equal-time correlation functions.                   
/            
\end{lstlisting}
%

The program allows for a number of different  updating schemes.  If no other variables are specified in the \texttt{VAR\_QMC} name space, then the program will run in its default mode, namely the sequential single spin-flip mode.   The additional, optional variables in   \texttt{VAR\_QMC}   include the following: 
\begin{lstlisting}[style=fortran]

&VAR_QMC                 ! Variables for the QMC run 
Propose_S0      = .true. ! Proposes single spin flip moves with probability exp(-S0) 
Global_moves    = .true. ! Allows for global moves in space and time 
N_Global        = 1      ! Number of global moves  per sweep 
Global_tau_moves= .true. ! Allows for global moves on a single time slice.  
N_Global_tau    = 10     ! Number of global moves that will be carried out on a 
                         ! single time slice
Nt_sequential_start = 1  ! One can combine sequential and global moves on 
                         ! a time slice.  
Nt_sequential_end =      ! The program will carry our sequential local moves in the
                         ! range [Nt_sequential_start, Nt_sequential_end] and then
                         ! N_Global_tau global moves
/   
\end{lstlisting}
Note that if \texttt{Nt\_sequential\_start}  and \texttt{Nt\_sequential\_end}  are not specified and that the variable \texttt{Global\_tau\_moves}  is set to true, then  the program will  carry out only global moves, by setting  \\  \texttt{Nt\_sequential\_start=1}  and \texttt{Nt\_sequential\_end=0}. 

If the program is compiled with the parallel tempering flag, then the additional name space \texttt{VAR\_TEMP} has to be included in the parameter file.
\begin{lstlisting}[style=fortran,escapechar=\%]

&VAR_TEMP                      ! Variables for parallel tempering
N_exchange_steps      = 6      ! Number of exchange moves %[see Eq.~\eqref{eq:exchangestep}]%
N_Tempering_frequency = 10     ! The frequency in units of sweeps at which the
                               ! exchange moves will be carried 
mpi_per_parameter_set = 2      ! Number of mpi-processes per parameter set
Tempering_calc_det    = .true. ! Specifies whether the fermion weight has to be taken
                               ! into account while tempering. The default is .true.,
                               ! and it can be set to .false. if the parameters that
                               ! get varied only enter the Ising action S_0
/
\end{lstlisting}

Additionally, in order for the maximum entropy code, described in Sec.~\ref{sec:maxent}, to be used, the namelist \texttt{VAR\_Max\_Stoch} should also be defined:
\begin{lstlisting}[style=fortran]

&VAR_Max_Stoch               ! Variables for Stochastic Maximum entropy
Ngamma     = 400             ! # of Dirac delta-functions for parametrization
Om_st      = 0               ! Frequency range lower bound
Om_en      = 8               ! Frequency range upper bound
NDis       = 2000            ! # of boxes for histogram
Nbins      = 250             ! # of bins for Monte Carlo
Nsweeps    = 70              ! # of sweeps per bin
NWarm      = 20              ! The Nwarm first bins will be ommitted
N_alpha    = 14              ! # of tempertures
alpha_st   = 1.d0            ! smallest inverse temperature
R          = 1.2d0           ! increment for inverse temperature (see above) 
Channel    = "P"             ! T0       : Zero temperature
                             ! P        : Finite temperarure particle 
                             ! PH       : Finite temperarure particle-hole
                             ! PP       : Finite temperarure particle-particle 
Checkpoint = .false.         !.true.    : dump files will be produced so as to be able
                             !            to restart the simulation
                             !.false.   : dump files will not be produced 
Tolerance  = 0.1d0           ! Data points for which the relative error exceeds the
                             ! tolerance threshold will be omitted.
/
\end{lstlisting}


%------------------------------------------------------------
\subsubsection{Output: Observables} \label{sec:output_obs}
%------------------------------------------------------------
%
\begin{table}[h]
   \begin{tabular}{@{} l l @{}}\toprule
   File & Description \\\midrule
   \path{info} & After completion of the simulation, this file documents the parameters of\\
   & the model, as well as the QMC run and simulation metrics (precision,\\
   & acceptance rate, wallclock time).\\
   \path{X_scal} & Results of equal-time measurements of scalar observables. \\
   & The placeholder \path{X} stands for the observables \path{Kin}, \path{Pot}, \path{Part}, and \path{Ener}. \\
   \path{Y_eq, Y_tau} & Results of equal-time and time-displaced measurements of correlation\\
   & functions. The placeholder \path{Y} stands for \path{Green}, \path{SpinZ}, \path{SpinXY}, and \path{Den}. \\   
   \path{confout_<thread number>} & Output files (one per MPI instance) for the HS and Ising configuration. \\\bottomrule
   \end{tabular}
   \caption{Overview of the standard output files. See Sec.~\ref{sec:obs} for the definitions of observables and correlation functions. \label{table:output}}
\end{table}
%
The standard output files are listed in Table~\ref{table:output}. 
The output of the measured data is organized in bins. One bin corresponds to the arithmetic average 
over a fixed number of individual measurements which depends 
on the chosen measurement interval \path{[LOBS_ST,LOBS_EN]} on the imaginary-time axis and on the number \path{NSweep} of Monte Carlo sweeps. If the user runs an MPI parallelized version of the code, the average also extends over the number of MPI threads. The formatting of a single bin's output depends on the observable type, \path{Obs_vec} or \path{Obs_Latt}:
\begin{itemize}
\item Observables of type \path{Obs_vec}:
For each additional bin, a single new line is added to the output file.
In case of an observable with \path{N_size} components, the formatting is 
\begin{verbatim}
N_size + 1    <measured value, 1> ... <measured value, N_size>    <measured sign>
\end{verbatim}
The counter variable \path{N_size+1} refers to the number of measurements per line, including the phase measurement. 
This format is required by the error analysis routine (see Sec.~\ref{sec:analysis}). 
Scalar observables like kinetic energy, potential energy, total energy and particle number are treated as a vector 
of size \path{N_size=1}.

\item Observables of type \path{Obs_Latt}:
For each additional bin, a new data block is added to the output file. 
The block consists of the expectation values [Eq.~(\ref{eqn:o})] contributing to the background part [Eq.~(\ref{eqn:s_back})] of the correlation function,
and the correlated part [Eq.~(\ref{eqn:s_corr})] of the correlation function.
For imaginary-time displaced correlation functions, the formatting of the block is given by:
\begin{alltt}
<measured sign>  <N_orbital>  <N_unit_cell>  <N_time_slices>  <dtau>
do alpha = 1, N_orbital
    \(\langle\hat{O}\sb{\alpha}\rangle \)
enddo
do i = 1, N_unit_cell
   <reciprocal lattice vector k(i)>
   do tau = 1, N_time_slices
      do alpha = 1, N_orbital
         do beta = 1, N_orbital
            \(\langle{S}\sb{\alpha,\beta}\sp{(\mathrm{corr})}(k(i),\tau)\rangle\)
         enddo
      enddo
   enddo
enddo
\end{alltt}
The same block structure is used for equal-time correlation functions, except for the entries  \path{<N_time_slices>} and \path{<dtau>}, which are then omitted.
Using this structure for the bins as input, the full correlation function $S_{\alpha,\beta}(\vec{k},\tau)$ [Eq.~(\ref{eqn:s})] is then calculated by calling the error analysis routine (see Sec.~\ref{sec:analysis}).
\end{itemize}

%
%------------------------------------------------------------
\subsubsection{Output: Precision} \label{sec:output_prec}
%------------------------------------------------------------
%


\red{[THIS SECTION MAYBE BELONGS INTO THE "RUNNING"...]}

The finite-temperature, auxiliary-field QMC algorithm is known to be numerically unstable, as discussed in Sec.~\ref{sec:stable}.
The numerical instabilities arise from the imaginary-time propagation, which invariably leads to exponentially small and exponentially large scales.
As shown in Ref.~\cite{Assaad08_rev}, scales can be omitted in the ground state algorithm -- thus rendering it very stable --  but have to be taken into account in the  finite-temperature code.

Numerical stabilization of the code is a delicate procedure that has been pioneered in Ref.~\cite{White89}  for the finite-temperature algorithm and in Refs.~\cite{Sugiyama86,Sorella89} for the zero-temperature projective algorithm.
It is important to be aware of the fragility of the numerical stabilization and that there is no guarantee that it will work for a given model. It is therefore crucial to always check the file \texttt{info}, which, apart from runtime data, contains important information concerning the stability of the code, in particular \texttt{Precision Green}.
If the numerical stabilization fails, one possible measure is to reduce the value of the parameter \texttt{Nwrap} in the parameter file, which will however also impact performance -- see Sec.~\ref{sec:optimize} for further optimization tips.

For performing the stabilization of the involved matrix multiplications we rely on routines from LAPACK. Notice that results are very likely to change
%significantly
depending on the specific implementation of the library used\footnote{The linked library should implement at least the LAPACK-3.4.0 interface.}.
In order to deal with this possibility, we offer a simple baseline which can be used as a quick check as tho whether results depend on the library used for linear algebra routines. Namely, we have included QR-decomposition related routines of the LAPACK-3.6.1 reference implementation from \url{http://www.netlib.org/lapack/}, which you can use by 
%including the switch \texttt{-DQRREF} into the \texttt{STABCONFIGURATION} string in the 
running the script \path{configureHPC.sh}, (described in Sec.~\ref{sec:running}), with the flag \texttt{STAB1} and recompiling ALF\footnote{This flag may trigger compiling issues, in particular, the Intel ifort compiler version 10.1 fails for all optimization levels.}.

In order to provide further flexibility, we offer various stabilization schemes that can be selected through the appropriate flags when running \texttt{configureHPC.sh}: \red{[MAKE A TABLE INSTEAD?]} \texttt{STAB1}, for using the reference stabilization scheme;
\texttt{STAB2}, which sets a stabilization scheme based on the QR decomposition, but not using the LAPACK reference implementation and with additional normalizations;
\texttt{STAB3}, for the newest and fastest stabilization, which separates large and small scales -- it generally works well, but there are models for which it fails;
and \texttt{LOG}, for using log storage for internal scales.

Typical values for the numerical precision can be found in the examples of Sec.~\ref{sec:ex} (see Sec.~\ref{sec:prec_charge} and \ref{sec:prec_spin}).

%------------------------------------------------------------
\subsection{Scripts}\label{sec:scripts}
%------------------------------------------------------------
%

\red{[IMPROVE or eliminate?]}

\begin{table}[h]
   \begin{tabular}{@{} l l l @{}}\toprule
   Script & Description & Section\\\midrule
   \path{Start/out_to_in.sh} & Copies the output field configurations to the respective input files. & \ref{sec:running} \\
   \path{Start/analysis.sh} & Starts the error analysis. & \ref{sec:analysis}\\\bottomrule
   \end{tabular}
   \caption{Overview of the bash script files. 
      \label{table:scripts}}
\end{table}
%

\section{Using the Code}\label{sec:running}
% Copyright (c) 2016 2017 The ALF project.
% This is a part of the ALF project documentation.
% The ALF project documentation by the ALF contributors is licensed
% under a Creative Commons Attribution-ShareAlike 4.0 International License.
% For the licensing details of the documentation see license.CCBYSA.

% !TEX root = Doc.tex
%
%-------------------------------------------------------------------------------------
\subsection{Running the code}\label{sec:running}
%-------------------------------------------------------------------------------------
%
In this section we describe the steps how to compile and run the code, as well as how to perform the error analysis of the data.
%
%-------------------------------------------------------------------------------------
\subsubsection{Compilation}
\label{sec:compilation}
%-------------------------------------------------------------------------------------
%
The environment variables and the directives to compile the code are set in the following makefile \texttt{Makefile}:
\lstset{style=bash}
\begin{lstlisting}

# -DMPI selects MPI.
# -DTEMPERING selects tempering mode.  MPI has to be switched on.
# -DSTAB1   Alternative stabilization, using the singular value decomposition.
# -DSTAB2   Alternative stabilization, lapack QR with  manual pivoting.
#           Packed form of QR factorization is not used.
# -DSTAB3   Alternative stabilization, using QR  with pivoting.
#           Internally, scales larger and smaller one are distinguished.
# -DLOG     Alternative stabilization, using QR  with pivoting.
#           Internally, scales are stored on log axsis to allow larger beta and
#           larger and smaller have to be distinguished.
# (no flag) Default  stabilization, using lapack QR with pivoting. 
#           Packed form of QR factorization  is used. 
# -DQRREF   Enables reference lapack implementation of QR decomposition.
# Recommendation: just use the -DMPI flag if you want to run in parallel or 
#                 leave it empty for serial jobs.  
#                 The default stabilization, no flag, is generically the best. 
#                 Consider using -DLOG if you run into overflows
PROGRAMCONFIGURATION = -DMPI 
PROGRAMCONFIGURATION = 
f90 = gfortran
export f90
F90OPTFLAGS = -O3 -Wconversion  -fcheck=all
F90OPTFLAGS = -O3
export F90OPTFLAGS
F90USEFULFLAGS = -cpp -std=f2003
F90USEFULFLAGS = -cpp
export F90USEFULFLAGS
FL = -c ${F90OPTFLAGS} ${PROGRAMCONFIGURATION}
export FL
DIR = ${CURDIR}
export DIR
Libs = ${DIR}/Libraries/
export Libs
LIB_BLAS_LAPACK = -llapack -lblas
export LIB_BLAS_LAPACK

all: lib ana program

lib:
	cd Libraries && $(MAKE)
ana:
	cd Analysis && $(MAKE)
program:
	cd Prog && $(MAKE)


clean: cleanall
cleanall: cleanprog cleanlib cleanana
cleanprog:
	cd Prog && $(MAKE) clean
cleanlib:
	cd Libraries && $(MAKE) clean
cleanana:
	cd Analysis && $(MAKE) clean
help:
	@echo "The following are some of the valid targets of this Makefile"
	@echo "all, program, lib, ana, clean, cleanall, cleanprog, cleanlib,
	       cleanana"

\end{lstlisting}
In the above, the GNU Fortan compiler \texttt{gfortran} is set.\footnote{A known issue with the alternative Intel Fortran compiler \texttt{ifort} is the handling of automatic, temporary arrays 
which \texttt{ifort} allocates on the stack. For large system sizes and/or low temperatures this may lead to 
a runtime error. One solution is to demand allocation of arrays above a certain size on the heap instead of the stack. 
This is accomplished by the \texttt{ifort} compiler flag \texttt{-heap-arrays [n]} where \texttt{[n]} is the minimal size (in kilobytes, for example \texttt{n=1024}) of arrays 
that are allocated on the heap.}
We provide a set of options for compilation of the QMC code. The present options are \texttt{-DMPI}, \texttt{-DQRREF}, \texttt{-DSTAB1}, and \texttt{-DSTAB2}. 
They can be included in the string variable \texttt{PROGRAMCONFIGURATION} by the user, as shown above.
The program can be compiled and ran either in single-thread mode (default) or 
in multi-threading mode (define \texttt{-DMPI}) using the MPI standard for parallelization. The remaining three compiler options select a particular stabilization scheme for the matrix multiplications (see Sec.~\ref{sec:output_prec}).
To compile the libraries, the analysis routines and the QMC program at once, just execute the single command:
\begin{verbatim}
make
\end{verbatim}
To clean up all directories and remove the object files and executables, execute the command \texttt{make clean}. As can be seen in the above makefile, there exist also rules to compile/clean up the library, the analysis routines and the QMC program separately.  

%
%-------------------------------------------------------------------------------------
\subsubsection{Starting a simulation}
%-------------------------------------------------------------------------------------
%
To start a simulation from scratch, the following files have to be present: \texttt{parameters} and \texttt{seeds}. 
To run a single-thread simulation, for example by using the parameters of one of the  Hubbard models described in Sec.~\ref{sec:ex}, issue the command
\begin{verbatim}
./Prog/Examples.out
\end{verbatim}
To restart the code using an existing simulation as a starting point, first run the script \texttt{out\_to\_in.sh} to set 
the input configuration files.
%
%-------------------------------------------------------------------------------------
\subsubsection{Error analysis}
%-------------------------------------------------------------------------------------
%
Note that the error analysis script requires the presence of the environment variable \path{DIR} which defines the path to the error analysis programs.
So before starting the error analysis, one has to make this variable available which is done by the script \path{setenv.sh}. The command is
\begin{verbatim}
source ./setenv.sh
\end{verbatim}
To perform an error analysis based on the Jackknife resampling method (Sec.~\ref{sec:jack})  of the Monte Carlo bins for all observables run the script \texttt{analysis.sh} 
(see Sec.~\ref{sec:analysis}). In case that the parameter \path{N_auto} is set to a finite value the script will also trigger the computation of autocorrelation functions (Sec.~\ref{sec:autocorr}).


\section{The plain vanilla Hubbard model on the square lattice} \label{sec:vanilla}
% Copyright (c) 2016-2019 The ALF project.
% This is a part of the ALF project documentation.
% The ALF project documentation by the ALF contributors is licensed
% under a Creative Commons Attribution-ShareAlike 4.0 International License.
% For the licensing details of the documentation see license.CCBYSA.

% !TEX root = doc.tex

All the data structures necessary to implement a given model have been introduced in the previous sections. Here we show how to implement the Hubbard model  by specifying the lattice, the hopping, the interaction, the trial wave function  (if  required), and the observables.  Consider  the  \textit{plain vanilla}  Hubbard model  written as: 
\begin{equation}
\label{eqn_hubbard_Mz}
\mathcal{H}=
- t 
\sum\limits_{\langle \ve{i}, \ve{j} \rangle,  \sigma={\uparrow,\downarrow}} 
  \left(  \hat{c}^{\dagger}_{\ve{i}, \sigma} \hat{c}^{\phantom\dagger}_{\ve{j},\sigma}  + \hc \right) 
- \frac{U}{2}\sum\limits_{\ve{i}}\left[
\hat{c}^{\dagger}_{\ve{i}, \uparrow} \hat{c}^{\phantom\dagger}_{\ve{i}, \uparrow}  -   \hat{c}^{\dagger}_{\ve{i}, \downarrow} \hat{c}^{\phantom\dagger}_{\ve{i}, \downarrow}  \right]^{2}   
-  \mu \sum_{\ve{i},\sigma } \hat{c}^{\dagger}_{\ve{i}, \sigma}  \hat{c}^{\phantom\dagger}_{\ve{i},\sigma}. 
\end{equation} 
Here $ \langle \ve{i}, \ve{j} \rangle $    denotes nearest neighbors. 
We can make contact with the general form of the Hamiltonian  [see Eq.~\eqref{eqn:general_ham}] by setting: 
$N_{\mathrm{fl}} = 2$, $N_{\mathrm{col}} \equiv \texttt{N\_SUN}     =1 $, 
 $M_T    =    1$, 
 \begin{equation}
  T^{(ks)}_{x y}   = 
  \left\{ 
 \begin{array}{ll}
       -t         & \text{if } x,y \text{ are nearest neighbors} \\
       -\mu    & \text{if } x = y \\
       0         &  \text{otherwise},
 \end{array}
  \right.
 \end{equation}
 $M_V   =  N_{\text{unit-cell}} $,  $U_{k}       =   \frac{U}{2}$, 
 $V_{x y}^{(k, s=1)} =  \delta_{x,y} \delta_{x,k}  $,  $V_{x y}^{(k, s=2)} =  - \delta_{x,y} \delta_{x,k}  $,  $\alpha_{ks}   = 0  $ and $M_I       = 0 $.   
The coupling of the HS fields to the $z$-component of the magnetization breaks the SU(2) spin symmetry. Nevertheless, the $z$-component of the spin remains a good quantum number such that the imaginary-time propagator -- for a given HS field -- is block  diagonal in this quantum number. This corresponds to the flavor index running from $1$ to $2$,  labeling spin up and spin down degrees of freedom. We note that  in this formulation the  hopping matrix can be flavor dependent such that a Zeeman  magnetic field can be introduced.  If the chemical potential is set to zero, this will not generate a negative sign problem \cite{Wu04,Milat04,Bercx09}.    
The code that we describe below  can be found in the submodule \path{Prog/Hamiltonians/Hamiltonian_plain_vanilla_hubbard_smod.F90}. This file may be a good starting point for implementing a new model Hamiltonian. 

%------------------------------------------------------------
\subsection{Setting the Hamiltonian:  \texttt{Ham\_set} }
%------------------------------------------------------------

The main program will call the subroutine \texttt{Ham\_set} in the submodule \path{Hamiltonian_plain_vanilla_hubbard_smod.F90}.
The latter  subroutine  defines the  public variables
\begin{lstlisting}[style=fortran]
Type(Operator),     dimension(:,:), allocatable :: Op_V  ! Interaction
Type(Operator),     dimension(:,:), allocatable :: Op_T  ! Hopping
Type(WaveFunction), dimension(:),   allocatable :: WF_L  ! Left trial wave function
Type(WaveFunction), dimension(:),   allocatable :: WF_R  ! Right trial wave function
Type(Fields)        :: nsigma                            ! Fields
Integer             :: Ndim                              ! Number of sites
Integer             :: N_FL                              ! number of flavors
Integer             :: N_SUN	                         ! Number of colors 
Integer             :: Ltrot                             ! Total number of trotter silces
Integer             :: Thtrot                            ! Number of trotter slices 
                                                         ! reserved for projection
Logical             :: Projector                         ! Projector code
Integer             :: Group_Comm                        ! Group communicator for MPI
Logical             :: Symm                              ! Symmetric trotter 
\end{lstlisting}
which specify the model.  The  routine \texttt{Ham\_set}  will first  read the parameter file \texttt{parameters} (see Sec.~\ref{sec:input}); then set the lattice: \texttt{Call Ham\_latt};  set the hopping: \texttt{Call Ham\_hop};  
 set the interaction: \texttt{call Ham\_V}; and if required, set the trial wave function: \texttt{call Ham\_trial}.

%------------------------------------------------------------
\subsection{The lattice: \texttt{Ham\_latt}} \label{U_PV_Ham_latt}
%------------------------------------------------------------

The routine, which sets the square lattice, reads:
\begin{lstlisting}[style=fortran]
a1_p(1) = 1.0  ; a1_p(2) = 0.d0
a2_p(1) = 0.0  ; a2_p(2) = 1.d0
L1_p    = dble(L1)*a1_p
L2_p    = dble(L2)*a2_p
Call Make_Lattice(L1_p, L2_p, a1_p, a2_p, Latt)
Latt_unit%Norb = 1
Latt_unit%N_coord = 2
allocate(Latt_unit%Orb_pos_p(Latt_unit%Norb,2))
Latt_unit%Orb_pos_p(1, :) = [0.d0, 0.d0]
Ndim = Latt%N*Latt_unit\%Norb

\end{lstlisting}
In its last line, the routine sets the total number of single particle states per flavor and color:
\texttt{Ndim = Latt\%N*Latt\_unit\%Norb}.

%------------------------------------------------------------
\subsection{The hopping: \texttt{Ham\_hop}} \label{U_PV_Ham_hop}
%------------------------------------------------------------

The hopping matrix is implemented as follows. 
We allocate an array of dimension $1\times N_{\mathrm{fl}}$ of type operator  called \texttt{Op\_T} and set the  dimension for the hopping  matrix to $N=N_{\mathrm{dim}}$. The operator allocation and initialization is performed by the subroutine \texttt{Op\_make}: 
\begin{lstlisting}[style=fortran]
do nf = 1,N_FL
   call Op_make(Op_T(1,nf),Ndim)
enddo
\end{lstlisting}
Since the hopping  does not  break down into small blocks, we have ${\bm P}=\mathds{1}$   and  
\begin{lstlisting}[style=fortran]
Do nf = 1, N_FL
  Do i = 1,Latt%N
     Op_T(1,nf)%P(i) = i
  Enddo
Enddo
\end{lstlisting}
We set the hopping matrix  with 
\begin{lstlisting}[style=fortran]
Do nf = 1, N_FL
   Do I = 1, Latt%N
      Ix = Latt%nnlist(I,1,0)
      Iy = Latt%nnlist(I,0,1)
      Op_T(1,nf)%O(I,  Ix) = cmplx(-Ham_T,    0.d0, kind(0.D0))
      Op_T(1,nf)%O(Ix, I ) = cmplx(-Ham_T,    0.d0, kind(0.D0))
      Op_T(1,nf)%O(I,  Iy) = cmplx(-Ham_T,    0.d0, kind(0.D0))
      Op_T(1,nf)%O(Iy, I ) = cmplx(-Ham_T,    0.d0, kind(0.D0))
      Op_T(1,nf)%O(I,  I ) = cmplx(-Ham_chem, 0.d0, kind(0.D0))
   Enddo
   Op_T(1,nf)%g     = -Dtau
   Op_T(1,nf)%alpha = cmplx(0.d0,0.d0, kind(0.D0))
   Call Op_set(Op_T(1,nf))
Enddo
\end{lstlisting}
Here, the integer function \texttt{Latt\%nnlist(I,n,m)} is defined in the lattice module and returns the index of the lattice site $ \vec{I} +  n \vec{a}_1 +  m \vec{a}_2$.
Note that periodic boundary conditions are 
already taken into account.  The hopping parameter \texttt{Ham\_T}, as well as the chemical potential \texttt{Ham\_chem} are read from the parameter file.  
To completely define the hopping  we further set: \texttt{Op\_T(1,nf)\%g = -Dtau }, \texttt{Op\_T(1,nf)\%alpha = cmplx(0.d0,0.d0, kind(0.D0))} and call the routine  \texttt{Op\_set(Op\_T(1,nf))}  so as to generate  the unitary transformation and eigenvalues as specified in Table \ref{table:operator}.  Recall that for the hopping, the variable  \texttt{Op\_set(Op\_T(1,nf))\%type}  takes its default value of 0.  
Finally, note that, although a checkerboard decomposition is not used here, it can be implemented by considering a larger number of sparse hopping matrices.  


%------------------------------------------------------------
\subsection{The interaction: \texttt{Ham\_V}}\label{U_PV_Ham_V} 
%------------------------------------------------------------
To implement  the interaction, we allocate an array of \texttt{Operator} type. The array is called  \texttt{Op\_V} and has dimensions $N_{\mathrm{dim}}\times N_{\mathrm{fl}}=N_{\mathrm{dim}} \times 2$. 
We set the dimension for the interaction term to  $N=1$, and  allocate and initialize this array of type  \texttt{Operator} by repeatedly calling the subroutine \texttt{Op\_make}: 

\begin{lstlisting}[style=fortran]
Allocate(Op_V(Ndim,N_FL))
do nf = 1,N_FL
   do i  = 1, Ndim
      Call Op_make(Op_V(i,nf), 1)
   enddo
enddo
Do nf = 1,N_FL
   X = 1.d0
   if (nf == 2)  X = -1.d0
   Do i = 1,Ndim
      nc = nc + 1
      Op_V(i,nf)%P(1)   = I
      Op_V(i,nf)%O(1,1) = cmplx(1.d0, 0.d0, kind(0.D0))
      Op_V(i,nf)%g      = X*SQRT(CMPLX(DTAU*ham_U/2.d0, 0.D0, kind(0.D0))) 
      Op_V(i,nf)%alpha  = cmplx(0.d0, 0.d0, kind(0.D0))
      Op_V(i,nf)%type   = 2
      Call Op_set( Op_V(i,nf) )
   Enddo
Enddo
\end{lstlisting}
The code above makes it explicit that there is a sign difference between the coupling of the HS field in the two flavor sectors. 

%------------------------------------------------------------
\subsection{The trial wave function: \texttt{Ham\_Trial}} \label{U_PV_Ham_Trial}
%------------------------------------------------------------
\label{Sec:Plain_vanilla_trial}
As  argued in Sec.~\ref{sec:trial_wave_function}, it is useful to generate the trial wave function from a non-interacting trial Hamiltonian.   Here we will  use the same left and right  flavor-independent trial wave functions that correspond to the ground state of: 
\begin{equation}
   \hat{H}_T    = - t \sum_{\ve{i}} \left[  \left( 1 + (-1)^{i_x + i_y}  \delta \right)  \hat{c}^{\dagger}_{\ve{i}}   \hat{c}^{\phantom\dagger}_{\ve{i} +\ve{a}_x}  +  
   							\left(1 - \delta \right)  \hat{c}^{\dagger}_{\ve{i}}   \hat{c}^{\phantom\dagger}_{\ve{i} +\ve{a}_y}    + \hc  \right]   \equiv   \sum_{\ve{i},\ve{j}}  \hat{c}^{\dagger}_{\ve{i}}   h_{\ve{i},\ve{j}}  \hat{c}^{\phantom\dagger}_{\ve{i}}.
\end{equation}
For the half-filled case, the  dimerization $\delta  = 0^{+} $  opens up a gap at  half-filling,   thus generating the desired  non-degenerate  trial wave function  that has the same symmetries (particle-hole  for instance) as  the   trial  Hamiltonian.

Diagonalization  of  $ h_{\ve{i},\ve{j}}$,      $U^{\dagger} h  U  = \mathrm{Diag} \left(   \epsilon_1, \cdots, \epsilon_{N_{\mathrm{dim}}} \right) $     with  $\epsilon_i  <  \epsilon_j $  for $i < j$, allows us  to define the  trial wave function.  In particular, for the half-filled case, we set 
\begin{lstlisting}[style=fortran,escapechar=\#]
Do s = 1, N_fl
   Do x = 1,Ndim
      Do n = 1, N_part
         WF_L(s)%P(x,n)  = # $U_{x,n}$ #
         WF_R(s)%P(x,n)  = # $U_{x,n}$ #
      Enddo
   Enddo
Enddo
\end{lstlisting}
with \texttt{N\_part = Ndim/2}.     The  variable \texttt{Degen}   belonging to the \texttt{WaveFunction}  type  is given by  \texttt{Degen}$=\epsilon_{N_{\mathrm{Part}} +1 } - \epsilon_{N_{\mathrm{Part}}  }$.   This quantity should be greater than zero  for non-degenerate trial wave functions. 

%------------------------------------------------------------
\subsection{Observables}
%------------------------------------------------------------

At this point, all the information for starting the simulation has been provided.  The code will sequentially go through  the operator list  \texttt{Op\_V}  and update the  fields.  Between  time slices  \texttt{LOBS\_ST}  and  \texttt{LOBS\_EN} the main program will call the routine  \texttt{Obser(GR,Phase,Ntau)}, which handles equal-time correlation functions, and, if \texttt{Ltau=1}, the routine \texttt{ObserT(NT,  GT0,G0T,G00,GTT, PHASE)} which handles imaginary-time displaced correlation functions. 

Both \texttt{Obser} and \texttt{ObserT} should be provided by the user, who can either implement themselves the observables they want to compute or use the predefined structures of Chap.~\ref{Predefined_chap}. Here we describe how to proceed in order to define an observable. 

%------------------------------------------------------------------------------------
\subsubsection[Allocating space for the observables: \texttt{Alloc\_obs}]{Allocating space for the observables: \texttt{Alloc\_obs(Ltau)}} \label{Alloc_obs_sec}
%-------------------------------------------------------------------------------------

For  four scalar  or vector observables,  the user will have to  declare the following: 
\begin{lstlisting}[style=fortran]
Allocate ( Obs_scal(4) )
Do I = 1,Size(Obs_scal,1)
   select case (I)
   case (1)
      N = 2;  Filename ="Kin"
   case (2)
      N = 1;  Filename ="Pot"
   case (3)
      N = 1;  Filename ="Part"
   case (4)
      N = 1,  Filename ="Ener"
   case default
      Write(6,*) ' Error in Alloc_obs '  
   end select
   Call Obser_Vec_make(Obs_scal(I), N, Filename)
enddo
\end{lstlisting}
Here,   \texttt{Obs\_scal(1)}   contains a vector  of two observables  so as to account for the $x$- and $y$-components of the kinetic energy, for example.  

For equal-time correlation functions  we allocate  \texttt{Obs\_eq}  of type \texttt{Obser\_Latt}.  Here we include the calculation of spin-spin and density-density correlation functions alongside equal-time Green functions. 
\begin{lstlisting}[style=fortran]
Allocate ( Obs_eq(5) )
Do I = 1,Size(Obs_eq,1)
   select case (I)
   case (1)
      Filename = "Green"
   case (2)
      Filename = "SpinZ"
   case (3)
      Filename = "SpinXY"
   case (4)
      Filename = "SpinT"
   case (5)
      Filename = "Den"
   case default
      Write(6,*) "Error in Alloc_obs"
   end select
   Nt = 1
   Channel = "--"
   Call Obser_Latt_make(Obs_eq(I), Nt, Filename, Latt, Latt_unit, Channel, dtau)
Enddo
\end{lstlisting} 
Be aware that \texttt{Obser\_Latt\_make} does not copy the Bravais lattice \texttt{Latt} 
and unit cell \texttt{Latt\_unit}, but links them through pointers to be more memory 
efficient. One can have different lattices attached to different observables by declaring 
additional instances of \texttt{Type(Lattice)} and \texttt{Type(Unit\_cell)}.
 For equal-time correlation functions, we set \texttt{Nt = 1} and \texttt{Channel} specification is not necessary.

If \texttt{Ltau = 1}, then the code allocates space for time displaced quantities. The 
same structure as for equal-time correlation functions is used, albeit with 
\texttt{Nt = Ltrot + 1} and the channel should be set. Whith \texttt{Channel="PH"}, for instance, the analysis algorithm assumes the observable to be particle-hole symmetric. For more details on this parameter, see Sec.~\ref{sec:maxent}.

At the beginning of each bin, the main program will set the bin observables to zero by calling  the routine \texttt{Init\_obs(Ltau)}. The user does not have to edit this routine. 
 
%-------------------------------------------------------------------------------------
\subsubsection[Measuring equal-time observables: \texttt{Obser}]{Measuring equal-time observables: \texttt{Obser(GR,Phase,Ntau)}} \label{sec:EqualTimeobs}
%-------------------------------------------------------------------------------------

Having allocated the necessary memory, we proceed to define the observables. The equal-time  Green function,
\begin{equation}
	 \texttt{GR(x,y},\sigma{\texttt)}  = \langle \hat{c}^{\phantom{\dagger}}_{x,\sigma} \hat{c}^{\dagger}_{y,\sigma}  \rangle,
\end{equation}
the  phase factor \texttt{phase} [Eq.~(\ref{eqn:phase})], and time slice \texttt{Ntau}   are provided by the main program.  

Here,   $x$ and $y$ label  both unit cell as well as the orbital within the unit cell. For the Hubbard model described here, $x$ corresponds to the unit cell.  The Green function  does not depend on the color index, and is diagonal in flavor.  For the SU(2) symmetric implementation  there is only one flavor, $\sigma = 1$ and the Green function is  independent on the spin index.  This renders the calculation of the observables particularly easy.   

An explicit calculation of the   potential energy  $ \langle U \sum_{\vec{i}}  \hat{n}_{\vec{i},\uparrow}   \hat{n}_{\vec{i},\downarrow}  \rangle $ reads 
\begin{lstlisting}[style=fortran]
Obs_scal(2)%N        = Obs_scal(2)%N + 1
Obs_scal(2)%Ave_sign = Obs_scal(2)%Ave_sign + Real(ZS,kind(0.d0))
Do i = 1,Ndim
  Obs_scal(2)%Obs_vec(1)= Obs_scal(2)%Obs_vec(1) +(1-GR(i,i,1))*(1-GR(i,i,2))*Ham_U*ZS*ZP
Enddo
\end{lstlisting} 
Here  $ \texttt{ZS} = \sgn(C) $  [see Eq.~(\ref{Sign.eq})],  $ \texttt{ZP} =   \frac{e^{-S(C)}} {\Re \left[e^{-S(C)} \right]}   $ [see Eq.~(\ref{eqn:phase})] and  \texttt{Ham\_U}  corresponds to the Hubbard  $U$ term.

Equal-time correlations  are also computed in this routine. As an explicit example, we  consider the equal-time density-density correlation:
\begin{equation}
	 \langle \hat{n}_{\vec{i}}   \hat{n}_{\vec{j}} \rangle   -  \langle \hat{n}_{\ve{i} }\rangle  \langle    \hat{n}_{\ve{j}}  \rangle,
\end{equation} 
with
\begin{equation}
	 \hat{n}_{\vec{i}}  =   \sum_{\sigma} \hat{c}^{\dagger}_{\ve{i},\sigma} \hat{c}^{\phantom\dagger}_{\ve{i},\sigma}.
\end{equation}
For the calculation of such quantities, it is convenient to  define: 
\begin{equation}
\label{GRC.eq}
	\texttt{GRC(x,y,s)}   =  \delta_{x,y}  - \texttt{GR(y,x,s)  }
\end{equation}
such that \texttt{GRC(x,y,s)}    corresponds to  $ \langle \langle  \hat{c}_{x,s}^{\dagger}\hat{c}_{y,s}^{\phantom\dagger} \rangle \rangle $. 
In the program code, the calculation of the equal-time density-density correlation function looks as follows:
\begin{lstlisting}[style=fortran]
Obs_eq(4)%N = Obs_eq(4)%N + 1           ! Even if it is redundant, each observable  
                                        ! carries its own counter and sign.
Obs_eq(4)%Ave_sign = Obs_eq(4)%Ave_sign + Real(ZS,kind(0.d0))  
Do I = 1,Ndim
   Do J = 1,Ndim                       
      imj = latt%imj(I,J)
      Obs_eq(4)%Obs_Latt(imj,1,1,1) =  Obs_eq(4)%Obs_Latt(imj,1,1,1) + &
                     &     ( (GRC(I,I,1)+GRC(I,I,2)) * (GRC(J,J,1)+GRC(J,J,2))       + &
                     &        GRC(I,J,1)*GR(I,J,1)   +  GRC(I,J,2)*GR(I,J,2)  ) * ZP * ZS 
   Enddo
   Obs_eq(4)%Obs_Latt0(1) = Obs_eq(4)%Obs_Latt0(1) + (GRC(I,I,1)+GRC(I,I,2))*ZP*ZS
Enddo
\end{lstlisting} 
At the end of each bin the main program calls the routine \texttt{ Pr\_obs(LTAU)}. This routine appends the result for the current bins to the corresponding file, with the appropriate suffix. 

%-------------------------------------------------------------------------------------
\subsubsection[Measuring time-displaced observables: \texttt{ObserT}]{Measuring time-displaced observables: \texttt{ObserT(NT, GT0, G0T, G00, GTT, PHASE)}}  \label{sec:TimeDispObs}
%-------------------------------------------------------------------------------------
%
This subroutine is called by the main program at the beginning of each sweep, provided that \texttt{LTAU}  is set to $1$. The variable \texttt{NT} runs from \texttt{0}  to \texttt{Ltrot} and denotes the imaginary time difference. For a given time  displacement, the main program provides:
\begin{align}
\begin{aligned}
\label{Time_displaced_green.eq}
\texttt{GT0(x,y,s) }  &=   \phantom{+} \langle \langle \hat{c}^{\phantom\dagger}_{x,s} (Nt \Delta \tau)   \hat{c}^{\dagger}_{y,s} (0)   \rangle \rangle \;=\; \langle \langle \mathcal{T} \hat{c}^{\phantom\dagger}_{x,s} (Nt \Delta \tau)   \hat{c}^{\dagger}_{y,s} (0)   \rangle \rangle   \\
\texttt{G0T(x,y,s) }   &=  -   \langle \langle   \hat{c}^{\dagger}_{y,s} (Nt \Delta \tau)    \hat{c}^{\phantom\dagger}_{x,s} (0)    \rangle \rangle \;=\;
    \langle \langle \mathcal{T} \hat{c}^{\phantom\dagger}_{x,s} (0)    \hat{c}^{\dagger}_{y,s} (Nt \Delta \tau)   \rangle \rangle    \\
  \texttt{G00(x,y,s) }  &=    \phantom{+} \langle \langle \hat{c}^{\phantom\dagger}_{x,s} (0)   \hat{c}^{\dagger}_{y,s} (0)   \rangle \rangle     \\
    \texttt{GTT(x,y,s) }  &=   \phantom{+} \langle \langle \hat{c}^{\phantom\dagger}_{x,s} (Nt \Delta \tau)   \hat{c}^{\dagger}_{y,s} (Nt \Delta \tau)   \rangle \rangle.
\end{aligned}
\end{align}
In the above we have omitted the color index since  the  Green functions are color independent.  The time-displaced spin-spin correlations 
$ 4 \langle \langle \hat{S}^{z}_{\vec{i}} (\tau)  \hat{S}^{z}_{\vec{j}} (0)\rangle \rangle   $ 
are then given by: 
\begin{multline}
	4 \langle \langle \hat{S}^{z}_{\vec{i}} (\tau)  \hat{S}^{z}_{\vec{j}} (0)\rangle \rangle
	=  ( \texttt{GTT(I,I,1)} -  \texttt{GTT(I,I,2)} ) * ( \texttt{G00(J,J,1)} -  \texttt{G00(J,J,2)} )     \\  
	-   \; \texttt{G0T(J,I,1)}*\texttt{GT0(I,J,1)}  -  \texttt{G0T(J,I,2)}* \texttt{GT0(I,J,2)}
\end{multline}

The handling of time-displaced correlation functions is identical to that of equal-time correlations. 

%-------------------------------------------------------------------------------------
\subsection{Numerical precision}\label{sec:prec_spin}
%-------------------------------------------------------------------------------------

Information on the numerical stability is included in the following lines of the corresponding file \texttt{info}. 
For a  \textit{short} simulation on a $4 \times 4$  lattice at $U/t=4$ and $\beta t = 10$  we obtain
\begin{lstlisting}[basicstyle=\ttfamily\small,columns=fullflexible,keepspaces=true]
Precision Green  Mean, Max :   5.0823874429126405E-011  5.8621144596315844E-006
Precision Phase  Max       :   0.0000000000000000    
Precision tau    Mean, Max :   1.5929357848647394E-011  1.0985132530727526E-005 
\end{lstlisting}
showing the mean and maximum difference between the \textit{wrapped}  and from scratched computed equal and time-displaced  Green functions \cite{Assaad08_rev}.
A stable code  should produce results where the mean difference is smaller than the  stochastic error. The above example  shows a very stable simulation since the Green function  is of order one. 

\subsection{Running the code and testing}

To test the code, one can carry out high precision simulations. After compilation, the executable \texttt{ALF.out} is found in the directory \texttt{\$ALF\_DIR/Prog/} and can be run from any directory containing the files \texttt{parameters} and \texttt{seeds} (See Sec.~\ref{sec:files}).

Alternatively, as we do bellow, it may be convenient to use \texttt{pyALF} to compile and run the code, especially when using one of the scripts or notebooks available.  

\paragraph*{One-dimensional case} 

The \texttt{pyALF} python script   \href{https://git.physik.uni-wuerzburg.de/ALF/pyALF/-/blob/master/Scripts/Hubbard_Plain_Vanilla.py}{\texttt{Hubbard\_Plain\_Vanilla.py}}   runs the projective version of the code for the four-site Hubbard model.  At $\theta t =10$, $\Delta \tau t = 0.05 $ with the symmetric Trotter  decomposition, we obtain after 40 bins of 2000 sweeps each the total energy:   
\begin{equation*}
       \langle  \hat{H}   \rangle = -2.103750  \pm      0.004825,
 \end{equation*}
and the exact result is  
\begin{equation*}
\langle  \hat{H}   \rangle_{\texttt{Exact}}    = -2.100396.
\end{equation*}

\paragraph*{Two-dimensional case}  
For the two-dimensional case,   with similar parameters, we obtain the results listed in Table~\ref{tab:2dplain}.
\begin{table}[h!]
\begin{center}
\begin{tabular}{l l l}
\toprule
             &  QMC  & Exact  \\ \midrule
Total energy & -13.618   $\pm $  0.002 &  -13.6224  \\
 $\ve{Q}=(\pi,\pi)$ spin correlations &  \phantom{-1}3.630     $ \pm $   0.006     & \phantom{-1}3.64 \\ 
 \bottomrule
\end{tabular}
\caption{Test results for the \texttt{Hubbard\_Plain\_Vanilla} code on a two-dimensional lattice with default parameters.} \label{tab:2dplain}
\end{center}
\end{table}
The exact results stem from Ref.~\cite{Parola91}     and the slight discrepancies from the exact results can be  assigned to the finite value of $\Delta \tau$.  Note that all the simulations were carried out with the default value of the Hubbard interaction, $U/t =4$. 

\section{Predefined Structures}\label{sec:predefined}
% Copyright (c) 2016-2019 The ALF project.
% This is a part of the ALF project documentation.
% The ALF project documentation by the ALF contributors is licensed
% under a Creative Commons Attribution-ShareAlike 4.0 International License.
% For the licensing details of the documentation see license.CCBYSA.

% !TEX root = doc.tex





ALF includes modules providing predefined structures which the user can combine together or use as templates for defining new structures, namely: 
\begin{itemize}
	\item lattices and unit cells -- \texttt{Predefined\_Latt\_mod.F90}
	\item hopping Hamiltonians -- \texttt{Predefined\_Hop\_mod.F90 }
	\item interaction Hamiltonians -- \texttt{Predefined\_Int\_mod.F90}
	\item observables -- \texttt{Predefined\_Obs\_mod.F90 }
	\item trial wave functions -- \texttt{Predefined\_Trial\_mod.F90 }
\end{itemize}
which are defined using the data structures defined in the Sec.~\ref{sec:imp}, as described in this section.



%%%%%%%%%%%%%%%%%%%%%%%%%%%%
% !TEX root = doc.tex
% Copyright (c) 2017 The ALF project.
% This is a part of the ALF project documentation.
% The ALF project documentation by the ALF contributors is licensed
% under a Creative Commons Attribution-ShareAlike 4.0 International License.
% For the licensing details of the documentation see license.CCBYSA.
%
%-----------------------------------------------------------------------------------
\subsection{Generic hopping matrix elements}\label{sec:generic_hopping}
%-----------------------------------------------------------------------------------


Here we compute the hopping matrix element  between two sites of a given lattice  in the presence of twisted boundary conditions and  orbital magnetic field. 
The generic Hopping Hamiltonian will read: 
\begin{equation}
	   \hat{H}_T = \sum_{(i,\delta), (j,\delta'), s, \sigma}    T_{(i,\delta), (j,\delta')}^{(s)}    c^{\dagger}_{(i,\delta),s,\sigma }   e^{\frac{2 \pi i}{\Phi_0} \int_{i + \delta}^{j + \delta'}  \vec{A}(\vec{l})  d \vec{l}} c^{}_{(j,\delta'),s,\sigma }
\end{equation}
with boundary conditions 
\begin{equation}
	c^{\dagger}_{(i + L_i,\delta) ,s,\sigma }   =  e^{- 2 \pi i\frac{\Phi_i}{\Phi_0}} \, e^{\frac{2 \pi i }{\Phi_0} \chi_{L_1} ( i + \delta ) } \, c^{\dagger}_{(i,\delta) ,s,\sigma }.
\end{equation}
The vector potential accounts for an orbital magnetic field that is implemented  in the Landau  gauge:  $\vec{A}(\vec{x})  =  -B(y,0,0) $ with $ \vec{x} = (x,y,z)$. $\Phi_0$ corresponds to the flux  quanta and the scalar function $\chi$ is defined  through as:
\begin{equation}
	\vec{A}( \vec{x} + \vec{L}_{i} )  = \vec{A}( \vec{x} )   +  \vec{\nabla} \chi_{L_{\alpha}}(\vec{x}). 
\end{equation}

 Provided that the bare hopping Hamiltonian, $T$,  is invariant under lattice translations, $\hat{H}_T$ commutes with magnetic translations  that satisfy the  Algebra: 
\begin{equation}
	\hat{T}_{\vec{a}} \hat{T}_{\vec{b}} =  e^{ \frac{2 \pi i}{\Phi_0}   \vec{B} \cdot \left( \vec{a} \times \vec{b} \right) }  \hat{T}_{\vec{b}} \hat{T}_{\vec{a}}. 
\end{equation}
On the  torus, the uniqueness of the wave functions requires that  $\hat{T}_{\vec{L}_1} \hat{T}_{\vec{L}_2}  =   \hat{T}_{\vec{L}_2} \hat{T}_{\vec{L}_1} $ such
that
\begin{equation}
	 \frac{\vec{B} \cdot \left( \vec{a} \times \vec{b}  \right) }{\Phi_0 } = N_{\Phi}   
\end{equation}
with  $N_\Phi $ an integer  The variable \texttt{N\_Phi},   specified in the parameter file,   denotes the number of flux quanta piercing the lattice.    The variables \texttt{Phi\_1}  and   \texttt{Phi\_2} also   in the parameter file denote  the twists  -- in units of the flux quanta  --  along the $\vec{L}_1$ and  $\vec{L}_2$ directions.     There are gauge  equivalent ways to insert the  twist in the boundary conditions. In the above we  have inserted   twist as a boundary condition such  for example setting  \texttt{Phi\_1=0.5}  corresponds to anti-periodic boundary conditions along the $L_1$  axis.   Alternatively we  can  consider the 
Hamiltonian:
\begin{equation}
	   \hat{H}_T = \sum_{(i,\delta), (j,\delta'), s, \sigma}    T_{(i,\delta), (j,\delta')}^{(s)}    \tilde{c}^{\dagger}_{(i,\delta),s,\sigma }   e^{\frac{2 \pi i}{\Phi_0} \int_{i + \delta}^{j + \delta'} \left(  \vec{A}(\vec{l})  + \vec{A}_{\phi} \right)  d \vec{l}} \tilde{c}^{}_{(j,\delta'),s,\sigma }
\end{equation}
with boundary conditions 
\begin{equation}
	\tilde{c}^{\dagger}_{(i + L_i,\delta) ,s,\sigma }   =  e^{\frac{2 \pi i }{\Phi_0} \chi_{L_1} ( i + \delta ) } \, \tilde{c}^{\dagger}_{(i,\delta) ,s,\sigma }.
\end{equation}
Here 
\begin{equation}
	\vec{A}_{\phi} =\frac{  \phi_1  |\vec{a}_1|} { 2 \pi |\vec{L}_1| } \vec{b}_1 +  \frac{  \phi_2  |\vec{a}_2|}{2 \pi  |\vec{L}_2| } \vec{b}_2
\end{equation}
and $\vec{b}_i$  correspond to the reciprocal lattice vectors satisfying  $ \vec{a}_i  \cdot  \vec{b}_j  = 2 \pi \delta_{i,j} $.   The logical variable $\texttt{bulk} $ chooses between these two  gauge equivalent ways  on inserting the twist angle. If \texttt{bulk=\.true\.}    then  we use periodic boundary conditions  --  in the absence of an orbital field -- otherwise  twisted boundaries are used.  

%%%%%%%%%%%%%%%%%%%%%%%%%%%%


\subsection{Interaction Vertices}

In its most general form, an interaction Hamiltonian expressed in terms of sums of perfect squares can be written, as presented in Section~\ref{sec:intro}, as a sum of $M_V$ vertices: %Eq.~\eqref{eqn:general_ham_v}:

\begin{align*}
\hat{\mathcal{H}}_{V} &=  \sum\limits_{k=1}^{M_V}U_{k}
\left\{ \sum\limits_{\sigma=1}^{N_{\mathrm{col}}}
\sum\limits_{s=1}^{N_{\mathrm{fl}}} \left[ \left(
\sum\limits_{x,y}^{N_{\mathrm{dim}}} \hat{c}^{\dagger}_{x \sigma s}V_{xy}^{(k s)}\hat{c}^{\phantom\dagger}_{y \sigma s}\right)  +\alpha_{k s}  \right] \right\}^{2}
\equiv    \sum\limits_{k=1}^{M_V}U_{k}   \left(\hat{V}^{(k)} \right)^2 \tag{\ref{eqn:general_ham_v}}\\
&\equiv    \sum\limits_{k=1}^{M_V}\mathcal{H}_{V,k}.
\end{align*}
The module \texttt{Predefined\_Int\_mod.F90} implements some of the most common of such interaction vertices $\mathcal{H}_{V,k}$, as detailed below.


\subsubsection{The $SU(N)$ Hubbard interaction}

The $SU(N)$ Hubbard interaction on a given site $i$ is given by 
\begin{align}
%\label{eqn_hubbard_sun}
\hat{\mathcal{H}}_{V,i} =
+ \frac{U}{N_{\mathrm{col}}}\left[
\sum\limits_{\sigma=1}^{N_{\mathrm{col}}}
\left(  c^{\dagger}_{i \sigma} c^{\phantom\dagger}_{i\sigma}  -1/2 \right) \right]^{2}.
\end{align} 
Assuming that no other term in the Hamiltonian breaks the $SU(N) $  color symmetry,  then this interaction term  conveniently corresponds to  a single 
operator  which is defined in the subroutine \texttt{Predefined\_Int\_U\_SUN}.   
%which corresponds to the general form of Eq.~\eqref{eqn:general_ham_v} by setting: 
%$N_{\mathrm{fl}} = 1$,  $M_V = N_{\text{unit-cell}} $,  $U_{k} =  -\frac{U}{N_{\mathrm{col}}}$,  $V_{x y}^{(ks)} =  \delta_{x,y} \delta_{x,k}$, and $\alpha_{ks} = -\frac{1}{2}$; and which is defined in the subroutine \texttt{Predefined\_Int\_U\_SUN} by a single operator:

\begin{lstlisting}[style=fortran]

Op%P(1)   = i
Op%O(1,1) = cmplx(1.d0,  0.d0, kind(0.D0))
Op%alpha  = cmplx(-0.5d0,0.d0, kind(0.D0))
Op%g      = SQRT(CMPLX(-DTAU*U/(DBLE(N_SUN)), 0.D0, kind(0.D0))) 
Op%type   = 2

\end{lstlisting}

To relate to  Eq.~\eqref{eqn:general_ham_v} we have,   $V_{x y}^{(is)} =  \delta_{x,y} \delta_{x,i}$, $\alpha_{is} = -\frac{1}{2}$ and $U_{k} =  \frac{U}{N_{\mathrm{col}}}$.   Here  the flavor index, $s$,  plays no role. 


\subsubsection{The $M_z$-Hubbard interaction}

The $M_z$-Hubbard interaction is given by 
\begin{align}
%\label{eqn_hubbard_Mz}
\hat{\mathcal{H}}_{V,k} = - \frac{U}{2}\sum\limits_{x}\left[
c^{\dagger}_{x, \uparrow} c^{\phantom\dagger}_{x \uparrow}  -   c^{\dagger}_{x, \downarrow} c^{\phantom\dagger}_{x \downarrow}  \right]^{2},
\end{align} 
which corresponds to the general form of Eq.~\eqref{eqn:general_ham_v} by setting: 
$N_{\mathrm{fl}} = 2$, $N_{\mathrm{col}} \equiv \texttt{N\_SUN} =1 $,  $M_V =  N_{\text{unit-cell}} $,  $U_{k} = \frac{U}{2}$, 
$V_{x y}^{(k, s=1)} =  \delta_{x,y} \delta_{x,k}  $,  $V_{x y}^{(k, s=2)} =  - \delta_{x,y} \delta_{x,k}  $, and $\alpha_{ks}   = 0  $; and which is defined in the subroutine \texttt{Predefined\_Int\_U\_MZ} by two operators:
\begin{lstlisting}[style=fortran]
        
Op_up%P(1)   = I
Op_up%O(1,1) = cmplx(1.d0, 0.d0, kind(0.D0))
Op_up%alpha  = cmplx(0.d0, 0.d0, kind(0.D0))
Op_up%g      = SQRT(CMPLX(DTAU*U/2.d0, 0.D0, kind(0.D0))) 
Op_up%type   = 2

Op_do%P(1)   = I
Op_do%O(1,1) = cmplx(1.d0, 0.d0, kind(0.D0))
Op_do%alpha  = cmplx(0.d0, 0.d0, kind(0.D0))
Op_do%g      = -SQRT(CMPLX(DTAU*U/2.d0, 0.D0, kind(0.D0))) 
Op_do%type   = 2

\end{lstlisting}


\subsubsection{The $SU(N)$ $V$ interaction}

The interaction term of the generalized t-V model, given by 
\begin{align}
\hat{\mathcal{H}}_{V,k} =
-\frac{V}{N_\mathrm{col}}\left[ \sum_{s=1}^{N_\mathrm{col}}\left( c^{\dagger}_{i,s} c_{j,s} + c^{\dagger}_{j,s} c_{i,s} \right) \right]^2,
\end{align} 
is coded in the subroutine \texttt{Predefined\_Int\_V\_SUN} by a single symmetric operator:
\begin{lstlisting}[style=fortran]

Op%P(1)   = I
Op%P(2)   = J
Op%O(1,2) = cmplx(1.d0 ,0.d0, kind(0.D0)) 
Op%O(2,1) = cmplx(1.d0 ,0.d0, kind(0.D0))
Op%g      = SQRT(CMPLX(DTAU*V/real(N_SUN,kind(0.d0)), 0.D0, kind(0.D0))) 
Op%alpha  = cmplx(0.d0, 0.d0, kind(0.D0))
Op%type   = 2

\end{lstlisting}


\subsubsection{The Fermion-Ising Coupling}

The interaction between the Ising and a fermion degree of freedom, given by
\begin{align}
%\label{eqn_hubbard_sun_Ising}
\hat{\mathcal{H}}_{V,k} =
\hat{Z}_{i,j} \xi  \sum_{s=1}^{N_\mathrm{col}}( c^{\dagger}_{i,s} c^{\phantom\dagger}_{j,s} + c^{\dagger}_{j,s} c^{\phantom\dagger}_{i,s} ),
\end{align} 
where $\xi$ determines the coupling strength, is implemented in the subroutine \texttt{Predefined\_Int\_Ising\_SUN}:
\begin{lstlisting}[style=fortran]

Op%P(1)   = I
Op%P(2)   = J
Op%O(1,2) = cmplx(1.d0 ,0.d0, kind(0.D0)) 
Op%O(2,1) = cmplx(1.d0 ,0.d0, kind(0.D0)) 
Op%g      = cmplx(-dtau*xi,0.D0,kind(0.D0))
Op%alpha  = cmplx(0d0,0.d0, kind(0.D0)) 
Op%type   = 1

\end{lstlisting}



\subsubsection{The Long-Range Coulomb Repulsion}

The Long-Range Coulomb (LRC) interaction can be written as
\begin{align}
\hat{\mathcal{H}}_{V,k} =
\frac{1} { N } \sum_{\vec{i},\vec{j}}  \left(  \hat{n}_{\vec{i}} -  \frac{N}{2}  \right)  V_{\vec{i},\vec{j}} \left(  \hat{n}_{\vec{j}} -  \frac{N}{2}  \right), 
\end{align} 
where
\begin{align}
\hat{n}_{\vec{i}} = \sum_{\sigma=1}^{N}  \hat{c}^{\dagger}_{\vec{i},\sigma}  \hat{c}^{}_{\vec{i},\sigma}
\end{align} 
and
\begin{equation}
V_{\vec{i}, \vec{j}}   =   U \left\{
\begin{array}{ll}  
1          &   \text{ if } \vec{i} - \vec{j}    = 0 \\
\frac{\alpha   \;   d_\mathrm{min}}{ |   \vec{i} - \vec{j} | } &     \text{ otherwise }
\end{array}
\right. .
\end{equation}
Here $d_\mathrm{min}$ is the minimal distance between two orbitals.     The code uses the following  HS decomposition:
\begin{equation}
e^{-\Delta \tau \hat{H}_{V,k} }  =  \int \prod_{\vec{i}} d \phi_{\vec{i}}   e^{ - \frac{N \Delta \tau} {4} \phi_{\pmb{i}} V^{-1}_{\pmb{i},\pmb{j}}  \phi_{\pmb{j}} - \sum_{\pmb{i}}  i \Delta \tau \phi_i \left( n_{i} - \frac{N}{2} \right) }.
\end{equation}

The implementation follows Ref.~\cite{Hohenadler14}  but now supports various lattice geometries.    The definition of  the Coulomb repulsion is as follows. 
A general lattice site  \texttt{I,n}   where \texttt{I: 1...Latt\%N} is the unit cell and \texttt{ n = 1 ...Latt\_unit\%NORB}  the orbital  is given by: 
\begin{lstlisting}[style=fortran]
X_p(:) = Latt%list(I,1)*latt%a1_p(:)  + Latt%list(I,2)*latt%a2_p(:) 
+   Latt_unit%Orb_pos_p(no_j,:)
\end{lstlisting}
or in more compact notation $ \vec{i}  + \vec{\delta}_i $.   By definition \texttt{Latt\_unit\%Orb\_pos\_p(1,:)=0}.
The Coulomb repulsion between points   $ \vec{i}  + \vec{\delta}_i $   and $ \vec{j}  + \vec{\delta}_j $   reads: 
\begin{equation}
V(\vec{i}  + \vec{\delta}_i ,  \vec{j}  + \vec{\delta}_j  )  =  \frac{U d_\mathrm{min} \alpha}{  |  \overline{\vec{i} - \vec{j}} + \vec{\delta}_i - \vec{\delta}_j  |}.
\end{equation}
Here  we use periodic boundary conditions such that  $\overline{\vec{i} - \vec{j}}$  is an element of the real space lattice. Note that this is encoded in the array \texttt{Latt\%imj(I,J)}.

The LRC interaction is implemented in the subroutine \texttt{Predefined\_Int\_LRC}:
\begin{lstlisting}[style=fortran]

Op%P(1)   = I
Op%O(1,1) = cmplx(1.d0  ,0.d0, kind(0.D0))
Op%alpha  = cmplx(-0.5d0,0.d0, kind(0.D0))
Op%g      = cmplx(0.d0  ,Dtau, kind(0.D0)) 
Op%type   = 3

\end{lstlisting}


\subsubsection{The $J_z$-$J_z$ Interaction}

Another predefined vertex is:
\begin{align}
\hat{\mathcal{H}}_{V,k} =
- \frac{|J_z|}{2}  \left( S^{z}_i - \sgn|J_z| S^{z}_j \right)^2 =
J_z  S^{z}_i  S^{z}_j  - \frac{|J_z|}{2} (S^{z}_i)^2 - \frac{|J_z|}{2}(S^{z}_j)^2 
\end{align} 
which, if particle fluctuations are frozen on the $i$ and $j$ sites, then $(S^{z}_i)^2 = 1/4$ and the interactions corresponds to a $J_z$-$J_z$ ferro or antiferro coupling.

The implementation of the interaction in \texttt{Predefined\_Int\_Jz} defines two operators:
\begin{lstlisting}[style=fortran]

Op_up%P(1)   = I
Op_up%P(2)   = J
Op_up%O(1,1) = cmplx(        1.d0  ,0.d0, kind(0.D0))
Op_up%O(2,2) = cmplx(- Jz/Abs(Jz)  ,0.d0, kind(0.D0))
Op_up%alpha  = cmplx(0.d0, 0.d0, kind(0.D0))
Op_up%g      = SQRT(CMPLX(DTAU*Jz/8.d0, 0.D0, kind(0.D0))) 
Op_up%type   = 2

Op_do%P(1)   = I
Op_do%P(2)   = J
Op_do%O(1,1) = cmplx(        1.d0  ,0.d0, kind(0.D0))
Op_do%O(2,2) = cmplx(- Jz/Abs(Jz)  ,0.d0, kind(0.D0))
Op_do%alpha  = cmplx(0.d0, 0.d0, kind(0.D0))
Op_do%g      = -SQRT(CMPLX(DTAU*Jz/8.d0, 0.D0, kind(0.D0))) 
Op_do%type   = 2

\end{lstlisting}



\subsection{Template}

\red{Maybe discard this subsection.}\\
	
\red{Maybe merge this with Chapter on Examples and Models/Model Classes.\\}

 \red{TODO} Go through everything one has to defined/set in order to define a new Hamiltonian. \red{TODO}\\

We'd perhaps want to provide a \emph{minimum} Hamiltonian, probably written in pseudo-code, from which one could write their own.\\ \\

\noindent A \emph{complete} list of all the code that would have to be changed for the \emph{most general} case could read as:

\begin{itemize}
	\item Hamiltonian -- one-body, imaginary-time propagator for a given configuration of HS and  Ising fields
	\item Table~\ref{table:hamiltonian}:
	\begin{itemize}
		\item \texttt{Ham\_Set}
		\item \texttt{Ham\_V}
		\item \texttt{S0}
		\item \texttt{Setup\_Ising\_action}
		\item \texttt{Global\_move}
		\item \texttt{Delta\_S0\_global}
		\item \texttt{Global\_move\_tau}
		\item \texttt{Alloc\_obs}
		\item \texttt{Obser}
		\item \texttt{ObserT}
	\end{itemize}
	\item Check Table~\ref{table:operator} (Operator type).
	\item Table~\ref{table:lattice} and \ref{table:unit_cell}:
	\begin{itemize}
		\item \texttt{Lattice\%a1\_p}, \texttt{Lattice\%a2\_p}
		\item \texttt{Lattice\%L1\_p}, \texttt{Lattice\%L2\_p}
		\item \texttt{Lattice\%N}
		\item \texttt{Norb}
		\item \texttt{N\_coord}
		\item \texttt{Orb\_pos(1..Norb,2)}
	\end{itemize}
\end{itemize}


\section{Model Classes}\label{sec:model_classes}
% Copyright (c) 2016-2019 The ALF project.
% This is a part of the ALF project documentation.
% The ALF project documentation by the ALF contributors is licensed
% under a Creative Commons Attribution-ShareAlike 4.0 International License.
% For the licensing details of the documentation see license.CCBYSA.
% !TEX root = Doc.tex
\subsection{  Model Classes }

I suggest the following for the organization of the models.    Five independent Hamiltonian   files would suffice.


\subsubsection{Hubbard models   \texttt{tU\_mod.F90}}

\begin{equation}
    \sum_{\vec{i},\vec{j},\sigma=1}^{N}  \hat{c}^{\dagger}_{\vec{i},\sigma } T_{\vec{i},\vec{j}} \hat{c}^{\phantom\dagger}_{\vec{i},\sigma }     +  \frac{U}{N} \sum_{\vec{i}} \left(\sum_{\sigma=1}^{N}  \left[   \hat{c}^{\dagger}_{\vec{i},\sigma } 
    \hat{c}^{\phantom\dagger}_{\vec{i},\sigma }  - 1/2  \right] \right)^2 
\end{equation}
This Hamiltonian would
\begin{itemize} 
\item support   square,  honeycomb,  $\pi$-flux.  It would be very nice to have bilayer versions of these lattices as well. 
\item Breakdown of the $U(2N)$ symmetry to $U(N) \times U(N)$ so as to accommodate magnetic fields and pinning fields.  
\end{itemize}


\subsubsection{Hubbard models   \texttt{tV\_mod.F90}}

This would include the $SU(N)$  $t-V$ models on various lattices.  The defining property of this set of Hamiltonians would be the enlarged O(2N) symmetry.  Again  this   module should support our standard bipartitie lattices. 


\subsubsection{Hubbard models   \texttt{LRC\_mod.F90}}

This is the long range Coulomb. See above.    Again we should include the   standard lattices. 

\subsubsection{Hubbard models   \texttt{Z2\_mod.F90}}
I suggest to work on the  Hamiltonian of Ref.~\ref{Z2.Sec} since this is the most general  model I can think of.   Would be nice to add a Hubbard-$U$ term.  It is actually not so easy to generalize this model to 
arbitrary lattices, so that for the moment, I would concentrate only on the square lattice. 

\subsubsection{Hubbard models   \texttt{Kondo\_mod.F90}}
Kondo lattice model on various lattices.  I still have to think about the best way of doing things here. 
\section{Maximum Entropy}\label{sec:maxent}
% Copyright (c) 2016-2019 The ALF project.
% This is a part of the ALF project documentation.
% The ALF project documentation by the ALF contributors is licensed
% under a Creative Commons Attribution-ShareAlike 4.0 International License.
% For the licensing details of the documentation see license.CCBYSA.

% !TEX root = doc.tex

\section{Maximum entropy }

\subsection{General setup}
Generically, the maximum entropy code computes the  image  $A(\omega) $ for a given  data  set $g(\tau) $  and kernel $K(\tau,\omega) $:
\begin{equation}
g(\tau) =  \int_{\omega_\text{start}}^{\omega_\text{end}} d {\omega} K(\tau,\omega) A(\omega).
\end{equation} 
The  ALF-package includes a standard implementation of the stochastic MaxEnt as formulated in the article of K. Beach Ref.~\cite{Beach04a}. Here we will comment on the workflow.  The module 
\texttt{Libraries/Modules/\allowbreak{}maxent\_stoch.f90} contains a general implementation and the wrapper is in \texttt{Analysis/Max\_SAC.f90}. 

The stochastic MaxEnt is essentially a parallel tempering Monte Carlo simulation.    For a discrete set of $\tau_i$ points,  $i \in 1 \cdots n $ the energy reads
\begin{equation}
  \chi^{2}(A) =  \sum_{i,j=1}^{n}   \left[ g(\tau_i)  –    \overline{g(\tau_i)} \right] C^{-1}(\tau_i,\tau_j) \left[    g(\tau_j)  –  \overline{g(\tau_j)} \right] 
\end{equation} with $ \overline{g(i)} =\int d{\omega} K(\tau_{i},\omega)  A(\omega)$ and  $C$ the covariance matrix. 
The set  of inverse temperatures  we will consider  in the parallel tempering reads:
$ \alpha_m = \alpha_{st}  R^{m}, \; \; m = 1 \cdots N_{\alpha} $.   The phase space corresponds to all possible spectral functions with given sum rule and required positivity.  Finally,  the partition function reads
$Z =  \int{DA} e^{-\alpha \chi^{2}(A)}$.  

In the code, the spectral function is parametrized  by a  set of Dirac $\delta$ functions: 
\begin{equation}
      A(\omega)  = \sum_{i=1}^{N_{\gamma}} a_{i} \delta \left( \omega - \omega_i \right).
\end{equation}
To produce a histogram of  $ A(\omega) $ we divide  the frequency range in \texttt{Ndis} intervals. 
The Green function is read from the file \texttt{g\_dat}  corresponding to the  output of the the  \texttt{cov\_tau.f90} analysis program.  
Below, we summarize the  parameters   in   name-lists  \texttt{VAR\_Max\_Stoch }  and  \texttt{VAR\_errors }   in the  \texttt{parameters}  file   required  to run the maxent code .  
\lstset{style=fortran}
\begin{lstlisting} 

&VAR_Max_Stoch               ! Variables for Stochastic Maximum entropy
Ngamma                       ! # of  Dirac functions for parametrization
Om_st                        ! Frequency range lower bound
Om_en                        ! Frequency range upper bound
NDis                         ! # of boxes for histogram
Nbins                        ! # of bins for Monte Carlo
Nsweeps                      ! # of sweeps per bin
NWarm                        ! The Nwarm first bins will be ommitted
N_alpha                      ! # of tempertures
alpha_st                     ! smallest inverse temperature
R                            ! increment for inverse temperature (see above) 
Channel                      ! T0       : Zero temperature
                             ! P        : Finite temperarure particle 
                             ! PH       : Finite temperarure particle-hole
                             ! PP       : Finite temperarure particle-particle 
Checkpoint                   !.true.    : dump files will be produced so  
                             !            as to be able to restart the simulation
                             !.false.   : dump files will not be produced 
Tolerance                    !Data points for which the relative error
                             !exceeds the tolerance threshold will be omitted.
/

&VAR_erros                   ! Variables for the error analyis
....                         !
N_cov                        ! =1  Covariance will be taken into account
                             ! =0  Covariance will not be taken into account 
/
\end{lstlisting}
Note that the variable  \texttt{N\_cov}  req

\noindent
\textbf{Output files} \\
The code produces  the following output files.
\begin{itemize}
\item The files  \texttt{Aom\_n}  correspond to the average spectral function at inverse  temperature  $ \alpha_n $. This corresponds to
$  \langle A_n(\omega) \rangle =   \frac{1}{Z}   \int DA(\omega)    e^{-\alpha_n \chi^{2}(A)  } A(\omega). $
The file contains three colums  $ \omega, \;  \langle A_n(\omega) \rangle , \;  \Delta \langle A_n(\omega) \rangle $.

\item The files \texttt{Aom\_ps\_n}   contain the average image over  the  inverse   temperatures  $ \alpha_n $ to $ \alpha_{N_\gamma} $  see Ref.~\cite{Beach04a} for more details.   
 The first three columns have the same meaning as for the files \texttt{Aom\_n}

\item The file \texttt{Green} contains the Green function. The three columns correspond to $ \omega, \;   \text{ Re} G(\omega), \;  \text{  Im} G(\omega)  $.  This is obtained from the spectral function through:
\begin{equation}
 G(\omega) =  -\frac{1}{\pi} \int d \Omega   \frac{A(\Omega)}{\omega – \Omega + i \delta}
 \end{equation}
where  $ \delta =  \Delta \omega$ with $ \Delta \omega = (\omega_\text{end} -  \omega_\text{start})/\text{Ndis}$ and the image corresponds to that of the file \texttt{Aom\_ps\_m} with $ m = N_{\alpha} -10 $. 
The first column of the  \texttt{Green}  file is a place holder for post-processing. The last three columns   correspond to $\omega, \text{Re} G(\omega) ,   - \text{Im} G(\omega)/\pi $. 

\item  One of the most important files is the file  \texttt{energies}. It contains there columns:  $ \alpha_n, \langle \chi^2 \rangle, \Delta \langle \chi^2 \rangle $.

\item   \texttt{best\_fit}  gives the values of $a_i$ and $\omega_i$   (recall that $ A(\omega)  = \sum_{i=1}^{N_{\gamma}} a_{i} \delta \left( \omega - \omega_i \right)$) corresponding to the last configuration of the  lowest temperature run.

\item  The File \texttt{data\_out}  is a crosscheck. It plots   $ \tau,  g(\tau),  \Delta g(\tau), \int d \omega  K(\tau, \omega) A(\omega) $ where the image  corresponds to the best fit (i.e. the lowest temperature). 
This file will give you  a feeling on how good the fit actually is.  Note that  \texttt{data\_out} contains only the data points that have  passed the tolerance test. 


\item There are two \texttt{dump} files which are generated. Since  the MaxEnt is a  Monte Carlo code, one  would like to be able to continue a simulation to improve. The data in the dump files will allow you to pursue the simulation without loosing the first run(s).   These files are  only generated if the variable  \texttt{checkpoint} is set to true. 
 \end{itemize}

The essential question is: which image should one use. There is no real answer to this question in the context of the stochastic MaxEnt. The only rule of thumb is to consider temperatures for which the \( \chi^2 \) is  comparable to the number of data points.


\subsection{Single particle quantities}
For the single-particle Green function, 

\begin{equation} 
	\langle \hat{c}^{\phantom\dagger}_{k} (\tau)  \hat{c}^{\dagger}_{k} (0)   \rangle   = \int d \omega  K_p(\tau,\omega)   A_p(k, \omega) 
\end{equation}
with 
\begin{equation}
K_{p}(\tau,\omega) =    \frac{1}{\pi} \frac{e^{-\tau \omega} }  {  1 + e^{-\beta\omega} }
\end{equation}
and in the Lehmann representation, 
 \begin{equation}
   A_p(k, \omega) = \frac{ \pi}{Z} \sum_{n,m} e^{-\beta E_n } \left( 1 + e^{-\beta \omega}\right) | \langle n | c_n | m  \rangle |^{2} \delta \left( E_m - E_n - \omega \right)  
\end{equation}  
Here $ \left( \hat{H} - \mu \hat{N} \right) | n \rangle = E_n | n \rangle  $.

Note that  $ A_p(k, \omega)  = - \text{Im} G^{\text{ret}} (k, \omega) $ with 
\begin{equation}
	G^{\text{ret}} (k, \omega)  = -i \int d t \Theta(t)  e^{i \omega t} \langle \left\{ \hat{c}^{\phantom\dagger}_{k} (t), \hat{c}^{\dagger}_{k} (0) \right\} \rangle
\end{equation}
Finally the sum rule reads:
\begin{equation}
	\int d \omega  A_p(k, \omega)  = \pi \langle  \left\{ \hat{c}^{\phantom\dagger}_{k} , \hat{c}^{\dagger}_{k}  \right\}   \rangle = \pi 
\end{equation}
Using the \texttt{Max\_Sac.f90}  with \texttt{Channel="P"}   will  load the above Kernel in the MaxEnt library.  Note that in this case the back  transformation is set to unity.  
Note that since for each  configuration of fields,  $ \langle  \langle \hat{c}^{\phantom\dagger}_{k} (\tau=0)  \hat{c}^{\dagger}_{k} (0)   \rangle  \rangle_{C} +   
\langle \langle \hat{c}^{\phantom\dagger}_{k} (\tau=\beta)  \hat{c}^{\dagger}_{k} (0)   \rangle \rangle_{C} = 
\langle \langle \left\{ \hat{c}^{\phantom\dagger}_{k},   \hat{c}^{\dagger}_{k}    \right\} \rangle \rangle_{C}   = 1$.  Hence if both  the $\tau=0$ anf $\tau=\beta$ data points are included, the covariance matrix will have a zero eigenvalue and the $\chi^{2}$. measure is not defined. Hence for the particle channel, the program omits the $\tau=\beta$ data point.     One should also not that there are special  particle-hole symmetric  cases where the $\tau=0$ data point shows no  fluctuations. In this case, the 
code equally omits the $\tau=0$ data point. 
\subsection{Particle-hole quantities }

\noindent
\textbf{Imaginary time formulation.}
 For particle-hole quantities such as spin-spin or charge-charge correlations, 
the  Kernel reads:
\begin{equation}
	\langle \hat{S}(q,\tau) \hat{S}(-q,0) \rangle  = \frac{1}{\pi} 
   \int {\text d} \omega  \frac{e^{- \tau \omega} }{ 1 - e^{-\beta  \omega} } \chi''(q,\omega).
\end{equation}
This follows directly from the  Lehmann representation: 
\begin{equation}
 \chi''(q,\omega)  = \frac{\pi}{Z} \sum_{n,m} e^{-\beta E_n} |\langle n | \hat{S}(q) | \rangle m |^2 
\delta ( \omega + E_n - E_m) \left( 1 - e^{-\beta  \omega} \right) 
\end{equation}
Since the linear response to a Hermitian perturbation  is real, $\chi''(q,\omega)  = - \chi''(-q,-\omega)$.  Hence for systems with inversion symmetry -- that 
we will consider here -- $\langle \hat{S}(q,\tau) \hat{S}(-q,0) \rangle $ is a symmetric function around $\beta= \tau/2$.  The analysis  file \texttt{cov\_tau\_ph.f90} produced at compilation
time will use this  to define an improved estimator. 

The  Stochastic MaxEnt requires a sum rule, such that   the Kernel and image have to be adequately redefined. 
Consider: 
\begin{equation}
	\text{coth}(\beta \omega/2) \chi''(q,\omega)
\end{equation}
For this quantitiy, we have the sum rule since: 
\begin{equation}
	\int {\text d} \omega 	\text{coth}(\beta \omega/2) \chi''(q,\omega) = 
  2 \pi \langle \hat{S}(q,\tau=0) \hat{S}(-q,0) \rangle
\end{equation}
which is just the first point in the data. 

Hence,
\begin{equation}
	\langle \hat{S}(q,\tau) \hat{S}(-q,0) \rangle  =  
       \int {\text d} \omega  \underbrace{ \frac{1}{\pi} \frac{e^{- \tau \omega} }
            { 1 - e^{-\beta  \omega} } \text{tanh}(\beta \omega/2)  }_{K_{pp}(\tau,\omega)} 
       \underbrace{ \text{coth}(\beta \omega/2)   \chi''(q,\omega) }_{A(\omega)} 
\label{Kpp.eq}
\end{equation}
and one  computes $A(\omega)$. Note that since $\chi'' $ is an odd function of $\omega$  one restricts the integration range  positive values of $\omega$. 
Hence: 
\begin{equation}
	\langle \hat{S}(q,\tau) \hat{S}(-q,0) \rangle  =  
       \int_{0}^{\infty}  {\text d} \omega \underbrace{\left( K(\tau,\omega)  + K(\tau,-\omega) \right)}_{K_{ph}(\tau,\omega)}  A(\omega).
\end{equation}
In the code, $\omega_\text{start}$ is set to zero by default and the Kernel $K_{ph}$ is used in the code and is defined in the  routine \texttt{XKER\_ph}. 
In general,  one would like to produce the  dynamical structure factor that relates to the susceptibility according to
\begin{equation}
 S(q,\omega)  = \chi''(q,\omega)/\left( 1 - e^{-\beta  \omega} \right). 
\end{equation}

In the code the routine \texttt{BACK\_TRANS\_ph}   transforms the image $A$ to the desired quantity.
\begin{equation}
	S(q,\omega) = \frac{A(\omega)}{1 + e^{-\beta \omega} }  
\end{equation}

\noindent
\textbf{Matsubara frequency formulation.}
The ALF  library uses  imaginary time. It is however possible to formulate the MaxEnt in  Matsubara frequencies.
Consider:
\begin{equation}
  \chi(q,i\Omega_m) = \int_0^{\beta} {\text d} \tau  e^{i \Omega_m \tau}
	\langle S(q,\tau) S(-q,0) \rangle  = \frac{1}{\pi}
   \int {\text d} \omega  \frac{\chi''(q,\omega)}{ \omega - i \Omega_m }.
\end{equation}
Using the fact that $\chi''(q,\omega) = -\chi''(-q,-\omega) = -\chi''(q,-\omega)$ one obtains:
\begin{equation}
\begin{gathered}
  \chi(q,i\Omega_m) = 
	\frac{1}{\pi}
   \int_0^{\infty} {\text d} \omega \left(\frac{1}{ \omega - i \Omega_m } - \frac{1}{ -\omega - i \Omega_m } \right)
         \chi''(q,\omega) \\
    = \frac{2}{\pi} \int_0^{\infty} {\text d} \omega \frac{\omega^2}{ \omega^2  + \Omega_m^2 } 
  \frac{\chi''(q,\omega)}{\omega} 
   \equiv \int_0^{\infty} {\text d} \omega K(\omega,i\Omega_m) A(q,\omega)
\end{gathered}
\end{equation}
with
\begin{equation}
   K(\omega,i\Omega_m) = \frac{\omega^2}{ \omega^2  + \Omega_m^2 } 
\end{equation}
and
\begin{equation}
A(q,\omega) =  \frac{2}{\pi}   \frac{\chi''(q,\omega)}{\omega} 
\end{equation}
The above definitions are useful since the image satisfies the sum rule:
\begin{equation}
\int_0^{\infty} {\text d} \omega A(q,\omega) =  \frac{1}{\pi}  \int_{-\infty}^{\infty} {\text d} \omega 
   \frac{\chi''(q,\omega)}{\omega}   \equiv \chi(q,i\Omega_m=0)
\end{equation}


\subsection{Particle-Particle quantities}

Similarly to the particle-hole channel  the particle-particle channel is also a bosonic correlation function. Here however we do not assume that the 
imaginary time data is symmetric around   the $\tau = \beta/2$ point.  We use the Kernel $K_{pp}$ define in Eq.~\ref{Kpp.eq}  and consider the whole frequency range. 
The back transformation  yields
\begin{equation}
 \frac{\chi''(\omega)} {\omega}   = \frac{\text{tanh} \left( \beta \omega/2 \right) }{ \omega }   A(\omega) 
\end{equation}



\subsection{Zero temperature, projective code}

 In the zero temperature limit,  the spectral function associated to an operator $\hat{O} $    reads:
 \begin{equation}
 	  A_o(\omega)    = \pi  \sum_{n}    | \langle n  | \hat{O} | 0 \rangle |^2 \delta( E_n - E_0 - \omega) 
 \end{equation}
 such that 
 \begin{equation}
 	\langle 0 | \hat{O}^{\dagger}(\tau) \hat{O}^{}(0) | 0 \rangle =  \int d  \omega  K_0(\tau,\omega) A_0(\omega) 
 \end{equation}
 with 
 \begin{equation}
 	K_0(\tau,\omega)  = \frac{1}{\pi}e^{-\tau \omega}.
 \end{equation}
 The zeroth moment of the spectral function reads, 
 \begin{equation}
  \int d \omega A_o(\omega) = \pi \langle 0 | \hat{O}^{\dagger}(0) \hat{O}^{}(0) | 0 \rangle, 
 \end{equation}
 and hence corresponds to the first data point. 
 In the zero-temperature limit one does not distinguish between  particle, particle-hole, or particle-particle channels.
 Using the \texttt{Max\_Sac.f90}  with \texttt{Channel="T0"}   will  load the above Kernel in the MaxEnt library. In this case the back  transformation is set to unity. 
 The code will also cut-off the tail of the  imaginary time correlation function  if the relative error is greater that the variable \texttt{Tolerance}. 

\section{Conclusions and Future Directions}\label{sec:con}
% Copyright (c) 2016 The ALF project.
% This is a part of the ALF project documentation.
% The ALF project documentation by the ALF contributors is licensed
% under a Creative Commons Attribution-ShareAlike 4.0 International License.
% For the licensing details of the documentation see license.CCBYSA.
% !TEX root = doc.tex

In its present form, the  auxiliary-field QMC code of the ALF project allows us to simulate a large class of non-trivial models, both efficiently and at minimal  programming cost.  ALF 2.0 contains many advanced functionalities, including a projective formulation, various updating schemes, better control of Trotter errors, predefined structures that facilitate reuse, a large class of models, continuous fields and, finally, stochastic analytical continuation code. Also the usability of the code has improved in comparison with ALF 1.0. In particular the \href{https://git.physik.uni-wuerzburg.de/ALF/pyALF}{pyALF} project provides a Python interface to the ALF which substantially facilitates running the code for established models.  This ease of use renders ALF 2.0 a  powerful  tool to for benchmarking new algorithms. 

There are further capabilities that we would like to see in future versions of ALF. Introducing time-dependent Hamiltonians, for instance, will require some rethinking, but will allow, for example, to access entanglement properties of interacting fermionic systems \cite{Broecker14,Assaad13a,Assaad15}. Moreover, the auxiliary field approach is not the only method to simulate fermionic systems.
It would be desirable to include additional lattice fermion algorithms such as the CT-INT \cite{Rubtsov05,Assaad07}.
Lastly, at the more technical level, improved IO (e.g., HDF5 support), post-processing, object oriented programming, as well as increased compatibility with other software projects are all certainly improvements to look forward to. 

\addcontentsline{toc}{section}{Acknowledgments}
% Copyright (c) 2016 The ALF project.
% This is a part of the ALF project documentation.
% The ALF project documentation by the ALF contributors is licensed
% under a Creative Commons Attribution-ShareAlike 4.0 International License.
% For the licensing details of the documentation see license.CCBYSA.
% !TEX root = Doc.tex
%-------------------------------------------------------------------------------------
\section*{Acknowledgments} 
%-------------------------------------------------------------------------------------

We are very grateful to  S.~Beyl, M.~Hohenadler,  F.~Parisen Toldin,  M.~Raczkowski, T.~Sato, J.~Schwab, Z.~Wang, and M.~Weber  for constant support during the development of this project. 
\mycomment{We equally thank G.~Hager, M.~Wittmann, and G.~Wellein for useful discussions and support.}
FFA would also like to thank T.~Lang   and Z.~Y.~Meng for  developments of the auxiliary field code as well as T.~Grover. 
MB thanks the Bavarian Competence Network for Technical and Scientific High Performance Computing (KONWIHR) for financial support. FG  and JH thank the SFB-1170 for  financial support under projects Z03 and C01.  FFA thanks the DFG-funded FOR1807 and FOR1346 for partial financial support.
Part of the optimization of the code was carried out during  the  Porting and Tuning Workshop 2016 offered by the Forschungszentrum J\"ulich.
Calculations  to extensively test this package were carried out both on  SuperMUC at the  Leibniz Supercomputing Centre and on  JURECA  \cite{Jureca16} at the J\"ulich Supercomputing Centre.  We thank both institutions for generous allocation of computing time.
 %The authors gratefully acknowledge the computing time granted by the John von Neumann Institute for Computing (NIC) and provided on the supercomputer JURECA \cite{Jureca16} at Jülich Supercomputing Centre (JSC). The authors gratefully acknowledge the Gauss Centre for Supercomputing e.V. (www.gauss-centre.eu) for funding this project by providing computing time on the GCS Supercomputer SuperMUC at the Leibniz Supercomputing Centre (LRZ, www.lrz.de).
\begin{appendix}
% Copyright (c) 2016, 2020 The ALF project.
% This is a part of the ALF project documentation.
% The ALF project documentation by the ALF contributors is licensed
% under a Creative Commons Attribution-ShareAlike 4.0 International License.
% For the licensing details of the documentation see license.CCBYSA.

% !TEX root = doc.tex

%-------------------------------------------------------------------------------------
\section{Practical implementation of Wick decomposition of 2n-point correlation functions of two imaginary times } \label{sec:wick}
%-------------------------------------------------------------------------------------

In this Appendix,  we briefly  outline how to compute 2n point correlation functions   of the form: 
\begin{align}
\label{Time_dispalced_gen.eq}
	\lim_{\epsilon \rightarrow 0  } & \sum_{\sigma_1, \sigma'_1, \cdots, \sigma_n, \sigma'_n,  s_1, s'_1  \cdots s_n,  s'_n  }  f( \sigma_1, \sigma'_1, \cdots, \sigma_n, \sigma'_n,  s_1, s'_1  \cdots s_n,  s'_n ) 
	\nonumber    \\
         &  \, \, \, \,        \langle \langle {\cal T}  \left( c^{\dagger}_{x_1,\sigma_1,s_1}(\tau_{1,\epsilon}) c^{\phantom\dagger}_{x'_{1},\sigma'_1,s'_1}(\tau'_{1,\epsilon}) - a_1  \right)  \cdots 
	    \left( c^{\dagger}_{x_n,\sigma_n,s_n}(\tau_{n,\epsilon}) c^{\phantom\dagger}_{x'_{n},\sigma'_n,s'_m} (\tau'_{n,\epsilon}) - a_n  \right)   \rangle \rangle_C
\end{align}
Here,  $ \sigma $ is a color  index and $s$ a flavor index such that 
\begin{equation}
	\langle \langle {\cal T}  c^{\dagger}_{x,\sigma,s}(\tau) c^{\phantom\dagger}_{x',\sigma',s'}(\tau')  \rangle \rangle_C  = 
	\langle \langle {\cal T}  c^{\dagger}_{x,s}(\tau) c^{\phantom\dagger}_{x',s}(\tau')  \rangle \rangle_C  \, \, \delta_{s,s'} \delta_{\sigma,\sigma'}.
\end{equation}
That is, the single particle Green function is diagonal in the flavor index  and color  independent.   To  define the time ordering we will assume  that all times differ  but that $  \lim_{\epsilon \rightarrow 0 }    \tau_{n,\epsilon}  $   as well as $ \lim_{\epsilon \rightarrow 0 }    \tau'_{n,\epsilon} $  take the values $0$  or $\tau$.  
Let
\begin{equation}
	G(I,J,s)   =  \lim_{\epsilon \rightarrow 0 }\langle \langle   \mathcal{T}   c^{\dagger}_{x_I,s}(\tau_{I,\epsilon}) c^{\phantom\dagger}_{x'_{J},s}(\tau'_{J,\epsilon})  \rangle \rangle_{C} 
\end{equation}
The $G(I,J,s)  $  are  uniquely defined by the time  displaced correlation   functions  that enter  the \texttt{ObserT}   routine in the Hamiltonian files.  They are defined in Eq.~\ref{Time_displaced_green.eq} and read: 
\begin{align}
\begin{aligned}
\texttt{GT0(x,y,s) }  &=   \phantom{+} \langle \langle \hat{c}^{\phantom\dagger}_{x,s} (\tau)   \hat{c}^{\dagger}_{y,s} (0)   \rangle \rangle_C \;=\; \langle \langle \mathcal{T} \hat{c}^{\phantom\dagger}_{x,s} (\tau)   \hat{c}^{\dagger}_{y,s} (0)   \rangle \rangle_C   \\
\texttt{G0T(x,y,s) }   &=  -   \langle \langle   \hat{c}^{\dagger}_{y,s} (\tau)    \hat{c}^{\phantom\dagger}_{x,s} (0)    \rangle \rangle_C \;=\;
    \langle \langle \mathcal{T} \hat{c}^{\phantom\dagger}_{x,s} (0)    \hat{c}^{\dagger}_{y,s} (\tau)   \rangle \rangle_C  \\
  \texttt{G00(x,y,s) }  &=    \phantom{+} \langle \langle \hat{c}^{\phantom\dagger}_{x,s} (0)   \hat{c}^{\dagger}_{y,s} (0)   \rangle \rangle_C    \\
    \texttt{GTT(x,y,s) }  &=   \phantom{+} \langle \langle \hat{c}^{\phantom\dagger}_{x,s} (\tau)   \hat{c}^{\dagger}_{y,s} (\tau)   \rangle \rangle_C.
\end{aligned}
\end{align}
For instance, let  $\tau_{I,\epsilon}   > \tau'_{J,\epsilon}  $  and $ \lim_{\epsilon \rightarrow 0 } \tau_{I,\epsilon}   = \lim_{\epsilon \rightarrow 0 }\tau'_{J,\epsilon} = \tau$. Then 
\begin{equation}
	G(I,J,s)   =  \langle \langle  c^{\dagger}_{x_I,s}(\tau) c^{\phantom\dagger}_{x'_{J},s}(\tau)  \rangle \rangle_{C}  =   \delta_{x_I,x'_J} -  GTT(x'_J,x_I,s).
\end{equation}

Using the formulation of Wicks theorem of Eq.~\ref{Wick.eq},  Eq.~\ref{Time_dispalced_gen.eq}  reads: 
\begin{align}
	& \sum_{\sigma_1, \sigma'_1, \cdots, \sigma_n, \sigma'_n,  s_1, s'_1  \cdots s_n,  s'_n  }  f( \sigma_1, \sigma'_1, \cdots, \sigma_n, \sigma'_n,  s_1, s'_1  \cdots s_n,  s'_n ) 
	\\
	& \det  
\begin{bmatrix}
   G(1,1,s_1) \delta_{s_1,s'_1} \delta_{\sigma_1,\sigma'_1} - \alpha_1 & 
   G(1,2,s_1) \delta_{s_1,s'_2} \delta_{\sigma_1,\sigma'_2}   \phantom{ - \alpha_1}         & \dots   &   
   G(1,n,s_1) \delta_{s_1,s'_n} \delta_{\sigma_1,\sigma'_n}   \phantom{ - \alpha_1} \\
   G(2,1,s_2) \delta_{s_2,s'_1} \delta_{\sigma_2,\sigma'_1}  \phantom{ - \alpha_1} &   
   G(2,2,s_2) \delta_{s_2,s'_2} \delta_{\sigma_2,\sigma'_2} - \alpha_2  & \dots  &
    G(2,n,s_2) \delta_{s_2,s'_n} \delta_{\sigma_2,\sigma'_n} \phantom{ - \alpha_2}  \\
    \vdots & \vdots &  \ddots & \vdots \\
    G(n,1,s_n) \delta_{s_n,s'_1} \delta_{\sigma_n,\sigma'_1}   \phantom{- \alpha_n} & 
    G(n,2,s_n) \delta_{s_n,s'_2} \delta_{\sigma_n,\sigma'_2}   \phantom{- \alpha_n} & \dots  & 
     G(n,n,s_n) \delta_{s_n,s'_n} \delta_{\sigma_n,\sigma'_n}   - \alpha_n  
 \end{bmatrix}.   \nonumber 
\end{align}
The  symbolic evaluation of the  determinant   as well as the sum over the color and flavor indices can be carried out with Mathematica.  This  produces a  long expression in terms of the   functions $G(I,J,s)$ that can then be  included in the code.   The Mathematica notebooks  that we use  can be found in the directory  \texttt{Mathematica}  of  the ALF  directory.   As an open source alternative to Mathematica, the user can use the  Sympy  Python library. 

% Copyright (c) 2016 The ALF project.
% This is a part of the ALF project documentation.
% The ALF project documentation by the ALF contributors is licensed
% under a Creative Commons Attribution-ShareAlike 4.0 International License.
% For the licensing details of the documentation see license.CCBYSA.
% !TEX root = Doc.tex
%-------------------------------------------------------------------------------------
\subsection{Performance, memory requirements and parallelization}
%-------------------------------------------------------------------------------------


As mentioned in the  introduction, the auxiliary field QMC algorithm scales linearly in inverse temperature $\beta$ and cubic in the volume $N_{\text{dim}}$. Using fast updates,  a single spin flip  requires $(N_{\text{dim}})^2$ operations to update the Green function upon acceptance.  As there are $L_{\text{Trotter}}\times N_{\text{dim}}$ spins to be visited, the total computational cost for one sweep is of the order of $\beta (N_{\text{dim}})^3$. This operation  dominates the performance, see Fig.~\ref{fig_scaling_size}. A profiling analysis of our code shows that 80-90\% of the CPU time is spend in ZGEMM calls of the BLAS library provided in the MKL package by Intel. Consequently, the single-core performance is next to optimal.

\begin{figure}[h]
	\begin{center}
		\includegraphics[scale=.8]{Figures/Size_scaling_ALF_2.pdf}
	\end{center}
	\caption{\label{fig_scaling_size}Volume scaling behavior of the auxiliary field QMC code of the ALF project on SuperMUC (phase 2/Haswell nodes) at the LRZ in Munich. The number of sites $N_{\text{dim}}$ corresponds to the system volume.
	The plot confirms that the leading scaling order is due to matrix multiplications such that the runtime is dominated by calls to ZGEMM. }
\end{figure}

For the implementation which scales linearly in $\beta$, one has to store $L_{\text{Trotter}}/\texttt{NWrap}$ intermediate propagation matrices of dimension $N\times N$. For large lattices and/or low temperatures this dominates the total memory requirements that can exceed 2~GB memory for a sequential version.

At the heart of Monte Carlo schemes lies a random walk through the given configuration space. This is easily parallalized via MPI by associating one random walker to each MPI task. For each task, we start from a random configuration and have to invest the autocorrelation time $T_\mathrm{auto}$ to produce an equilibrated configuration.
Additionally we can also profit from an OpenMP parallelized version of the BLAS/LAPACK library for an additional speedup, which also effects equilibration overhead $N_\text{MPI}\times T_\text{auto} / N_\text{OMP}$, where $N_{\text{MPI}}$ is the number of cores and $N_{\text{OMP}}$ the number of OpenMP threads.
For a given number of independent measurements  $N_\text{meas}$, we  therefore need a wall-clock time given by
\begin{equation}\label{eqn:scaling}
T  =  \frac{T_\text{auto}}{N_\text{OMP}} \left( 1   +    \frac{N_\text{meas}}{N_\text{MPI}}  \right) \,.
\end{equation}
As we typically have $ N_\text{meas}/N_\text{MPI} \gg 1 $, 
the speedup is expected to be almost perfect, in accordance with
the performance test results for the auxiliary field
QMC code  on SuperMUC (see Fig.~\ref{fig_scaling} (left)).

For many problem sizes, 2~GB memory per MPI task (random walker) suffices such that we typically start as many MPI tasks as there are physical cores per node. Due to the large amount of CPU time spent in MKL routines, we do not profit from the hyper-threading option. For large systems, the memory requirement increases and this is tackled by increasing the amount of OpenMP threads to decrease the stress on the memory system and to simultaneously reduce the equilibration overhead (see Fig.~\ref{fig_scaling} (right)). For the displayed speedup, it was crucial to pin the MPI tasks as well as the OpenMP threads in a pattern which keeps the threads as compact as possible to profit from a shared cache. This also explains the drop in efficiency from 14 to 28 threads where the OpenMP threads are spread over both sockets. 

We store the field configurations of the random walker as checkpoints, such that a long simulation can be easily split into several short simulations. This procedure allows us to take advantage of chained jobs using the dependency chains provided by the batch system.

\begin{figure}[H]
	\begin{center}
		\includegraphics[scale=0.6]{Figures/MPI_scaling_ALF_2.pdf}
		\includegraphics[scale=0.6]{Figures/OMP_scaling_ALF_2.pdf}
	\end{center}
	\caption{\label{fig_scaling} MPI (left) and OpenMP (right) scaling behavior of the auxiliary field QMC code of the ALF project on SuperMUC (phase 2/Haswell nodes) at the LRZ in Munich.
		The MPI performance data was normalized to 28 cores and was obtained using a problem size of $N_{\text{dim}}=400$. This is a medium to small system size that is the least favorable in terms of MPI synchronization effects.
		The OpenMP performance data was obtained using a problem size of $N_{\text{dim}}=1296$. Employing 2 and 4 OpenMP threads introduces some synchronization/management overhead such that the per-core performance is slightly reduced, compared to the single thread efficiency. Further increasing the amount of threads to 7 and 14 keeps the efficiency constant. The drop in performance of the 28 thread configuration is due to the architecture as the threads are now spread over both sockets of the node. To obtain the above results, it was crucial to pin the processes in a fashion that keeps the OpenMP threads as compact as possible.}
\end{figure}

%Next to the entire computational time is spent in BLAS routines such that the performance of the code will depend on the particular  implementation of this library. 
%We have found that the code performs well, and that  an efficient  OpenMP  version of the library  can be obtained merely by   loading the corresponding BLAS and LAPACK routines. 
%\mycomment{MB: Do we want to say more about OpenMP here, i.e. that it can be useful when warm-up time is a problem (and getting many CPUs is not). 
%In all other cases, the MPI parallelization is always better than the trivial OpenMP parallelization of library algos.}

% Copyright (c) 2016 The ALF project.
% This is a part of the ALF project documentation.
% The ALF project documentation by the ALF contributors is licensed
% under a Creative Commons Attribution-ShareAlike 4.0 International License.
% For the licensing details of the documentation see license.CCBYSA.

% !TEX root = Doc.tex
%-------------------------------------------------------------------------------------
\section*{License}
%-------------------------------------------------------------------------------------
When we were discussing how to make the ALF code generally available to the world we quickly
agreed that it should be open source so that people can benefit and we have the the hope that the code 
proves useful to others and makes some contribution to the scientific community.
Nevertheless we are all scientists and we have to make ends meet in our careers. Therefore we felt that the scientific practice 
of giving a citation back if one has benefitted from another person's work is something that we felt we can reciprocally hope for, from our users.
To facilitate the communication with our users we have set up our project's homepage \url{alf.physik.uni-wuerzburg.de}
and we hope that it gives us the tools to create a small but vibrant community around the code and provides a suitable
entrypoint for future contributors.
The homepage is also the place where the original source files can be found.
With the coming public release it was necessary to add copyright headers to our source files and to think about the licensing
of our software and therefore the question was on the table of how to make those ideas part of our licensing scheme.
We felt that the Creative Commons licenses are a good way to share our documentation and it is also
accepted well with publishers. Therefore this documentation is licensed to you under a CC-BY-SA license.
This means you can share it and redistribute it as long as you cite the original source and
license your changes under the same license. The details are in the file license.CCBYSA that you shou have received with this documentation.
The source code itself is licensed under a GPL license to keep the source as well as any future work in the community.
To express our desire for a proper attribution we decided to make this a visible part of the license.
To that end we have exercised the rights of section 7 of GPL version 3 and have amended
the license terms with an additional paragraph that expresses our wish that if an author has benfitted from this code
that he/she should consider giving back a citation as specified on \url{alf.physik.uni-wuerzburg.de}.
This is not something that is meant to restrict your freedom of use, but something that we strongly expect to be good scientific conduct.
The original GPL license can be found in the file license.GPL and the additional terms can be found in license.additional.
In favour to our users, \textit{ALF} contains part of the lapack implementation version 3.6.1 from \url{http://www.netlib.org/lapack}.
Lapack is licensed under the modified BSD license whose full text can be found in license.BSD.\\
With that being said, we hope that ALF will prove to you to be a suitable and highly performant tool that enables
you to perform Monte Carlo studies of solid state models of unprecedented complexity.\\
\\
The ALF project's contributors.\\
                        
%-------------------------------------------------------------------------------------
\subsection*{COPYRIGHT}
%-------------------------------------------------------------------------------------

Copyright \textcopyright ~2016, The \textit{ALF} Project.\\
The ALF Project Documentation 
is licensed under a Creative Commons Attribution-ShareAlike 4.0 International License.
You are free to share and benefit from this documentation as long as this license is preserved
and proper attribution to the authors is given. For details see the ALF project
homepage \url{alf.physik.uni-wuerzburg.de} and the file \texttt{license.CCBYSA}.

\end{appendix}
\bibliography{./fassaad,./doc}
\printindex

\nolinenumbers

\end{document}
