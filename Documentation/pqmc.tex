% Copyright (c) 2016-2019 The ALF project.
% This is a part of the ALF project documentation.
% The ALF project documentation by the ALF contributors is licensed
% under a Creative Commons Attribution-ShareAlike 4.0 International License.
% For the licensing details of the documentation see license.CCBYSA.

% !TEX root = doc.tex

The projective  approach is the method of choice if  one is interested in ground state properties.     The starting point is a pair of trial wave functions,  $| \Psi_{T,L/R} \rangle  $,  that are  not orthogonal to the ground state,  $| \Psi_0 \rangle  $:
\begin{equation}
  \langle \Psi_{T,L/R}  | \Psi_0 \rangle  \neq 0. 
\end{equation}
The ground state expectation value of  any  observable  $\hat{O} $ can then be computed by  propagation along the imaginary time axis:
  \begin{equation}
	 \frac{ \langle \Psi_0 | \hat{O} | \Psi_0 \rangle }{ \langle \Psi_0 | \Psi_0 \rangle}   = \lim_{\theta \rightarrow \infty}  
	 \frac{ \langle \Psi_{T,L} | e^{-\theta \hat{H}}  e^{-(\beta - \tau)\hat{H}  }\hat{O} e^{- \tau  \hat{H} }   e^{-\theta \hat{H}} | \Psi_{T,R} \rangle } 
	        { \langle \Psi_{T,L} | e^{-(2 \theta + \beta) \hat{H}  } | \Psi_{T,R} \rangle } 
\end{equation}
The simulations are carried out at large  but finite values of  $\Theta$ so as to guarantee convergence to the ground  state within the statistical uncertainly.   $\beta$ denotes an imaginary time range where  observables 
(time displaced and equal time) can be measured.   The trial wave function is determined up to a phase, and the program will use this gauge choice to  guarantee  that
\begin{equation}
	 \langle \Psi_{T,L} | \Psi_{T,R} \rangle  > 0.
\end{equation}
To use  the projective version of the code, one has to  set  \texttt{projector=.true.}  in the  namespace  of the parameter file dedicated to the specific model. One also has to specify the value of the projection parameter  \texttt{Theta} as well as the
imaginary time interval \texttt{Beta} in which observables will be measured.    Note that time displaced correlation functions  are computed for a $\tau$  interval ranging from $0$ to $\beta$.  The implicit assumption  in this formulation, is that  the projection  parameter  \texttt{Theta}  suffices to reach the ground state.    Since the  computational time scales linearly with  \texttt{Theta}   large projections parameters are computationally not expensive. 


\subsection{ Specification of the trial wave function }

For each flavor, one needs to specify a left and right trial wave function. In the ALF, it is assumed that it is the ground state of  single particle trial  Hamiltonians $\hat{H}_{T, L/R}$, and hence a single Slater determinant.   To be more specific we consider a  single particle Hamiltonian with the  same symmetries   (color and flavor) as the original Hamiltonian: 
\begin{equation}
\hat{H}_{T,L/R} = 
\sum\limits_{\sigma=1}^{N_{\mathrm{col}}}
\sum\limits_{s=1}^{N_{\mathrm{fl}}}
\sum\limits_{x,y}^{N_{\mathrm{dim}}}
\hat{c}^{\dagger}_{x \sigma   s} h_{xy}^{(s, L/R)} \hat{c}^{\phantom\dagger}_{y \sigma s}.
\end{equation}
Ordering the eigenvalues  of the Hamiltonian in ascending order gives  yields the ground state: 
\begin{equation}
	 | \Psi_{T,L/R} \rangle    =     \prod_{\sigma=1}^{N_{\mathrm{col}}}  \prod_{s=1}^{N_{\mathrm{fl}}}      \prod_{n=1}^{N_{\mathrm{part},s}} 
	 \left( \sum_{x=1}^{N_{\mathrm{dim}}}    \hat{c}^{\dagger}_{x \sigma   s} U^{(s, L/R)}_{x,n} \right) 
	  | 0 \rangle 
\end{equation} 
where 
\begin{equation}
	U^{\dagger,(s, L/R)}h^{(s, L/R)}  U^{(s, L/R)}   = \mathrm{Diag} \left(   \epsilon_1^{(s, L/R)}, \cdots, \epsilon_{N_{\mathrm{dim}}}^{(s, L/R)} \right).
\end{equation}
The trial wave function is hence  completely defined by the set of orthogonal vectors  $ U^{(s, L/R)}_{x,n}  $    for  $ n $ ranging from  $ 1 $ to  the number of particles   in each flavor sector,  $N_{\mathrm{part},s}$.    This information  is stored in the \texttt{WaveFunction}   type defined in the module \texttt{WaveFunction\_mod}.    Note  that  owing to the SU(N$_{\mathrm{col}}$) symmetry the color index is not necessary to define  the trial wave function.  The user will have to specify to specify  the trial wave function in the following way:


\begin{lstlisting}[style=fortran]
Do s = 1, N_fl
   Do x = 1,Ndim
      Do n = 1, N_part(s)
         WF_L(s)%P(x,n)  = U^{(s, L)}_{x,n}
         WF_R(s)%P(x,n)  = U^{(s, R)}_{x,n}
      Enddo
   Enddo
Enddo
\end{lstlisting}
In the above    \texttt{WF\_L}  and \texttt{WF\_R}  are arrays  of length $N_{\mathrm{fl}}$ and of type \texttt{WaveFunction}.   Generically,   the  unitary matrix    will  be generated by a
diagonalization routine such that  if the ground state for the given particle number is degenerate, the trial wave function  has a degree of ambiguity  and does not necessarily share the symmetries of the Hamiltonian $\hat{H}_{T, L/R}$.   Since symmetries are  the key to show the absence of negative sign problem,   violating them in the choice of the trial wave function can very well lead to a  sign problem.   It is hence recommended to define the  trail Hamiltonians,  $\hat{H}_{T, L/R}$, such that the ground state  for the given  particle number is non-degenerate. The  real number   \texttt{WL\_L/R(s)\%Degen}   corresponds to the energy difference between the last occupied and fist un-occupied single particle state.  If greater than zero  the trial wave function is non-degenerate, and hence has all the symmetry properties of the trial Hamiltonians, $\hat{H}_{T, L/R}$.    If the \texttt{projector}   variable is set to true, this quantity will be listed in the 
\texttt{info}   file. 

\subsection{Some technical aspects of the projective code.}
If one is interested solely in zero temperature properties, the projective code offers many advantages.   This comes from the related facts that the Green function matrix is a projector, and that scales can be omitted. 
  
In the projective algorithm, one will show  \cite{Assaad08_rev} that 
\begin{equation}\label{eqn:GreenT0_eq}
G(x,\sigma,s,\tau| x',\sigma,s,\tau)  =    \left[ 1 -  U^{>}_{(s)}(\tau)  \left(   U^{<}_{(s)}(\tau) U^{>}_{(s)}(\tau)  \right)^{-1}  U^{<}_{(s)}(\tau) \right]_{x,x'}
\end{equation}
with 
\begin{equation}
  U^{>}_{(s)}(\tau)  =    \prod_{\tau'=1}^{\tau} \bm{B}_{\tau'}^{(s)}   P^{(s),R}  \; \; 
\text{   and    }  \; \; 
  U^{<}_{(s)}(\tau)  =    P^{(s),L, \dagger} \prod_{\tau'=L_{\text{Trotter}} }^{\tau+1} \bm{B}_{\tau'}^{(s)}.   
\end{equation} 
Here  $P^{(s),L/R}$  corresponds to the $N_{\mathrm{dim}} \times N_{\mathrm{part},s} $  submatrix of $U^{(s),L/R}$.      To see that scales can be omitted, we carry out a singular value decomposition: 
\begin{equation}
	U^{>}_{(s)}(\tau)  =\tilde{U}^{>}_{(s)}(\tau)   d^{>} v^{>}   \; \; \text{   and    }  \; \;  U^{<}_{(s)}(\tau)  = v^{<}  d^{<} \tilde{U}^{<}_{(s)}(\tau)   
\end{equation}
such that $ \tilde{U}^{>}_{(s)}(\tau) $ corresponds to a set of column-wise orthogonal vectors.  One will readily show that  scales can be omitted since:
\begin{equation}
G(x,\sigma,s,\tau| x',\sigma,s,\tau)  =    \left[ 1 -  \tilde{U}^{>}_{(s)}(\tau)  \left(   \tilde{U}^{<}_{(s)}(\tau) \tilde{U}^{>}_{(s)}(\tau)  \right)^{-1}  \tilde{U}^{<}_{(s)}(\tau) \right]_{x,x'}.
\end{equation}
Hence for the projective code,  stabilization is never an issue, and arbitrarily large projection parameters  can be reached.  
 
The form of the Green function matrix implies that it is a projector $G^2 = G$.   This property has been used in Ref.~\cite{Feldbach00}  to very efficiently compute the imaginary time displaced correlation functions.  



\subsection{Comparison of finite and projective codes.}

The finite temperature code  operates in the grand canonical ensemble whereas  in the projective   approach  the particle number is fixed.   On finite lattices,   the comparison between both approaches can only  be made at a temperature scale   below which  a finite sized charge gap  emerges.  In Fig.~\ref{PQMC.fig}    we consider a semi-metallic phase  as realized by    the Hubbard model on the Honeycomb lattice  at $U/t=2$. As apparent   at a scale below which  charge fluctuations are  suppressed  both  algorithms yield identical results. 
        
\begin{figure}
\center
\includegraphics[width=0.49\textwidth]{Figures/Projector/Proj_ener.pdf}
\includegraphics[width=0.49\textwidth]{Figures/Projector/Proj_kin.pdf} \\
\includegraphics[width=0.49\textwidth]{Figures/Projector/Proj_chi.pdf}

	\caption{Comparison between the finite temperature and projective code for     Hubbard model on a $6 \times 6 $  Honeycomb lattice at $U/t=2$ and with periodic boundary conditions.   For the projective code,  blue and black symbols,  we have kept $\beta t = 1$ and  varied $\theta$. In all cases   we have used  $\Delta \tau t = 0.1$, no checkerboard decomposition,  and used a symmetric Trotter decomposition.  For this lattice size and choice of boundary conditions, the non-interacting ground state is degenerate since the Dirac   points  belong to the discrete set  of crystal momenta.  To generate the trial wave function we  have lifted this degeneracy by including a K\'ekul\' e mass term  \cite{Lang13} that breaks translation symmetry  (black symbols)  or by adding a  next-next nearest neighbor hopping  breaks the symmetry nematically  and shifts the Dirac points away from the zone boundary  \cite{Ixert14}. As apparent both choices of  trial wave functions yield the same answer  that compares very well with the finite temperature code at temperature scales below the finite size charge gap.   }
	\label{PQMC.fig}
\end{figure}

