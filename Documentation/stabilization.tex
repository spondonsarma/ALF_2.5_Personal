% !TEX root = doc.tex
% Copyright (c) 2017 The ALF project.
% This is a part of the ALF project documentation.
% The ALF project documentation by the ALF contributors is licensed
% under a Creative Commons Attribution-ShareAlike 4.0 International License.
% For the licensing details of the documentation see license.CCBYSA.
%
%-----------------------------------------------------------------------------------
\subsection{Stabilization - a peculiarity of the BSS algorithm}\label{sec:stable}
%-----------------------------------------------------------------------------------
%
From the partition function in Eq.~\eqref{eqn:partition_2} it can be seen that, for the calculation of the Monte Carlo weight and of the observables, a long product of matrix exponentials has to be formed.
In addition to that, we need to be able to extract the single-particle Green function  for a given flavor index at, say, time slice $\tau = 0$.  As  mentioned above (cf. Eq.~\eqref{eqn:Green_eq}), this quantity is given by: 
\begin{equation}
\bm{G}= \left( \mathds{1} + \prod_{ \tau= 1}^{L_{\text{Trotter}}} \bm{B}_\tau \right)^{-1},
\end{equation}
which can be recast as the more familiar linear algebra problem of finding a solution for the linear system
\begin{equation}
\left(\mathds{1} + \prod_\tau \bm{B}_\tau\right) x = b.
\end{equation}
The matrices $\bm{B}_\tau \in \mathbb{C}^{n\times n}$ depend on the lattice size as well as other physical parameters that can be chosen such that a matrix norm of $\bm{B}_\tau$ can be unbound in size.
From standard perturbation theory for linear systems, the computed solution $\tilde{x}$ would 
contain a relative error
\begin{equation}
\frac{|\tilde{x} - x|}{|x|} = \mathcal{O}\left(\epsilon \kappa_p\left(\mathds{1} + \prod_\tau \bm{B}_\tau\right)\right),
\end{equation}
where $\epsilon$ denotes the machine precision, which is $2^{-53}$ for IEEE double-precision numbers, and $\kappa_p(\bm{M})$ is the condition number of the matrix $\bm{M}$ with respect to the matrix $p$-norm. Due to $\prod_ \tau \bm{B}_\tau$ containing exponentially large and small scales, as can be seen in Eq.~\eqref{eqn:partition_2}, a straightforward inversion turns out to be completely ill-suited. That would lead the condition number, as a function of increasing inverse temperature, to grow exponentially, rendering the computed solution $\tilde{x}$ meaningless.

In order to circumvent this, more sophisticated methods have to be employed. As a first step, assuming that the multiplication of \texttt{NWrap} $\bm{B}$ matrices has an acceptable condition number and, for simplicity, that \texttt{NWrap} is a divisor of $L_{\text{Trotter}}$, we can write:
%\begin{equation}
%\bm{G} = \left( \mathds{1} + \prod\limits_{ i = 0}^{\frac{L_{\text{Trotter}}} {\texttt{NWrap} -1}}       \underbrace{\prod_{\tau=1}^{\texttt{NWrap}} \bm{B}_{i  \cdot  \texttt{NWrap}+ \tau} }_{ \equiv \mathcal{\bm{B}}_i}\right)^{-1}.
%\end{equation}
\begin{equation}
\bm{G} = \left( \mathds{1} + \prod\limits_{ i = 1}^{\frac{L_{\text{Trotter}}} {\texttt{NWrap}}}       \underbrace{\prod_{\tau=1}^{\texttt{NWrap}} \bm{B}_{(i-1)  \cdot  \texttt{NWrap}+ \tau} }_{ \equiv \mathcal{\bm{B}}_i}\right)^{-1}.
\end{equation}
The default stabilization strategy in the auxiliary-field QMC implementation of the ALF project, is then to form a product of QR-decompositions, which was proven to be weakly backwards stable in \cite{Bai2011}.
The key idea is to efficiently separate the scales of a matrix from the orthogonal part of a matrix.
This can be achieved using a QR decomposition of a matrix $\bm{A}$ in the form $\bm{A}_i = \bm{Q}_i \bm{R}_i$. The matrix $\bm{Q}_i$ is unitary and hence in the usual $2$-norm it holds that $\kappa_2(\bm{Q}_i) = 1$.
To get a handle on the condition number of $\bm{R}_i$ we will form the
diagonal matrix
\begin{equation}
(\bm{D}_i)_{n,n} = |(\bm{R}_i)_{n,n}|
\label{eq:diagnorm}
\end{equation}
and set $\tilde{\bm{R}}_i = \bm{D}_i^{-1} \bm{R}_i$
This gives the decomposition
\begin{equation}
\bm{A}_i = \bm{Q}_i \bm{D}_i \tilde{\bm{R}}_i.
\end{equation}
$\bm{D}_i$ now contains the row norms of the original $\bm{R}_i$ matrix and hence attempts to separate off the total scales of the problem from $\bm{R}_i$.
This is similar in spirit to the so-called matrix equilibration which tries to improve the condition number of a matrix through suitably chosen column and row scalings.
Due to a theorem by van der Sluis \cite{vanderSluis1969} we know that the choice in Eq.~\eqref{eq:diagnorm} is almost optimal among all diagonal matrices $\bm{D}$ from the space of diagonal matrices $\mathcal{D}$, in the sense that
\begin{equation*}
\kappa_p((\bm{D}_i)^{-1} \bm{R}_i ) \leq n^{1/p} \min_{\bm{D} \in \mathcal{D}} \kappa_p(\bm{D}^{-1} \bm{R}_i).
\end{equation*}
Now, given an initial decomposition of $\bm{A}_{j-1} = \prod_i \mathcal{\bm{B}}_i = \bm{Q}_{j-1} \bm{D}_{j-1} \bm{T}_{j-1}$ an update
$\mathcal{\bm{B}}_j \bm{A}_{j-1}$ is formed in the following three steps:
\begin{enumerate}
\item Form $ \bm{M}_j = (\mathcal{\bm{B}}_j \bm{Q}_{j-1}) \bm{D}_{j-1}$. Note the parentheses.
\item Do a QR decomposition of $\bm{M}_j = \bm{Q}_j \bm{D}_j \bm{R}_j$. This gives the final $\bm{Q}_j$ and $\bm{D}_j$.
\item Form the updated $\bm{T}$ matrices $\bm{T}_j = \bm{R}_j \bm{T}_{j-1}$.
\end{enumerate}
%While this provides provides a stable method to calculate the involved matrix product
%it can be pretty expensive. Therefore the user can specify to skip a certain number of 
%QR Decompositions and perform plain multiplications instead. This is specified in the parameters file by the \path{NWrap} parameter.
%\path{NWrap}~=~1 corresponds to always performing QR decompositions whereas larger integers give longer intervals where no QR decomposition will be performed.
The effectiveness of the stabilization \emph{has} to be judged for every simulation from the output file \path{info} (Sec.~\ref{sec:output_obs}). For most simulations there are two values to look out for:
\begin{itemize}
\item \texttt{Precision Green}
\item \texttt{Precision Phase}
\end{itemize}
The Green function, as well as the average phase, are usually numbers with a magnitude of $\mathcal{O} (1)$. 
For that reason we recommend that \path{NWrap} is chosen such that the mean precision is of the order of $10^{-8}$ or better (or further recommendations see Sec.~\ref{sec:optimize}).
We include typical values of \texttt{Precision Phase} and of the mean and the maximal values of \texttt{Precision Green} in the example simulations discussed in Sec.~\ref{sec:prec_charge} and Sec.~\ref{sec:prec_spin}.


%-----------------------------------------------------------------------------------
\subsection{Stabilization - ensuring Hermitian evolution}\label{sec:hermitian}
%-----------------------------------------------------------------------------------
%

\red{\large TODO}\\ \\

parameter \texttt{Symm = .true.}: \\

Green's functions will be symmetrized before being  sent to the Obser, ObserT subroutines. 

In particular, the transformation,  $ \tilde{G} =  e^{-\Delta \tau T /2 } G e^{\Delta \tau T /2 } $
 will be carried out  and $ \tilde{G} $  will be sent to the Obser and ObserT subroutines.  Note that
if you want to use this  feature, then you have to be sure the hopping and interaction terms are decomposed
symmetrically. If Symm is true, the propagation reads:
\begin{align}
\prod_{\tau} \; \;  \prod_{n=N_T}^{1}e^{T_n/2} \prod_{n=1}^{N_V}e^{V_n(\tau)}  \prod_{n=1}^{N_T}e^{T_n/2}
\end{align}

