% !TEX root = doc.tex
% Copyright (c) 2017 The ALF project.
% This is a part of the ALF project documentation.
% The ALF project documentation by the ALF contributors is licensed
% under a Creative Commons Attribution-ShareAlike 4.0 International License.
% For the licensing details of the documentation see license.CCBYSA.
%
%-----------------------------------------------------------------------------------
\subsection{Stabilization - a peculiarity of the BSS algorithm}\label{sec:stable}
%-----------------------------------------------------------------------------------
%
From \eqref{eqn:partition_2} it can be seen that for the calculation of the Monte Carlo weight
and for the observables a long product of matrix exponentials has to be formed.
On top of that we need to be able to extract the single-particle Green function  for a given flavor index at say time slice $\tau = 0$.  As  mentioned above in Eq.~(\ref{eqn:Green_eq}), this quantity is given by: 
\begin{equation}
\bm{G}= \left( \mathds{1} + \prod_{ \tau= 1}^{L_{\text{Trotter}}} \bm{B}_\tau \right)^{-1}.
\end{equation}
To boil this down to more familiar terms from linear algebra we remark that we can recast this problem as the question to the solution of the linear system
\begin{equation}
(\mathds{1} + \prod_\tau \bm{B}_\tau) x = b.
\end{equation}
The $\bm{B}_\tau$ depend on the system size as well as other physical parameters that can be chosen such that a matrix norm of $B_i$ can have any number.
From standard perturbation theory for linear systems it is known that the computed solution $\tilde{x}$ would 
contain a relative error of
\begin{equation}
\frac{|\tilde{x} - x|}{|x|} = \mathcal{O}\left(\epsilon \kappa(\mathds{1} + \prod_\tau \bm{B}_\tau)\right).
\end{equation}
Here $\epsilon$ denotes the machine precision which is $2^{-53}$ for IEEE double precision numbers
and $\kappa(\bm{M})$ is the condition number of the matrix $\bm{M}$.
The important fact that makes straight-forward inversion so badly suited  stems from the fact that $  \prod_ \tau \bm{B}_\tau $ contains exponentially large and small scales as can be seen in Eq.~\eqref{eqn:partition_2}.  Thereby, as a function of increasing inverse temperature, 
the condition number  will grow exponentially so that the computed solution $\tilde{x}$
will often contain no correct digits at all.
To circumvent this, more sophisticated methods have to be employed. We will first of all assume that the multiplication of \texttt{NWRAP} $\bm{B}$ matrices has an acceptable condition number.
Assuming for simplicity that \texttt{NWRAP} is a multiple of $L_{\text{Trotter}}$, we can write:
\begin{equation}
\bm{G} = \left( \mathds{1} + \prod\limits_{ i = 0}^{\frac{L_{\text{Trotter}}} {\texttt{NWrap} -1}}       \underbrace{\prod_{\tau=1}^{\texttt{NWrap}} \bm{B}_{i  \cdot  \texttt{NWrap}+ \tau} }_{ \equiv \mathcal{\bm{B}}_i}\right)^{-1}.
\end{equation}

Within the auxiliary field QMC implementation of the ALF project, we are by default employing
the strategy of forming a product of QR-decompositions which was proven to be weakly backwards stable in \cite{Bai2011}.
The key idea is to efficiently separate the scales of a matrix from the orthogonal part of a matrix.
This can be achieved using a QR decomposition of a matrix $\bm{A}$ in the form $\bm{A}_i = \bm{Q}_i \bm{R}_i$. The matrix $\bm{Q}_i$ is unitary and hence $\kappa(\bm{Q}_i) = 1$.
To get a handle on the condition number of $\bm{R}_i$ we will form the
diagonal matrix
\begin{equation}
(\bm{D}_i)_{n,n} = |(\bm{R}_i)_{n,n}|
\label{eq:diagnorm}
\end{equation}
and set $\tilde{\bm{R}}_i = \bm{D}_i^{-1} \bm{R}_i$
This gives the decomposition
\begin{equation}
\bm{A}_i = \bm{Q}_i \bm{D}_i \tilde{\bm{R}}_i.
\end{equation}
$\bm{D}_i$ now contains the row norms of the original $\bm{R}_i$ matrix and hence attempts to separate off the total scales of the problem from $\bm{R}_i$.
This is similar in spirit to the so-called matrix equilibration which tries to improve the condition number of a matrix by suitably chosen column and row scalings.
Due to a theorem by van der Sluis \cite{vanderSluis1969} we know that the choice in \eqref{eq:diagnorm} is almost optimal among all diagonal matrices $\bm{D}$ from the space of diagonal matrices 
$\mathcal{D}$ in the sense that
\begin{equation*}
\kappa_p((\bm{D}_i)^{-1} \bm{R}_i ) \leq n^{1/p} \min_{\bm{D} \in \mathcal{D}} \kappa_p(\bm{D}^{-1} \bm{R}_i).
\end{equation*}
Now, given an initial decomposition of $\bm{A}_{j-1} = \prod_i \mathcal{\bm{B}}_i = \bm{Q}_{j-1} \bm{D}_{j-1} \bm{T}_{j-1}$ an update
$\mathcal{\bm{B}}_j \bm{A}_{j-1}$ is formed in the following three steps:
\begin{enumerate}
\item Form $ \bm{M}_j = (\mathcal{\bm{B}}_j \bm{Q}_{j-1}) \bm{D}_{j-1}$. Note the parentheses.
\item Do a QR decomposition of $\bm{M}_j = \bm{Q}_j \bm{D}_j \bm{R}_j$. This gives the final $\bm{Q}_j$ and $\bm{D}_j$.
\item Form the updated $\bm{T}$ matrices $\bm{T}_j = \bm{R}_j \bm{T}_{j-1}$.
\end{enumerate}
%While this provides provides a stable method to calculate the involved matrix product
%it can be pretty expensive. Therefore the user can specify to skip a certain number of 
%QR Decompositions and perform plain multiplications instead. This is specified in the parameters file by the \path{NWrap} parameter.
%\path{NWrap}~=~1 corresponds to always performing QR decompositions whereas larger integers give longer intervals where no QR decomposition will be performed.
The effectiveness of the stabilization \emph{has} to be judged for every simulation from the info
file. For most simulations there are two values to look out for:
\begin{itemize}
\item \path{Precision Green}
\item \path{Precision Phase}
\end{itemize}
The Green's function as well as the average phase are usually numbers with a magnitude of $\mathcal{O} (1)$. 
For that reason we recommend that \path{Nwrap} is chosen such that the mean precision is of the order of $10^{-8}$ or better.  
\mycomment{Think about formulation}
