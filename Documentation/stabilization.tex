\subsection{Stabilization - A Peculiarity of the BSS Algorithm}
From \eqref{eqn:partition_2} it can be seen that for the calculation of the Monte Carlo weight
and for the observables a long product of matrix exponentials has to be formed.
On top of that we need to be able to extract the single particle Greensfunction from 
the equation 
\begin{equation}
G = \left( 1 + \prod_{i = 1}^{N_\tau} B(\tau_i, \tau_{i+1})\right)^{-1}
\end{equation}
To boil this down to more familiar terms from linear algebra we remark that we can recast this problem as the question to the solution of the linear system
\begin{equation}
(1 + \prod_i B_i) x = b.
\end{equation}
The $B_i$ depend on the system size as well as other physical parameters that can be chosen such that a matrix norm of $B_i$ can have any number.
From standard perturbation theory for linear systems it is known that the computed solution $\tilde{x}$ would 
contain a relative error of
\begin{equation}
\frac{|\tilde{x} - x|}{|x|} = \mathcal{O}\left(\epsilon \kappa(1 + \prod_i B_i)\right).
\end{equation}
Here $\epsilon$ denotes the machine precision which is $2^{-53}$ for IEEE double precision numbers
and $\kappa(M)$ is the condition number of the matrix $M$.
The important fact that makes straight-forward inversion so badly suited is the point that
each of the $B_i$ contains exponentially large and small scales as can be seen in \eqref{eqn:partition_2}.
The condition number in the above expression is determined by the product of the $B_i$
and hence can be virtually unbounded in size. This means that the so computed solution $\tilde{x}$
will often contain no correct digits at all.
To circumvent that more sophisticated methods have to be employed. ALF is by default employing
the strategy of forming a product of QR-decompositions which was proven to be weakly backwards stable in \cite{Bai2011}.
The key idea is to efficiently separate the scales of a matrix from the orthogonal part of a matrix.
This can be achieved using a QR decomposition of the $B_i = Q_i \tilde{R_i}$. Now $Q_i$ is a unitary matrix and hence $\kappa(Q_i) = 1$.
To get a handle on the condition number of $\tilde{R}_i$ we will form the
diagonal matrix $(D_i)_{jj} = |(\tilde{R}_i)_{jj}|$ and rescale $\tilde{R}_i$ accordingly, $\tilde{R}_i = D_i R_i$.
This gives the decomposition
\begin{equation}
B_i = Q_i D_i R_i
\end{equation}
$D_i$ now contains the row norms of the original $\tilde{R}_i$ matrix and hence separates off the total scales of the problem since $R_i$ is now only of modest condition number.
So given an initial decomposition of $B_{j-1} = Q_{j-1} D_{j-1} T_{j-1}$ any product 
of $B$ matrices is formed in the following two steps:
\begin{enumerate}
\item Form $ M_j = (B_j Q_{j-1}) D_{j-1}$. Note the parentheses.
\item Do a QR decomposition of $M_j = Q_j D_j R_j$.
\item Form the updated $R$ matrices $T_j = R_j T_{j-1}$.
\end{enumerate}
While this provides provides a stable method to calculate the involved matrix product
it can be pretty expensive. Therefore the user can specify to skip a certain number of 
QR Decompositions and perform plain multiplications instead. This is specified in the parameters file by the \path{NWrap} parameter.
\path{NWrap}~=~1 corresponds to always performing QR decompositions whereas larger integers give longer intervals where no QR decomposition will be performed.
The effectiveness of the stabilization \emph{HAS} to be judged for every simulation from the info
file. For most simulations there are two values to look out for:
\begin{itemize}
\item \path{Precision Green}
\item \path{Precision Phase}
\end{itemize}
The Green's function as well as the average phase are usually numbers with a magnitude of $\mathcal{O} (1)$. 
For that reason we recommend that \path{Nwrap} is chosen such that the precision is between $10^{-10} - 10^{-16}$.
