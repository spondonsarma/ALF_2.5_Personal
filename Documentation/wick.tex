% Copyright (c) 2016, 2020 The ALF project.
% This is a part of the ALF project documentation.
% The ALF project documentation by the ALF contributors is licensed
% under a Creative Commons Attribution-ShareAlike 4.0 International License.
% For the licensing details of the documentation see license.CCBYSA.

% !TEX root = doc.tex

%-------------------------------------------------------------------------------------
\section{Practical implementation of Wick decomposition of 2n-point correlation functions of two imaginary times } \label{sec:wick}
%-------------------------------------------------------------------------------------

In this Appendix,  we briefly  outline how to compute 2n point correlation functions   of the form: 
\begin{align}
	\lim_{\epsilon \rightarrow 0  } & \sum_{\sigma_1, \sigma'_1, \cdots, \sigma_n, \sigma'_n,  s_1, s'_1  \cdots s_n,  s'_n  }  f( \sigma_1, \sigma'_1, \cdots, \sigma_n, \sigma'_n,  s_1, s'_1  \cdots s_n,  s'_n ) 
	\nonumber    \\
         &  \, \, \, \,        \langle \langle {\cal T}  \left( c^{\dagger}_{x_1,\sigma_1,s_1}(\tau_{1,\epsilon}) c^{\phantom\dagger}_{x'_{1},\sigma'_1,s'_1}(\tau'_{1,\epsilon}) - a_1  \right)  \cdots 
	    \left( c^{\dagger}_{x_n,\sigma_n,s_n}(\tau_{n,\epsilon}) c^{\phantom\dagger}_{x'_{n},\sigma'_n,s'_m} (\tau'_{n,\epsilon}) - a_n  \right)   \rangle \rangle_C
\end{align}
Here,  $ \sigma $ is a color  index and $s$ a flavor index such that 
\begin{equation}
	\langle \langle {\cal T}  c^{\dagger}_{x,\sigma,s}(\tau) c^{\phantom\dagger}_{x',\sigma',s'}(\tau')  \rangle \rangle_C  = 
	\langle \langle {\cal T}  c^{\dagger}_{x,s}(\tau) c^{\phantom\dagger}_{x',s}(\tau')  \rangle \rangle_C  \, \, \delta_{s,s'} \delta_{\sigma,\sigma'}.
\end{equation}
That is, the single particle Green function is diagonal in the flavor index  and color  independent.   To  define the time ordering we will assume  that all times differ  but that $  \lim_{\epsilon \rightarrow 0 }    \tau_{n,\epsilon}  $   as well as $ \lim_{\epsilon \rightarrow 0 }    \tau'_{n,\epsilon} $  take the values $0$  or $\tau$.  

{\color{red}  ToDo   1) Define $G(I,J,s)$  2) Relation to input  $G0T ..$, and then the final  determinant formula, and how you carry out the sum   with Mathematica.   Sympy as open source alternative.   }  


\begin{align}
\begin{aligned}
\texttt{GT0(x,y,s) }  &=   \phantom{+} \langle \langle \hat{c}^{\phantom\dagger}_{x,s} (Nt \Delta \tau)   \hat{c}^{\dagger}_{y,s} (0)   \rangle \rangle \;=\; \langle \langle \mathcal{T} \hat{c}^{\phantom\dagger}_{x,s} (Nt \Delta \tau)   \hat{c}^{\dagger}_{y,s} (0)   \rangle \rangle   \\
\texttt{G0T(x,y,s) }   &=  -   \langle \langle   \hat{c}^{\dagger}_{y,s} (Nt \Delta \tau)    \hat{c}^{\phantom\dagger}_{x,s} (0)    \rangle \rangle \;=\;
    \langle \langle \mathcal{T} \hat{c}^{\phantom\dagger}_{x,s} (0)    \hat{c}^{\dagger}_{y,s} (Nt \Delta \tau)   \rangle \rangle    \\
  \texttt{G00(x,y,s) }  &=    \phantom{+} \langle \langle \hat{c}^{\phantom\dagger}_{x,s} (0)   \hat{c}^{\dagger}_{y,s} (0)   \rangle \rangle     \\
    \texttt{GTT(x,y,s) }  &=   \phantom{+} \langle \langle \hat{c}^{\phantom\dagger}_{x,s} (Nt \Delta \tau)   \hat{c}^{\dagger}_{y,s} (Nt \Delta \tau)   \rangle \rangle.
\end{aligned}
\end{align}



\begin{multline}
\langle \langle 	\mathcal{T}   c^{\dagger}_{\underline x_{1}}(\tau_{1}) c^{\phantom\dagger}_{{\underline x}'_{1}}(\tau'_{1})  
\cdots c^{\dagger}_{\underline x_{n}}(\tau_{n}) c^{\phantom\dagger}_{{\underline x}'_{n}}(\tau'_{n}) 
\rangle \rangle_{C} = \\
\det  
\begin{bmatrix}
   \langle \langle   \mathcal{T}   c^{\dagger}_{\underline x_{1}}(\tau_{1}) c^{\phantom\dagger}_{{\underline x}'_{1}}(\tau'_{1})  \rangle \rangle_{C} & 
    \langle \langle  \mathcal{T}   c^{\dagger}_{\underline x_{1}}(\tau_{1}) c^{\phantom\dagger}_{{\underline x}'_{2}}(\tau'_{2})  \rangle \rangle_{C}  & \dots   &   
    \langle \langle   \mathcal{T}   c^{\dagger}_{\underline x_{1}}(\tau_{1}) c^{\phantom\dagger}_{{\underline x}'_{n}}(\tau'_{n})  \rangle \rangle_{C}  \\
    \langle \langle   \mathcal{T}   c^{\dagger}_{\underline x_{2}}(\tau_{2}) c^{\phantom\dagger}_{{\underline x}'_{1}}(\tau'_{1})  \rangle \rangle_{C}  & 
      \langle \langle   \mathcal{T}   c^{\dagger}_{\underline x_{2}}(\tau_{2}) c^{\phantom\dagger}_{{\underline x}'_{2}}(\tau'_{2})  \rangle \rangle_{C}  & \dots  &
       \langle \langle   \mathcal{T}   c^{\dagger}_{\underline x_{2}}(\tau_{2}) c^{\phantom\dagger}_{{\underline x}'_{n}}(\tau'_{n})  \rangle \rangle_{C}   \\
    \vdots & \vdots &  \ddots & \vdots \\
    \langle \langle   \mathcal{T}   c^{\dagger}_{\underline x_{n}}(\tau_{n}) c^{\phantom\dagger}_{{\underline x}'_{1}}(\tau'_{1})  \rangle \rangle_{C}   & 
     \langle \langle   \mathcal{T}   c^{\dagger}_{\underline x_{n}}(\tau_{n}) c^{\phantom\dagger}_{{\underline x}'_{2}}(\tau'_{2})  \rangle \rangle_{C}   & \dots  & 
     \langle \langle   \mathcal{T}   c^{\dagger}_{\underline x_{n}}(\tau_{n}) c^{\phantom\dagger}_{{\underline x}'_{n}}(\tau'_{n})  \rangle \rangle_{C}
 \end{bmatrix}.
\end{multline}
Here, we have defined the super-index $\underline{ x} = \left\{   x,\sigma,s \right\}$.