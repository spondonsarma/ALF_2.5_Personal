% Copyright (c) 2016 2017 The ALF project.
% This is a part of the ALF project documentation.
% The ALF project documentation by the ALF contributors is licensed
% under a Creative Commons Attribution-ShareAlike 4.0 International License.
% For the licensing details of the documentation see license.CCBYSA.
% !TEX root = Model_classes.tex 
\label{Z2.Sec}
The Hamiltonian we will consider here reads
\begin{align}
	\hat{H} =& -  t_{Z_2} \sum_{\langle \vec{i}, \vec{j} \rangle, \sigma } \hat{\sigma}^z_{\langle \vec{i}, \vec{j} \rangle}
	\left(\hat{\Psi}^{\dagger}_{\vec{i},\sigma} \hat{\Psi}^{\phantom{\dagger}}_{\vec{j},\sigma}   + h.c. \right) - \mu \sum_{\vec{i},\sigma} \hat{\Psi}^{\dagger}_{\vec{i},\sigma} \hat{\Psi}^{\phantom{\dagger}}_{\vec{i},\sigma}  
	-g \sum_{\langle \vec{i}, \vec{j} \rangle } \hat{\sigma}^{x}_{\langle \vec{i}, \vec{j} \rangle }  +
	  K \sum_{\square} \prod_{\langle \vec{i}, \vec{j} \rangle \in \partial \square} \hat{\sigma}^{z}_{\langle \vec{i}, \vec{j} \rangle}  \nonumber \\
	& + J  \sum_{\langle \vec{i}, \vec{j} \rangle}  \hat{\tau}^z_{\pmb{i}}  \hat{\sigma}^{z}_{\langle \vec{i}, \vec{j} \rangle} \hat{\tau}^z_{\pmb{j}}   
	      -  h \sum_{ \vec{i} } \hat{\tau}^x_{\vec{i}}   - t  \sum_{\langle \vec{i}, \vec{j} \rangle, \sigma }   \hat{\tau}^z_{\pmb{i}}   \hat{\tau}^z_{\pmb{j}}  \left( \hat{\Psi}^{\dagger}_{\vec{i},\sigma} \hat{\Psi}^{\phantom{\dagger}}_{\vec{j},\sigma} 	+ h.c. \right) + \frac{U}{N}\sum_{\ve{i}} \left[ \sum_{\sigma}  ( \hat{\Psi}^{\dagger}_{\ve{i},\sigma}  \hat{\Psi}^{\phantom\dagger}_{\ve{i},\sigma} - 1/2 ) \right]^2 
\end{align}  
The model is defined on a square lattice, and describers fermions, 
\begin{align}
 \left\{ \hat{\Psi}^{\dagger}_{\vec{i},\sigma},  \hat{\Psi}^{\phantom\dagger}_{\vec{j},\sigma'} \right\}  = \delta_{\vec{i},\vec{j}} \delta_{\sigma,\sigma'}, \;  
\left\{ \hat{\Psi}^{\phantom\dagger}_{\vec{i},\sigma},  \hat{\Psi}^{\phantom\dagger}_{\vec{j},\sigma'} \right\}  =  0,  
\end{align}
coupled to  bond gauge fields 
\begin{align}
\hat{\sigma}^{z}_{\langle \vec{i}, \vec{j} \rangle}  = 
\begin{bmatrix}
1 & 0 \\
0 & -1 
\end{bmatrix},  
\hat{\sigma}^x_{\langle \vec{i}, \vec{j} \rangle}  = 
\begin{bmatrix}
0 & 1 \\
1 & 0 
\end{bmatrix},   
\left\{ \hat{\sigma}^{z}_{\langle \vec{i}, \vec{j} \rangle} , \hat{\sigma}^x_{\langle \vec{i}', \vec{j}'  \rangle} \right\}  =  2 \left( 1 -  \delta_{\langle \vec{i}, \vec{j}  \rangle, \langle \vec{i}', \vec{j}'  \rangle } \right) 
\hat{\sigma}^{z}_{\langle \vec{i}, \vec{j} \rangle}  \hat{\sigma}^x_{\langle \vec{i}', \vec{j}'  \rangle}
\end{align}
and  $Z_2$ matter fields:
\begin{align}
 \hat{\tau}_{\ve{i}}^{z} = 
\begin{bmatrix}
1 & 0 \\
0 & -1 
\end{bmatrix},  
\hat{\tau}^{x}_{\vec{i} }  = 
\begin{bmatrix}
0 & 1 \\
1 & 0 
\end{bmatrix}, 
\left\{ \hat{\tau}^{z}_{ \vec{i} } , \hat{\tau}^x_{\vec{i}'} \right\}  =  2 \left( 1 -  \delta_{ \vec{i}, \vec{i}' } \right) 
\hat{\tau}^{z}_{ \vec{i} }  \hat{\tau}^x_{ \vec{i}' }.
\end{align}
 Fermions,  gauge fields and $Z_2$ matter fields commute with each other. 
 
 Importantly,  the model has a local  $Z_2$ symmetry. Consider:
\begin{equation}
	\hat{Q}_{\vec{i}} =  (-1)^{\sum_{\sigma} \hat{\Psi}^{\dagger}_{\vec{i},\sigma} \hat{\Psi}^{\phantom{\dagger}}_{\vec{i},\sigma}   } 
	\;  \hat{\tau}^{x}_{\vec{i}}  \; \hat{\sigma}^{x}_{\vec{i},\vec{i} +  \vec{a}_x} \hat{\sigma}^{x}_{\vec{i},\vec{i} -  \vec{a}_x} \hat{\sigma}^{x}_{\vec{i},\vec{i} +  \vec{a}_y} \hat{\sigma}^{x}_{\vec{i}}.
\end{equation} 
One can  then show that  $\hat{Q}_{\vec{i}}^2 = 1 $ and that
\begin{equation}
	\left[   \hat{Q}_{\vec{i}}, \hat{H}  \right]  = 0. 
\end{equation} 
The above allows us to assign $Z_2$  charges to the operators.   Since $ \left\{ \hat{Q}_{\vec{i}},   \hat{\Psi}^{\dagger}_{\vec{i},\sigma} \right\} =    0 $ we can assign a $Z_2$ to the fermions.  Equivalently 
$\hat{\tau}^{z}_{\ve{i}}$   has a $Z_2$ charge and $\hat{\sigma}^{z}_{\vec{i},\vec{j}} $   carries  $Z_2$ charges at it's ends.   
 Since the total fermion number is conserved,   we can equally assign an electric charge to the  fermions.  

 Finally, since  the model has an SU(N) color symmetry.  In fact, at zero chemical potential and $U=0$,    the symmetry is enhanced to O(2N) \cite{Assaad16}.
Aspect of this Hamiltonian were investigated in Ref.~\cite{Assaad16,Gazit16,Gazit18,Gazit19,Hohenadler18,Hohenadler19}  and we refer  the interested user these papers for a discussion of the phases and phase transitions supported by the model.  

\subsubsection*{ QMC implementation } 

The key point to implement the model,  is to  define a new bond variable: 
\begin{equation}
	\hat{\mu}^{z}_{ \langle  \ve{i}, \ve{j}  \rangle }  =  \hat{\tau}^{z}_{ \ve{i}}\hat{\tau}^{z}_{\ve{j}  }. 
\end{equation}  
By construction, the $\hat{\mu}^{z}_{ \langle  \ve{i}, \ve{j}  \rangle } $ bond variables   have a zero flux constraint:
\begin{equation}
	\hat{\mu}^{z}_{ \langle  \ve{i}, \ve{i} + \ve{a}_x  \rangle }  \hat{ \mu}^{z}_{ \langle  \ve{i} + \ve{a}_x, \ve{i} + \ve{a}_x  + \ve{a}_y \rangle } 
	\hat{\mu}^{z}_{ \langle  \ve{i} + \ve{a}_x + \ve{a}_y, \ve{i} +  \ve{a}_y \rangle } \hat{\mu}^{z}_{ \langle  \ve{i} + \ve{a}_y, \ve{i}  \rangle }  = 1. 
\label{zero_flux.eq}
\end{equation} 
The sole knowledge of the  bond variables   $ \left\{ \mu^{z}_{ \langle  \ve{i}, \ve{j} \rangle }  \right\} $ ,   satisfying the above constraint, allows one to determine the  matter fields $ \left\{ \tau^{z}_{\ve{i}} \right\} $ 
only up to a global sign.    This sign is fixed by  keeping track of a single  matter field,   say  $\tau^{z}_{\ve{i}  = \ve{0}} $,

The model  can then be written as:
\begin{align}
	\hat{H} =& -  t_{Z_2} \sum_{\langle \vec{i}, \vec{j} \rangle, \sigma } \hat{\sigma}^z_{\langle \vec{i}, \vec{j} \rangle}
	\left(\hat{\Psi}^{\dagger}_{\vec{i},\sigma} \hat{\Psi}^{\phantom{\dagger}}_{\vec{j},\sigma}   + h.c. \right) - \mu \sum_{\vec{i},\sigma} \hat{\Psi}^{\dagger}_{\vec{i},\sigma} \hat{\Psi}^{\phantom{\dagger}}_{\vec{i},\sigma}  
	-g \sum_{\langle \vec{i}, \vec{j} \rangle } \hat{\sigma}^{x}_{\langle \vec{i}, \vec{j} \rangle }  +
	  K \sum_{\square} \prod_{\langle \vec{i}, \vec{j} \rangle \in \partial \square} \hat{\sigma}^{z}_{\langle \vec{i}, \vec{j} \rangle}  \nonumber \\
	& + J  \sum_{\langle \vec{i}, \vec{j} \rangle}  \hat{\mu}^z_{ \langle \pmb{i}, \ve{j} \rangle }  \hat{\sigma}^{z}_{\langle \vec{i}, \vec{j} \rangle}    
	      -  h \sum_{ \vec{i} } \hat{\mu}^{x}_{\ve{i},\ve{i} + \ve{a}_x } \hat{\mu}^{x}_{\ve{i} + \ve{a}_x, \ve{i} + \ve{a}_x + \ve{a}_y }   \hat{\mu}^{x}_{\ve{i} + \ve{a}_x + \ve{a}_y, \ve{i} + \ve{a}_y  }
	         \hat{\mu}^{x}_{\ve{i} + \ve{a}_y, \ve{i}  }	  \\      
	&        - t  \sum_{\langle \vec{i}, \vec{j} \rangle, \sigma }   \hat{\mu}^z_{\pmb{i},\ve{j}}    \left( \hat{\Psi}^{\dagger}_{\vec{i},\sigma} \hat{\Psi}^{\phantom{\dagger}}_{\vec{j},\sigma} 	+ h.c. \right) + \frac{U}{N}\sum_{\ve{i}} \left[ \sum_{\sigma}  ( \hat{\Psi}^{\dagger}_{\ve{i},\sigma}  \hat{\Psi}^{\phantom\dagger}_{\ve{i},\sigma} - 1/2 ) \right]^2 
\end{align}  
subject to the constraint of Eq.~\ref{zero_flux.eq}.  
Note that $\left\{ \hat{\mu}^{z}_{\langle \vec{i}, \vec{j} \rangle} , \hat{\mu}^x_{\langle \vec{i}', \vec{j}'  \rangle} \right\}  =  2 \left( 1 -  \delta_{\langle \vec{i}, \vec{j}  \rangle, \langle \vec{i}', \vec{j}'  \rangle } \right) 
\hat{\mu}^{z}_{\langle \vec{i}, \vec{j} \rangle}  \hat{\mu}^x_{\langle \vec{i}', \vec{j}'  \rangle} $    such that applying $\hat{\mu}^x_{\langle \vec{i}', \vec{j}'  \rangle}$  on an eigenstate of  $\hat{\mu}^{z}_{\langle \vec{i}, \vec{j} \rangle}$  flips the 
field. 

To formulate the Monte Carlo, we work in a basis in which  $\hat{\mu}^{z}_{ \langle  \ve{i}, \ve{j} \rangle }$,  $ \hat{\tau}^{z}_{   \ve{0} } $  and  $ \hat{\sigma}^{z}_{ \langle  \ve{i}, \ve{j} \rangle }$    are diagonal: 
\begin{align}
	  \hat{\mu}^{z}_{ \langle  \ve{i}, \ve{j} \rangle } | \sigma, \mu, \tau_0 \rangle  &=  \mu_{ \langle  \ve{i}, \ve{j} \rangle } | \sigma, \mu, \tau_0 \rangle     \\
	 \hat{\sigma}^{z}_{ \langle  \ve{i}, \ve{j} \rangle } | \sigma, \mu, \tau_0 \rangle & =  \sigma_{ \langle  \ve{i}, \ve{j} \rangle } | \sigma, \mu, \tau_0 \rangle     \\
	  \hat{\tau}^{z}_{   \ve{0} } | \sigma, \mu, \tau_0 \rangle  &=  \tau_{  \ve{0} } | \sigma, \mu, \tau_0 \rangle   
\end{align}











 
\begin{itemize}
\item  Include an attractive $U$-term. This breaks the O(4) symmetry down to SU(2)$\times$SU(2)   and selects the $\hat{Q}_{\vec{i}} =1 $ sector.   Note that a repulsive U should  can also be included and should produce equivalent results under particle-hole symmetry.
\item  Include dynamics, so as to study the dynamics of the OSM to  FL* phase. Note that the FL* phase will ultimately be unstable to the AFM* phase, but the energy scale is expected to be extremely small at small U.  
\item  Include a projective version, with different left and right wave functions. The right imposes translation invariance and  the left  the constraint.  That is,  we choose the right trial wave function to be the ground state of 
\begin{equation}
	\hat{H}_T^{R}  = -  t  \sum_{\langle \vec{i}, \vec{j} \rangle, \sigma } 
	  \left( \hat{\Psi}^{\dagger}_{\vec{i},\sigma} \hat{\Psi}^{\phantom{\dagger}}_{\vec{j},\sigma}    + h.c. \right)  
	  - h \sum_{\vec{i} } \left( \hat{X}_{\vec{i},\vec{i} +  \vec{a}_x }+  \hat{X}_{\vec{i},\vec{i} +  \vec{a}_y}  + \tau^{x}_{\vec{i}} \right)
\end{equation}
and the left one to be the ground state of
\begin{equation}
	\hat{H}_T^{L}  =  - U  \sum_{ \vec{i},  \sigma } 
	    e^{i \vec{Q} \cdot \vec{i} } \hat{\Psi}^{\dagger}_{\vec{i},\sigma} \hat{\Psi}^{\phantom{\dagger}}_{\vec{i},\sigma}  
	  - h \sum_{\vec{i} } \left( \hat{X}_{\vec{i},\vec{i} +  \vec{a}_x }+  \hat{X}_{\vec{i},\vec{i} +  \vec{a}_y}  + \tau^{x}_{\vec{i}} \right)
\end{equation}
with $\vec{Q} = ( \pi,\pi ) $. 
\end{itemize}  
