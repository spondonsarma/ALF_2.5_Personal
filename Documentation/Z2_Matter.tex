% Copyright (c) 2016 2020 The ALF project.
% This is a part of the ALF project documentation.
% The ALF project documentation by the ALF contributors is licensed
% under a Creative Commons Attribution-ShareAlike 4.0 International License.
% For the licensing details of the documentation see license.CCBYSA.
% !TEX root = doc.tex 


\subsection{$\Ztwo$ lattice gauge theories coupled to fermion and $\Ztwo$ matter  \texttt{Hamiltonian\_Z2\_smod.F90}} \label{Z2.Sec}

The Hamiltonian we will consider here reads
\begin{align}
	\hat{H} =& -  t_{\Ztwo} \sum_{\langle \vec{i}, \vec{j} \rangle, \sigma } \hat{\sigma}^z_{\langle \vec{i}, \vec{j} \rangle}
	\left(\hat{\Psi}^{\dagger}_{\vec{i},\sigma} \hat{\Psi}^{\phantom{\dagger}}_{\vec{j},\sigma}   + \hc \right) - \mu \sum_{\vec{i},\sigma} \hat{\Psi}^{\dagger}_{\vec{i},\sigma} \hat{\Psi}^{\phantom{\dagger}}_{\vec{i},\sigma}  
	-g \sum_{\langle \vec{i}, \vec{j} \rangle } \hat{\sigma}^{x}_{\langle \vec{i}, \vec{j} \rangle } \nonumber \\
	  & +K \sum_{\square} \prod_{\langle \vec{i}, \vec{j} \rangle \in \partial \square} \hat{\sigma}^{z}_{\langle \vec{i}, \vec{j} \rangle}  
	 + J  \sum_{\langle \vec{i}, \vec{j} \rangle}  \hat{\tau}^z_{\pmb{i}}  \hat{\sigma}^{z}_{\langle \vec{i}, \vec{j} \rangle} \hat{\tau}^z_{\pmb{j}}   
	      -  h \sum_{ \vec{i} } \hat{\tau}^x_{\vec{i}} \nonumber \\
	& - t  \sum_{\langle \vec{i}, \vec{j} \rangle, \sigma }   \hat{\tau}^z_{\pmb{i}}   \hat{\tau}^z_{\pmb{j}}  \left( \hat{\Psi}^{\dagger}_{\vec{i},\sigma} \hat{\Psi}^{\phantom{\dagger}}_{\vec{j},\sigma} 	+ \hc \right) + \frac{U}{N}\sum_{\ve{i}} \left[ \sum_{\sigma}  \left( \hat{\Psi}^{\dagger}_{\ve{i},\sigma}  \hat{\Psi}^{\phantom\dagger}_{\ve{i},\sigma} - 1/2\right) \right]^2.
\end{align}  
The model is defined on a square lattice, and describes  fermions, 
\begin{align}
 \left\{ \hat{\Psi}^{\dagger}_{\vec{i},\sigma},  \hat{\Psi}^{\phantom\dagger}_{\vec{j},\sigma'} \right\}  = \delta_{\vec{i},\vec{j}} \delta_{\sigma,\sigma'}, \;  
\left\{ \hat{\Psi}^{\phantom\dagger}_{\vec{i},\sigma},  \hat{\Psi}^{\phantom\dagger}_{\vec{j},\sigma'} \right\}  =  0,  
\end{align}
coupled to  bond gauge fields, 
\begin{align}
\hat{\sigma}^{z}_{\langle \vec{i}, \vec{j} \rangle}  = 
\begin{bmatrix}
1 & 0 \\
0 & -1 
\end{bmatrix},  
\hat{\sigma}^x_{\langle \vec{i}, \vec{j} \rangle}  = 
\begin{bmatrix}
0 & 1 \\
1 & 0 
\end{bmatrix},   
\left\{ \hat{\sigma}^{z}_{\langle \vec{i}, \vec{j} \rangle} , \hat{\sigma}^x_{\langle \vec{i}', \vec{j}'  \rangle} \right\}  =  2 \left( 1 -  \delta_{\langle \vec{i}, \vec{j}  \rangle, \langle \vec{i}', \vec{j}'  \rangle } \right) 
\hat{\sigma}^{z}_{\langle \vec{i}, \vec{j} \rangle}  \hat{\sigma}^x_{\langle \vec{i}', \vec{j}'  \rangle}
\end{align}
and  $\Ztwo$ matter fields:
\begin{align}
 \hat{\tau}_{\ve{i}}^{z} = 
\begin{bmatrix}
1 & 0 \\
0 & -1 
\end{bmatrix},\quad
\hat{\tau}^{x}_{\vec{i} }  = 
\begin{bmatrix}
0 & 1 \\
1 & 0 
\end{bmatrix},\quad 
\left\{ \hat{\tau}^{z}_{ \vec{i} }, \hat{\tau}^x_{\vec{i}'} \right\}  =  2 \left( 1 -  \delta_{ \vec{i}, \vec{i}' } \right) 
\hat{\tau}^{z}_{ \vec{i} }  \hat{\tau}^x_{ \vec{i}' }.
\end{align}
 Fermions,  gauge fields and $\Ztwo$ matter fields commute with each other. 
 
 Importantly,  the model has a local  $\Ztwo$ symmetry. Consider:
\begin{equation}
	\hat{Q}_{\vec{i}} =  (-1)^{\sum_{\sigma} \hat{\Psi}^{\dagger}_{\vec{i},\sigma} \hat{\Psi}^{\phantom{\dagger}}_{\vec{i},\sigma}   } 
	\;  \hat{\tau}^{x}_{\vec{i}}  \; \hat{\sigma}^{x}_{\vec{i},\vec{i} +  \vec{a}_x} \hat{\sigma}^{x}_{\vec{i},\vec{i} -  \vec{a}_x} \hat{\sigma}^{x}_{\vec{i},\vec{i} +  \vec{a}_y} \hat{\sigma}^{x}_{\vec{i}}.
\end{equation} 
One can  then show that  $\hat{Q}_{\vec{i}}^2 = 1 $ and that
\begin{equation}
	\left[   \hat{Q}_{\vec{i}}, \hat{H}  \right]  = 0. 
\end{equation} 
The above allows us to assign $\Ztwo$  charges to the operators.   Since $ \left\{ \hat{Q}_{\vec{i}},   \hat{\Psi}^{\dagger}_{\vec{i},\sigma} \right\} =    0 $ we can assign a $\Ztwo$  charge to the fermions.  Equivalently 
$\hat{\tau}^{z}_{\ve{i}}$   has a $\Ztwo$ charge and $\hat{\sigma}^{z}_{\vec{i},\vec{j}} $   carries  $\Ztwo$ charges at its ends.   
 Since the total fermion number is conserved,   we can assign an electric charge to the  fermions.
Finally, the model has an SU(N) color symmetry.
In fact, at zero chemical potential and $U=0$, the symmetry is enhanced to $O(2N)$ \cite{Assaad16}.
Aspects of this Hamiltonian were investigated in Refs.~\cite{Assaad16,Gazit16,Gazit18,Gazit19,Hohenadler18,Hohenadler19} and we refer the interested user to these papers for a discussion of the phases and phase transitions supported by the model.

\subsubsection*{QMC implementation} 

The name space for this model class reads: 

\begin{lstlisting}[style=fortran,escapechar=\#,breaklines=true]
&VAR_Z2_Matter             !! Variables for the Z_2 class
ham_T          = 1.0        ! Hopping for fermions
ham_TZ2        = 1.0        ! Hopping for orthogonal fermions
ham_chem       = 0.0        ! Chemical potential for fermions
ham_U          = 0.0        ! Hubbard for fermions
Ham_J          = 1.0        ! Hopping Z2 matter fields
Ham_K          = 1.0        ! Plaquette term for gauge fields
Ham_h          = 1.0        ! sigma^x-term for matter
Ham_g          = 1.0        ! tau^x-term for gauge
Dtau           = 0.1d0      ! Thereby Ltrot=Beta/dtau
Beta           = 10.d0      ! Inverse temperature
Projector      = .False.    ! To enable projective code
Theta          = 10.0       ! Projection parameter 
/
\end{lstlisting}


We note that the implementation is such that  if \texttt{Ham\_T=0}   (\texttt{Ham\_TZ2=0}) then all the terms involving the matter field ($\Ztwo$  gauge field) are automatically set to zero.  
We warn the user that autocorrelation and warmup times can be  large for this model class.
At this point,  the model is only implemented for the square lattice  and does not support a symmetric Trotter decomposition.

The key point to implement the model is to  define a new bond variable: 
\begin{equation}
	\hat{\mu}^{z}_{ \langle  \ve{i}, \ve{j}  \rangle }  =  \hat{\tau}^{z}_{ \ve{i}}\hat{\tau}^{z}_{\ve{j}  }. 
\end{equation}  
By construction, the $\hat{\mu}^{z}_{ \langle  \ve{i}, \ve{j}  \rangle } $ bond variables   have a zero flux constraint:
\begin{equation}
	\hat{\mu}^{z}_{ \langle  \ve{i}, \ve{i} + \ve{a}_x  \rangle }  \hat{ \mu}^{z}_{ \langle  \ve{i} + \ve{a}_x, \ve{i} + \ve{a}_x  + \ve{a}_y \rangle } 
	\hat{\mu}^{z}_{ \langle  \ve{i} + \ve{a}_x + \ve{a}_y, \ve{i} +  \ve{a}_y \rangle } \hat{\mu}^{z}_{ \langle  \ve{i} + \ve{a}_y, \ve{i}  \rangle }  = 1. 
\label{zero_flux.eq}
\end{equation} 


Consider a basis where  $\hat{\mu}^{z}_{ \langle  \ve{i}, \ve{j}  \rangle } $ and $  \hat{\tau}^{z}_{ \ve{i}}$  are diagonal with eigenvalues  $\mu_{ \langle  \ve{i}, \ve{j}  \rangle } $ and $  {\tau}_{ \ve{i}}$  respectively. 
The map from  $ \left\{ \tau_{\vec{i}}  \right\} $ to $ \left\{ \mu_{\langle \vec{i}, \vec{j} \rangle } \right\} $  is unique.
The reverse however  is valid only up to a global sign.
To pin down this sign (and thereby  the  relative signs between different time slices)  we store the fields $ \mu_{\langle \vec{i},\vec{j} \rangle } $ at every time slice as well as the value of the Ising field at a reference site $\tau_{\vec{i} = \ve{0}}$. Within the ALF, this can be done by adding a dummy operator in the \texttt{Op\_V} list to carry this degree of freedom.    With this extra degree of freedom we can switch  between the two representations without loosing any information.   To compute the Ising part of the action it is certainly more transparent to work  with the $ \left\{ \tau_{\vec{i}}  \right\} $  variables. For the  fermion determinant,  the $ \left\{ \mu_{\langle \vec{i}, \vec{j} \rangle } \right\} $   are more convenient.

Since flipping  $\hat{\tau}^{z}_{ \ve{i}} $  amounts to changing the sign of the four  bond variables emanating from site $\ve{i}$, the identity:
\begin{equation}
\hat{\tau}^x_{\ve{i}}  = \hat{\mu}^{x}_{\ve{i},\ve{i} + \ve{a}_x } \hat{\mu}^{x}_{\ve{i} + \ve{a}_x, \ve{i} + \ve{a}_x + \ve{a}_y }   \hat{\mu}^{x}_{\ve{i} + \ve{a}_x + \ve{a}_y, \ve{i} + \ve{a}_y  }
\end{equation}
holds.  
Note that $\left\{ \hat{\mu}^{z}_{\langle \vec{i}, \vec{j} \rangle} , \hat{\mu}^x_{\langle \vec{i}', \vec{j}'  \rangle} \right\}  =  2 \left( 1 -  \delta_{\langle \vec{i}, \vec{j}  \rangle, \langle \vec{i}', \vec{j}'  \rangle } \right) 
\hat{\mu}^{z}_{\langle \vec{i}, \vec{j} \rangle}  \hat{\mu}^x_{\langle \vec{i}', \vec{j}'  \rangle} $, such that applying $\hat{\mu}^x_{\langle \vec{i}, \vec{j}  \rangle}$  on an eigenstate of  $\hat{\mu}^{z}_{\langle \vec{i}, \vec{j} \rangle}$  flips the field. 


The model  can then be written as:
\begin{align}
	\hat{H} =& - t_{\Ztwo}\! \sum_{\langle \vec{i}, \vec{j} \rangle, \sigma } \hat{\sigma}^z_{\langle \vec{i}, \vec{j} \rangle}
	\left(\hat{\Psi}^{\dagger}_{\vec{i},\sigma} \hat{\Psi}^{\phantom{\dagger}}_{\vec{j},\sigma} \!+ \hc \right) - \mu \sum_{\vec{i},\sigma} \hat{\Psi}^{\dagger}_{\vec{i},\sigma} \hat{\Psi}^{\phantom{\dagger}}_{\vec{i},\sigma}  
	-g \sum_{\langle \vec{i}, \vec{j} \rangle } \hat{\sigma}^{x}_{\langle \vec{i}, \vec{j} \rangle }  +
	  K \sum_{\square} \prod_{\langle \vec{i}, \vec{j} \rangle \in \partial \square} \hat{\sigma}^{z}_{\langle \vec{i}, \vec{j} \rangle}  \nonumber \\
	& + J  \sum_{\langle \vec{i}, \vec{j} \rangle}  \hat{\mu}^z_{ \langle \pmb{i}, \ve{j} \rangle }  \hat{\sigma}^{z}_{\langle \vec{i}, \vec{j} \rangle}    
	      -  h \sum_{ \vec{i} } \hat{\mu}^{x}_{\ve{i},\ve{i} + \ve{a}_x } \hat{\mu}^{x}_{\ve{i} + \ve{a}_x, \ve{i} + \ve{a}_x + \ve{a}_y }   \hat{\mu}^{x}_{\ve{i} + \ve{a}_x + \ve{a}_y, \ve{i} + \ve{a}_y  }
	         \hat{\mu}^{x}_{\ve{i} + \ve{a}_y, \ve{i}  }	\nonumber  \\      
	&        - t  \sum_{\langle \vec{i}, \vec{j} \rangle, \sigma }   \hat{\mu}^z_{\pmb{i},\ve{j}}    \left( \hat{\Psi}^{\dagger}_{\vec{i},\sigma} \hat{\Psi}^{\phantom{\dagger}}_{\vec{j},\sigma} 	+ \hc \right) + \frac{U}{N}\sum_{\ve{i}} \left[ \sum_{\sigma}  ( \hat{\Psi}^{\dagger}_{\ve{i},\sigma}  \hat{\Psi}^{\phantom\dagger}_{\ve{i},\sigma} - 1/2 ) \right]^2 
\end{align}  
subject to the constraint of Eq.~\eqref{zero_flux.eq}.  

To formulate the Monte Carlo, we work in a basis in which  $\hat{\mu}^{z}_{ \langle  \ve{i}, \ve{j} \rangle }$,  $ \hat{\tau}^{z}_{   \ve{0} } $  and  $ \hat{\sigma}^{z}_{ \langle  \ve{i}, \ve{j} \rangle }$    are diagonal: 
\begin{align}
	  \hat{\mu}^{z}_{ \langle  \ve{i}, \ve{j} \rangle } |  \underline{s} \rangle  =  \mu_{ \langle  \ve{i}, \ve{j} \rangle }   |  \underline{s} \rangle,\quad 
	 \hat{\sigma}^{z}_{ \langle  \ve{i}, \ve{j} \rangle }|  \underline{s} \rangle  =  \sigma_{ \langle  \ve{i}, \ve{j} \rangle } |  \underline{s} \rangle,\quad    
	  \hat{\tau}^{z}_{   \ve{0} }  |  \underline{s} \rangle   =  \tau_{  \ve{0} } |  \underline{s} \rangle
\end{align}
with $ \underline{s} = \left( \left\{  \mu_{ \langle  \ve{i}, \ve{j} \rangle }  \right\},  \left\{  \sigma_{ \langle  \ve{i}, \ve{j} \rangle }  \right\},  \tau_{\ve{0}}  \right) $. 
In this basis,
\begin{equation}
   Z  =  \sum_{\underline {s}_1, \cdots, \underline {s}_{L_{\tau}}}  e^{-S_0( \left\{ \underline{s}_\tau \right\})} \text{Tr}_F   \left[    \prod_{\tau=1}^{L_{\tau}} e^{- \Delta \tau \hat{H}_F(\underline{s}_{\tau}) } \right],
 \end{equation}
 where 
 \begin{align*}
 	 S_0( \left\{ \underline{s}_\tau \right\})  = - \ln  \left[  \prod_{\tau=1}^{L_{\tau}}    \langle \underline{s}_{\tau+1}   |  e^{-\Delta \tau   \hat{H}_I} |  \underline{s}_{\tau}  \rangle  \right], 
\end{align*}
\begin{align*}
         \hat{H}_I  = &  -g \sum_{\langle \vec{i}, \vec{j} \rangle } \hat{\sigma}^{x}_{\langle \vec{i}, \vec{j} \rangle }  +
	                        K \sum_{\square} \prod_{\langle \vec{i}, \vec{j} \rangle \in \partial \square} \hat{\sigma}^{z}_{\langle \vec{i}, \vec{j} \rangle}  
	     + J  \sum_{\langle \vec{i}, \vec{j} \rangle}  \hat{\mu}^z_{ \langle \pmb{i}, \ve{j} \rangle }  \hat{\sigma}^{z}_{\langle \vec{i}, \vec{j} \rangle}     \\
	     & -  h \sum_{ \vec{i} } \hat{\mu}^{x}_{\ve{i},\ve{i} + \ve{a}_x } \hat{\mu}^{x}_{\ve{i} + \ve{a}_x, \ve{i} + \ve{a}_x + \ve{a}_y }   \hat{\mu}^{x}_{\ve{i} + \ve{a}_x + \ve{a}_y, \ve{i} + \ve{a}_y}  	
\end{align*}
and 
 \begin{align*}
   \hat{H}_F(\underline{s})  = 
   	     & - t_{\Ztwo} \sum_{\langle \vec{i}, \vec{j} \rangle, \sigma } \sigma_{\langle \vec{i}, \vec{j} \rangle}
	\left(\hat{\Psi}^{\dagger}_{\vec{i},\sigma} \hat{\Psi}^{\phantom{\dagger}}_{\vec{j},\sigma}   + \hc \right) - \mu \sum_{\vec{i},\sigma} \hat{\Psi}^{\dagger}_{\vec{i},\sigma} \hat{\Psi}^{\phantom{\dagger}}_{\vec{i},\sigma}  \\
         & - t  \sum_{\langle \vec{i}, \vec{j} \rangle, \sigma }   \mu_{\pmb{i},\ve{j}}    \left( \hat{\Psi}^{\dagger}_{\vec{i},\sigma} \hat{\Psi}^{\phantom{\dagger}}_{\vec{j},\sigma} 	+ \hc \right)  
         +\frac{U}{N}\sum_{\ve{i}} \left[ \sum_{\sigma}  ( \hat{\Psi}^{\dagger}_{\ve{i},\sigma}  \hat{\Psi}^{\phantom\dagger}_{\ve{i},\sigma} - 1/2 ) \right]^2. 
 \end{align*}
In the above, $  | \underline{s}_{L_\tau+1}  \rangle  = | \underline{s}_{1}  \rangle   $.  With a further  HS transformation of the Hubbard term (see Sec.~\ref{Hubbard_SUN_HS.eq})  the model is readily implemented in the ALF.    Including this HS field, $l$,  [see Eq.~\eqref{HS_squares}] yields  the configuration space: 
\begin{equation}
	C = \left(  \big\{  \mu_{ \langle  \ve{i}, \ve{j} \rangle,\tau }  \big\},  \big\{  \sigma_{ \langle  \ve{i}, \ve{j} \rangle, \tau }  \big\},    \big\{ \tau_{\ve{0},\tau}  \big\},  \big\{ l_{\ve{i}, \tau}  \big\}  \right) 
\end{equation} 
where the variables $\mu$, $\tau$ and $\sigma$ take the values $\pm 1$  and $l$  the values $\pm1, \pm 2$. 

The initial configuration as well as the  moves have to respect the zero flux constraint of Eq.~\eqref{zero_flux.eq}. Therefore, single spin flips of the $\mu$ fields  are prohibited and the minimal move one can carry out on  a given time slice is the following. We randomly choose a site $\vec{i} $ and  propose a move where:
$ \mu_{\vec{i},\vec{i} +  \vec{a}_x} \rightarrow - \mu_{\vec{i},\vec{i} +  \vec{a}_x} $,  $ \mu_{\vec{i},\vec{i} -  \vec{a}_x} \rightarrow - \mu_{\vec{i},\vec{i} -  \vec{a}_x} $,
$ \mu_{\vec{i},\vec{i} +  \vec{a}_y} \rightarrow - \mu_{\vec{i},\vec{i} +  \vec{a}_y} $ and $ \mu_{\vec{i},\vec{i} -  \vec{a}_y} \rightarrow - \mu_{\vec{i},\vec{i} -  \vec{a}_y} $.  One can carry out such moves by using the global move in real space option presented in Sec.~\ref{sec:global_space} and \ref{sec:input}.

\subsubsection{Projective approach} 
The program also supports a zero temperature implementation.
Our  choice  of the trial wave  function does not break any symmetries of the model and reads: 
\begin{equation}
	| \Psi_T \rangle  =    | \Psi^{F}_T \rangle \,  \otimes_{\langle \ve{i},\ve{j} \rangle}  | + \rangle_{\langle \ve{i},\ve{j} \rangle }     \,  \otimes_{ \ve{i} }  | + \rangle_{ \ve{i} }.   
\end{equation}
For the fermion part we use a Fermi sea with small dimerization to avoid the negative sign problem at half-filling (see Sec.~\ref{Sec:Plain_vanilla_trial}).  For the Ising part the trial   wave function  is diagonal in the $ \hat{\sigma}^{x}_{\langle \ve{i}, \ve{j} \rangle} $ and $\hat{\tau}^{x}_{\ve{i}} $   operators: 
\begin{equation}
	\hat{\sigma}^{x}_{\langle \ve{i}, \ve{j} \rangle}  | + \rangle_{\langle \ve{i},\ve{j} \rangle}  = | + \rangle_{\langle \ve{i},\ve{j} \rangle }  \quad\text{and}\quad  \hat{\tau}^{x}_{\ve{i}} | + \rangle_{ \ve{i} } = | + \rangle_{ \ve{i} }.
\end{equation}

An alternative  choice would  be to choose a  charge density wave  fermionic trial wave function.     This   violates the partial particle-hole symmetry of the model at $U=\mu=0$ and effectively imposes the constraint $\hat{Q}_{\ve{i}} =1 $.

\subsubsection{Observables} 
Apart from the  standard  observables discussed in Sec.~\ref{sec:predefined_observales}  the code computes additionally  
\begin{equation*} 
	\big\langle  \hat{\sigma}^{x}_{ \langle \ve{i},\ve{j} \rangle } \big\rangle    \quad\text{and}\quad  \big\langle  \hat{\tau}^{x}_{ \ve{j}  } \big\rangle,
\end{equation*}
which are written to file  \texttt{X\_scal}; 
\begin{equation*}
	\big\langle  \hat{\sigma}^{z}_{ \langle \ve{i},\ve{i} + \ve{a}_x \rangle }   \hat{\sigma}^{z}_{ \langle \ve{i} + \ve{a}_x ,\ve{i} + \ve{a}_x +  \ve{a}_y   \rangle }  
	\hat{\sigma}^{z}_{ \langle \ve{i}+ \ve{a}_x +  \ve{a}_y ,\ve{i} + \ve{a}_y \rangle }   \hat{\sigma}^{z}_{ \langle \ve{i} + \ve{a}_y,\ve{i}  \rangle }   \big\rangle
\end{equation*}
	and
\begin{equation*}
	\big\langle  \hat{\mu}^{z}_{ \langle \ve{i},\ve{i} + \ve{a}_x \rangle }   \hat{\mu}^{z}_{ \langle \ve{i} + \ve{a}_x ,\ve{i} + \ve{a}_x +  \ve{a}_y   \rangle }  
	\hat{\mu}^{z}_{ \langle \ve{i}+ \ve{a}_x +  \ve{a}_y ,\ve{i} + \ve{a}_y \rangle }   \hat{\mu}^{z}_{ \langle  \ve{i} + \ve{a}_y,\ve{i}  \rangle }   \big\rangle,
\end{equation*}
written to file \texttt{Flux\_scal}; and also $ \langle \hat{Q}_{\ve{i}} \rangle $ (file \texttt{Q\_scal}).  Note that the flux over a plaquette  of the $\hat{\mu}^{z}_{ \langle \ve{i},\ve{j} \rangle  } $   is equal to unity by construction  so that this observable  provides a sanity check.    The file \texttt{Q\_eq} contains the two-point correlation $ \langle \hat{Q}_{\ve{i}} \hat{Q}_{\ve{j}} \rangle -  \langle \hat{Q}_{\ve{i}} \rangle \langle \hat{Q}_{\ve{j}} \rangle $  and \texttt{Greenf\_eq}  the equal-time fermion   Green function 
$ \langle  \hat{\tau}^{z}_{\ve{i}}  \hat{\Psi}^{\dagger}_{\ve{i},\sigma}  \hat{\tau}^{z}_{\ve{j}}  \hat{\Psi}^{\phantom\dagger}_{\ve{j},\sigma}   \rangle $.


\subsubsection{A test case: $\Ztwo$ slave spin formulation of the SU(2) Hubbard model }

In this subsection, we  demonstrate that the code can be used to  simulate the attractive Hubbard model in the  $\Ztwo$-slave spin formulation \cite{Ruegg10}:
\begin{equation}
        \hat{H} = -t \sum_{\langle \vec{i}, \vec{j} \rangle, \sigma }\hat{c}^{\dagger}_{\vec{i},\sigma} \hat{c}^{\phantom{\dagger}}_{\vec{j},\sigma}   -  U  \sum_i
        \left( \hat{n}_{\vec{i}, \uparrow} - 1/2\right)  \left( \hat{n}_{\vec{i}, \downarrow} - 1/2\right).
\end{equation}
In the $\Ztwo$ slave spin  representation, the physical fermion, $\hat{c}_{\vec{i},\sigma} $,   is fractionalized into  an Ising spin carrying $\Ztwo$ charge and a fermion, $\hat{\Psi}_{\vec{i},\sigma} $, carrying $\Ztwo$ and  global $U(1)$ charge:
\begin{equation}
        \hat{c}^{\dagger}_{\vec{i},\sigma}  = \hat{\tau}^{z}_{\vec{i}} \hat{\Psi}^{\dagger}_{\vec{i},\sigma}.
\end{equation}
To ensure that we remain in the correct Hilbert space, the constraint:
\begin{equation}
        \hat{\tau}^{x}_{i}   - (-1)^{\sum_{\sigma}\hat{\Psi}^{\dagger}_{i,\sigma}  \hat{\Psi}^{\phantom{\dagger}}_{i,\sigma}  }  = 0
\end{equation}
has to be imposed locally. Since $\left(  \tau^{x}_{\vec{i}}\right)^2 = 1 $, the latter is   equivalent to
 \begin{equation}
        \hat{Q}_{\vec{i}} = \tau^{x}_{\vec{i}}  (-1)^{\sum_{\sigma}\hat{\Psi}^{\dagger}_{\vec{i},\sigma}  \hat{\Psi}^{\phantom{\dagger}}_{\vec{i},\sigma}  }   = 1.
 \end{equation}
 Using 
 \begin{equation}
 	(-1)^{\sum_{\sigma}\hat{\Psi}^{\dagger}_{\vec{i},\sigma}  \hat{\Psi}^{\phantom{\dagger}}_{\vec{i},\sigma}  }    = \prod_{\sigma} ( 1 - 2 \hat{\Psi}^{\dagger}_{\vec{i},\sigma}  \hat{\Psi}^{\phantom{\dagger}}_{\vec{i},\sigma} ) =  4\prod_{\sigma} ( \hat{c}^{\dagger}_{\vec{i},\sigma}  \hat{c}^{\phantom{\dagger}}_{\vec{i},\sigma}  - 1/2 ),
 \end{equation}
 the  $\Ztwo$ slave spin representation of  the Hubbard model now reads:
 \begin{equation}
         \hat{H}_{\Ztwo}= -t \sum_{\langle \vec{i}, \vec{j} \rangle, \sigma }  \hat{\tau}^{z}_{\vec{i}}  \hat{\tau}^{z}_{\vec{j}} \hat{\Psi}^{\dagger}_{\vec{i},\sigma} \hat{\Psi}^{\phantom{\dagger}}_{\vec{j},\sigma}   -  \frac{U}{4}  \sum_{\vec{i}}  \hat{\tau}^{x}_{\vec{i}}.
 \end{equation}
 Importantly, the constraint  commutes with Hamiltonian:
 \begin{equation}
        \left[ \hat{H}_{\Ztwo}, \hat{Q}_{\vec{i}} \right] = 0.
 \end{equation}
Hence  one can foresee that the constraint will be dynamically imposed (we expect a finite-temperature Ising phase  transition below which $\hat{Q}_{\ve{i}}$ orders) and that at  $T=0$ on a finite lattice both models should give the same results.


A test run for the $8\times 8 $ lattice at $U/t = 4$ and $\beta t = 40$ gives:
\begin{center}
\begin{tabular}{c c c}
 \toprule
   k                      & $\langle n_k \rangle_{H} $     &  $\langle n_k \rangle_{H_{\Ztwo}}$\\
  \midrule
   $(0,0)$                & $1.93348548  \pm  0.00011322$  &  $ 1.93333895  \pm  0.00010405 $ \\
   $(\pi/4, \pi/4)$       & $1.90120688  \pm  0.00014854$  &  $ 1.90203726  \pm  0.00017943 $ \\
   $(\pi/2, \pi/2)$       & $0.99942957  \pm  0.00091377$  &  $ 1.00000000  \pm  0.00000000 $ \\
   $(3\pi/4, 3\pi/4)$     & $0.09905425  \pm  0.00015940$  &  $ 0.09796274  \pm  0.00017943 $ \\
   $(\pi,\pi)$            & $0.06651452  \pm  0.00011321$  &  $ 0.06666105  \pm  0.00010405 $ \\
  \bottomrule
\end{tabular}

\end{center}
\vspace*{0.5cm}
Here a Trotter time step of  $\Delta \tau t = 0.05$ was used in order to minimize the systematic error   which should be different  between the two codes.   The Hamiltonian is invariant under a partial particle-hole transformation (see Ref.~\cite{Assaad16}).
Since $\hat{Q}_{\ve{i}} $ is odd under this  transformation, $\langle \hat{Q}_{\ve{i}}  \rangle =0$.
 To asses whether the constraint is well imposed,  the code,  for this special case, computes the correlation function:
\begin{equation}
	   S_Q(\ve{q})  = \sum_{\ve{i}}    \langle \hat{Q}_{\ve{i}} \hat{Q}_{\ve{0}}   \rangle.
\end{equation}
For the above run we obtain  $ S_Q(\ve{q} = \ve{0}) = 63.4 \pm 1.7 $  which,   for this $ 8\times 8$ lattice, complies with a ferromagnetic ordering of the 
Ising  $\hat{Q}_{\ve{i}}$  variables.    The pyALF python script that produces this data can be found in  \href{https://git.physik.uni-wuerzburg.de/ALF/pyALF/-/blob/master/Scripts/Z2_Matter.py}{\texttt{Z2\_Matter.py}}.
This code  was used in Refs.~\cite{Hohenadler18,Hohenadler19}.  


 
%\begin{itemize}
%\item  Include an attractive $U$-term. This breaks the O(4) symmetry down to SU(2)$\times$SU(2)   and selects the $\hat{Q}_{\vec{i}} =1 $ sector.   Note that a repulsive U should  can also be included and should produce equivalent results under particle-hole symmetry.
%\item  Include dynamics, so as to study the dynamics of the OSM to  FL* phase. Note that the FL* phase will ultimately be unstable to the AFM* phase, but the energy scale is expected to be extremely small at small U.  
%\item  Include a projective version, with different left and right wave functions. The right imposes translation invariance and  the left  the constraint.  That is,  we choose the right trial wave function to be the ground state of 
%\begin{equation}
%	\hat{H}_T^{R}  = -  t  \sum_{\langle \vec{i}, \vec{j} \rangle, \sigma } 
%	  \left( \hat{\Psi}^{\dagger}_{\vec{i},\sigma} \hat{\Psi}^{\phantom{\dagger}}_{\vec{j},\sigma}    + \hc \right)  
%	  - h \sum_{\vec{i} } \left( \hat{X}_{\vec{i},\vec{i} +  \vec{a}_x }+  \hat{X}_{\vec{i},\vec{i} +  \vec{a}_y}  + \tau^{x}_{\vec{i}} \right)
%\end{equation}
%and the left one to be the ground state of
%\begin{equation}
%	\hat{H}_T^{L}  =  - U  \sum_{ \vec{i},  \sigma } 
%	    e^{i \vec{Q} \cdot \vec{i} } \hat{\Psi}^{\dagger}_{\vec{i},\sigma} \hat{\Psi}^{\phantom{\dagger}}_{\vec{i},\sigma}  
%	  - h \sum_{\vec{i} } \left( \hat{X}_{\vec{i},\vec{i} +  \vec{a}_x }+  \hat{X}_{\vec{i},\vec{i} +  \vec{a}_y}  + \tau^{x}_{\vec{i}} \right)
%\end{equation}
%with $\vec{Q} = ( \pi,\pi ) $. 
%\end{itemize}  
