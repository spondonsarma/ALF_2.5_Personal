% Copyright (c) 2016-2019 The ALF project.
% This is a part of the ALF project documentation.
% The ALF project documentation by the ALF contributors is licensed
% under a Creative Commons Attribution-ShareAlike 4.0 International License.
% For the licensing details of the documentation see license.CCBYSA.
% !TEX root = doc.tex

\subsubsection{Long range Coulomb}

The model we consider here reads: 
\begin{equation}
	\hat{H}  = -t \sum_{\vec{i},\vec{j},\sigma=1}^{N}   \hat{c}^{\dagger}_{\vec{i},\sigma}  T_{\vec{i},\vec{j}} \hat{c}^{}_{\vec{j},\sigma}   +     
	\frac{1} { N } \sum_{\vec{i},\vec{j}}  \left(  \hat{n}_{\vec{i}} -  \frac{N}{2}  \right)  V_{\vec{i},\vec{j}} \left(  \hat{n}_{\vec{j}} -  \frac{N}{2}  \right) \; \; 
	\text{ with }  \hat{n}_{\vec{i}} = \sum_{\sigma=1}^{N}  \hat{c}^{\dagger}_{\vec{i},\sigma}  \hat{c}^{}_{\vec{i},\sigma}
\end{equation}
The interaction is specified in the following way: 
\begin{equation}
	V_{\vec{i}, \vec{j}}   =   U \left\{
	\begin{array}{ll}  
	1          &   \text{ if } \vec{i} - \vec{j}    = 0 \\
	\frac{\alpha   \;   d_{min}}{ |   \vec{i} - \vec{j} | } &     \text{ otherwise }
	\end{array}
\right.
\end{equation}
Here $d_{min}$ is the minimal distance between two orbitals.     The code uses the following  HS decomposition:
\begin{equation}
e^{-\Delta \tau \hat{H}_V }  =  \int \prod_{\vec{i}} d \phi_{\vec{i}}   e^{ - \frac{N \Delta \tau} {4} \phi_{\pmb{i}} V^{-1}_{\pmb{i},\pmb{j}}  \phi_{\pmb{j}} - \sum_{\pmb{i}}  i \Delta \tau \phi_i \left( n_{i} - \frac{N}{2} \right) } 
\end{equation}

The implementation follows Ref.~\cite{Hohenadler14}  but now supports various lattice geometries.    The definition of  the Coulomb repulsion is as follows. 
A general lattice site  \texttt{I,n}   where \texttt{I: 1...Latt\%N} is the unit cell and \texttt{ n = 1 ...Latt\_unit\%NORB}  the orbital  is given by: 
\begin{lstlisting}[style=fortran]
X_p(:) = Latt%list(I,1)*latt%a1_p(:)  + Latt%list(I,2)*latt%a2_p(:) 
          +   Latt_unit%Orb_pos_p(no_j,:)
\end{lstlisting}
or in more compact notation $ \vec{i}  + \vec{\delta}_i $.   By definition \texttt{Latt\_unit\%Orb\_pos\_p(1,:)=0}.
The Coulomb repulsion between points   $ \vec{i}  + \vec{\delta}_i $   and $ \vec{j}  + \vec{\delta}_j $   reads: 
\begin{equation}
	V(\vec{i}  + \vec{\delta}_i ,  \vec{j}  + \vec{\delta}_j  )  =  \frac{U d_{min} \alpha}{  |  \overline{\vec{i} - \vec{j}} + \vec{\delta}_i - \vec{\delta}_j  |}  
\end{equation}
Here  we use periodic boundary conditions such that  $\overline{\vec{i} - \vec{j}}$  is an element of the real space lattice. Note that this is encoded in the array \texttt{Latt\%imj(I,J)}.


