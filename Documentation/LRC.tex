% Copyright (c) 2016-2020 The ALF project.
% This is a part of the ALF project documentation.
% The ALF project documentation by the ALF contributors is licensed
% under a Creative Commons Attribution-ShareAlike 4.0 International License.
% For the licensing details of the documentation see license.CCBYSA.
% !TEX root = Model_classes.tex 

The model we consider  here is  defined  for \texttt{N\_FL=1},  arbitrary values of \texttt{N\_SUN}  and  supports all the predefined lattices.  It  reads: 
\begin{equation}
\hat{H}=  \sum_{i,j}  \sum_{\sigma =1}^{N}  T_{i, j}    \hat{c}^{\dagger}_{i, \sigma }   e^{\frac{2 \pi i}{\Phi_0} \int_{i}^{j}  
     \vec{A}(\ve{l})  d \ve{l}} \hat{c}^{}_{j,\sigma}   +
     \frac{1} { N } \sum_{i,j}  \left(  \hat{n}_{i} -  \frac{N}{2}  \right)  V_{i,j} \left(  \hat{n}_{j} -  \frac{N}{2}  \right)
      - \mu \sum_{i} \hat{n}_{i} 
\end{equation}
In the above,  $i = (\ve{i}, \delta_i) $  and $j = (\ve{j}, \delta_j) $  are super-indices encoding the unit-cell and orbital and 
$  \hat{n}_{i} = \sum_{\sigma =1}^{N} \hat{c}^{\dagger}_{i,\sigma } \hat{c}^{\phantom\dagger}_{i,\sigma} $ For simplicity, the interaction is specified by  two  parameters, $U$ and $\alpha$ that monitor the  strength of the onsite interaction and the magnitude of the Coulomb tail  respectively.  
\begin{equation}
	V_{i, j}     \equiv V(\vec{i}  + \vec{\delta}_i ,  \vec{j}  + \vec{\delta}_j  )  =   U \left\{
	\begin{array}{ll}  
	1          &   \text{ if }  i = j    \\
	\frac{\alpha   \;   d_\mathrm{min}}{  {  || \vec{i} - \vec{j} + \vec{\delta}_i - \vec{\delta}_j  ||}   } &     \text{ otherwise }
	\end{array}
\right.
\end{equation}
Here $d_\mathrm{min}$ is the minimal distance between two orbitals.      On a  torus, some care  has be taken in  defining the distance. Let the lattice size be given by the vectors $\ve{L}_1$  and  $\ve{L}_2$  (see Sec.~\ref{sec:predefined_lattices}).  Then 
\begin{equation}
	|| \ve{i} || = \min_{n_1,n_2 \in \mathbb{Z} }  | \ve{i} - n_1 \ve{L}_1 -  n_2 \ve{L}_2 | 
\end{equation}
The implementation follows Ref.~\cite{Hohenadler14}  but now supports various lattice geometries.  We use  the following  HS decomposition:
\begin{equation}
e^{-\Delta \tau \hat{H}_V }  \propto \int \prod_{i} d \phi_{i}   e^{ - \frac{N \Delta \tau} {4} \sum_{i,j} \phi_{i} V^{-1}_{i,j}  \phi_{j} - \sum_{i}  i \Delta \tau \phi_i \left( \hat{n}_{i} - \frac{N}{2} \right) } 
\end{equation}
where $\Phi_i $ is a  real  variable, $V$ is symmetric, and importantly has to be  positive definite  for the Gaussian integration to be defined. 
The partition function  reads:
\begin{equation}
	Z \propto \int \prod_{i} d \phi_{i, \tau} \overbrace{e^{ - \frac{N \Delta \tau} {4} \sum_{i,j} \phi_{i,\tau} V^{-1}_{i,j}  \phi_{j,\tau}} }^{W_B(\phi)}\underbrace{\text{Tr} \left[   \prod_{\tau}   
	 e^{-\Delta \tau \hat{H}_T}  e^{- \sum_{i}  i \Delta \tau \phi_{i,\tau} \left( \hat{n}_{i} - \frac{N}{2} \right) }\right]}_{W_F(\phi)}.
\end{equation}
such that the weight splits into a bosonic and fermionic part. 
 
%In the above, the field has acquired time index. 
%{\color{red}   There is a constant to be fixed in the above equation}
 
For the update, it is convenient to  work in a basis where  $V$   is diagonal: 
\begin{equation}
	\text{ Diag}  \left( \lambda_1, \cdots ,\lambda_{\texttt{Ndim}} \right)    =  O^{T} V O 
\end{equation}
 with $O^T O = 1$  and define:
 \begin{equation}
 	  \eta_{i}^{\phantom\dagger}   =  \sum_{j} O^{T}_{i,j} \phi_{j}^{\phantom\dagger}.
 \end{equation}
 On a given time slice we propose a  new field configuration with the probability: 
 \begin{equation}
 	T^{0} ( \ve{\eta}  \rightarrow \ve{\eta'} ) =   \prod_{i} \left[  P P_B(\eta'_i)  + (1-P) \delta( \eta_i - \eta'_{i})   \right] 
 \end{equation}
 where 
 \begin{equation}
 	 P_B(\eta_i)   \propto e^{ - \frac{N \Delta \tau} {4 \lambda_i}  \eta_{i}^2 }
 \end{equation}
$ P \in \left[0,1 \right] $ and $\delta $  the Dirac $\delta$-function.      Using the Metropolis-Hasting  algorithm,   we accept a  move according to the probability: 
\begin{equation}
here
\end{equation}


 That is,  we  carry out  simple sampling of the field with probability $P$  and leave the field unchanged with probability 
$(1-P)$.  $P$ is a free parameter that   does not change the final result but that allows to adjust the acceptance.  



 \red{Todo \\ 
 1) Provide a minimal description of the updating scheme  \\
 2) Namespace \\
 3) Testing: \\
 $ 4 \times 4 $ lattice at $ \beta t = 5$ and $U/t = 4$    \texttt{Hamiltonian\_Hubbard\_mod.F90}    gives $ E = -13.188896      \pm  0.001698 $  and
  \texttt{Hamiltonian\_LRC\_mod.F90}  $ E =  -13.198512    \pm  0.040029 $.   Hubbard uses  \texttt{Mz = True} \\
  4) Provide the Python file for this   comparison. } 
  
% The definition of  the Coulomb repulsion is as follows. 
%A general lattice site  \texttt{I,n}   where \texttt{I: 1...Latt\%N} is the unit cell and \texttt{ n = 1 ...Latt\_unit\%NORB}  the orbital  is given by: 
%\begin{lstlisting}[style=fortran]
%X_p(:) = Latt%list(I,1)*latt%a1_p(:)  + Latt%list(I,2)*latt%a2_p(:) 
%          +   Latt_unit%Orb_pos_p(no_j,:)
%\end{lstlisting}
%or in more compact notation $ \vec{i}  + \vec{\delta}_i $.   By definition \texttt{Latt\_unit\%Orb\_pos\_p(1,:)=0}.
%The Coulomb repulsion between points   $ \vec{i}  + \vec{\delta}_i $   and $ \vec{j}  + \vec{\delta}_j $   reads: 
%\begin{equation}
%	V(\vec{i}  + \vec{\delta}_i ,  \vec{j}  + \vec{\delta}_j  )  =  \frac{U d_\mathrm{min} \alpha}{  |  \overline{\vec{i} - \vec{j}} + \vec{\delta}_i - \vec{\delta}_j  |}  
%\end{equation}



