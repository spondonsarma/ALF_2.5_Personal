% Copyright (c) 2016-2020 The ALF project.
% This is a part of the ALF project documentation.
% The ALF project documentation by the ALF contributors is licensed
% under a Creative Commons Attribution-ShareAlike 4.0 International License.
% For the licensing details of the documentation see license.CCBYSA.
% !TEX root = doc.tex 


\subsection{Models with long range Coulomb interactions \texttt{LRC\_mod.F90}}


The model we consider  here is  defined  for \texttt{N\_FL=1},  arbitrary values of \texttt{N\_SUN}  and  all the predefined lattices.  It  reads: 
\begin{equation}
\hat{H}=  \sum_{i,j}  \sum_{\sigma =1}^{N}  T_{i, j}    \hat{c}^{\dagger}_{i, \sigma }   e^{\frac{2 \pi i}{\Phi_0} \int_{i}^{j}  
     \vec{A}(\ve{l})  d \ve{l}} \hat{c}^{}_{j,\sigma}   +
     \frac{1} { N } \sum_{i,j}  \left(  \hat{n}_{i} -  \frac{N}{2}  \right)  V_{i,j} \left(  \hat{n}_{j} -  \frac{N}{2}  \right)
      - \mu \sum_{i} \hat{n}_{i}.
\end{equation}
In the above, $i = (\ve{i}, \ve{\delta}_i) $  and $j = (\ve{j}, \ve{\delta}_j) $  are super-indices encoding the unit-cell and orbital and 
$  \hat{n}_{i} = \sum_{\sigma =1}^{N} \hat{c}^{\dagger}_{i,\sigma } \hat{c}^{\phantom\dagger}_{i,\sigma} $ For simplicity, the interaction is specified by  two  parameters, $U$ and $\alpha$ that monitor the  strength of the onsite interaction and the magnitude of the Coulomb tail  respectively:
\begin{equation}
	V_{i, j}     \equiv V(\vec{i}  + \vec{\delta}_i ,  \vec{j}  + \vec{\delta}_j  )  =   U \left\{
	\begin{array}{ll}  
	1          &   \text{ if }  i = j    \\
	\frac{\alpha   \;   d_\mathrm{min}}{  {  || \vec{i} - \vec{j} + \vec{\delta}_i - \vec{\delta}_j  ||}   } &     \text{ otherwise }.
	\end{array}
\right.
\end{equation}
Here $d_\mathrm{min}$ is the minimal distance between two orbitals.      On a  torus, some care  has be taken in  defining the distance. Namely, with the lattice size given by the vectors $\ve{L}_1$  and  $\ve{L}_2$  (see Sec.~\ref{sec:predefined_lattices}),
\begin{equation}
	|| \ve{i} || = \min_{n_1,n_2 \in \mathbb{Z} }  | \ve{i} - n_1 \ve{L}_1 -  n_2 \ve{L}_2 | .
\end{equation}
The implementation of the model follows Ref.~\cite{Hohenadler14}, but supports various lattice geometries.  We use  the following  HS decomposition:
\begin{equation}
e^{-\Delta \tau \hat{H}_V }  \propto \int \prod_{i} d \phi_{i}   e^{ - \frac{N \Delta \tau} {4} \sum_{i,j} \phi_{i} V^{-1}_{i,j}  \phi_{j} - \sum_{i}  i \Delta \tau \phi_i \left( \hat{n}_{i} - \frac{N}{2} \right) } ,
\end{equation}
where $\phi_i $ is a  real  variable, $V$ is symmetric and, importantly, has to be  positive definite  for the Gaussian integration to be defined. 
The partition function  reads:
\begin{equation}
	Z \propto \int \prod_{i} d \phi_{i, \tau} \overbrace{e^{ - \frac{N \Delta \tau} {4} \sum_{i,j} \phi_{i,\tau} V^{-1}_{i,j}  \phi_{j,\tau}} }^{W_B(\phi)}\underbrace{\text{Tr} \left[   \prod_{\tau}   
	 e^{-\Delta \tau \hat{H}_T}  e^{- \sum_{i}  i \Delta \tau \phi_{i,\tau} \left( \hat{n}_{i} - \frac{N}{2} \right) }\right]}_{W_F(\phi)},
\end{equation}
such that the weight splits into bosonic and fermionic parts. 
 
%In the above, the field has acquired time index. 
%{\color{red}   There is a constant to be fixed in the above equation}
 
For the update, it is convenient to  work in a basis where $V$ is diagonal:
\begin{equation}
	\text{ Diag}  \left( \lambda_1, \cdots ,\lambda_{\texttt{Ndim}} \right)    =  O^{T} V O 
\end{equation}
 with $O^T O = 1$  and define:
 \begin{equation}
 	  \eta_{i,\tau}^{\phantom\dagger}   =  \sum_{j} O^{T}_{i,j} \phi_{j,\tau}^{\phantom\dagger}.
 \end{equation}
 On a given time slice $\tau_u$ we propose a  new field configuration with the probability: 
 \begin{equation}
 	T^{0} ( \eta  \rightarrow  \eta' ) = 
	\left\{ 
	\begin{array}{ll} 
	  \prod_{i} \left[  P P_B(\eta'_{i,\tau_u})  + (1-P) \delta( \eta_{i,\tau_u} - \eta'_{i,\tau_u})   \right]  & \text{  for  } \tau = \tau_u \\
	  \delta( \eta_{i,\tau} - \eta'_{i,\tau})   & \text{  for  } \tau \neq \tau_u 
	  \end{array}
	  \right.
 \end{equation}
 where 
 \begin{equation}
 	 P_B(\eta_{i,\tau})   \propto e^{ - \frac{N \Delta \tau} {4 \lambda_i}  \eta_{i,\tau}^2 },
 \end{equation}
$ P \in \left[0,1 \right] $ and $\delta $ denotes the Dirac $\delta$-function.    That is, we  carry out  simple sampling of the field with probability $P$  and leave the field unchanged with probability 
$(1-P)$.  $P$ is a free parameter that   does not change the final result but that allows one to adjust the acceptance.    We then use  the Metropolis-Hasting    acceptance-rejection  scheme   and accept the move  with probability 
\begin{equation}
   \text{min} \left(     \frac{T^{0} ( \eta'  \rightarrow \eta ) W_B(\eta') W_F(\eta')  }{ T^{0} ( \eta \rightarrow \eta' ) W_B(\eta) W_F(\eta) } , 1 \right)     = \text{min} \left(     \frac{W_F(\eta')  }{ W_F(\eta) } , 1 \right),
\end{equation}
where 
\begin{equation}
  W_B(\eta)  = e^{ - \frac{N \Delta \tau} {4 } \sum_{i,\tau} \eta_{i,\tau}^2/\lambda_i } 
  \;\text{ and }\;
  W_F(\eta)  = \text{Tr} \left[   \prod_{\tau}   
     e^{-\Delta \tau \hat{H}_T}  e^{- \sum_{i,j}  i \Delta \tau O_{i,j}\eta_{j,\tau} \left( \hat{n}_{i} - \frac{N}{2} \right) }\right].
\end{equation}


Since a local change on  a single time slice in the $\eta$ basis corresponds to a non-local space update  in the $\phi$ basis, we  use the routine for global updates in space to carry out the update (see Sec.~\ref{sec:global_space}). 

\subsubsection*{ QMC implementation } 

The name space for this model class  reads: 

\begin{lstlisting}[style=fortran,escapechar=\#,breaklines=true]
&VAR_LRC                   !! Variables for the  Long Range Coulomb class
ham_T          = 1.0       ! Specifies the hopping and chemical potential
ham_T2         = 1.0       ! For bilayer systems
ham_Tperp      = 1.0       ! For bilayer systems
ham_chem       = 1.0       ! Chemical potential
ham_U          = 4.0       ! On-site interaction
ham_alpha      = 0.1       ! Coulomb tail magnitude
Percent_change = 0.1       ! Parameter P 
/
\end{lstlisting}

By setting $\alpha$ to zero we can test this code against the Hubbard code.   For a   $ 4 \times 4 $ square  lattice at $ \beta t = 5$, $U/t = 4$, and half-band filling,     \texttt{Hamiltonian\_Hubbard\_mod.F90}  gives $ E = -13.1889 \pm  0.0017 $  and \texttt{Hamiltonian\_LRC\_mod.F90}, $E = -13.199 \pm  0.040 $.    Note that for the  Hubbard code we have used  the default \texttt{Mz = .True.}.   This  option   breaks SU(2) spin symmetry for a given HS configuration, but produces very precise values of the energy. On the other hand,  the LRC code  is an SU(2) invariant code (as would be  choosing \texttt{Mz = .False.})  and  produces  more  fluctuations in the double occupancy.   This  partly explains the difference in  error  bars between the two codes.    To produce this data, one  can run the pyALF  python script \href{https://git.physik.uni-wuerzburg.de/ALF/pyALF/-/blob/master/Scripts/LRC.py}{\texttt{LRC.py}}.
  
% The definition of  the Coulomb repulsion is as follows. 
%A general lattice site  \texttt{I,n}   where \texttt{I: 1...Latt\%N} is the unit cell and \texttt{ n = 1 ...Latt\_unit\%NORB}  the orbital  is given by: 
%\begin{lstlisting}[style=fortran]
%X_p(:) = Latt%list(I,1)*latt%a1_p(:)  + Latt%list(I,2)*latt%a2_p(:) 
%          +   Latt_unit%Orb_pos_p(no_j,:)
%\end{lstlisting}
%or in more compact notation $ \vec{i}  + \vec{\delta}_i $.   By definition \texttt{Latt\_unit\%Orb\_pos\_p(1,:)=0}.
%The Coulomb repulsion between points   $ \vec{i}  + \vec{\delta}_i $   and $ \vec{j}  + \vec{\delta}_j $   reads: 
%\begin{equation}
%	V(\vec{i}  + \vec{\delta}_i ,  \vec{j}  + \vec{\delta}_j  )  =  \frac{U d_\mathrm{min} \alpha}{  |  \overline{\vec{i} - \vec{j}} + \vec{\delta}_i - \vec{\delta}_j  |}  
%\end{equation}



