\section{Module Hamiltonian}
\textit{Detailed description of the module Hamiltonian since it will be modified by the users}

The module contains the following subroutines:
\subsection{ham\_set}
It calls the subroutines
\begin{itemize}
\item ham\_latt
\item ham\_hop
\item ham\_v
\end{itemize}
It reads in the file
\begin{itemize}
\item parameters
\end{itemize}
It sets the variables {\tt ltrot,n\_fl,n\_sun}.
If compiled as a MPI-program, it broadcasts all variables that define the lattice, the model and the simulation process.


\subsection{ham\_latt}
It sets the lattice, by calling the subroutine
\begin{itemize}
\item make\_lattice(l1\_p, l2\_p, a1\_p,  a2\_p, latt)
\end{itemize}

\subsection{ham\_hop}
Setup of the hopping amplitudes between the vertices of the graph (lattice sites and unit cell orbitals). 
It calls the subroutines
\begin{itemize}
\item op\_make(op\_t(nc,n),ndim
\item op\_set(op\_t(nc,n))
\end{itemize}

\subsection{ham\_v}
It calls the subroutines
\begin{itemize}
\item op\_make(op\_v(i,nf),1)
\item op\_set( op\_v(nc,nf) )
\end{itemize}
\subsection{s0(n,nt)}
It is defined as $s0(n,nt)=1.d0$. Why? It is superfluous.

\subsection{alloc\_obs(ltau)}
Allocation of equal time and time-resolved quantities.

\subsection{init\_obs(ltau)}
Initializes equal time and time-resolved quantities with zero.

\subsection{obser(gr,phase,ntau)}
Includes the definition of all equal-time observables (scalars and correlation functions) that are built from the single-particle Green function based on 
Wick's theorem.

\subsection{pr\_obs(ltau)}
Output (print) of the observables.

\subsection{obsert(nt,gt0,g0t,g00,gtt,phase)}
Includes the definition of time-resolved observables that are built from the time-resolved single-particle Green function based on 
Wick's theorem.