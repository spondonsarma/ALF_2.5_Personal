\section{List of files}
\textit{all files that constitute the code, with a brief description}
\subsection{cgr1.f90 \& cgr2.f90}
Stable computation of the physical single-particle equal time Green function $G(\tau)$.
\subsection{control\_mod.f90}
Includes a set of auxilliary routines, regarding the flow of the simulation. 
Examples are initialization of performance variables, precision tests and controlled termination of the code.
\subsection{gperp.f90}
\subsection{Hamiltonian\_Hub.f90}
Here, the physical simulation parameters (the model parameters) and the lattice parameters are read in. 
The lattice, the non-interacting and the interacting part of the Hubbard Hamiltonian are set according to the parameters.
\subsection{Hop\_mod.f90}
\subsection{inconfc.f90}
The auxiliary-field QMC method is based on a Hubbard-Stratonovich decomposition of the interaction term. This decomposition introduces a space-time array of (discrete) configurations of auxiliary fields, i.e. Ising spins. 
In this routine, an existing configuration is read in, check on the dimensionality are made and, in case no prior configuration exist, a random configuration of Ising spins is set up.
\subsection{main.f90}
\subsection{nranf.f90}
\subsection{Operator.f90}
\subsection{outconfc.f90}
\subsection{print\_bin\_mod.f90}
\subsection{tau\_m.f90}
\subsection{truncation.f90}
\subsection{UDV\_WRAP.f90}
\subsection{upgrade.f90}
\subsection{wrapgrdo.f90 \& wrapgrup.f90}
Single-particle equal-time Green functions are the central object of the code. The physical single-particle equal-time Green function $G(\tau)$ is updated in wrapgrup.f90 (up propagation, from $\tau=0$ to $\tau=LTROT$) 
and in wrapgrdo.f90 (down propagation, from $\tau=LTROT$ to $\tau=0$). The update is sequentially, over all (interacting) lattice sites or lattice bonds.
\subsection{wrapul.f90 \& wrapur.f90}
To stabilize the simulation at the time slice $\tau_{2}=i n_{stab}$, the Green function has to be recomputed regularly, based on the stable matrices at $\tau_{1}=(i-1) n_{stab}$.
These stable matrices result from a singular-value-decomposition of the propagation matrix. They are computed in wrapur.f90 (up propagation, from $\tau=0$ to $\tau=LTROT$) and wrapul.f90 
 (down propagation).