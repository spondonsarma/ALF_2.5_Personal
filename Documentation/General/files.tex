\section{List of files}
\textit{all files that constitute the code, with a brief description}
\subsection{cgr1.f90 \& cgr2.f90}
Stable computation of the physical single-particle equal time Green function $G(\tau)$.
\subsection{control\_mod.f90}
Includes a set of auxiliary routines, regarding the flow of the simulation. 
Examples are initialization of performance variables, precision tests and controlled termination of the code.
\subsection{gperp.f90}
\subsection{Hamiltonian\_Hub.f90}
Here, the physical simulation parameters (the model parameters) and the lattice parameters are read in. 
The lattice, the non-interacting and the interacting part of the Hubbard Hamiltonian are set according to the parameters and the chosen Hubbard-Stratonovich decomposition.
\subsection{Hop\_mod.f90}
\subsection{inconfc.f90}
The auxiliary-field QMC method is based on a Hubbard-Stratonovich decomposition of the interaction term. This decomposition introduces a space-time array of (discrete) configurations of auxiliary fields, i.e. Ising spins. 
In this routine, an existing configuration is read in, checks on its dimensionality are made and, in case no prior configuration exists, a random configuration of Ising spins is set up.
\subsection{main.f90}
Top-level part of the program. Here, the program flow which consists of initialization, sweeps through the space-time lattice, and finalizing the program, is coded.
\subsection{nranf.f90}
Auxiliary routine controlling the evaluation of random numbers. 
\subsection{Operator.f90}
The algorithms is centered around evaluation of single-particle operators, represented as square matrices. 
In this routine, the abstract type Operator is defined, including information on the coupling strength, the sites that participate in the single-particle hopping process, and the type of Hubbard-Stratonovich transformation. 
This routine collects all program relevant operations that are applied to the type Operator, like initializations or multiplications.
\subsection{outconfc.f90}
At the end of the simulation, the last configuration of Hubbard-Stratonovich variables, together with the last set of random numbers  is written to the file confout. 
Prior to the start of a new simulation of the identical space-time dimesion, one can (manually) copy the file confout to the file confin and make the new run use the old configuration. 
Doing this saves warmup time compared to a complete random (unphysical) configuration.
\subsection{print\_bin\_mod.f90}
Here the way to write the measure bins to the respective output files is coded.
A bin is an average over many individual measurements. The bin defines the unit of Monte Carlo time.
\subsection{tau\_m.f90}
\subsection{UDV\_WRAP.f90}
\subsection{upgrade.f90}
The update of the Hubbard-Stratonovich configuration is done sequentially in the space-time lattice, i.e. one Hubbard-Stratonovich Ising spin after the other. 
In this routine, an update (i.e. a spin flip) is accepted or rejected. The decision is made using the Metropolis method of importance sampling.
\subsection{wrapgrdo.f90 \& wrapgrup.f90}
Single-particle equal-time Green functions are the central object of the code. The physical single-particle equal-time Green function $G(\tau)$ is updated in wrapgrup.f90 (up propagation, from $\tau=0$ to $\tau=LTROT$) 
and in wrapgrdo.f90 (down propagation, from $\tau=LTROT$ to $\tau=0$). The update is sequentially, over all (interacting) lattice sites or lattice bonds.
\subsection{wrapul.f90 \& wrapur.f90}
To stabilize the simulation at the time slice $\tau_{2}=i n_{stab}$, the Green function has to be recomputed regularly, based on the stable matrices at an earlier stabilization point, $\tau_{1}=(i-1) n_{stab}$.
These stable matrices result from a singular-value-decomposition of the propagation matrix. They are computed in wrapur.f90 (up propagation) and wrapul.f90  (down propagation).