% !TEX root = Doc.tex
\section{File structure}\label{sec:files}
%
\begin{table}[h]
   \begin{tabular}{l l}
   Directory & Description \\\hline
   \texttt{Prog/} & main program and subroutines  \\
  \texttt{Libraries/} &  mathematical routines \\  
  \texttt{Analysis/} & routines for error analysis and  analytic continuation\\
  \texttt{Test\_Hub\_Template/} & template for the Hubbard model\\
  \texttt{Documentation/} & \mycomment{to include}\\
%   \texttt{configure.sh} & sets the global variables for compilation \\
   \end{tabular}
   \caption{Overview of the directories. \label{table:files}}
\end{table}
%
The code package consists of the program directories \texttt{Prog/} and \texttt{Libraries/}, 
the analysis routines in \texttt{Analysis/} and \texttt{Test\_Hub\_Template/}, a template for the implementation of the Hubbard model (table~\ref{table:files}).
%
\subsection{Input files}\label{sec:input}
%
\begin{table}[h]
   \begin{tabular}{l l}
   File & Description \\\hline
  \texttt{parameters} &  sets the parameters for lattice, model and the QMC process.\\
  \texttt{seeds} & integer numbers to initialize the random number generator.
   \end{tabular}
   \caption{Overview of the input files in \texttt{Test\_Hub\_Template/Start/} for a simulation from scratch. \label{table:input}}
\end{table}
%
The input files are listed in table~\ref{table:input}. 
The parameter file \texttt{Test\_Hub\_Template/Start/parameters} has the following form, 
using the $SU(2)$-symmetric Hubbard model on a square lattice as an example (see Sec.~\ref{sec:walk1} for a detailed walkthrough):
%
\begin{lstlisting} 

===============================================================================
!  Variables for the Hubb program
!-------------------------------------------------------------------------------
&VAR_lattice
L1 = 4                    ! Length in direction a_1
L2 = 4                    ! Length in direction a_2
Lattice_type = "Square"	  ! a_1 = (1,0) and a_2=(0,1)
Model = "Hubbard_SU2"     ! Sets Nf = 1, N_sun = 2
! Model = "Hubbard_Mz"     ! alternatively: set Nf = 2, N_sun = 1
/
&VAR_Hubbard              ! Variables for the Hubbard model
ham_T   = 1.D0            ! Hopping parameter
ham_chem= 0.D0            ! chemical potential
ham_U   = 4.D0            ! Hubbard interaction
Beta    = 5.D0            ! inverse temperature
dtau    = 0.1D0           ! Thereby Ltrot=Beta/dtau
/
&VAR_QMC                  ! Variables for the QMC run
Nwrap   = 10              ! Stabilization. Green functions will be computed from scratch 
                          ! after each time interval  Nwrap*Dtau
NSweep  = 500             ! Number of sweeps
NBin    = 2               ! Number of bins
Ltau    = 1               ! 1 for calculation of time desplaced Green functions. 0 otherwise
LOBS_ST = 1               ! Start measurments at time slice LOBS_ST
LOBS_EN = 50              ! End   measurments at time slice LOBS_EN
CPU_MAX = 0.1             ! Code will stop after CPU_MAX hours. 
                          ! If not specified, code will stop after Nbin bins.
/                         ! slash terminates namelist statement - DO NOT REMOVE
\end{lstlisting}
%
\subsection{Output files} \label{sec:output}
%
\begin{table}[h]
   \begin{tabular}{l l}
   File & Description \\\hline
   \texttt{info} & contains model and QMC parameters as well as simulation metrics\\
   & (precision, acceptance rate, CPU time)\\
   \texttt{X\_eq} & Results of equal-time measurements.
   The placeholder \texttt{X} stands for the observables \\
   &    \texttt{Kin, Pot, Part, Ener} and the correlation functions \texttt{Green, SpinZ, SpinXY, Den}. \\
   \texttt{X\_tau} & Results of time-displaced measurements.
   The placeholder \texttt{X} stands for the correlation functions \\
   &    \texttt{Green, SpinZ, SpinXY, Den}. \\   
   \end{tabular}
   \caption{Overview of the output files. 
  See Sec.~\ref{sec:obs} for the definitions of observables and correlation functions. \label{table:files}}
\end{table}
%
