\documentclass[10pt,Arial]{article}
\usepackage{graphicx}
\usepackage{a4wide}
\usepackage[utf8]{inputenc}
\usepackage{bbm}               % for getting a nice 1
\usepackage[fleqn]{amsmath} % math environments and more by the AMS
\usepackage{amssymb}
\usepackage{wasysym}
\usepackage{color}
\usepackage{xspace}
\usepackage{bm}
\usepackage{subfigure}
\usepackage{titlesec}
\usepackage{blindtext}
\usepackage{tcolorbox}



\definecolor{dark-gray}{gray}{0.4}
\definecolor{light-gray}{gray}{0.9}
\definecolor{babyblueeyes}{rgb}{0.63, 0.79, 0.95}
\titleformat{\section}
{\color{blue}\normalfont\Large\bfseries}
{\color{blue}\thesection}{1em}{}
\titleformat{\subsection}
{\color{dark-gray}\normalfont\large\bfseries}
{\color{dark-gray}\thesubsection}{1em}{}


% \begin{tcolorbox}[width=\textwidth,colback={babyblueeyes},title={},colbacktitle=yellow,coltitle=blue]    
% text ...
% \end{tcolorbox}

\begin{document}
\title{Documentation for the General QMC code}
\author{Martin Bercx}
\maketitle
% \textit{The outline follows the ALPS documentation}

\section{Using the code}
\textit{Example simulation, tutorial: where to find and how to start}\\

\subsection{Parameter files}
\textit{describe the input parameters, give sample values for the stabilization parameters}
\subsection{Analysis files}
\textit{how the analysis of Monte Carlo data is done}
\section{List of files}
\textit{all files that constitute the code, with a brief description}


\section{Module Hamiltonian}
\textit{Detailed description of the module Hamiltonian since it will be modified by the users}

The module contains the following subroutines:
\subsection{ham\_set}
It calls the subroutines
\begin{itemize}
\item ham\_latt
\item ham\_hop
\item ham\_v
\end{itemize}
It reads in the file
\begin{itemize}
\item parameters
\end{itemize}
It sets the variables {\tt ltrot,n\_fl,n\_sun}.
If compiled as a MPI-program, it broadcasts all variables that define the lattice, the model and the simulation process.


\subsection{ham\_latt}
It sets the lattice, by calling the subroutine
\begin{itemize}
\item make\_lattice(l1\_p, l2\_p, a1\_p,  a2\_p, latt)
\end{itemize}

\subsection{ham\_hop}
Setup of the hopping amplitudes between the vertices of the graph (lattice sites and unit cell orbitals). 
It calls the subroutines
\begin{itemize}
\item op\_make(op\_t(nc,n),ndim
\item op\_set(op\_t(nc,n))
\end{itemize}

\subsection{ham\_v}
It calls the subroutines
\begin{itemize}
\item op\_make(op\_v(i,nf),1)
\item op\_set( op\_v(nc,nf) )
\end{itemize}
\subsection{s0(n,nt)}
It is defined as $s0(n,nt)=1.d0$. Why? It is superfluous.

\subsection{alloc\_obs(ltau)}
Allocation of equal time and time-resolved quantities.

\subsection{init\_obs(ltau)}
Initializes equal time and time-resolved quantities with zero.

\subsection{obser(gr,phase,ntau)}
Includes the definition of all equal-time observables (scalars and correlation functions) that are built from the single-particle Green function based on 
Wick's theorem.

\subsection{pr\_obs(ltau)}
Output (print) of the observables.

\subsection{obsert(nt,gt0,g0t,g00,gtt,phase)}
Includes the definition of time-resolved observables that are built from the time-resolved single-particle Green function based on 
Wick's theorem.

\section{Necessary background information}

\subsection{Definition of the physical Hamiltonian and its implementation}

The physical Hamiltonians that we can simulate have the general form:
\begin{equation}
\label{eqn_general_ham1}
\mathcal{H}
=\sum\limits_{s=1}^{N_{fl}}\sum\limits_{\bm{x},\bm{y}}
c^{\dagger}_{\bm{x} s}M_{\bm{x}\bm{y}}c^{\phantom\dagger}_{\bm{y}s}
-\sum\limits_{k=1}^{M}U_{k}\left[
\sum\limits_{s=1}^{N_{fl}}\sum\limits_{\bm{x},\bm{y}}
\left( 
c^{\dagger}_{\bm{x}s}T^{(k)}_{\bm{x}\bm{y}}c^{\phantom\dagger}_{\bm{y}s}-\alpha_{k}
\right)
\right]^{2}\;.
\end{equation}
The indices $\bm{x},\bm{y}$ are multi-indices that label sites and spin states: $\bm{x}=(i,\sigma)$, 
where $i=1,\cdots N_{sites}$ and $\sigma=1,\cdots N_{sun}$, so
\begin{equation}
\sum\limits_{\bm{x},\bm{y}}\equiv
\sum\limits_{i=1,j=1}^{N_{sites}}\sum\limits_{\sigma=1,\sigma^{\prime}=1}^{N_{sun}}\;.
\end{equation}
Note, that  we introduced \textit{two} different labels for the number of spin states (flavours): 
$N_{fl}$ and $N_{sun}$.
The number of correlated sites which is a subset of all sites, is labelled by $M$  ($M\leq N_{sites}$).

I suggest to use a more intuitive notation and to label the hopping matrix by $T$ and the interaction matrix by $V$:
\begin{equation}
\label{eqn_general_ham2}
\mathcal{H}
=\sum\limits_{s=1}^{N_{fl}}\sum\limits_{\bm{x},\bm{y}}
c^{\dagger}_{\bm{x}s}T_{\bm{x}\bm{y}}c^{\phantom\dagger}_{\bm{y}s}
-\sum\limits_{k=1}^{M}U_{k}\left[
\sum\limits_{s=1}^{N_{fl}}\sum\limits_{\bm{x},\bm{y}}
\left( 
c^{\dagger}_{\bm{x}s}V^{(k)}_{\bm{x}\bm{y}}c^{\phantom\dagger}_{\bm{y} s}-\alpha_{k}
\right)
\right]^{2}\;,
\end{equation}

\subsection{Which information does the type \textit{operator} contain?}

\section{Tutorial to set up the Hubbard model}
The $SU(2)$ symmetric Hubbard model is given by
\begin{equation}
\label{eqn_hubbard_sun}
\mathcal{H}=
-t\sum\limits_{\langle i,j\rangle,\sigma} 
\left(c^{\dagger}_{i,\sigma} c^{\phantom\dagger}_{j,\sigma} + \text{H.c.}
\right)
+ \frac{U}{2}\sum\limits_{i}\left[
\sum\limits_{\sigma}
( c^{\dagger}_{i\sigma} c^{\phantom\dagger}_{i\sigma}) -1 \right]^{2}\;.
\end{equation}
To bring Eq.~(\ref{eqn_hubbard_sun}) in the general form (\ref{eqn_general_ham2}), we define:
\begin{eqnarray}
N_{fl}&=&1\nonumber\\
N_{sun}&=&2\nonumber\\
T_{\bm{x}\bm{y}}&=&-t\delta_{\langle i,j\rangle}\delta_{\sigma,\sigma^{\prime}}\nonumber\\
M&=&N_{sites}\nonumber\\
U_{k}&=&-U/2\nonumber\\
V_{\bm{x}\bm{y}}^{(k)}&=&\delta_{i,j}\delta_{i,k}\delta_{\sigma,\sigma^{\prime}}\nonumber\\
\alpha_{k}&=&1/(N_{sites}N_{sun})^{2}\;.
\end{eqnarray}

\section{Tutorial to set up a lattice}

\section{Installation}
\subsection{Dependencies}
\textit{which software and libraries are needed and which version}
\begin{itemize}
\item libraries: LAPACK, BLAS, EISPACK, NAG \textit{They are included in the package, but NAG is not  public-domain (?)}
\item tools: cmake
\item compiler: gfortran or ifort
\end{itemize}

\subsection{Build the GQMC program from source code}
\textit{configuration, compile and Installation}
In the top level directory, where the README file resides, do:
\begin{verbatim}
mkdir build
cd build
cmake ..
make
\end{verbatim}
  
\section{Reference manual}
\section{License}
Use of the GQMC code requires citation of the paper ...
The GQMC code is available for academic and non-commercial use under the terms of the license ...
For commercial licenses, please contact the GQMC development team. 

\section{ideas}
FAQ, walkthroughs,

\end{document}
