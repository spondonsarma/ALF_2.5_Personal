% !TEX root = Doc.tex
\section{Walkthrough: the $SU(2)$-Hubbard model on a square lattice}
In this section, we describe the subroutine \texttt{Hamiltonian\_Hub.f90} which is an implementation of the Hubbard model on the square lattice. 
The $SU(2)$-symmetric Hubbard model is given by
\begin{equation}
\label{eqn_hubbard_sun}
\mathcal{H}=
\sum\limits_{\sigma=1}^{2} 
\sum\limits_{x,y } 
  c^{\dagger}_{x \sigma} T_{x,y}c^{\phantom\dagger}_{y \sigma} 
+ \frac{U}{2}\sum\limits_{x}\left[
\sum\limits_{\sigma=1}^{2}
\left(  c^{\dagger}_{x \sigma} c^{\phantom\dagger}_{x \sigma}  -1/2 \right) \right]^{2}\;.
\end{equation}

\subsection{The lattice  and basic parameters}
Here we set $\vec{a}_1 =  (1,0) $ and $\vec{a}_2 =  (0,1) $  and for an $L_1 \times L_2$  lattice  $\vec{L}_1 = L_1 \vec{a}_1$ and  $\vec{L}_2 = L_2 \vec{a}_2$.     
With this choice   the call to  \texttt{ Call Make\_Lattice( L1, L2, a1,  a2, Latt )} will generate the lattice   \texttt{Latt} of type \texttt{Lattice} such that  $N_{dim}   =N_{unit\;cell} \equiv Latt\%N$. 

In order to bring the general Hamiltonian (\ref{eqn_general_ham2}) to this form, we set
\begin{eqnarray}
N_{fl}         &=&  1 \nonumber\\
N_{col} \equiv N_{SUn}       &=&  2 \nonumber\\
M_T      & = &   1 \nonumber \\
T^{(ks)}_{x y}        &=&    T_{x,y}  \nonumber\\
M_V              &=&  N_{dim} \nonumber\\
U_{k}          &=&   -\frac{U}{2} \nonumber\\
V_{x y}^{(ks)} &=&  \delta_{x,y} \nonumber\\
\alpha_{ks}     &=&  \frac{1}{2} \nonumber \\
M_I              &=&  0
\end{eqnarray}
Since $N_{fl}=1$ for $SU(N)$-symmetric Hubbard models, we will drop the flavor index $\sigma$ in the following.  

\subsection{Hopping term}
The hopping matrix is implemented as follows. 
We allocate an array of dimension $1\times 1$, called \texttt{Op\_T}. It therefore contains only a single \texttt{Operator} structure.
We set the effective dimension for the hopping term: $N=N_{dim}$. 
And we allocate and initialize this structure by a single call to the subroutine \texttt{Op\_make}: 
\begin{verbatim}
call Op_make(Op_T(1,1),Ndim)
\end{verbatim}

Since the effective dimension is identical to the total dimension, it follows trivially, that ${\bm P}_{T}=\mathds{1}$ and ${\bm O}_{T}={\bm T}$. 


\mycomment{Note that although a checkerboard decomposition is not yet used for the Hubbard model, in principle it can be implemented.}

\subsection{Interaction term}
To implement this interaction, we allocate an array of \texttt{Operator} structures. The array is called  \texttt{Op\_V} and has dimensions $N_{dim}\times N_{fl}=N_{dim}\times 1$. 
We set the effective dimension for the interaction term: $N_{eff}=1$. 
And we allocate and initialize this array of structures by repeatedly calling the subroutine \texttt{Op\_make}: 
\begin{verbatim}
N_dim = Latt%N
N_fl = 1
N_eff = 1

do nf = 1, N_FL
do i  = 1, Latt%N
call Op_make(Op_V(i,nf),N_eff)
enddo
enddo
\end{verbatim}
For each lattice site $i$, the projection matrices ${\bm P}_{V}^{(i)}$ are of dimension $1\times N_{dim} $ and have one non-vanishing entry: $(P_{V}^{(i)})_{1j}=\delta_{ij}$. 
The effective matrices are scalars in this example: ${\bm O}_{V}^{(i)}=1$.\\


\begin{table}[h]
   \begin{tabular}{l l}
    Name of variable in the code & Description \\\hline
    \texttt{Ndim}    & Spacial dimension of the lattice (total number of sites) \\
    \texttt{Latt\%N} & Number of unit cells of the underlying Bravais lattice  \\
    \texttt{Op\_T}   & Array of structure variables that bundles all variables\\
                     & needed to define the hopping operator.\\
    \texttt{Op\_V}   & Array of structure variables that bundles all variables\\
                     & needed to define the two-particle interaction operator.\\ 
    \texttt{N\_sun}  & Number of fermion colors \mycomment{spin states of the $SU(N_{sun})$-symmetric fermions}\\
    \texttt{N\_fl}   & Number of fermion flavors\\
   \end{tabular}
   \caption{Common variables that are set in the Hamiltonian, operator and lattice modules of the code. 
   \mycomment{!!! We have a missmatch in the labelling: $N_{col}=\texttt{N\_sun}$ !!!}
   \label{tab:definitions}}
\end{table}

\subsection{Definition of the square lattice}
This is set in the subroutine \texttt{Ham\_latt}.
The square lattice is already implemented. In principle, one can specify other lattice geometries and use them by specifying the keyword \texttt{Lattice\_type} in the parameter file.



\subsection{Observables for the Hubbard model}


To do next:
\begin{itemize}
\item dicuss the measurements: what observables exit and how do I add a new one?
\item  discuss the implementation of the lattice.
\item discuss the Hubbard-Stratonovich decompositions (this is related to the coupling in the operator structure), discuss also the spin-symmetry-breaking HS-decomposition for the Hubbard model.
\end{itemize}

