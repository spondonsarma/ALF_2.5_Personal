% !TEX root = Doc.tex
\section{Walkthrough: the $SU(2)$-Hubbard model on a square lattice}
To implement a Hamiltonian, the user has to provide  a module   which  specifies the lattice, the model, as well as the observables  he/she  wishes to compute. 
In this section, we describe the module \texttt{Hamiltonian\_Hub.f90} which is an implementation of the Hubbard model on the square lattice. 
 The Hamiltonian reads 
\begin{equation}
\label{eqn_hubbard_sun}
\mathcal{H}=
\sum\limits_{\sigma=1}^{2} 
\sum\limits_{x,y =1 }^{N_{unit\; cells }} 
  c^{\dagger}_{x \sigma} T_{x,y}c^{\phantom\dagger}_{y \sigma} 
+ \frac{U}{2}\sum\limits_{x}\left[
\sum\limits_{\sigma=1}^{2}
\left(  c^{\dagger}_{x \sigma} c^{\phantom\dagger}_{x \sigma}  -1/2 \right) \right]^{2}\;.
\end{equation} 
We can make contact with the general form of the Hamiltonian by setting: 
$N_{fl} = 1$, $N_{col} \equiv N_{SUn}     =2 $,   $M_T    =    1$,  $T^{(ks)}_{x y}   =  T_{x,y}$,  $M_V   =  N_{Unit\; cells} $,  $U_{k}       =   -\frac{U}{2}$, 
 $V_{x y}^{(ks)} =  \delta_{x,y} \delta_{x,k}$,  $\alpha_{ks}   =  \frac{1}{2}  $ and $M_I       = 0 $.



\subsection{Setting the Hamiltonian.  \texttt{Ham\_set} }
The main program will call the subroutine   \texttt{Ham\_set} in the module \texttt{Hamiltonian\_Hub.f90}.
This  subroutine  defines the  public variables
\begin{lstlisting}

  Type (Operator), dimension(:,:), allocatable  :: Op_V
  Type (Operator), dimension(:,:), allocatable  :: Op_T
  Integer, allocatable :: nsigma(:,:)
  Integer              :: Ndim,  N_FL,  N_SUN,  Ltrot

\end{lstlisting}
which specify the model.  This  routine will first  read the parameter file,  then set the lattice, \texttt{Call Ham\_latt},  set the hopping \texttt{Call Ham\_hop}  and set the interaction
\texttt{call Ham\_V}.  
The parameters are read in from the file \texttt{parameters}: 
\begin{lstlisting}

===============================================================================
!  Variables for the Hubb program
!-------------------------------------------------------------------------------
&VAR_lattice
L1=4                    ! Length in direction a_1
L2=4                    ! Length in direction a_2
Lattice_type = "Square"	! a_1 = (1,0) and a_2=(0,1)
Model = "Hubbard_SU2"   ! Sets  Nf = 1, N_sun = 2
/
&VAR_Hubbard            ! Variables for the Hubbard model
ham_T   =1.D0
ham_chem=0.D0
ham_U   =4.D0
Beta    =5.D0
dtau    =0.1D0          ! Thereby Ltrot=Beta/dtau
/
&VAR_QMC      ! Variables for the QMC run
Nwrap   = 10  ! Stabilization. Green functions will be computed from scratch 
              ! after each time interval  Nwrap*Dtau
NSweep  = 500 ! Number of sweeps
NBin    = 2   ! Number of bins
Ltau    = 1   ! 1 for calculation of time desplaced. 0 otherwise
LOBS_ST = 1   ! Start measurments at time slice LOBS_ST
LOBS_EN =50   ! End   measurments at time slice LOBS_EN
CPU_MAX= 0.1  ! Code will stop after CPU_MAX hours. 
              ! If not specified, code will stop after Nbin bins.
/

\end{lstlisting}
%\end{verbatim}
Here we have three name lists relevant for  defining the lattice, model parameters as well  as the Monte Carlo run.   Thereby, \texttt{Ltrot=Beta/dtau}. 

\subsubsection{The lattice.   \texttt{Call Ham\_latt} }
The choice \texttt{Lattice\_type = "Square"} sets $\vec{a}_1 =  (1,0) $ and $\vec{a}_2 =  (0,1) $  and for an $L_1 \times L_2$  lattice  $\vec{L}_1 = L_1 \vec{a}_1$ and  $\vec{L}_2 = L_2 \vec{a}_2$.     The call to  \texttt{ Call Make\_Lattice( L1, L2, a1,  a2, Latt)} will generate the lattice   \texttt{Latt} of type \texttt{Lattice} such that  $N_{dim}   =N_{unit\;cell} \equiv Latt\%N$. 


\subsubsection{Hopping term. \texttt{Call Ham\_hop}}
The hopping matrix is implemented as follows. 
We allocate an array of dimension $1\times 1$ of type operator  called \texttt{Op\_T} and set the  dimension for the hopping  matrix to $N=N_{dim}$. One  allocates and initializes this type by a single call to the subroutine \texttt{Op\_make}: 
\begin{lstlisting}

call Op_make(Op_T(1,1),Ndim)

\end{lstlisting}
Since the hopping  does not  break down into small blocks ${\bm P}=\mathds{1}$   and  
\begin{lstlisting}

Do i= 1,Latt%N
  Op_T(1,1)%P(i) = i
Enddo

\end{lstlisting}
We set the hopping matrix  with 
\begin{lstlisting}

DO I = 1, Latt%N
   Ix = Latt%nnlist(I,1,0)
   Iy = Latt%nnlist(I,0,1)
   Op_T(1,1)%O(I  ,Ix) = cmplx(-Ham_T,   0.d0,kind(0.D0))
   Op_T(1,1)%O(Ix,I  ) = cmplx(-Ham_T,   0.d0,kind(0.D0))
   Op_T(1,1)%O(I  ,Iy) = cmplx(-Ham_T,   0.d0,kind(0.D0))
   Op_T(1,1)%O(Iy, I ) = cmplx(-Ham_T,   0.d0,kind(0.D0))
   Op_T(1,1)%O(I  ,I ) = cmplx(-Ham_chem,0.d0,kind(0.D0))
ENDDO

\end{lstlisting}
Here,    the integer  function \texttt{  j=  Latt\%nnlist(I,n,m)}   is defined in the lattice module and returns the index of the lattice site $ \vec{I} +  n \vec{a}_1 +  m \vec{a}_2$. Note that periodic boundary conditions are 
already taken into account.  The hopping parameter, \texttt{Ham\_T} as well as the chemical potential \texttt{Ham\_chem} are read from the parameter file.  
%Finally,   \texttt{ Op_T(1,1)\%g = -Dtau }

\mycomment{Note that although a checkerboard decomposition is not  used here,  it can be implemented by considering a larger number of sparse Hopping matrices.}

\subsubsection{Interaction term. \texttt{Call Ham\_V}}
To implement this interaction, we allocate an array of \texttt{Operator} type. The array is called  \texttt{Op\_V} and has dimensions $N_{dim}\times N_{fl}=N_{dim} \times 1$. 
We set the dimension for the interaction term to  $N=1$, and  allocate and initialize this array of type  \texttt{Operator} by repeatedly calling the subroutine \texttt{Op\_make}: 

\begin{lstlisting}

do i  = 1,Latt%N
   call Op_make(Op_V(i,1),1)
enddo

\end{lstlisting}
For each lattice site $i$, the  matrices ${\bm P}$ are of dimension $1\times N_{dim} $ and have only one non-vanishing entry. Thereby we can set:

\begin{lstlisting}

Do i = 1,Latt%N
   Op_V(i,1)%P(1)   = i
   Op_V(i,1)%O(1,1) = cmplx(1.d0,0.d0, kind(0.D0))
   Op_V(i,1)%g      = sqrt(cmplx(-dtau*ham_U/(dble(N_SUN)),0.D0,kind(0.D0)))
   Op_V(i,1)%alpha  = cmplx(-0.5d0,0.d0, kind(0.D0))
   Op_V(i,1)%type   = 2
Enddo

\end{lstlisting}
so as to completely define the interaction term. 


\subsection{Observables}

At this point, all the information   for the simulation to  start has been provided.  The code will sequentially go through  the operator list  \texttt{Op\_V}  and update the  fields.   Between  time slices   \texttt{LOBS\_ST}  and 
\texttt{LOBS\_EN}   the main program will call the routine  \texttt{Obser(GR,Phase,Ntau)}   which is also provided by the user.  For each configuration of the  fields Wicks theorem holds so that it suffices 
to  know the single particle Green function so as to compute  any observable. 
The main program provides the  equal time Green function: 
\texttt{GR(Ndim,Ndim,N\_FL)}, the phase \texttt{PHASE} and the time slice on which the measurement is being carried out \texttt{Ntau}.   The Green function is defined as: 
\begin{equation}
	 GR(x,y,\sigma)  = \langle c^{\phantom{\dagger}}_{x,\sigma} c^{\dagger}_{y,\sigma}  \rangle
\end{equation}
Space for observables in allocated in the  subroutine \texttt{Call Alloc\_obs}. At the beginning of each bin, the  observables are set to zero \texttt{Call Init\_obs} and at the end of each bin the observables are written out on disc \texttt{Call Pr\_obs}.



\section{Walkthrough: the $SU(2)$-Hubbard model on a square lattice coupled to a transverse Ising field}




The model we consider here  is very similar to the  above,  but has an additional coupling to a transverse field. 
\begin{equation}
\label{eqn_hubbard_sun_Ising}
\mathcal{H}=
\sum\limits_{\sigma=1}^{2} 
\sum\limits_{x,y } 
  c^{\dagger}_{x \sigma} T_{x,y}c^{\phantom\dagger}_{y \sigma} 
+ \frac{U}{2}\sum\limits_{x}\left[
\sum\limits_{\sigma=1}^{2}
\left(  c^{\dagger}_{x \sigma} c^{\phantom\dagger}_{x \sigma}  -1/2 \right) \right]^{2}   
+  \xi \sum_{\sigma,\langle x,y \rangle} \hat{Z}_{\langle x,y \rangle}  \left( c^{\dagger}_{x \sigma} c^{\phantom\dagger}_{y \sigma}  + h.c. \right) + h \sum_{\langle x,y \rangle} \hat{X}_{\langle x,y \rangle}
\end{equation}
We can make contact with the general form of the Hamiltonian by setting: 
$N_{fl} = 1$, $N_{col} \equiv N_{SUn}     =2 $,   $M_T    =    1$,  $T^{(ks)}_{x y}   =  T_{x,y}$,  $M_V   =  N_{Unit\; cells} \equiv N_{dim}$,  $U_{k}       =   -\frac{U}{2}$, 
 $V_{x y}^{(ks)} =  \delta_{x,y} \delta_{x,k}$,  $\alpha_{ks}   =  \frac{1}{2}  $ and $M_I       = 2 N_{Unit\; cells} $.
The modifications  required to generalize the Hubbard model code to the above model are two-fold.  Firstly, one has to specify the function \texttt{Real (Kind=8) function S0(n,nt)} and  secondly  modify the interaction \texttt{Call Ham\_V}.


\subsection{Interaction term. \texttt{Call Ham\_V}}
The dimension of   \texttt{Op\_V}  is now  $(M_V + M_I)\times N_{fl}=(3*N_{dim}) \times 1$. 
We set the effective dimension  for the Hubbard term to  $N=1$  and to  $N=2$ for the Ising term. The allocation of this array of operators reads: 

\begin{lstlisting}

do i  = 1,Ndim
  call Op_make(Op_V(i,1),1)
enddo
do i  =  Ndim+1, 3*Ndim
  call Op_make(Op_V(i,1),2)
enddo

\end{lstlisting}
As for the Hubbard case, the first \texttt{Ndim}  operators read: 

\begin{lstlisting}

Do i = 1,Ndim
    Op_V(i,1)%P(1)   = i
    Op_V(i,1)%O(1,1) = cmplx(1.d0  ,0.d0, kind(0.D0))
    Op_V(i,1)%g      = sqrt(cmplx(-dtau*ham_U/(DBLE(N_SUN)), 0.D0, kind(0.D0)))
    Op_V(i,1)%alpha  = cmplx(-0.5d0,0.d0, kind(0.D0))
    Op_V(i,1)%type   = 2
Enddo

\end{lstlisting}
The  next \texttt{2*Ndim} operators run through the 2N bonds of the square lattice and   are given by:
\begin{lstlisting}

Do nc = 1,N_coord   ! Coordination number = 2
   Do i = 1,Ndim
      j = i + nc*Ndim
      Op_V(j,1)%P(1)    =  i 
      If (nc == 1) Op_V(j,1)%P(2)   = Latt%nnlist(i,1,0)
      If (nc == 2) Op_V(j,1)%P(2)   = Latt%nnlist(i,0,1)
      Op_V(j,1)%O(1,2) = cmplx(1.d0  ,0.d0, kind(0.D0))
      Op_V(j,1)%O(2,1) = cmplx(1.d0  ,0.d0, kind(0.D0))
      Op_V(j,1)%g      = cmplx(-dtau*ham_xi, 0.D0, kind(0.D0)))
      Op_V(j,1)%alpha  = cmplx(0d0,0.d0, kind(0.D0))
      Op_V(j,1)%type   = 1
    Enddo
Enddo

\end{lstlisting}

Here,  \texttt{ham\_xi} defines the coupling strength  between the Ising  and fermion degree of freedom.

\subsection{The function \texttt{Real (Kind=8) function S0(n,nt)} }
As mentioned above,  a configuration is given by
\begin{equation}
	C = \left\{   s_{i,\tau} ,  l_{j,\tau}  \text{ with }  i=1\cdots M_I,  j = 1\cdots M_V,  \tau=1,L_{Trotter}  \right\}
\end{equation}
and is stored in the  integer array \texttt{nsigma(M\_V + M\_I, Ltrot)}.  With the above ordering of Hubbard and Ising interaction terms,  and a for a given  imaginary time, the first $Ndim$ 
fields corresponds to the Hubbard  interaction and the next 2*Ndim ones to the Ising interaction.    The first   argument of the function \texttt{S0}, n,  corresponds to the index of the operator  string 
\texttt{Op\_V(n,1)}. If \texttt{Op\_V(n,1)\%type = 2},    \texttt{S0(n,nt)}  returns 1.  If   \texttt{Op\_V(n,1)\%type = 1}  then function \texttt{S0}  returns
\begin{equation}
\frac{e^{-S_{0,I} \left(  s_{1,\tau},  \cdots,  - s_{m,\tau},  \cdots s_{M_I,\tau}   \right) } }{e^{-S_{0,I}  \left(  s_{1,\tau},  \cdots,   s_{m,\tau},  \cdots s_{M_I,\tau}   \right)   } }	
\end{equation}
That is,  $ \texttt{S0(n,nt)} $  returns the ratio of the new to old weight  Ising Hamiltonian upon flipping a single Ising spin $ s_{m,\tau} $. Note that in  this specific case  \texttt{ m = n -  Ndim } 
 

%\begin{table}[h]
%   \begin{tabular}{l l}
%    Name of variable in the code & Description \\\hline
%    \texttt{Ndim}    & Spacial dimension of the lattice (total number of sites) \\
%    \texttt{Latt\%N} & Number of unit cells of the underlying Bravais lattice  \\
%    \texttt{Op\_T}   & Array of structure variables that bundles all variables\\
%                     & needed to define the hopping operator.\\
%    \texttt{Op\_V}   & Array of structure variables that bundles all variables\\
%                     & needed to define the two-particle interaction operator.\\ 
%    \texttt{N\_sun}  & Number of fermion colors \mycomment{spin states of the $SU(N_{sun})$-symmetric fermions}\\
%    \texttt{N\_fl}   & Number of fermion flavors\\
%   \end{tabular}
%   \caption{Common variables that are set in the Hamiltonian, operator and lattice modules of the code. 
%   \mycomment{!!! We have a missmatch in the labelling: $N_{col}=\texttt{N\_sun}$ !!!}
%   \label{tab:definitions}}
%\end{table}
%
%\subsection{Definition of the square lattice}
%This is set in the subroutine \texttt{Ham\_latt}.
%The square lattice is already implemented. In principle, one can specify other lattice geometries and use them by specifying the keyword \texttt{Lattice\_type} in the parameter file.
%
%
%
%\subsection{Observables for the Hubbard model}
%
%
%To do next:
%\begin{itemize}
%\item dicuss the measurements: what observables exit and how do I add a new one?
%\item  discuss the implementation of the lattice.
%\item discuss the Hubbard-Stratonovich decompositions (this is related to the coupling in the operator structure), discuss also the spin-symmetry-breaking HS-decomposition for the Hubbard model.
%\end{itemize}
%
