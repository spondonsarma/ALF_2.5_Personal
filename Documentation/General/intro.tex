% !TEX root = doc.tex
\section{Introduction}\label{sec:intro}

The auxiliary field quantum Monte Carlo approach is the algorithm of choice to simulate a variety of correlated electron systems in the solid state and beyond.  The phenomena  one can investigate in detail include correlation effects in in the bulk and surfaces of topological insulators, quantum phase transitions between semimetals (Dirac fermions)  and insulators,  deconfined quantum critical points, topologically ordered phases, heavy fermion systems, nematic and magnetic quantum phase transitions in metals,   superconductivity in spin orbit split bands, SU(N) symmetric models,  etc.  This ever growing list of phenomena,  is based on  recent symmetry based insights that allow one to  find  sign free formulations of the  problem thus allowing solutions in polynomial time.    The aim of this project is to introduce a general formulation of the finite temperature  auxiliary field method  so as to quickly be able to play with different model Hamiltonians  at  minimal programming cost.  
	
		The first and most important  part is to define a general Hamiltonian  that  can  accommodate a large class of models  (see Sec.~\ref{sec:def}). Our approach is to express the model as a sum of one-body terms, a sum of two-body terms each written as a perfect square of a one one body term, as well as one-body terms  coupled with an Ising field with  dynamics to be specified by the user.   Symmetry considerations  are  imperative to enhance the speed of the code.   We thereby include a {\it color} index  reflecting  and SU(N) color symmetry as  well as a flavour index  reflecting  the fact that  after Hubbard Stratonovitch transformation,  the  Fermionic determinant is block diagonal in this index.     The form of the interaction in terms of sums of perfect squares allows us to use standard and generic forms of  discrete    Hubbard-Stratonovitch transformations.  The action for a given Hubbard Stratonovitch transformation  has to be specified by the user. 
		
               To define the action and thereby the model,  we will need a lattice, operators, as well as  observables  which we  wish to compute. The structures we have opted for are describes in Sec..~\ref{sec:def}
		 The general Hamiltonian operator is written down in Sec.~\ref{sec:def}, followed by 
a brief outline  of the quantum Monte Carlo algorithm. 
In Sec.~\ref{sec:imp}, we discuss the implementation of a model, introducing the \texttt{Operator} data structure which is the building block of the Hamiltonian. And we disuss the implementation of the lattice and the observables.
Section ~\ref{sec:io} is about actually running the code. We describe input and output files, the analysis protocoll and the compilation procedure. 
In Sec.~\ref{sec:walk1} and \ref{sec:walk2} two detailed walkthroughs are performed: the $SU(2)$-symmetric Hubbard  on a square lattice (Sec.~\ref{sec:walk1}) and the same model, but additionally coupled to a transverse field Ising model (Sec.~\ref{sec:walk2}).


