% !TEX root = doc.tex
% Please do not remove this.
\section{Introduction}\label{sec:intro}
The auxiliary field quantum Monte Carlo approach has by now proven to be a key algorithm to simulate a variety of  electron systems where correlations effects play a dominant role.  This includes correlation effects in topological band structures, quantum phase transitions between semimetals and insulators, deconfined quantum critical points, topologically ordered phases, heavy fermion systems, nematic and magnetic quantum phase transitions in metals,   superconductivity in the presence of spin orbit coupling. This list of ever growing phenomena is based on  recent  symmetry based insights, which allows one to  find formulations that avoid the so called negative sign problem.   The aim of this project is to introduce a general formulation of the auxiliary methods, 

 \mycomment{placeholder for a general introduction, mentioning the purpose and the powers of the general QMC code.}
This documentation is organized as follows. The general Hamiltonian operator is written down in Sec.~\ref{sec:def}, followed by 
a brief outline  of the quantum Monte Carlo algorithm. 
In Sec.~\ref{sec:imp}, we discuss the implementation of a model, introducing the \texttt{Operator} data structure which is the building block of the Hamiltonian. And we disuss the implementation of the lattice and the observables.
Section ~\ref{sec:io} is about actually running the code. We describe input and output files, the analysis protocoll and the compilation procedure. 
In Sec.~\ref{sec:walk1} and \ref{sec:walk2} two detailed walkthroughs are performed: the $SU(2)$-symmetric Hubbard  on a square lattice (Sec.~\ref{sec:walk1}) and the same model, but additionally coupled to a transverse field Ising model (Sec.~\ref{sec:walk2}).


