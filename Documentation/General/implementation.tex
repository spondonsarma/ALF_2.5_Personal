\section{Implementation of a model Hamiltonian} 
To implement a Hamiltonian which belongs to the class of Hamiltonians defined by Eq.~(\ref{eqn_general_ham2}), 
the user only has to write/modify a single subroutine. A template is given by \texttt{Hamiltonian\_template.f90}. 
Existing model subroutines are \texttt{Hamiltonian\_Hubb.f90}.

\subsection{The \texttt{Operator} variable}

In the code implementation, we define a structure called \texttt{Operator}. 
This structure variable \texttt{Operator} bundles several components that are needed to define and use an operator matrix in the program.
In Fortran a structure variable like this is called a derived type. 
The components it contains are: 
\begin{itemize}
\item the projector ${\bm P}_{V}$,
\item the matrix ${\bm O}_V$, 
\item the effective dimension $N_{eff}$,
\item and a couple of auxiliary matrices and scalars.
\end{itemize}
In general, we will not only have one structure variable \texttt{Operator}, but a whole  array of these structures.

 

\begin{table}[h]
   \begin{tabular}{l l}
    Name of variable in the code & Description \\\hline
    \texttt{Op\_V\%N}            & effective dimension $N_{eff}$\\
    \texttt{Op\_V\%O}            &  matrix  $\mathbf{O}_{V}$\\
    \texttt{Op\_V\%U}            &  matrix containing the eigenvectors of $\mathbf{O}_{V}$  \\
    \texttt{Op\_V\%E}            &  eigenvalues of $\mathbf{O}_{V}$ \\
    \texttt{Op\_V\%P}            &  projection matrix $\mathbf{P}_{V}$ \\
    \texttt{Op\_V\%N\_non\_zero} &  number of non-vanishing eigenvalues of $\mathbf{O}_{V}$ \\
    \texttt{Op\_V\%g}            &  coupling strength in Hubbard-Stratonovich transformation \\  
    \texttt{Op\_V\%alpha}        &  constant \\
    \texttt{Op\_V\%type}         &  integer parameter to set the type of\\
                                 &  Hubbard-Stratonovich transformation  \mycomment{Possible Issue: type is also a Fortran keyword}
   \end{tabular}
   \caption{Components of the \texttt{Operator} structure variable \texttt{Op\_V}.
   \label{tab:definitions}}
\end{table}

\subsection{The observables}

\subsection{The lattice}
