% !TEX root = Doc.tex
\section{Running the code}\label{sec:running}
In this section we describe the steps to compile and run the code and to perform the error analysis of the data.


\subsection{Compilation}
The environment variables are defined in the bash script \texttt{set\_env.sh} as follows:
\lstset{style=custombash}
\begin{lstlisting}

export DIR=`pwd`
export f90="gfortran"                         # compiler
export FL="-c -O3"                            # compiler flags
export Libs=${DIR}"/Libraries/"
export LIB_BLAS_LAPACK="-llapack -lblas"
export Enable_MPI=false                       # compilation with or without MPI directives

if $Enable_MPI ; then
   echo '#define MPI' > Prog_8/machine
else
   echo '#define noMPI' > Prog_8/machine
fi

\end{lstlisting}
The program can be compiled and ran either in single-thread mode (define \texttt{export Enable\_MPI=false}) or 
in multithreading mode (define \texttt{export Enable\_MPI=true}) using the MPI standard for parallelization.
To compile the libraries, the analysis programs and the quantum Monte Carlo program, the followin	g steps should be executed:
\begin{enumerate}
\item Export  the environment variables:
\begin{verbatim}
source set_env.sh
\end{verbatim}
\item Compile the libraries and the error analysis routines
\begin{verbatim}
cd Libraries
make
cd ..
cd Analysis
make
cd ..
\end{verbatim}
\item Compile the quantum Monte Carlo code
\begin{verbatim}
cd Prog
make
cd ..
\end{verbatim}
\end{enumerate}


\subsection{Starting a simulation}
To start a simulation from scratch, the following files have to be present: \texttt{parameters} and \texttt{seeds}. 
To run a single-thread simulation for the Hubbard model, issue the command
\begin{verbatim}
./Prog/Hubb.out
\end{verbatim}
To restart the code using an existing simulation as a starting point, first run the script\\ \texttt{Test\_Hub\_Template/Start/out\_to\_in.sh} to set 
the input configuration files.

\subsection{Error analyis}
To perform an error analysis, based on the jackknife scheme, of the Monte Carlo bins for all observables run the script \texttt{analysis.sh}.

 



