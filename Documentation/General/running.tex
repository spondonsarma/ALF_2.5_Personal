% !TEX root = Doc.tex
\section{Running the code}\label{sec:io}
In this section we describe the steps to compile and run the code and to perform the error analysis of the data.


\subsection{Compilation}
The following steps will compile the sample code, resulting in the  binary program file \texttt{Hubb\_Template.out}, and also compile the 
analysis program.
\begin{enumerate}
\item Export  environment variables:
\begin{verbatim}
source configure.sh
\end{verbatim}
\item Compile the libraries: 
\begin{verbatim}
cd Libraries
make
cd ..
\end{verbatim}
\item Compile the analysis codes: 
\begin{verbatim}
cd Analysis
make
cd ..
\end{verbatim}
\item 
Enable or disable MPI parallelization by modifying the file \texttt{Prog/machine}.
\mycomment{We should eventually move the choice of MPI or not entirely to the script \texttt{configure.sh}.}
\item Compile the simulation code:
\begin{verbatim}
cd Prog
make
cd ..
\end{verbatim}
\end{enumerate}


\subsection{Starting a simulation}
To start a simulation from scratch, the following files have to be present: \texttt{parameters} and \texttt{seeds}. 
To run a single-thread simulation for the Hubbard model, issue the command
\begin{verbatim}
./Prog/Hubb.out
\end{verbatim}
To run the code in MPI parallelized mode, use the settings of your MPI implementation. 
To restart the code using an existing simulation as a starting point, first run the script \texttt{Test\_Hub\_Template/Start/out\_to\_in.c} to set 
the input configuration files.

\subsection{Error analyis}

To perform an error analysis, based on the jackknife scheme, of the Monte Carlo bins for all observables run the script \texttt{analysis.c}.

% 
% 
% output files; \texttt{info,confout\_x,ener, Green\_eq, Den\_eq, SpinZ\_eq, SpinXY\_eq, Green\_tau, Den\_tau, SpinZ\_tau, SpinXY\_tau}\\
% input files: \texttt{parameters, seeds,confin\_x}
% shell script files: \texttt{out\_to\_in.c, analysis.c}
% 



