\section{Tutorial to set up the $SU(2)$-Hubbard model on a square lattice}
The $SU(2)$ symmetric Hubbard model is given by
\begin{equation}
\label{eqn_hubbard_sun}
\mathcal{H}=
-t\sum\limits_{\langle i,j\rangle,\sigma} 
\left(c^{\dagger}_{i,\sigma} c^{\phantom\dagger}_{j,\sigma} + \text{H.c.}
\right)
+ \frac{U}{2}\sum\limits_{i}\left[
\sum\limits_{\sigma}
\left(  c^{\dagger}_{i\sigma} c^{\phantom\dagger}_{i\sigma}  -1/2 \right) \right]^{2}\;.
\end{equation}
To bring Eq.~(\ref{eqn_hubbard_sun}) in the general form (\ref{eqn_general_ham2}), we set
\begin{eqnarray}
N_{fl}&=&1\nonumber\\
N_{sun}&=&2\nonumber\\
T_{x y}&=&-t\delta_{\langle i,j\rangle}\delta_{\sigma,\sigma^{\prime}}\nonumber\\
M&=&N_{sites}\nonumber\\
U_{k}&=&-U/2\nonumber\\
V_{x y}^{(k)}&=&
\delta_{x,y}\delta_{i,k}=
\delta_{i,j}\delta_{\sigma,\sigma^{\prime}}\delta_{i,k}\nonumber\\
\alpha_{k}&=&1/(N_{sites}N_{sun})^{2}=1/(2 N_{sites})^{2}\;.
\end{eqnarray}

In the following, we skip the $N_{sun}$-spin degree of freedom which is present in the multi-indices $x,y$ of the matrices $\bm{T}$ and ${\bm V}^{(k)}$. 
So $N_{dim}=N_{sites}$. 

\mycomment{What is the role of the $N_{sun}$ index in the code?
\begin{itemize}
\item it appears in the coupling $g$ in the \texttt{Operator} structure.
\item it appears as a normalization constant in the definition of observables
\item it appears as exponent in the calculation of the phase factor and the update ratio 
\end{itemize}
Is the  code limited to $SU(N)$ symmetric models with respect to the $N_{sun}$ degree of freedom? }

Note that in this example $N_{dim}=Latt\%N$ since there is only one spacial orbital per unit cell of the underlying Bravais lattice.

\subsection{Hopping term}
The hopping matrix is implemented as follows. 
We allocate an array of dimension $1\times 1$, called \texttt{Op\_T}. It therefore contains only a single \texttt{Operator} structure.
We set the effective dimension (here, $N_{eff}=N_{dim}$), and allocate and initialize this structure by a single call to the subroutine \texttt{Op\_make}: 
\begin{verbatim}
call Op_make(Op_T(1,1),Ndim)
\end{verbatim}
Since the effective dimension is identical to the total dimension, it follows trivially, that ${\bm P}_{T}=\mathds{1}$ and ${\bm O}_{T}={\bm T}$. 
\mycomment{Although a checkerboard decomposition is not yet used for the Hubbard model, in principle it can be implemented.}

\subsection{Interaction term}
To implement this interaction, we allocate an array of \texttt{Operator} structures. The array is called  \texttt{Op\_V} and has dimensions $N_{dim}\times N_{fl}=N_{sites}\times 1$. 
We set the effective dimension, $N_{eff}=1$, and allocate and initialize this array of structures by repeatedly calling the subroutine \texttt{Op\_make}: 
\begin{verbatim}
N_sites = Latt%N
N_fl = 1
N_eff = 1

do nf = 1, N_FL
do i  = 1, Latt%N
call Op_make(Op_V(i,nf),N_eff)
enddo
enddo
\end{verbatim}
For each lattice site $i$, the projection matrices ${\bm P}_{V}^{(i)}$ are of dimension $1\times N_{dim} $ and have one non-vanishing entry: $(P_{V}^{(i)})_{1j}=\delta_{ij}$. 
The effective matrices are again trivial: ${\bm O}_{V}^{(i)}=1$.\\


\begin{table}[h]
   \begin{tabular}{l l}
    Name of variable in the code & Description \\\hline
    \texttt{Ndim}    & Spacial dimension of the lattice (total number of sites) \\
                     & \mycomment{what about the $N_{sun}$?} \\
    \texttt{Latt\%N} & Number of unit cells of the underlying Bravais lattice  \\
    \texttt{Op\_T}   & Array of structure variables that bundles all variables\\
                     & needed to define the hopping operator.\\
    \texttt{Op\_V}   & Array of structure variables that bundles all variables\\
                     & needed to define the two-particle interaction operator.\\ 
    \texttt{N\_sun}    & Number of spin states of the $SU(N_{sun})$-symmetric fermions\\
    \texttt{N\_fl}      & Number of spin flavors\\
   \end{tabular}
   \caption{Common variables that are set in the Hamiltonian, operator and lattice modules of the code.
   \label{tab:definitions}}
\end{table}

\subsection{Definition of the lattice}
This is set in the subroutine \texttt{Ham\_latt}.
The square lattice is already implemented. In principle, one can specify other lattice geometries and use them by specifying the keyword \texttt{Lattice\_type} in the parameter file.



\subsection{Observables for the Hubbard model}


To do next:
\begin{itemize}
\item dicuss the measurements: what observables exit and how do I add a new one?
\item  discuss the implementation of the lattice.
\item discuss the Hubbard-Stratonovich decompositions (this is related to the coupling in the operator structure), discuss also the spin-symmetry-breaking HS-decomposition for the Hubbard model.
\end{itemize}

