\section{Tutorial to set up the $SU(2)$-Hubbard model on a square lattice}
The $SU(2)$ symmetric Hubbard model is given by
\begin{equation}
\label{eqn_hubbard_sun}
\mathcal{H}=
-t\sum\limits_{\langle i,j\rangle,\sigma} 
\left(c^{\dagger}_{i,\sigma} c^{\phantom\dagger}_{j,\sigma} + \text{H.c.}
\right)
+ \frac{U}{2}\sum\limits_{i}\left[
\sum\limits_{\sigma}
\left(  c^{\dagger}_{i\sigma} c^{\phantom\dagger}_{i\sigma}  -1/2 \right) \right]^{2}\;.
\end{equation}
To bring Eq.~(\ref{eqn_hubbard_sun}) in the general form (\ref{eqn_general_ham2}), we set
\begin{eqnarray}
N_{fl}&=&1\nonumber\\
N_{sun}&=&2\nonumber\\
T_{x y}&=&-t\delta_{\langle i,j\rangle}\delta_{\sigma,\sigma^{\prime}}\nonumber\\
M&=&N_{sites}\nonumber\\
U_{k}&=&-U/2\nonumber\\
V_{x y}^{(k)}&=&
\delta_{x,y}\delta_{i,k}=
\delta_{i,j}\delta_{\sigma,\sigma^{\prime}}\delta_{i,k}\nonumber\\
\alpha_{k}&=&1/(N_{sites}N_{sun})^{2}=1/(2 N_{sites})^{2}\;.
\end{eqnarray}

In the following, we skip the $N_{sun}$-spin degree of freedom which is present in the multi-indices $x,y$ of the matrices $\bm{T}$ and ${\bm V}^{(k)}$. 
So $N_{dim}=N_{sites}$. \mycomment{Is the code limited to $SU(N)$ symmetric models with respect to the $N_{sun}$ degree of freedom? }
So both $\bm{T}$ and ${\bm V}^{(k)}$  have dimension $N_{sites}$.
\mycomment{Note that in this example $N_{dim}=Latt\%N$ since there is only one spacial orbital per unit cell of the underlying Bravais lattice}

\subsection{Hopping interaction}
The hopping matrix is implemented as follows. 
We allocate an array of dimension $1\times 1$, called \texttt{Op\_T}. It therefore contains one \texttt{Operator} structure.
We specify the effective dimension ($N_{dim}$), allocate and initialize this structure by calling the subroutine \texttt{Op\_make}: \texttt{call Op\_make(Op\_T(1,1),Ndim)}.
It follows trivially, that ${\bm P}_{T}=\mathds{1}$ and ${\bm O}_{T}=({\bm T}_{ij})$ 
\mycomment{Although a checkerboard decomposition is not yet used for the Hubbard model, in principle it can be implemented.}

\subsection{Two-particle interaction}
To implement this interaction, we allocate an array of \texttt{Operator} structures. The array is called  \texttt{Op\_V} and has dimensions $N_{dim}\times 1$. 
We specify the effective dimension ($1$), allocate and initialize this array of structures by calling the subroutine \texttt{Op\_make}: 
\begin{verbatim}
do i = 1, Latt%N
call Op_make(Op_V(i,1),1)
enddo
\end{verbatim}
For each lattice site $i$, the projection matrices ${\bm P}_{V}^{(i)}$ are of dimension $1\times N_{dim} $ and have one non-vanishing entry: $(P_{V}^{(i)})_{1j}=\delta_{ij}$. 
The effective matrices are again trivial: ${\bm O}_{V}^{(i)}=1$.\\



To do next:
\begin{itemize}
\item dicuss the measurements: what observables exit and how do I add a new one?
\item  discuss the implementation of the lattice.
\end{itemize}

