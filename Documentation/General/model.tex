

\section{Definition of the model Hamiltonian}
The class of solvable models includes  Hamiltonians $\mathcal{H}$ having the following general form:
${\mathcal{H}=\mathcal{H}_{T}+\mathcal{H}_{V}}$, where
\begin{eqnarray}
\label{eqn_general_ham2}
\mathcal{H}_{T}
&=&
\sum\limits_{s=1}^{N_{fl}}
\sum\limits_{\sigma=1}^{N_{col}}
\sum\limits_{x,y}
c^{\dagger}_{x \sigma   s}T_{xy}^{(s)} c^{\phantom\dagger}_{y \sigma s}\\
\mathcal{H}_{V}
&=&
-
\sum\limits_{k=1}^{M}U_{k}
\left\{
\sum\limits_{s=1}^{N_{fl}}
\sum\limits_{\sigma=1}^{N_{col}}
\left[
\sum\limits_{x,y}
\left(
c^{\dagger}_{x \sigma s}V_{xy}^{(k s)}c^{\phantom\dagger}_{y \sigma s}
\right)
-\alpha_{k s} 
\right]
\right\}^{2}\;.
\end{eqnarray}
The indices have the following meaning:
\begin{itemize}
\item The number of fermion \textit{flavors} is set by $N_{fl}$. 
\item The number of fermion \textit{colors} is set by $N_{col}$. \mycomment{ Does it set the symmetry group of the fermions, namely 
the dimension of the special unitary group $SU(N_{sun})$?}
\item The indices $x,y$ label lattice sites where $x,y=1,\cdots N_{dim}$. 
$N_{dim}$ is the total number of spacial vertices: $N_{dim}=N_{unit\;cell} N_{orbital}$, where $N_{unit\;cell}$ is the number of unit cells of the underlying Bravais lattice and $N_{orbital}$ is the number of (spacial) orbitals per unit cell \mycomment{Check the definition of $N_{orbital}$ in the code.} 
\item Therefore, the  matrices $\bm{T}^{(s)}$ and $\bm{V}^{(ks)}$ are  of dimension $N_{dim}\times N_{dim}$
\item The number of correlated sites which is a subset of all sites, is labelled by $M$  ($M\leq N_{dim}$).
\mycomment{Be more general here and speak of correlated blocks?}
\end{itemize}
Note that the matrices  $\bm{T}^{(s)}$ and $\bm{V}^{(ks)}$ explicitly depend on the flavor index $s$ but not on the color index $\sigma$. 
Using this symmetry property is essential for an efficient code implementation. The color index $\sigma$ only appears 
\begin{itemize}
\item in the coupling $g$ in the \texttt{Operator} structure (see Sec.~\ref{}).
\item as normalization constant in the definition of observables (see Sec.~\ref{})
\item as exponent in the calculation of the phase factor and the Monte Carlo update ratio.
\end{itemize}



\subsection{Structure of the hopping matrix  ${\bf T}$ and the interaction matrices ${\bf V}^{(k)}$}

In general, the matrices ${\bf T}^{(s)}$ and ${\bf V}^{(ks)}$ are sparse matrices. 
This property is used to minimize computational cost and storage.
In the following, we discuss the implementation of the single-particle matrix representation ${\bf V}^{(ks)}$ of the interaction operator. 
The same logic applies for the hopping matrix ${\bf T}^{(s)}$.

We denote a subset of $N_{eff}$ \mycomment{(in the code, $N_{eff}$ is called just $N$)} degrees of freedom \mycomment{here: sites} by the set  $[z_{1},\cdots  z_{N_{eff}}]$ and define it to contain only vertices for which an interaction term is defined:
\begin{equation}
V^{(ks)}_{x y}\neq 0\quad \text{only if} \quad x,y \in [z_{1}^{(ks)},\cdots  z_{N_{eff}^{(ks)}}^{(ks)}]\;.
\end{equation}
We define the projection matrices $\mathbf{P}^{(ks)}_{V}$ of dimension $N_{eff}^{(ks)}\times N_{dim}$:
\begin{equation}
(P_{V}^{(ks)})_{i,z}=\delta_{z_{i}^{(ks)},z}\;,
\end{equation}
where $i\in [1,\cdots N_{eff}^{(ks)}]$ and $z\in [1,\cdots N_{dim}]$. The matrix operator $\bm{P}^{(ks)}_{V}$ picks out the non-vanishing entries of $\bm{V}^{(ks)}$, 
which are contained in the rank-$N_{eff}^{(ks)}$  matrix $\bm{O}_{V}^{(ks)}$:
\begin{equation}
\bm{V}^{(ks)}=\bm{P}^{(ks) T}_{V} \bm{O}_{V}^{(ks)}\bm{P}^{(ks)}_{V}\;,
\end{equation}
and
\begin{equation}
V_{xy}^{(ks)}=(P^{(ks)}_{V})_{ix} \left[O_{V}^{(ks)}\right]_{ij}(P_{V}^{(ks)})_{jy}=\sum\limits_{i,j}^{N_{eff}^{(ks)}} \delta_{z_{i}^{(ks)},x}  \left[O_{V}^{(ks)}\right]_{ij} \delta_{z_{j}^{(ks)},y} \;.
\end{equation}
\mycomment{Comment that the P matrices have only one non-vanishing entry per column.}
To set the  interaction part, we therefore have to specify the following:
\begin{itemize}
\item the matrix elements $\left[O_{V}^{(k)}\right]_{ij}$
\item the set $[z_{1}^{(k)},\cdots  z_{N_{eff}^{(k)}}^{(k)}]$ 
\item the interaction strenghts $U_{k}$
\item the numbers  $\alpha_{k}$.
\end{itemize}

\mycomment{Be more specific here what really has to specified in the actual code.}
The same logic also applies to the implementation of the hopping interaction \mycomment{be more specific}.

\subsection{The Hubbard-Stratonovich decomposition} 
Consider a single-particle (in other words bilinear) operator $O_{i}$.
One obtains an approximation to the evolution operator by the following series expansion \cite{AssaadBook08}
\begin{equation}
\label{eqn_2_HS}
e^{-\Delta\tau O^{2}_{i} } = \sum\limits_{s=\pm1,\pm2} \gamma(s) e^{i \sqrt{\Delta\tau}\eta(s)O_{i}} + \mathcal{O}(\Delta\tau^{4})\;,
\end{equation}
with 
%
\begin{eqnarray}
\gamma(\pm 1) = (1+\sqrt{6}/3)/4\;,\;\gamma(\pm 2) = (1-\sqrt{6}/3)/4\;,\nonumber\\
\eta(\pm 1) =\pm \sqrt{2(3-\sqrt{6})}\;,\;\eta(\pm 2) =\pm \sqrt{2(3+\sqrt{6})}\;.
\end{eqnarray}
%
Eq.~(\ref{eqn_2_HS}) can be easily proven by expanding its right hand side  to eighth order in $O_{i}$. 
The transformation introduces therefore two Ising fields $s$ per lattice site $i$, taking the values $\pm 1$ and $\pm 2$.
\mycomment{same label as the flavor index}
