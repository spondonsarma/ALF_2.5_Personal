
% !TEX root = Doc.tex
\section{Definition of the model Hamiltonian}

\mycomment{Notation: Hats for second quantized operators, bold for matrices. \\
  Structure:  \\ 1) We first want to define the model.  \\ 
                      2)  implementation of the QMC.  \\ 
                   3) Data structure  \\ 
                   4) Practical implementation and some  simple test cases. }

 
The class of solvable models includes  Hamiltonians $\mathcal{H}$ having the following general form:
${\mathcal{H}=\mathcal{H}_{T}+\mathcal{H}_{V}} +  \mathcal{H}_{I} +   \mathcal{H}_{0,I}  $, where
\begin{eqnarray}
\label{eqn_general_ham2}
\mathcal{H}_{T}
&=&
\sum\limits_{k=1}^{M_T}
\sum\limits_{s=1}^{N_{fl}}
\sum\limits_{\sigma=1}^{N_{col}}
\sum\limits_{x,y}
c^{\dagger}_{x \sigma   s}T_{xy}^{(k s)} c^{\phantom\dagger}_{y \sigma s}  \equiv  \sum\limits_{k=1}^{M_T} \hat{T}^{(k)}\\
\mathcal{H}_{V}
&=&
-
\sum\limits_{k=1}^{M_V}U_{k}
\left\{
\sum\limits_{s=1}^{N_{fl}}
\sum\limits_{\sigma=1}^{N_{col}}
\left[
\left(
\sum\limits_{x,y}
c^{\dagger}_{x \sigma s}V_{xy}^{(k s)}c^{\phantom\dagger}_{y \sigma s}
\right)
-\alpha_{k s} 
\right]
\right\}^{2}  \equiv   -
\sum\limits_{k=1}^{M_V}U_{k}   \left(\hat{V}{(k)} \right)^2\\ 
\mathcal{H}_{I}  & = &
\sum\limits_{k=1}^{M_I} \hat{\sigma}^{z}_{k} 
\left\{
\sum\limits_{s=1}^{N_{fl}}
\sum\limits_{\sigma=1}^{N_{col}}
\left[
\sum\limits_{x,y}
c^{\dagger}_{x \sigma s} I_{xy}^{(k s)}c^{\phantom\dagger}_{y \sigma s}
\right]
\right\} \equiv \sum\limits_{k=1}^{M_I} \hat{\sigma}^{z}_{k}    \hat{I}^{(k)} 
\;.
\end{eqnarray}
The indices have the following meaning:
\begin{itemize}
\item The number of fermion \textit{flavors} is set by $N_{fl}$.  After the Hubbard Stratonovitch transformation, the action will be block diagonal in the flavor index. 
\item The number of fermion \textit{colors} is set by $N_{col}$.    The Hamiltonian is invariant under  SU($N_{col}$)  rotations. Note that  In the code $ N_{col} \equiv NSUN $. 
%\mycomment{ Does it set the symmetry group of the fermions, namely 
%the dimension of the special unitary group $SU(N_{sun})$?}
\item The indices $x,y$ label lattice sites where $x,y=1,\cdots N_{dim}$. 
$N_{dim}$ is the total number of spacial vertices: $N_{dim}=N_{unit\;cell} N_{orbital}$, where $N_{unit\;cell}$ is the number of unit cells of the underlying Bravais lattice and $N_{orbital}$ is the number of (spacial) orbitals per unit cell \mycomment{Check the definition of $N_{orbital}$ in the code.} 
\item Therefore, the  matrices $\bm{T}^{(k s)}$, $\bm{V}^{(ks)}$  and $\bm{I}^{(ks)}$ are  of dimension $N_{dim}\times N_{dim}$
\item The number of interaction terms  is labelled by $M_V$   and $M_I$.   $M_T> 1 $ would allow for a checkerboard decomposition. 
\item $\hat{\sigma}^z_k$ is an Ising variable which couples to a general one-body term. 
\item  $\mathcal{H}_{0,I}$  gives the dynamics of the Ising variable. 
%\mycomment{Be more general here and speak of correlated blocks?}
\end{itemize}
Note that the matrices  $\bm{T}^{(s)}$ and $\bm{V}^{(ks)}$ explicitly depend on the flavor index $s$ but not on the color index $\sigma$. 
The color index $\sigma$ only appears only in  the  second quantized operators such that the Hamiltonian is manifestly SU($N_{col}$)    symmetric.


\subsection{Formulation of the QMC}  
The formulation of the  Monte Carlo simulation is based on the following.
\begin{itemize}
\item  We will work in  a basis  where  $\hat{\sigma}^z_k$ is diagonal. 
\item  We will discretize the imaginary time propagation: $\beta = \Delta \tau L_{\text{Trotter}} $
\item  We will the   discrete Hubbard Stratonovitch transformation: 
\begin{equation}
\label{HS_squares}
        e^{\Delta \tau  \lambda  A^2 } =
        \sum_{ l = \pm 1, \pm 2}  \gamma(l)
e^{ \sqrt{\Delta \tau \lambda }
       \eta(l)  O }
                + {\cal O} (\Delta \tau ^4)
\end{equation}
where the fields $\eta$ and $\gamma$ take the values:
\begin{eqnarray}
 \gamma(\pm 1) = 1 + \sqrt{6}/3, \; \; \gamma(\pm 2) = 1 - \sqrt{6}/3
\nonumber \\
 \eta(\pm 1 ) = \pm \sqrt{2 \left(3 - \sqrt{6} \right)},  \; \;
 \eta(\pm 2 ) = \pm \sqrt{2 \left(3 + \sqrt{6} \right)}.
\nonumber
\end{eqnarray}
\item From the above it follows that   the Monte Carlo configuration space is given by: 
\begin{equation}
	C = \left\{   s_{i,\tau} ,  l_{j,\tau}  \text{ with }  i=1\cdots M_I,  j = 1\cdots M_V,  \tau=1,L_{Trotter}  \right\}
\end{equation}
\end{itemize}

With the above, the partition function of the model can be written as follows.
\begin{eqnarray}
Z = \text {Tr}   e^{-\beta \mathcal{H} } =   \text {Tr}  \left[ e^{-\Delta \tau \mathcal{H}_{0,I}}   \prod_{k=1}^{M_T}   e^{-\Delta \tau \hat{T}^{(k)}}  
    \prod_{k=1}^{M_V}   e^{  \Delta \tau  U_k \left(  \hat{V}^{(k)} \right)^2}   \prod_{k=1}^{M_I}   e^{  -\Delta \tau  \hat{\sigma}_{k}  \hat{I}^{(k)}} 
   \right]^{L_{\text{Trotter}}}  \nonumber \\
   \sum_{C} \left( \prod_{j=1}^{M_V} \prod_{\tau=1}^{L_{Trotter}} \gamma_{j,\tau} \right) e^{-S_{0,I} \left( \left\{ s_{i,\tau} \right\}  \right) } 
    \text {Tr}_{F}   \prod_{\tau=1}^{L_{Trotter}} \left[   \prod_{k=1}^{M_T}   e^{-\Delta \tau \hat{T}^{(k)}}  
    \prod_{k=1}^{M_V}   e^{  \sqrt{ \Delta \tau  U_k} \eta_{k,\tau} \hat{V}^{(k)} }   \prod_{k=1}^{M_I}   e^{  -\Delta \tau s_{k,\tau}  \hat{I}^{(k)}}  \right] \nonumber
\end{eqnarray}
In the above,  $\text {Tr} $  runs over the Ising spins as well as over the fermionic degrees of freedom, and $ \text {Tr}_{F}  $ only over the  fermionc Fock space. 
$S_{0,I} \left( \left\{ s_{i,\tau} \right\}  \right)  $ is the action  corresponding to the Ising Hamiltonian,  and is only dependent on the Ising spins so that  it can be pulled out of the fermionic trace.
At this point,  and  since for a given configuration $C$  we are dealing with a free propagation, we can integrate out the fermions to obtain a determinant: 
\begin{equation}
 \text {Tr}_{F}   \prod_{\tau=1}^{L_{Trotter}} \left[   \prod_{k=1}^{M_T}   e^{-\Delta \tau \hat{T}^{(k)}}  
    \prod_{k=1}^{M_V}   e^{  \sqrt{ \Delta \tau  U_k} \eta_{k,\tau} \hat{V}^{(k)} }   \prod_{k=1}^{M_I}   e^{  -\Delta \tau s_{k,\tau}  \hat{I}^{(k)}}  \right]  = 
\end{equation}

 We will first  simplify the notation as: 





\begin{itemize}
\item in the coupling $g$ in the \texttt{Operator} structure (see Sec.~\ref{}).
\item as normalization constant in the definition of observables (see Sec.~\ref{})
\item as exponent in the calculation of the phase factor and the Monte Carlo update ratio.
\end{itemize}



\subsection{Structure of the hopping matrix  ${\bf T}$ and the interaction matrices ${\bf V}^{(k)}$}

In general, the matrices ${\bf T}^{(s)}$ and ${\bf V}^{(ks)}$ are sparse matrices. 
This property is used to minimize computational cost and storage.
In the following, we discuss the implementation of the single-particle matrix representation ${\bf V}^{(ks)}$ of the interaction operator. 
The same logic applies for the hopping matrix ${\bf T}^{(s)}$.

We denote a subset of $N_{eff}$ \mycomment{(in the code, $N_{eff}$ is called just $N$)} degrees of freedom \mycomment{here: sites} by the set  $[z_{1},\cdots  z_{N_{eff}}]$ and define it to contain only vertices for which an interaction term is defined:
\begin{equation}
V^{(ks)}_{x y}\neq 0\quad \text{only if} \quad x,y \in [z_{1}^{(ks)},\cdots  z_{N_{eff}^{(ks)}}^{(ks)}]\;.
\end{equation}
We define the projection matrices $\mathbf{P}^{(ks)}_{V}$ of dimension $N_{eff}^{(ks)}\times N_{dim}$:
\begin{equation}
(P_{V}^{(ks)})_{i,z}=\delta_{z_{i}^{(ks)},z}\;,
\end{equation}
where $i\in [1,\cdots N_{eff}^{(ks)}]$ and $z\in [1,\cdots N_{dim}]$. The matrix operator $\bm{P}^{(ks)}_{V}$ picks out the non-vanishing entries of $\bm{V}^{(ks)}$, 
which are contained in the rank-$N_{eff}^{(ks)}$  matrix $\bm{O}_{V}^{(ks)}$:
\begin{equation}
\bm{V}^{(ks)}=\bm{P}^{(ks) T}_{V} \bm{O}_{V}^{(ks)}\bm{P}^{(ks)}_{V}\;,
\end{equation}
and
\begin{equation}
V_{xy}^{(ks)}=(P^{(ks)}_{V})_{ix} \left[O_{V}^{(ks)}\right]_{ij}(P_{V}^{(ks)})_{jy}=\sum\limits_{i,j}^{N_{eff}^{(ks)}} \delta_{z_{i}^{(ks)},x}  \left[O_{V}^{(ks)}\right]_{ij} \delta_{z_{j}^{(ks)},y} \;.
\end{equation}
\mycomment{Comment that the P matrices have only one non-vanishing entry per column.}
To set the  interaction part, we therefore have to specify the following:
\begin{itemize}
\item the matrix elements $\left[O_{V}^{(k)}\right]_{ij}$
\item the set $[z_{1}^{(k)},\cdots  z_{N_{eff}^{(k)}}^{(k)}]$ 
\item the interaction strenghts $U_{k}$
\item the numbers  $\alpha_{k}$.
\end{itemize}

\mycomment{Be more specific here what really has to specified in the actual code.}
The same logic also applies to the implementation of the hopping interaction \mycomment{be more specific}.

\subsection{The Hubbard-Stratonovich decomposition} 
Consider a single-particle (in other words bilinear) operator $O_{i}$.
One obtains an approximation to the evolution operator by the following series expansion \cite{AssaadBook08}
\begin{equation}
\label{eqn_2_HS}
e^{-\Delta\tau O^{2}_{i} } = \sum\limits_{s=\pm1,\pm2} \gamma(s) e^{i \sqrt{\Delta\tau}\eta(s)O_{i}} + \mathcal{O}(\Delta\tau^{4})\;,
\end{equation}
with 
%
\begin{eqnarray}
\gamma(\pm 1) = (1+\sqrt{6}/3)/4\;,\;\gamma(\pm 2) = (1-\sqrt{6}/3)/4\;,\nonumber\\
\eta(\pm 1) =\pm \sqrt{2(3-\sqrt{6})}\;,\;\eta(\pm 2) =\pm \sqrt{2(3+\sqrt{6})}\;.
\end{eqnarray}
%
Eq.~(\ref{eqn_2_HS}) can be easily proven by expanding its right hand side  to eighth order in $O_{i}$. 
The transformation introduces therefore two Ising fields $s$ per lattice site $i$, taking the values $\pm 1$ and $\pm 2$.
\mycomment{same label as the flavor index}
