% Copyright (c) 2016 The ALF project.
% This is a part of the ALF project documentation.
% The ALF project documentation by the ALF contributors is licensed
% under a Creative Commons Attribution-ShareAlike 4.0 International License.
% For the licensing details of the documentation see license.CCBYSA

% !TEX root = doc.tex
%------------------------------------------------------------
\subsection{File structure}\label{sec:files}
%------------------------------------------------------------
%
\begin{table}[h]
	\begin{center}
	\begin{tabular}{@{} l l @{}}\toprule
   	Directory                             & Description \\\midrule
   	\path{Prog/}                          & Main program and subroutines  \\
   	\path{Libraries/}                     & Collection of mathematical routines \\  
  	\path{Analysis/}                      & Routines for error analysis \\
  	\path{Scripts_and_Parameters_files/}  & Helper scripts and the \path{Start/} directory, which contains \\ 
  	                                      & the files required to start a run \\
  	\path{Documentation/}                 & This documentation\\
  	\path{testsuite/}                     & A suite for automatic testing various parts of the code\\ \bottomrule
  	\hline
	\end{tabular}
   	\caption{Overview of the directories included in the ALF package.\label{table:files}}
   \end{center}
\end{table}
%

The code package, summarized in Table~\ref{table:files}, consists of the program directories \path{Prog/}, \path{Libraries/}, and \path{Analysis/}, as well as the directory \path{Scripts_and_Parameters_files/}, which contains supporting scripts and, in its subdirectory \path{Start}, the input files necessary for a run, described in the Sec.~\ref{sec:input}.
The routines available in the directory \path{Analysis/} are described in Sec.~\ref{sec:analysis}, and the testsuite in Sec.~\ref{sec:compilation}. 

Below we describe the structure of the input and output files of the QMC. Notice that the input/output files for the Analysis routines are described in Sec.~\ref{sec:analysis}.

%------------------------------------------------------------
\subsubsection{Input files}\label{sec:input}
%------------------------------------------------------------
%

The input files are listed in Table~\ref{table:input}. 
The parameter file \path{Start/parameters} has the following form --
using as an example the Hubbard model on a square lattice (see Sec.~\ref{sec:hubbard} for the general SU(N) Hubbard and Sec.~\ref{sec:vanilla} for a detailed walk-through on its plain vanilla version):
%
\begin{lstlisting}[style=fortran,escapechar=\#,breaklines=true]
!=======================================================================================
!  Input variables for a general ALF run
!---------------------------------------------------------------------------------------

&VAR_lattice               !! Parameters defining the specific lattice and base model
L1           = 6            ! Length in direction a_1
L2           = 6            ! Length in direction a_2
Lattice_type = "Square"     ! Sets a_1 = (1,0), a_2=(0,1), Norb=1, N_coord=2
Model        = "Hubbard"    ! Sets the Hubbard model, to be specified in &VAR_Hubbard
/

&VAR_Model_Generic         !! Common model parameters
Checkerboard = .T.          ! Whether checkerboard decomposition is used
Symm         = .T.          ! Whether symmetrization takes place
N_SUN        = 2            ! Number of colors
N_FL         = 1            ! Number of flavors
Phi_X        = 0.d0         ! Twist along the L_1 direction, in units of the flux quanta
Phi_Y        = 0.d0         ! Twist along the L_2 direction, in units of the flux quanta
Bulk         = .T.          ! Twist as a vector potential (.T.), or at the boundary (.F.)
N_Phi        = 0            ! Total number of flux quanta traversing the lattice
Dtau         = 0.1d0        ! Thereby Ltrot=Beta/dtau
Beta         = 5.d0         ! Inverse temperature
Projector    = .F.          ! Whether the projective algorithm is used
Theta        = 10.d0        ! Projection parameter
/

&VAR_QMC                   !! Variables for the QMC run
Nwrap               = 10    ! Stabilization. Green functions will be computed from 
                            ! scratch after each time interval Nwrap*Dtau
NSweep              = 20    ! Number of sweeps
NBin                = 5     ! Number of bins
Ltau                = 1     ! 1 to calculate time-displaced Green functions; 0 otherwise
LOBS_ST             = 0     ! Start measurements at time slice LOBS_ST
LOBS_EN             = 0     ! End measurements at time slice LOBS_EN
CPU_MAX             = 0.0   ! Code stops after CPU_MAX hours, if 0 or not
                            ! specified, the code stops after Nbin bins
Propose_S0          = .F.   ! Proposes single spin flip moves with probability exp(-S0) 
Global_moves        = .F.   ! Allows for global moves in space and time 
N_Global            = 1     ! Number of global moves per sweep 
Global_tau_moves    = .F.   ! Allows for global moves on a single time slice.  
N_Global_tau        = 1     ! Number of global moves that will be carried out on a 
                            ! single time slice
Nt_sequential_start = 0     ! One can combine sequential and global moves on a time slice
Nt_sequential_end   = -1    ! The program then carries out sequential local moves in the
                            ! range [Nt_sequential_start, Nt_sequential_end] followed by
                            ! N_Global_tau global moves
/

&VAR_errors                !! Variables for analysis programs
n_skip  = 1                 ! Number of bins that to be skipped.
N_rebin = 1                 ! Rebinning  
N_Cov   = 0                 ! If set to 1 covariance computed for non-equal-time
                            ! correlation functions
N_auto   = 0                ! If > 0  triggers  calculation of autocorrelation 
N_Back   = 1                ! If set to 1, substract background in correlation functions
/  

&VAR_TEMP                  !! Variables for parallel tempering
N_exchange_steps      = 6   ! Number of exchange moves #[see Eq.~\eqref{eq:exchangestep}]#
N_Tempering_frequency = 10  ! The frequency in units of sweeps at which the
                            ! exchange moves are carried out
mpi_per_parameter_set = 2   ! Number of mpi-processes per parameter set
Tempering_calc_det    = .T. ! Specifies whether the fermion weight has to be taken
                            ! into account while tempering. The default is .true.,
                            ! and it can be set to .F. if the parameters that
                            ! get varied only enter the Ising action S_0
/

&VAR_Max_Stoch             !! Variables for Stochastic Maximum entropy
Ngamma     = 400            ! Number of Dirac delta-functions for parametrization
Om_st      = -10.d0         ! Frequency range lower bound
Om_en      = 10.d0          ! Frequency range upper bound
NDis       = 2000           ! Number of boxes for histogram
Nbins      = 250            ! Number of bins for Monte Carlo
Nsweeps    = 70             ! Number of sweeps per bin
NWarm      = 20             ! The Nwarm first bins will be ommitted
N_alpha    = 14             ! Number of temperatures
alpha_st   = 1.d0           ! Smallest inverse temperature increment for inverse
R          = 1.2d0          ! temperature (see above) 
Checkpoint = .F.            ! Whether to produce dump files, allowing the simulation
                            ! to be resumed later on
Tolerance  = 0.1d0          ! Data points for which the relative error exceeds the
                            ! tolerance threshold will be omitted.
/

&VAR_Hubbard               !! Variables for the specific model
Mz        = .T.             ! When true, sets the M_z-Hubbard model: Nf=2, N_sun=1, HS field
                            ! couples to the z-component of magnetization; otherwise, HS field
                            ! couples to the density
ham_T     = 1.d0            ! Hopping parameter
ham_chem  = 0.d0            ! Chemical potential
ham_U     = 4.d0            ! Hubbard interaction
ham_T2    = 1.d0            ! For bilayer systems
ham_U2    = 4.d0            ! For bilayer systems
ham_Tperp = 1.d0            ! For bilayer systems
/
               
\end{lstlisting}
%
%!Model = "Hubbard_Mz"     ! Sets Nf=2, N_sun=1. HS field couples to the 
%! z-component of magnetization.  
%!Model="Hubbard_SU2_Ising"! Sets Nf_1, N_sun=2 and runs only for the square lattice
%! Hubbard model coupled to transverse Ising field
%&VAR_Ising                ! Model parameters for the Ising code
%Ham_xi = 1.d0             ! Only needed if Model="Hubbard_SU2_Ising"
%Ham_J  = 0.2d0
%Ham_h  = 2.d0
%/


\begin{table}[h]
	\begin{center}
	\begin{tabular}{@{} l l @{}}\toprule
		File & Description \\\midrule
		\path{parameters} &  Sets the parameters for lattice, model, QMC process, and the error analysis.\\
		\path{seeds} & List of integer numbers to initialize the random number generator and \\
		& to start a simulation from scratch.
		%\\
		%  \path{confin_<thread number>} & Input files for the HS and Ising configuration, used to continue a simulation.
		\\\bottomrule
	\end{tabular}
	\caption{
	%\red{[To be removed, at least after the files seeds is no longer necessary.]} 
	Overview of the input files required for a simulation, which can be found in the subdirectory \texttt{Scripts\_and\_Parameters\_files/Start/}. \label{table:input}}
\end{center}
\end{table}
%
\FloatBarrier

The program allows for a number of different  updating schemes.  If no other variables are specified in the \texttt{VAR\_QMC} name space, then the program will run in its default mode, namely the sequential single spin-flip mode.
%The additional, optional variables in   \texttt{VAR\_QMC}   include the following: 
%\begin{lstlisting}[style=fortran]
%&VAR_QMC                 ! Variables for the QMC run 
%Propose_S0      = .true. ! Proposes single spin flip moves with probability exp(-S0) 
%Global_moves    = .true. ! Allows for global moves in space and time 
%N_Global        = 1      ! Number of global moves  per sweep 
%Global_tau_moves= .true. ! Allows for global moves on a single time slice.  
%N_Global_tau    = 10     ! Number of global moves that will be carried out on a 
%                         ! single time slice
%Nt_sequential_start = 1  ! One can combine sequential and global moves on 
%                         ! a time slice.  
%Nt_sequential_end =      ! The program will carry our sequential local moves in the
%                         ! range [Nt_sequential_start, Nt_sequential_end] and then
%                         ! N_Global_tau global moves
%/   
%\end{lstlisting}
In particular, note that if \texttt{Nt\_sequential\_start}  and \texttt{Nt\_sequential\_end}  are not specified and that the variable \texttt{Global\_tau\_moves}  is set to true, then  the program will  carry out only global moves, by setting \texttt{Nt\_sequential\_start=1}  and \texttt{Nt\_sequential\_end=0}. 

If the program is not compiled with the parallel tempering flag, then the \texttt{VAR\_TEMP} name space can be omitted from the parameter file.
%\begin{lstlisting}[style=fortran,escapechar=\#]
%&VAR_TEMP                      ! Variables for parallel tempering
%N_exchange_steps      = 6      ! Number of exchange moves #[see Eq.~\eqref{eq:exchangestep}]#
%N_Tempering_frequency = 10     ! The frequency in units of sweeps at which the
%                               ! exchange moves will be carried 
%mpi_per_parameter_set = 2      ! Number of mpi-processes per parameter set
%Tempering_calc_det    = .true. ! Specifies whether the fermion weight has to be taken
%                               ! into account while tempering. The default is .true.,
%                               ! and it can be set to .false. if the parameters that
%                               ! get varied only enter the Ising action S_0
%/
%\end{lstlisting}

%Additionally, in order for the maximum entropy code, described in Sec.~\ref{sec:maxent}, to be used, the namelist \texttt{VAR\_Max\_Stoch} should also be defined:
%\begin{lstlisting}[style=fortran]
%&VAR_Max_Stoch               ! Variables for Stochastic Maximum entropy
%Ngamma     = 400             ! # of Dirac delta-functions for parametrization
%Om_st      = 0               ! Frequency range lower bound
%Om_en      = 8               ! Frequency range upper bound
%NDis       = 2000            ! # of boxes for histogram
%Nbins      = 250             ! # of bins for Monte Carlo
%Nsweeps    = 70              ! # of sweeps per bin
%NWarm      = 20              ! The Nwarm first bins will be ommitted
%N_alpha    = 14              ! # of temperatures
%alpha_st   = 1.d0            ! smallest inverse temperature
%R          = 1.2d0           ! increment for inverse temperature (see above) 
%Checkpoint = .false.         !.true.    : dump files will be produced so as to be able
%                             !            to restart the simulation
%                             !.false.   : dump files will not be produced 
%Tolerance  = 0.1d0           ! Data points for which the relative error exceeds the
%                             ! tolerance threshold will be omitted.
%/
%\end{lstlisting}


%------------------------------------------------------------
\subsubsection{Output files -- observables} \label{sec:output_obs}
%------------------------------------------------------------
%
\begin{table}[h]
	\begin{center}
   \begin{tabular}{@{} p{0.26\columnwidth}p{0.7\columnwidth} @{}}\toprule
   File               & Description \\\midrule
   \path{info}        & After completion of the simulation, this file documents the parameters of the model, as well as the QMC run and simulation metrics (precision, acceptance rate, wallclock time)\\
   \path{X_scal}      & Results of equal-time measurements of scalar observables \\
   & The placeholder \path{X} stands for the observables \path{Kin}, \path{Pot}, \path{Part}, and \path{Ener} \\
   \path{Y_eq, Y_tau} & Results of equal-time and time-displaced measurements of correlation functions. The placeholder \path{Y} stands for \path{Green}, \path{SpinZ}, \path{SpinXY}, \path{Den}, etc. \\   
   \path{Y_eq_info, Y_tau_info} & Additional info, like Bravais lattice and unit cell, for equal-time and time-displaced observable \\
   \path{confout_<thread number>} & Output files (one per MPI instance) for the HS and Ising configuration \\\bottomrule
   \end{tabular}
   \caption{Overview of the standard output files. See Sec.~\ref{sec:obs} for the definitions of observables and correlation functions. \label{table:output}}
\end{center}
\end{table}
%
The standard output files are listed in Table~\ref{table:output}. 
The output of the measured data is organized in bins. One bin corresponds to the arithmetic average 
over a fixed number of individual measurements which depends 
on the chosen measurement interval \path{[LOBS_ST,LOBS_EN]} on the imaginary-time axis and on the number \path{NSweep} of Monte Carlo sweeps. If the user runs an MPI parallelized version of the code, the average also extends over the number of MPI threads. The formatting of a single bin's output depends on the observable type, \path{Obs_vec} or \path{Obs_Latt}:
\begin{itemize}
\item Observables of type \path{Obs_vec}:
For each additional bin, a single new line is added to the output file.
In case of an observable with \path{N_size} components, the formatting is 
\begin{verbatim}
N_size + 1    <measured value, 1> ... <measured value, N_size>    <measured sign>
\end{verbatim}
The counter variable \path{N_size+1} refers to the number of measurements per line, including the phase measurement. 
This format is required by the error analysis routine (see Sec.~\ref{sec:analysis}). 
Scalar observables like kinetic energy, potential energy, total energy and particle number are treated as a vector 
of size \path{N_size=1}.

\item Observables of type \path{Obs_Latt}:
For each additional bin, a new data block is added to the output file. 
The block consists of the expectation values [Eq.~(\ref{eqn:o})] contributing to the background part [Eq.~(\ref{eqn:s_back})] of the correlation function,
and the correlated part [Eq.~(\ref{eqn:s_corr})] of the correlation function.
For imaginary-time displaced correlation functions, the formatting of the block is given by:
\begin{alltt}
<measured sign>  <N_orbital>  <N_unit_cell>  <N_time_slices>  <dtau>  <Channel>
do alpha = 1, N_orbital
    \(\langle\hat{O}\sb{\alpha}\rangle \)
enddo
do i = 1, N_unit_cell
   <reciprocal lattice vector k(i)>
   do tau = 1, N_time_slices
      do alpha = 1, N_orbital
         do beta = 1, N_orbital
            \(\langle{S}\sb{\alpha,\beta}\sp{(\mathrm{corr})}(k(i),\tau)\rangle\)
         enddo
      enddo
   enddo
enddo
\end{alltt}
The same block structure is used for equal-time correlation functions, except for the entries  \path{<N_time_slices>}, \path{<dtau>} and \path{<Channel>}, which are then omitted.
Using this structure for the bins as input, the full correlation function $S_{\alpha,\beta}(\vec{k},\tau)$ [Eq.~(\ref{eqn:s})] is then calculated by calling the error analysis routine (see Sec.~\ref{sec:analysis}).
\end{itemize}


%------------------------------------------------------------
%\subsection{Scripts}\label{sec:scripts}
%------------------------------------------------------------
%

%\begin{table}[h]
%   \begin{tabular}{@{} l l l @{}}\toprule
%   Script & Description & Section\\\midrule
%   \path{Start/analysis.sh} & Starts the error analysis. & \ref{sec:analysis}\\
%   \path{Start/out_to_in.sh} & Copies outputted field configurations into input files for further runs. & \ref{sec:running} \\\bottomrule
%   \end{tabular}
%   \caption{Overview of the bash script files and the corresponding reference sections. 
%      \label{table:scripts}}
%\end{table}
%
