% !TEX root = doc.tex
% Copyright (c) 2017-2020 The ALF project.
% This is a part of the ALF project documentation.
% The ALF project documentation by the ALF contributors is licensed
% under a Creative Commons Attribution-ShareAlike 4.0 International License.
% For the licensing details of the documentation see license.CCBYSA.
%
%-----------------------------------------------------------------------------------
\subsection{Predefined lattices} \label{sec:predefined_lattices}
%-----------------------------------------------------------------------------------


The types \texttt{Lattice} and \texttt{Unit\_cell}, described in Section~\ref{sec:latt}, allow us to define arbitrary one- and two-dimensional Bravais lattices. The subroutine \texttt{Predefined\_Latt} provides some of the most common lattices, as described bellow.

The subroutine is called as:
\begin{lstlisting}[style=fortran]
Predefined_Latt(Lattice_type, L1, L2, Ndim, List, Invlist, Latt, Latt_Unit)
\end{lstlisting}
which returns a lattice of size \texttt{L1$\times$L2} of the given \texttt{Lattice\_type}, as detailed in Table~\ref{table:predefined_lattices}.
%
\begin{center}
\begin{table}[h]
	\begin{tabular}{@{} p{0.13\columnwidth}  p{0.1\columnwidth} p{0.09\columnwidth} p{0.58\columnwidth}  @{}}
		\toprule
		Argument                 & Type       & Role   & Description \\
		\midrule
		\texttt{Lattice\_type}   & String     & Input  & lattice configuration, which can take the values:
		%\vspace{-\topsep}
		\begin{itemize}
			\setlength{\itemsep}{0pt} \setlength{\parskip}{0pt} \setlength{\parsep}{0pt}
			\item[-] Square
			\item[-] Honeycomb
			\item[-] Pi\_Flux  (Depreciated)
			\item[-] N\_leg\_ladder
			\item[-] Bilayer\_square
			\item[-] Bilayer\_honeycomb
			\vspace{-1.4\topsep} 
		\end{itemize} \\
	    %\vspace{-\topsep} \\ \vspace{-\topsep}
		\texttt{L1}, \texttt{L2} & Integer    & Input  & lattice sizes (set \texttt{L2=1} for 1D lattices)\\
		\texttt{Ndim}            & Integer    & Output & total number of orbitals\\
		\texttt{List}            & Integer    & Output & for every site index $\texttt{I} \in [1,\texttt{Ndim}]$, stores the corresponding lattice position, \texttt{List(I,1)}, and the (local) orbital index, \texttt{List(I,2)}\\
		\texttt{Invlist}         & Integer    & Output &  for every $\texttt{lattice\_position} \in [1,\texttt{Latt\%N}]$ and $\texttt{orbital} \in [1,\texttt{Norb}]$ stores the corresponding site index \texttt{I(lattice\_position,orbital)}\\
		\texttt{Latt}            & Lattice    & Output &  sets the lattice\\
		\texttt{Latt\_Unit}      & Unit\_cell & Output & sets the unit cell\\
		\bottomrule
	\end{tabular}
\caption{Arguments of the subroutine \texttt{Predefined\_Latt}.   Note that the Pi\_Flux    lattice  is depreciated for the moment since it can be emulated with the Square lattice with half a flux quanta piercing each plaquette. }		\label{table:predefined_lattices}
\end{table}
\end{center}





