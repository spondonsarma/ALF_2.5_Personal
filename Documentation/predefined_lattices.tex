% !TEX root = doc.tex
% Copyright (c) 2017-2020 The ALF project.
% This is a part of the ALF project documentation.
% The ALF project documentation by the ALF contributors is licensed
% under a Creative Commons Attribution-ShareAlike 4.0 International License.
% For the licensing details of the documentation see license.CCBYSA.
%
%-----------------------------------------------------------------------------------
\subsection{Predefined lattices} \label{sec:predefined_lattices}
%-----------------------------------------------------------------------------------


The types \texttt{Lattice} and \texttt{Unit\_cell}, described in Section~\ref{sec:latt}, allow us to define arbitrary one- and two-dimensional Bravais lattices. The subroutine \texttt{Predefined\_Latt} provides some of the most common lattices, namely Square, Honeycomb, and Pi Flux, as described bellow.

The subroutine is called as:
\begin{lstlisting}[style=fortran]
Predefined_Latt(Lattice_type, L1, L2, Ndim, List, Invlist, Latt, Latt_Unit)
\end{lstlisting}
taking 3 input variables:
\begin{itemize}
	\item \texttt{Lattice\_type}, which can take the values
	\begin{itemize}
		\item Square
		\item Honeycomb
		\item Pi\_Flux
	\end{itemize}
	\item \texttt{L1} and \texttt{L2} -- the lattice sizes (set \texttt{L2=1} for 1D lattices)
\end{itemize}
and returns:
\begin{itemize}
	\item \texttt{Ndim} -- total number of orbitals
	\item \texttt{List} -- for every site index $\texttt{I} \in [1,\texttt{Ndim}]$, stores the corresponding lattice position \texttt{List(I,1)} and the (local) orbital index \texttt{List(I,2)}
	\item \texttt{Invlist} -- for every $\textrm{"lattice position"} \in [1,\texttt{Latt\%N}]$ and $\textrm{"orbital"} \in [1,\texttt{Norb}]$ stores the corresponding site index \texttt{I(\textrm{orbital},\textrm{orbital})}
	\item \texttt{Latt} -- sets the lattice
	\item \texttt{Latt\_Unit} -- sets the unit cell
\end{itemize}
