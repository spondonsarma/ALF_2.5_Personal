% !TEX root = doc.tex
% Copyright (c) 2017-2020 The ALF project.
% This is a part of the ALF project documentation.
% The ALF project documentation by the ALF contributors is licensed
% under a Creative Commons Attribution-ShareAlike 4.0 International License.
% For the licensing details of the documentation see license.CCBYSA.
%
%-----------------------------------------------------------------------------------
\subsection{Predefined lattices} \label{sec:predefined_lattices}
%-----------------------------------------------------------------------------------


The types \texttt{Lattice} and \texttt{Unit\_cell}, described in Section~\ref{sec:latt}, allow us to define arbitrary one- and two-dimensional Bravais lattices. The subroutine \texttt{Predefined\_Latt} provides some of the most common lattices, as described bellow.

The subroutine is called as:
\begin{lstlisting}[style=fortran]
Predefined_Latt(Lattice_type, L1, L2, Ndim, List, Invlist, Latt, Latt_Unit)
\end{lstlisting}
which returns a lattice of size \texttt{L1$\times$L2} of the given \texttt{Lattice\_type}, as detailed in Table~\ref{table:predefined_lattices}. Notice that the orbital position \texttt{Latt\_Unit\%Orb\_pos\_p(1,:)} is set to zero unless otherwise specified.
%
\begin{table}[h]
	\begin{center}
	\begin{tabular}{@{} p{0.13\columnwidth}  p{0.1\columnwidth} p{0.09\columnwidth} p{0.58\columnwidth}  @{}}
		\toprule
		Argument                 & Type       & Role   & Description \\
		\midrule
		\texttt{Lattice\_type}   & String     & Input  & lattice configuration, which can take the values:
		\vspace{-\topsep} %sometimes dispensable
		\begin{itemize}
			\setlength{\itemsep}{0pt} \setlength{\parskip}{0pt} \setlength{\parsep}{0pt}
			\item[-] \texttt{Square}
			\item[-] \texttt{Honeycomb}
			\item[-] \texttt{Pi\_Flux}  (deprecated)
			\item[-] \texttt{N\_leg\_ladder}
			\item[-] \texttt{Bilayer\_square}
			\item[-] \texttt{Bilayer\_honeycomb}
			\vspace{-1.4\topsep} 
		\end{itemize} \\
	    %\vspace{-\topsep} \\ \vspace{-\topsep}
		\texttt{L1}, \texttt{L2} & Integer    & Input  & lattice sizes (set \texttt{L2=1} for 1D lattices)\\
		\texttt{Ndim}            & Integer    & Output & total number of orbitals\\
		\texttt{List}            & Integer    & Output & for every site index $\texttt{I} \in [1,\texttt{Ndim}]$, stores the corresponding lattice position, \texttt{List(I,1)}, and the (local) orbital index, \texttt{List(I,2)}\\
		\texttt{Invlist}         & Integer    & Output &  for every $\texttt{lattice\_position} \in [1,\texttt{Latt\%N}]$ and $\texttt{orbital} \in [1,\texttt{Norb}]$ stores the corresponding site index \texttt{I(lattice\_position,orbital)}\\
		\texttt{Latt}            & Lattice    & Output &  sets the lattice\\
		\texttt{Latt\_Unit}      & Unit\_cell & Output & sets the unit cell\\
		\bottomrule
	\end{tabular}
\caption{Arguments of the subroutine \texttt{Predefined\_Latt}.   Note that the Pi\_Flux lattice is deprecated for the moment since it can be emulated with the Square lattice with half a flux quanta piercing each plaquette.}		\label{table:predefined_lattices}
\end{center}
\end{table}

In order to easily keep track of the orbital and unit cell, \texttt{List} and \texttt{Invlist} make use of a super-index, defined as shown below:
\begin{lstlisting}[style=fortran]
nc = 0                                  ! Super-index labeling unit cell and orbital
Do I = 1,Latt%N                         ! Unit-cell index 
   Do no = 1,Norb                       ! Orbital index
      nc = nc + 1
      List(nc,1) = I                    ! Unit-cell of super index nc
      List(nc,2) = no                   ! Orbital of super index nc
      Invlist(I,no) = nc                ! Super-index for given unit cell and orbital
   Enddo
Enddo
\end{lstlisting}
With the above lists one can run through all the orbitals and at each time keep track of the unit-cell and orbital index. We note that when translation symmetry is completely absent one can work with a single unit cell, and the number of orbitals will then correspond to the number of lattice sites. 

\subsubsection{Square lattice}

The choice \texttt{Lattice\_type = "Square"} sets $\vec{a}_1 =  (1,0) $ and $\vec{a}_2 =  (0,1) $  and for an $L_1 \times L_2$  lattice  $\vec{L}_1 = L_1 \vec{a}_1$ and  $\vec{L}_2 = L_2 \vec{a}_2$:
\begin{lstlisting}[style=fortran]
a1_p(1) =  1.0  ; a1_p(2) =  0.d0
a2_p(1) =  0.0  ; a2_p(2) =  1.d0
L1_p    =  dble(L1)*a1_p
L2_p    =  dble(L2)*a2_p
\end{lstlisting}
Also, the number of orbitals per unit cell is given by \texttt{NORB=1} such that   $N_{\mathrm{dim}}   \equiv N_{\text{unit-cell}}   \cdot \texttt{NORB}  = \texttt{Latt\%N} \cdot \texttt{NORB}$, since $N_{\text{unit-cell}} = \texttt{Latt\%N}$.

\subsubsection{Honeycomb lattice}

In order to carry out simulations on the Honeycomb lattice, which is a triangular Bravais lattice with two orbitals per unit cell, we choose \path{Lattice_type = "Honeycomb"}, which sets
\begin{lstlisting}[style=fortran]
Norb    = 2
N_coord = 3
a1_p(1) =  1.D0   ; a1_p(2) =  0.d0
a2_p(1) =  0.5D0  ; a2_p(2) =  sqrt(3.D0)/2.D0             
L1_p    =  dble(L1) * a1_p
L2_p    =  dble(L2) * a2_p
\end{lstlisting}
The coordination number of this lattice is \texttt{ N\_coord=3 }  and  the number of orbitals per unit cell, \texttt{NORB=2}. The total number of orbitals is therefore \texttt{$N_{\mathrm{dim}}$=Latt\%N*NORB}.


\subsubsection{$\pi$-Flux lattice (deprecated)}

The Pi\_Flux lattice has been deprecated, since it can be emulated with the Square lattice with half a flux quanta piercing each plaquette. Nonetheless, the configuration is still available, and sets:
\begin{lstlisting}[style=fortran]
Latt_Unit%Norb    = 2
Latt_Unit%N_coord = 4
a1_p(1) =  1.D0   ; a1_p(2) =   1.d0
a2_p(1) =  1.D0   ; a2_p(2) =  -1.d0
Latt_Unit%Orb_pos_p(1,:) = 0.d0 
Latt_Unit%Orb_pos_p(2,:) = (a1_p(:) - a2_p(:))/2.d0 
L1_p    =  dble(L1) * (a1_p - a2_p)/2.d0
L2_p    =  dble(L2) * (a1_p + a2_p)/2.d0
\end{lstlisting}


\subsubsection{$N$-leg Ladder lattice}
The "\texttt{N\_leg\_ladder}" configuration sets:
\begin{lstlisting}[style=fortran]
Latt_Unit%Norb     = L2
Latt_Unit%N_coord  = 1
do no = 1,L2
   Latt_Unit%Orb_pos_p(no,1) = 0.d0 
   Latt_Unit%Orb_pos_p(no,2) = real(no-1,kind(0.d0))
enddo
a1_p(1) =  1.0   ; a1_p(2) =  0.d0
a2_p(1) =  0.0   ; a2_p(2) =  1.d0
L1_p    =  dble(L1)*a1_p
L2_p    =           a2_p
\end{lstlisting}


\subsubsection{Bilayer Square lattice}
The "\texttt{Bilayer\_square}" configuration sets:
\begin{lstlisting}[style=fortran]
Latt_Unit%Norb     = 2
Latt_Unit%N_coord  = 2
do no = 1,2
   Latt_Unit%Orb_pos_p(no,1) = 0.d0 
   Latt_Unit%Orb_pos_p(no,2) = 0.d0 
   Latt_Unit%Orb_pos_p(no,3) = real(1-no,kind(0.d0))
enddo
a1_p(1) =  1.0  ; a1_p(2) =  0.d0
a2_p(1) =  0.0  ; a2_p(2) =  1.d0
L1_p    =  dble(L1)*a1_p
L2_p    =  dble(L2)*a2_p
\end{lstlisting}


\subsubsection{Bilayer Honeycomb lattice}
The "\texttt{Bilayer\_honeycomb}" configuration sets:
\begin{lstlisting}[style=fortran]
Latt_Unit%Norb     = 4
Latt_Unit%N_coord  = 3
Latt_unit%Orb_pos_p = 0.d0
do n = 1,2
   Latt_Unit%Orb_pos_p(1,n) = 0.d0 
   Latt_Unit%Orb_pos_p(2,n) = (a2_p(n) - 0.5D0*a1_p(n) ) * 2.D0/3.D0
   Latt_Unit%Orb_pos_p(3,n) = 0.d0 
   Latt_Unit%Orb_pos_p(4,n) = (a2_p(n) - 0.5D0*a1_p(n) ) * 2.D0/3.D0
enddo
Latt_Unit%Orb_pos_p(3,3) = -1.d0
Latt_Unit%Orb_pos_p(4,3) = -1.d0
a1_p(1) =  1.D0   ; a1_p(2) =  0.d0
a2_p(1) =  0.5D0  ; a2_p(2) =  sqrt(3.D0)/2.D0
L1_p    =  dble(L1)*a1_p
L2_p    =  dble(L2)*a2_p
\end{lstlisting}


