% Copyright (c) 2016 The ALF project.
% This is a part of the ALF project documentation.
% The ALF project documentation by the ALF contributors is licensed
% under a Creative Commons Attribution-ShareAlike 4.0 International License.
% For the licensing details of the documentation see license.CCBYSA.

% !TEX root = doc.tex


\red{TO BE DELETED, with parts being reused in sections Predefined Structures and Model classes.}


%------------------------------------------------------------------------------------- 
\subsection{The $SU(2)$-Hubbard model on a square lattice coupled to a transverse Ising field}\label{sec:walk2}
%-------------------------------------------------------------------------------------

The model we consider here  is very similar to the  above,  but has an additional coupling to a transverse field: 
\begin{equation}
\begin{aligned}
\label{eqn_hubbard_sun_Ising}
\mathcal{H} =&
\sum\limits_{\sigma=1}^{2} 
\sum\limits_{x,y } 
  c^{\dagger}_{x \sigma} T_{x,y}c^{\phantom\dagger}_{y \sigma} 
+ \frac{U}{2}\sum\limits_{x}\left[
\sum\limits_{\sigma=1}^{2}
\left(  c^{\dagger}_{x \sigma} c^{\phantom\dagger}_{x \sigma}  -1/2 \right) \right]^{2}   
+  \xi \sum_{\sigma,\langle x,y \rangle} \hat{Z}_{\langle x,y \rangle}  \left( c^{\dagger}_{x \sigma} c^{\phantom\dagger}_{y \sigma}  + h.c. \right) \\ 
 &- h \sum_{\langle x,y \rangle} \hat{X}_{\langle x,y \rangle}   - J \sum_{\langle \langle x,y \rangle \langle x',y' \rangle \rangle} 
  \hat{Z}_{\langle x,y \rangle}   \hat{Z}_{\langle x',y' \rangle} 
  \end{aligned}
\end{equation}
We can make contact with the general form of the Hamiltonian by setting: 
$N_{\mathrm{fl}} = 1$, $N_{\mathrm{col}} \equiv \texttt{N\_SUN}     =2 $,   $M_T    =    1$,  $T^{(ks)}_{x y}   =  T_{x,y}$,  $M_V   =  N_{\text{unit-cell}} \equiv N_{\mathrm{dim}}$,  $U_{k}       =   -\frac{U}{2}$, 
 $V_{x y}^{(ks)} =  \delta_{x,y} \delta_{x,k}$,  $\alpha_{ks}   = - \frac{1}{2}  $ and $M_I       = 2 N_{\text{unit-cell}} $.  
 The last two terms of the  above Hamiltonian describe a transverse Ising field model on the bonds of the square lattice.  This  type of Hamiltonian  has  recently been extensively discussed  \cite{Schattner15,Xu16,Assaad16}.  Here we adopt the notation of Ref.~\cite{Assaad16}. Note that   $\langle \langle x,y \rangle \langle x',y' \rangle \rangle $ denotes nearest neighbor bonds.
The modifications  required to generalize the Hubbard model code to the above model are two-fold. 

First,  one has to specify the function \path{Real (Kind=8) function S0(n,nt)}, and  second,  modify the interaction \texttt{Call Ham\_V}.

A sample run for this model can be found in \path{Examples/Hubbard_SU2_Ising_Square/}.


%-------------------------------------------------------------------------------------
\subsubsection{The Ising term.}
%-------------------------------------------------------------------------------------
Since the Ising field lives on bonds we have to provide a data structure defining this quantity.  A bond has an anchor site as well as an orientation. The routine \path{Setup_Ising_action}   initializes  the arrays \path{L_bond} and  \path{L_bond_inv} that contain this information.
\begin{lstlisting}[style=fortran]
nc = 0
Do n_orientation = 1,N_coord
Do I = 1, Latt%N
   nc = nc + 1
   L_bond(I,n_orientation) = nc
   L_bond_inv(nc,1) = I    
   L_bond_inv(nc,2) = n_orientation
enddo
enddo
\end{lstlisting}
The two legs of the bond are given by  the anchor $\vec{I}$ and $\vec{I}+ \vec{a}_{n_\text{orientation}}$
%-------------------------------------------------------------------------------------
\subsubsection{The interaction term: \texttt{Call Ham\_V}}
%-------------------------------------------------------------------------------------

The dimension of   \texttt{Op\_V}  is now  $(M_I + M_V)\times N_{\mathrm{fl}}=((N_\text{coord} +  1 )N_{\text{dim}}) \times 1$ since each site has $N_\text{coord} =2$ bonds   for the square lattice.
\begin{lstlisting}[style=fortran]
do i  = 1,N_coord*Ndim                     ! Runs over bonds for Ising interaction.
  call Op_make(Op_V(i,1),2)
enddo
do i  =  N_coord*Ndim+1, (N_coord+1)*Ndim  ! Runs over sites for Hubbard interaction.
  call Op_make(Op_V(i,1),1)
enddo
\end{lstlisting}
The  first  \texttt{N\_coord*Ndim} operators run through the $2N$ bonds of the square lattice and   are given by:
\begin{lstlisting}[style=fortran]
Do nc = 1,Ndim*N_coord                     ! Runs over bonds. Coordination number = 2.
                                           ! For the square lattice Ndim = Latt%N
  
   I1 = L_bond_inv(nc,1)                   ! Anchor of the bond
	                                   ! L_bond_inv is setup in Setup_Ising_action
   if ( L_bond_inv(nc,2)  == 1 ) I2 = Latt%nnlist(I1,1,0)  ! Second site of the bond 
   if ( L_bond_inv(nc,2)  == 2 ) I2 = Latt%nnlist(I1,0,1) 
   Op_V(nc,1)%P(1) = I1
   Op_V(nc,1)%P(2) = I2
   Op_V(nc,1)%O(1,2) = cmplx(1.d0 ,0.d0, kind(0.D0))
   Op_V(nc,1)%O(2,1) = cmplx(1.d0 ,0.d0, kind(0.D0))
   Op_V(nc,1)%g      = cmplx(-dtau*Ham_xi,0.D0,kind(0.D0))
   Op_V(nc,1)%alpha  = cmplx(0d0,0.d0, kind(0.D0))
   Op_V(nc,1)%type   = 1
Enddo
\end{lstlisting}
Here,  \texttt{ham\_xi} defines the coupling strength  between the Ising  and fermion degree of freedom.
As for the Hubbard case, the last \texttt{Ndim}  operators read: 
\begin{lstlisting}[style=fortran]
nc = N_coord*Ndim 
Do i = 1, Ndim
    nc = nc + 1
    Op_V(nc,1)%P(1)   = i 
    Op_V(nc,1)%O(1,1) = cmplx(1.d0  ,0.d0, kind(0.D0))
    Op_V(nc,1)%g      = sqrt(cmplx(-dtau*ham_U/(DBLE(N_SUN)), 0.D0, kind(0.D0)))
    Op_V(nc,1)%alpha  = cmplx(-0.5d0,0.d0, kind(0.D0))
    Op_V(nc,1)%type   = 2
Enddo
\end{lstlisting}

%-------------------------------------------------------------------------------------
\subsubsection{The function \texttt{Real (Kind=8) function S0(n,nt)} }\label{sec:s0}
%-------------------------------------------------------------------------------------

As mentioned above,  a configuration now includes both HS spins and Ising spins and is given by
\begin{equation}
	C = \left\{   s_{i,\tau} ,  l_{j,\tau},  \text{ with }  i=1,\cdots, M_I;\;  j = 1,\cdots, M_V;\;  \tau=1,\cdots,L_\text{Trotter}  \right\}\:.
\end{equation}
This configuration is stored in the  integer array \texttt{nsigma(M\_V + M\_I, Ltrot)}.  With the above ordering of Hubbard and Ising interaction terms, and a for a given imaginary time, the first \texttt{2*Ndim} fields correspond to the Ising interaction and the next \path{Ndim} ones to the Hubbard interaction.
The first   argument of the function \texttt{S0}, namely \texttt{n},  corresponds to the index of the operator  string 
\texttt{Op\_V(n,1)}. If \texttt{Op\_V(n,1)\%type = 2} then   \texttt{S0(n,nt)}  returns 1. Note that \texttt{type=2} refers to spins that stem from a  HS transformation. 
If however  \texttt{Op\_V(n,1)\%type = 1}  then function \texttt{S0}  returns
\begin{equation}
\frac{e^{-S_{0,I} \left(  s_{1,\tau},  \cdots,  - s_{n,\tau},  \cdots s_{M_I,\tau}   \right) } }{e^{-S_{0,I}  \left(  s_{1,\tau},  \cdots,   s_{n,\tau},  \cdots s_{M_I,\tau}   \right)   } }	
\end{equation}
That is,   if $n \leq 2* \texttt{Ndim} $,    $ \texttt{S0(n,nt)} $  returns the ratio of the new weight to the old weight  of the  Ising Hamiltonian upon flipping a single Ising spin $ s_{n,\tau} $.  Otherwise, $ \texttt{S0(n,nt)} $   returns unity. 

