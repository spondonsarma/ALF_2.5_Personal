% !TEX root = doc.tex
% Copyright (c) 2017-2020 The ALF project.
% This is a part of the ALF project documentation.
% The ALF project documentation by the ALF contributors is licensed
% under a Creative Commons Attribution-ShareAlike 4.0 International License.
% For the licensing details of the documentation see license.CCBYSA.
%
%-----------------------------------------------------------------------------------
\subsection{Predefined interaction vertices} \label{sec:interaction_vertices}
%-----------------------------------------------------------------------------------

In its most general form, an interaction Hamiltonian expressed in terms of sums of perfect squares can be written, as presented in Section~\ref{sec:intro}, as a sum of $M_V$ vertices: %Eq.~\eqref{eqn:general_ham_v}:

\begin{align*}
\hat{\mathcal{H}}_{V} &=  \sum\limits_{k=1}^{M_V}U_{k}
\left\{ \sum\limits_{\sigma=1}^{N_{\mathrm{col}}}
\sum\limits_{s=1}^{N_{\mathrm{fl}}} \left[ \left(
\sum\limits_{x,y}^{N_{\mathrm{dim}}} \hat{c}^{\dagger}_{x \sigma s}V_{xy}^{(k s)}\hat{c}^{\phantom\dagger}_{y \sigma s}\right)  +\alpha_{k s}  \right] \right\}^{2}
\equiv    \sum\limits_{k=1}^{M_V}U_{k}   \left(\hat{V}^{(k)} \right)^2 \tag{\ref{eqn:general_ham_v}}\\
&\equiv    \sum\limits_{k=1}^{M_V}\hat{\mathcal{H}}_V^{(k)},
\end{align*}
which are encoded in one or more variables of type \texttt{Operator}, described in Sec.~\ref{sec:op}. We often use arrays of \texttt{Operator} type, which should be initialized by repeatedly calling the subroutine \texttt{Op\_make}.

The module \texttt{Predefined\_Int\_mod.F90} implements some of the most common of such interaction vertices $\hat{\mathcal{H}}_V^{(k)}$, as detailed in the remaining of this section, where we drop the superscript $(k)$ when unambiguous.


\subsubsection{$SU(N)$ Hubbard interaction}

The $SU(N)$ Hubbard interaction on a given site $i$ is given by 
\begin{align}
%\label{eqn_hubbard_sun}
\hat{\mathcal{H}}_{V,i} =
+ \frac{U}{N_{\mathrm{col}}}\left[
\sum\limits_{\sigma=1}^{N_{\mathrm{col}}}
\left(  \hat{c}^{\dagger}_{i \sigma} \hat{c}^{\phantom\dagger}_{i\sigma}  -1/2 \right) \right]^{2}.
\end{align} 
Assuming that no other term in the Hamiltonian breaks the $SU(N) $ color symmetry, then this interaction term conveniently corresponds to  a single operator, obtained by calling, for each of the $N_{\mathrm{dim}}$ sites $i$:
\begin{lstlisting}[style=fortran]
Call Predefined_Int_U_SUN( OP, I, N_SUN, DTAU, U )
\end{lstlisting}
which defines:
%which corresponds to the general form of Eq.~\eqref{eqn:general_ham_v} by setting: 
%$N_{\mathrm{fl}} = 1$,  $M_V = N_{\text{unit-cell}} $,  $U_{k} =  -\frac{U}{N_{\mathrm{col}}}$,  $V_{x y}^{(ks)} =  \delta_{x,y} \delta_{x,k}$, and $\alpha_{ks} = -\frac{1}{2}$; and which is defined in the subroutine \texttt{Predefined\_Int\_U\_SUN} by a single operator:

\begin{lstlisting}[style=fortran]
Op%P(1)   = I
Op%O(1,1) = cmplx(1.d0,  0.d0, kind(0.D0))
Op%alpha  = cmplx(-0.5d0,0.d0, kind(0.D0))
Op%g      = SQRT(CMPLX(-DTAU*U/(DBLE(N_SUN)), 0.D0, kind(0.D0))) 
Op%type   = 2

\end{lstlisting}

To relate to  Eq.~\eqref{eqn:general_ham_v} we have,   $V_{x y}^{(is)} =  \delta_{x,y} \delta_{x,i}$, $\alpha_{is} = -\frac{1}{2}$ and $U_{k} =  \frac{U}{N_{\mathrm{col}}}$.   Here  the flavor index, $s$,  plays no role. 


\subsubsection{$M_z$-Hubbard interaction}

\begin{lstlisting}[style=fortran]
Call Predefined_Int_U_MZ( OP_up, Op_do, I, DTAU, U )
\end{lstlisting}

The $M_z$-Hubbard interaction is given by 
\begin{align}
%\label{eqn_hubbard_Mz}
\hat{\mathcal{H}}_{V} = - \frac{U}{2}\sum\limits_{i}\left[
\hat{c}^{\dagger}_{i \uparrow} \hat{c}^{\phantom\dagger}_{i \uparrow}  -   \hat{c}^{\dagger}_{i \downarrow} \hat{c}^{\phantom\dagger}_{i \downarrow}  \right]^{2},
\end{align} 
which corresponds to the general form of Eq.~\eqref{eqn:general_ham_v} by setting: 
$N_{\mathrm{fl}} = 2$, $N_{\mathrm{col}} \equiv \texttt{N\_SUN} =1 $,  $M_V =  N_{\text{unit-cell}} $,  $U_{k} = \frac{U}{2}$, 
$V_{x y}^{(i, s=1)} =  \delta_{x,y} \delta_{x,i}  $,  $V_{x y}^{(i, s=2)} =  - \delta_{x,y} \delta_{x,i}  $, and $\alpha_{is}   = 0  $; and which is defined in the subroutine \texttt{Predefined\_Int\_U\_MZ} by two operators:
\begin{lstlisting}[style=fortran]
Op_up%P(1)   = I
Op_up%O(1,1) = cmplx(1.d0, 0.d0, kind(0.D0))
Op_up%alpha  = cmplx(0.d0, 0.d0, kind(0.D0))
Op_up%g      = SQRT(CMPLX(DTAU*U/2.d0, 0.D0, kind(0.D0))) 
Op_up%type   = 2

Op_do%P(1)   = I
Op_do%O(1,1) = cmplx(1.d0, 0.d0, kind(0.D0))
Op_do%alpha  = cmplx(0.d0, 0.d0, kind(0.D0))
Op_do%g      = -SQRT(CMPLX(DTAU*U/2.d0, 0.D0, kind(0.D0))) 
Op_do%type   = 2

\end{lstlisting}


\subsubsection{$SU(N)$ $V$-interaction}

\begin{lstlisting}[style=fortran]
Call Predefined_Int_V_SUN( OP, I, J, N_SUN, DTAU, V )
\end{lstlisting}

The interaction term of the generalized t-V model, given by 
\begin{align}
\hat{\mathcal{H}}_{V,i,j} =
-\frac{V}{N_\mathrm{col}}\left[ \sum_{\sigma=1}^{N_\mathrm{col}}\left( \hat{c}^{\dagger}_{i \sigma} \hat{c}^{\phantom\dagger}_{j \sigma} + \hat{c}^{\dagger}_{j \sigma} \hat{c}^{\phantom\dagger}_{i \sigma} \right) \right]^2,
\end{align} 
is coded in the subroutine \texttt{Predefined\_Int\_V\_SUN} by a single symmetric operator:
\begin{lstlisting}[style=fortran]
Op%P(1)   = I
Op%P(2)   = J
Op%O(1,2) = cmplx(1.d0 ,0.d0, kind(0.D0)) 
Op%O(2,1) = cmplx(1.d0 ,0.d0, kind(0.D0))
Op%g      = SQRT(CMPLX(DTAU*V/real(N_SUN,kind(0.d0)), 0.D0, kind(0.D0))) 
Op%alpha  = cmplx(0.d0, 0.d0, kind(0.D0))
Op%type   = 2

\end{lstlisting}


\subsubsection{Fermion-Ising coupling}

\begin{lstlisting}[style=fortran]
Call Predefined_Int_Ising_SUN( OP, I, J, DTAU, XI )
\end{lstlisting}

The interaction between the Ising and a fermion degree of freedom, given by
\begin{align}
%\label{eqn_hubbard_sun_Ising}
\hat{\mathcal{H}}_{V,i,j} =
\hat{Z}_{i,j} \xi  \sum_{\sigma=1}^{N_\mathrm{col}}\left( \hat{c}^{\dagger}_{i \sigma} \hat{c}^{\phantom\dagger}_{j \sigma} + \hat{c}^{\dagger}_{j \sigma} \hat{c}^{\phantom\dagger}_{i \sigma} \right),
\end{align} 
where $\xi$ determines the coupling strength, is implemented in the subroutine \texttt{Predefined\_Int\_Ising\_SUN}:
\begin{lstlisting}[style=fortran]
Op%P(1)   = I
Op%P(2)   = J
Op%O(1,2) = cmplx(1.d0 ,0.d0, kind(0.D0)) 
Op%O(2,1) = cmplx(1.d0 ,0.d0, kind(0.D0)) 
Op%g      = cmplx(-DTAU*XI,0.D0,kind(0.D0))
Op%alpha  = cmplx(0d0,0.d0, kind(0.D0)) 
Op%type   = 1

\end{lstlisting}



\subsubsection{Long-Range Coulomb repulsion}

\begin{lstlisting}[style=fortran]
Call Predefined_Int_LRC( OP, I, DTAU )
\end{lstlisting}

The Long-Range Coulomb (LRC) interaction can be written as
\begin{align}
\hat{\mathcal{H}}_{V} =
\frac{1} { N } \sum_{\vec{i},\vec{j}}  \left(  \hat{n}_{\vec{i}} -  \frac{N}{2}  \right)  V_{\vec{i},\vec{j}} \left(  \hat{n}_{\vec{j}} -  \frac{N}{2}  \right), 
\end{align} 
where
\begin{align}
\hat{n}_{\vec{i}} = \sum_{\sigma=1}^{N}  \hat{c}^{\dagger}_{\vec{i},\sigma}  \hat{c}^{}_{\vec{i},\sigma}
\end{align} 
and
\begin{equation}
V_{\vec{i}, \vec{j}}   =   U \left\{
\begin{array}{ll}  
1          &   \text{ if } \vec{i} - \vec{j}    = 0 \\
\frac{\alpha   \;   d_\mathrm{min}}{ |   \vec{i} - \vec{j} | } &     \text{ otherwise }
\end{array}
\right. .
\end{equation}
Here $d_\mathrm{min}$ is the minimal distance between two orbitals.     The code uses the following  HS decomposition:
\begin{equation}
e^{-\Delta \tau \hat{H}_{V,k} }  =  \int \prod_{\vec{i}} d \phi_{\vec{i}}   e^{ - \frac{N \Delta \tau} {4} \phi_{\pmb{i}} V^{-1}_{\pmb{i},\pmb{j}}  \phi_{\pmb{j}} - \sum_{\pmb{i}}  i \Delta \tau \phi_i \left( n_{i} - \frac{N}{2} \right) }.
\end{equation}

The implementation follows Ref.~\cite{Hohenadler14}  but now supports various lattice geometries.    The definition of  the Coulomb repulsion is as follows. 
A general lattice site  \texttt{I,n}   where \texttt{I: 1...Latt\%N} is the unit cell and \texttt{ n = 1 ...Latt\_unit\%NORB}  the orbital  is given by: 
\begin{lstlisting}[style=fortran]
X_p(:) = Latt%list(I,1)*latt%a1_p(:)  + Latt%list(I,2)*latt%a2_p(:) 
+   Latt_unit%Orb_pos_p(no_j,:)
\end{lstlisting}
or in more compact notation $ \vec{i}  + \vec{\delta}_i $.   By definition \texttt{Latt\_unit\%Orb\_pos\_p(1,:)=0}.
The Coulomb repulsion between points   $ \vec{i}  + \vec{\delta}_i $   and $ \vec{j}  + \vec{\delta}_j $   reads: 
\begin{equation}
V(\vec{i}  + \vec{\delta}_i ,  \vec{j}  + \vec{\delta}_j  )  =  \frac{U d_\mathrm{min} \alpha}{  |  \overline{\vec{i} - \vec{j}} + \vec{\delta}_i - \vec{\delta}_j  |}.
\end{equation}
Here  we use periodic boundary conditions such that  $\overline{\vec{i} - \vec{j}}$  is an element of the real space lattice. Note that this is encoded in the array \texttt{Latt\%imj(I,J)}.

The LRC interaction is implemented in the subroutine \texttt{Predefined\_Int\_LRC}:
\begin{lstlisting}[style=fortran]
Op%P(1)   = I
Op%O(1,1) = cmplx(1.d0  ,0.d0, kind(0.D0))
Op%alpha  = cmplx(-0.5d0,0.d0, kind(0.D0))
Op%g      = cmplx(0.d0  ,DTAU, kind(0.D0)) 
Op%type   = 3

\end{lstlisting}


\subsubsection{$J_z$-$J_z$ interaction}

\begin{lstlisting}[style=fortran]
Call Predefined_Int_Jz( OP_up, Op_do, I, J, DTAU, Jz )
\end{lstlisting}

Another predefined vertex is:
\begin{align}
\hat{\mathcal{H}}_{V,i,j} =
- \frac{|J_z|}{2}  \left( S^{z}_i - \sgn|J_z| S^{z}_j \right)^2 =
J_z  S^{z}_i  S^{z}_j  - \frac{|J_z|}{2} (S^{z}_i)^2 - \frac{|J_z|}{2}(S^{z}_j)^2 
\end{align} 
which, if particle fluctuations are frozen on the $i$ and $j$ sites, then $(S^{z}_i)^2 = 1/4$ and the interactions corresponds to a $J_z$-$J_z$ ferro or antiferro coupling.

The implementation of the interaction in \texttt{Predefined\_Int\_Jz} defines two operators:
\begin{lstlisting}[style=fortran]
Op_up%P(1)   = I
Op_up%P(2)   = J
Op_up%O(1,1) = cmplx(1.d0,              0.d0, kind(0.D0))
Op_up%O(2,2) = cmplx(-Jz/Abs(Jz),       0.d0, kind(0.D0))
Op_up%alpha  = cmplx(0.d0,              0.d0, kind(0.D0))
Op_up%g      = SQRT(CMPLX(DTAU*Jz/8.d0, 0.d0, kind(0.D0))) 
Op_up%type   = 2

Op_do%P(1)   = I
Op_do%P(2)   = J
Op_do%O(1,1) = cmplx(1.d0,               0.d0, kind(0.d0))
Op_do%O(2,2) = cmplx(-Jz/Abs(Jz),        0.d0, kind(0.d0))
Op_do%alpha  = cmplx(0.d0,               0.d0, kind(0.d0))
Op_do%g      = -SQRT(CMPLX(DTAU*Jz/8.d0, 0.d0, kind(0.d0))) 
Op_do%type   = 2

\end{lstlisting}
