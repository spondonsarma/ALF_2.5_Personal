% !TEX root = Doc.tex
\section{Implementation of the model} \label{sec:imp}
In general, the module \path{Hamiltonian} defines the model Hamiltonian, the lattice under consideration and the desired observables (Table~\ref{table:hamiltonian}). 
We have collected a number of example Hamiltonians, lattices and observables in the file  \path{Hamiltonian_Examples.f90}.  They are described in the Sec.~\ref{sec:walk1} - \ref{sec:walk2}.
To implement a user-defined model, only the module \path{Hamiltonian} has to be set up. Accordingly, this documentation focusses almost entirely  on this module and the subprograms it includes. 
The remaining parts of the code may be treated as as a black box.  

To specify the Hamiltonian, one needs  an  \path{Operator} and \path{Lattice} type as well as a type for the observables. These three data structures will be described in the following. 

%
\begin{table}[h]
    \begin{tabular}{@{} l l l @{}}\toprule
    Subprogram & Description & Section \\\midrule
    \hl{\texttt{Ham\_Set}}  & Reads in model and lattice parameters from the file \texttt{parameters}. \\
                       & And it sets the Hamiltonian by calling \texttt{Ham\_latt}, \texttt{Ham\_hop}, and \texttt{Ham\_V}. & \\
    \hl{\texttt{Ham\_hop}}  & Sets the hopping term  $\hat{\mathcal{H}}_{T}$ by calling \texttt{Op\_make} and \texttt{Op\_set}. & \ref{sec:op}, \ref{sec:specific}\\
    \hl{\texttt{Ham\_V}}    & Sets the interaction terms  $\hat{\mathcal{H}}_{V}$ and $\hat{\mathcal{H}}_{I}$ 
                         by calling \texttt{Op\_make} and \texttt{Op\_set}.& \ref{sec:op}, \ref{sec:specific}\\  
    \hl{\texttt{Ham\_Latt}} & Sets the lattice by calling \texttt{Make\_Lattice}.& \ref{sec:latt}\\
    \hl{\texttt{S0}}        & A function which returns an update ratio for the Ising term $\hat{\mathcal{H}}_{I,0}$. 
    & \ref{sec:s0} \\
    \hl{\texttt{Alloc\_obs}} & Asigns memory storage to the observables & \\
    \hl{\texttt{Obser}}      & Computes the scalar observables and equal-time correlation functions. & \ref{sec:obs} \\
    \hl{\texttt{ObserT}}     & Computes time-displaced correlation functions. & \ref{sec:obs}\\
    \texttt{Init\_obs}  & Initializes the observables to zero. & \\    
    \texttt{Pr\_obs}    & Writes the observables to the disk by calling \texttt{Print\_bin}. \\\bottomrule    
   \end{tabular}
   \caption{Overview of the subprograms of the  module \texttt{Hamiltonian} to define the Hamiltonian, the lattice and the observables. 
   The highlighted subroutines have to be modified by the user.
    \label{table:hamiltonian}}
\end{table}
%

\subsection{The \texttt{Operator} type}\label{sec:op}
The fundamental data structure in the code is the derived data type \path{Operator}. 
This type is used to define the Hamiltonian (\ref{eqn:general_ham}).
In general, the matrices ${\bf T}^{(ks)}$, ${\bf V}^{(ks)}$ and ${\bf I}^{(ks)}$ are sparse Hermitian matrices.
Consider the  matrix   ${\bm X}$ of dimension  $N_{\mathrm{dim}} \times N_{\mathrm{dim}}$, as an representative of each of the above three matrices .  Let us  denote  with  $ \left\{z_{1},\cdots,  z_{N}  \right\}$  a subset  of $N$ indices,  
for which
\begin{equation}
X_{x,y}  =
\left\{\begin{matrix}  X_{x,y}  &  \text{ if }   x,  y  \in \left\{ z_1, \cdots z_N \right\}\\ 
                                  0         &  \text{ otherwise } 
      \end{matrix}\right.
\end{equation}
 We define the $N \times N_{\mathrm{dim}}$ matrices $\mathbf{P}$  as
\begin{equation}
P_{i,x}=\delta_{z_{i},x}\;,
\end{equation}
where $i \in [1,\cdots, N ]$ and $ x  \in [1,\cdots, N_{\mathrm{dim}}]$. The matrix  $\bm{P}$ picks out the non-vanishing entries of $\bm{X}$, 
which are contained in the rank-$N$  matrix $\bm{O}$.  Thereby: 
\begin{equation}
\bm{X} =\bm{P}^{T} \bm{O} \bm{P}\;,
\end{equation}
such that:
\begin{equation}
X_{x,y} = \sum\limits_{i,j}^{N}  P_{i,x}  O_{i,j} P_{j,y}=\sum\limits_{i,j}^{N} \delta_{z_{i},x}  O_{ij} \delta_{z_{j},y} \;.
\end{equation}
Since  the  $\bm{P}$ matrices have only one non-vanishing entry per column,  they can be stored as a vector $\vec{P}$:
\begin{equation}
     P_i = z_i.
\end{equation}  
There are  many useful  identities which emerge from this  structure. For example: 
\begin{equation}
	e^{\bm{X}} =  e^{\bm{P}^{T} \bm{O} \bm{P}}   = \sum_{n=0}^{\infty}  \frac{\left( \bm{P}^{T} \bm{O} \bm{P} \right)^n}{n!} =  \bm{P}^{T} e^{ \bm{O} } \bm{P}
\end{equation}
since 
\begin{equation} 
	 \bm{P} \bm{P}^{T}= 1_{N\times N}.
\end{equation}

In the code, we define a structure called \path{Operator} to capture the above. 
This type \path{Operator} bundles several components that are needed to define and use an operator matrix in the program.  

\subsubsection{Specification of the model}\label{sec:specific}
%
\begin{table}[h]
    \begin{tabular}{@{} l l l @{}}\toprule
    Variable & Type & Description \\\midrule
    \hl{\texttt{Op\_X\%N}}       & Integer     &  effective dimension $N$ \\
    \hl{\texttt{Op\_X\%O}}       & Complex    &  matrix  $\mathbf{O}$  of dimension $N \times N$\\
    \hl{\texttt{Op\_X\%P}}       & Integer   &  projection matrix $\mathbf{P}$  encoded as a vector of dimension $N$.\\
    \hl{\texttt{Op\_X\%g}}       & Complex    &  coupling strength $g$ \\  
    \hl{\texttt{Op\_X\%alpha}}   & Complex  &  constant $\alpha$ \\
    \hl{\texttt{Op\_X\%type}}    & Integer   &  parameter to set the type of 
                                             HS transformation\\
                             &   &  (1 = Ising, 2 = Discrete HS, for perfect square)  \\ 
    \texttt{Op\_X\%U}            & Complex &  matrix containing the eigenvectors of $\mathbf{O}$  \\
    \texttt{Op\_X\%E}            & Real &  eigenvalues of $\mathbf{O}$ \\
    \texttt{Op\_X\%N\_non\_zero} & Integer &  number of non-vanishing eigenvalues of $\mathbf{O}$ \\\bottomrule
   \end{tabular}
   \caption{Member Variables of the \texttt{Operator}  type. 
   In the left column, the letter \texttt{X} is a placeholder for the letters \texttt{T} and \texttt{V}, 
   indicating hopping and interaction operators, respectively.
   The highlighted variables have to be specified by the user.
  %  One will have to specify $N$, $O$, $P$, $g$, $\alpha$ and the type.  The other variables will be automatically generated in the routine \texttt{Op\_Set}.  
    \label{table:operator}}
\end{table}
%
In order to specify the  Hamiltonian (\ref{eqn:general_ham}), we will  need several arrays of the object \path{Operator}. 
Its member variables are listed in Table~\ref{table:operator}.  
Since the implementation exploits the $SU(N_{\mathrm{col}})$ invariance of the Hamiltonian, we have dropped the color index $\sigma$ in the following.
\begin{itemize}
\item Hopping Hamiltonian (\ref{eqn:general_ham_t}): 
In this case $\bm{X}=\bm{T}^{(k,s)}$. The corresponding array of structure variables \texttt{Op\_T} is  \texttt{Op\_T(M$_T$,N$_{fl}$)} . 
Precisely, a single variable  \texttt{Op\_T}  describes the operator matrix:
\begin{equation}
            \left( \sum_{x,y}^{N_{\mathrm{dim}}} \hat{c}^{\dagger}_x T_{xy}^{(ks)} \hat{c}^{\phantom{\dagger}}_{y}  \right)  \;,
\end{equation} 
where $k=[1, M_{T}]$ and $s=[1, N_{\mathrm{fl}}]$.
We have $g=-\Delta \tau$, $\alpha = 0$, and the type variable $\texttt{Op\_T\%type}$  is irrelevant. 



\item Interaction Hamiltonian (\ref{eqn:general_ham_v}):
If the interaction is of perfect-square type, we set  ${\bm X}  = \bm{V}^{(k,s)}$ 
and  define the corresponding structure variables \texttt{Op\_V}  using the array \texttt{Op\_V(M\_V,N\_{fl})}.
A single variable  \texttt{Op\_V}  describes the operator matrix:
\begin{equation}
             \left[ \left( \sum_{x,y}^{N_{\mathrm{dim}}} \hat{c}^{\dagger}_x V_{x,y}^{(ks)} \hat{c}^{\phantom{\dagger}}_{y}  \right) - \alpha_{ks} \right]  \;,
\end{equation} 
where $k=[1, M_{V}]$ and $s=[1, N_{\mathrm{fl}}]$. For the perfect-square interaction, $\alpha = \alpha_{ks}$ and $g = \sqrt{\Delta \tau  U_k}$. 
The discrete Hubbard-Stratonovich decomposition is selected by setting the type variable to $\texttt{Op\_V\%type}=2$.

\item Ising interaction Hamiltonian (\ref{eqn:general_ham_i}):
In this case, $\bm{X}  = \bm{I}^{(k,s)} $ and we define the array\\ \texttt{Op\_V(M\_I,N\_{fl})}.  
A single variable  \texttt{Op\_V} then  describes the operator matrix:
\begin{equation}
            \left( \sum_{x,y}^{N_{\mathrm{dim}}} \hat{c}^{\dagger}_x I_{xy}^{(ks)} \hat{c}^{\phantom{\dagger}}_{y}  \right)  \;,
\end{equation} 
where $k=[1, M_{I}]$ and $s=[1, N_{\mathrm{fl}}]$.
The Ising interaction is specified by setting the type variable  $\texttt{Op\_V\%type=1}$, $\alpha = 0$ and $g = -\Delta \tau$.  

\item In case of a full interaction [perfect-square term (\ref{eqn:general_ham_v}) and Ising term (\ref{eqn:general_ham_i})], we  define  the corresponding doubled array \texttt{Op\_V(M$_V$+M$_I$,N$_{fl}$) } and set the variables separately for both ranges of the array according to the above.  

\end{itemize}
  %      There is another array   which defines the full interaction,  Ising as well as perfect square terms. For this  we define  the array \texttt{Op\_V(M$_V$+M$_I$,N$_{fl}$) }). In this context the variable \texttt{Op\_V\%type} specifies the interaction: Ising or  a perfect square.  If the interaction is of Ising type, then  $\bm{V}  = \bm{I}^{(k,s)} $, $\alpha = 0$ and $g = -\Delta \tau$.  
%   If the interaction is a perfect square type, then  $\bm{V}  = \bm{V}^{(k,s)} $, $\alpha = \alpha_{k,s}$ and $g = \sqrt{\Delta \tau  U_k}$.  

%The variable $\texttt{Op\_V\%type}  $  in the operator structure  is required to specify  the following. If the operator  correspond to an interaction part of the Hamiltonian  then for 
%$\texttt{Op\_V\%type} =1 $   the operator refers to an Ising  operator $ \bm{I}^{k,s}$ and for  $\texttt{Op\_V\%type} =2 $  to $\bm{V}^{ks} $
%\begin{itemize}
%\item the projector ${\bm P}$, encoded as the vector $\vec{P}$,
%\item the matrix ${\bm O}$ of dimension $N \times N$  
%\item the effective dimension $N$,
%\item and a couple of auxiliary matrices and scalars.
%\end{itemize}
%The precise definition of the Operator type reads:




\subsection{The \texttt{Lattice} type}\label{sec:latt}

We have a lattice module  which can generate one and two dimensional Bravais lattices.
Note that the orbital structure of each unit cell, has to be specified by the user in the Hamiltonian module. 
 The user has to specify unit vectors $\vec{a}_1$ and $\vec{a}_2$ as well as the size of the  lattice. The size is  characterized by  two vectors $\vec{L}_1$ and $\vec{L}_2$   and  the lattice is placed on a torus: 
\begin{equation}
	\hat{c}_{\vec{i} + \vec{L}_1 }  = \hat{c}_{\vec{i} + \vec{L}_2 }  = \hat{c}_{\vec{i}}
\end{equation}
The function call 
\begin{lstlisting} 
Call Make_Lattice( L1, L2, a1,  a2, Latt )
\end{lstlisting}
will generate the lattice   \texttt{Latt} of type \texttt{Lattice}.   Note that  the structure of the unit cell has to be provided by the user.    The reciprocal lattice vectors are defined by: 
\begin{equation}
\label{Latt.G.eq}
	\vec{a}_i  \cdot \vec{g}_i = 2 \pi \delta_{i,j}, 
\end{equation}
and the Brillouin zone corresponds to the Wigner Seitz cell of the lattice. 
With $\vec{k} = \sum_{i} \alpha_i  \vec{g}_i $, the  k-space quantization follows from: 
\begin{equation}
\begin{bmatrix}
	\vec{L}_1 \cdot \vec{g}_1  &  \vec{L}_1 \cdot \vec{g}_2  \\
	\vec{L}_2  \cdot \vec{g_1} & \vec{L}_2 \cdot  \vec{g}_2  
\end{bmatrix}
\begin{bmatrix}
   \alpha_1 \\
   \alpha_2
\end{bmatrix}
=
2 \pi 
\begin{bmatrix}
   n \\
   m
\end{bmatrix}
\end{equation}
such that 
\begin{eqnarray}
\label{k.quant.eq}
     \vec{k} =  n \vec{b}_1  + m \vec{b}_2 \text{  with  }   & &   \vec{b}_1 = \frac{2 \pi}{ (\vec{L}_1 \cdot \vec{g}_1)  (\vec{L}_2 \cdot  \vec{g}_2 )  - (\vec{L}_1 \cdot \vec{g}_2) (\vec{L}_2  \cdot \vec{g_1} ) }   \left[  (\vec{L}_2 \cdot  \vec{g}_2) \vec{g}_1 -   (\vec{L}_2  \cdot \vec{g_1} ) \vec{g}_2 \right] \text{   and  } \nonumber \\ 
        & & \vec{b}_2 = \frac{2 \pi}{ (\vec{L}_1 \cdot \vec{g}_1)  (\vec{L}_2 \cdot  \vec{g}_2 )  - (\vec{L}_1 \cdot \vec{g}_2) (\vec{L}_2  \cdot \vec{g_1} ) }   
           \left[  (\vec{L}_1 \cdot  \vec{g}_1) \vec{g}_2 -   (\vec{L}_1  \cdot \vec{g_2} ) \vec{g}_1 \right] 
\end{eqnarray}
% 
%
\begin{table}[h]
   \begin{tabular}{@{} l l l @{}}\toprule
    Variable  & Type & Description \\\midrule
     \hl{\texttt{Latt\%a1\_p}, \texttt{Latt\%a2\_p}}   & Real     & Unit vectors $\vec{a}_1$,  $\vec{a}_2$ \\ 
     \hl{\texttt{Latt\%L1\_p}, \texttt{Latt\%L2\_p}}   & Real     & Vectors $\vec{L}_1$, $\vec{L}_2$ that define the topology of the  lattice. \\
     									  &              &  Tilted lattices are  thereby possible to implement.  \\
    \texttt{Latt\%N}                                                 &   Integer &  Number of lattice points, $N_{\text{unit cell}}$   \\
    \texttt{Latt\%list}                                               & Integer &  maps each lattice point $i=1,\cdots, N_{\text{unit cell}}$ to a real space vector\\ 
                                                                             &   &  denoting the position of the unit cell: \\
                                                                             &   & $\vec{R}_i$ = \texttt{list(i,1)} $\vec{a}_1$ +  \texttt{list(i,2)} $\vec{a}_2$  $  \equiv i_1  \vec{a}_1 + i_2  \vec{a}_2 $ \\
    \texttt{Latt\%invlist}                                        &  Integer &   \texttt{Invlist}$(i_1,i_2) = i $ \\
    \texttt{Latt\%nnlist}                                         &  Integer &   $j = \texttt{nnlist} (i, n_1, n_2) $,  $n_1, n_2 \in [-1,1] $ \\
                                                                           &              &    $\vec{R}_j = \vec{R}_i + n_1 \vec{a}_1  + n_2 \vec{a}_2 $ \\
   \texttt{Latt\%imj}                                             &   Integer  &  $ \vec{R}_{imj(i,j)}  =  \vec{R}_i -  \vec{R}_j$.        $imj, i, j \in  1,\cdots, N_{\text{unit cell}}$\\
    \texttt{Latt\%BZ1\_p}, \texttt{Latt\%BZ2\_p}  &   Real     & Reciprocal space vectors $\vec{g}_i$   (See Eq.~\ref{Latt.G.eq})\\
    \texttt{Latt\%b1\_p}, \texttt{Latt\%b1\_p}       &   Real     &  k-quantization (See Eq.~\ref{k.quant.eq}) \\
    \texttt{Latt\%listk}                                           &  Integer &  maps each reciprocal lattice point $k=1,\cdots, N_{\text{unit cell}}$\\
                                                                          &    & to a reciprocal space vector\\
                                                                          &     & $\vec{k}_k= \texttt{listk(k,1)} \vec{b}_1 +  \texttt{listk(k,2)} \vec{b}_2  \equiv k_1  \vec{b}_1 +   k_2  \vec{b}_2 $\\
    \texttt{Latt\%invlistk}                                     &    Integer    &   \texttt{Invlistk}$(k_1,k_2) = k $ \\
   \texttt{Latt\%b1\_perp\_p},  \\ 
   \texttt{Latt\%b2\_perp\_p}                             &    Real         &  Orthonormal vectors to $\vec{b}_i$.  For internal use. \\\bottomrule
   \end{tabular}
   \caption{Components of the \texttt{Lattice} type for two-dimensional lattices using as example the default lattice name \texttt{Latt}.
   The highlighted variables have to be specified by the user.  Other components of the Lattice will be generated  when calling: \texttt{ Call Make\_Lattice( L1, L2, a1,  a2, Latt )}. 
    \label{table:lattice}}
\end{table}
%
The \path{Lattice}  module equally handles  the Fourier transformation.  For example  the  subroutine  \path{Fourier_R_to_K}   carries out the  transformation: 
\begin{equation}
	S(\vec{k}, :,:,:) =  \frac{1}{N_{unit \,cell}}  \sum_{\vec{i},\vec{j}}   e^{-i \vec{k} \cdot \left( \vec{i}-\vec{j} \right)} S(\vec{i}  - \vec{j}, :,:,:)
\end{equation}
and  \path{Fourier_K_to_R}  the  inverse Fourier transform 
 \begin{equation}
	S(\vec{r}, :,:,:) =  \frac{1}{N_{unit \,cell}}  \sum_{\vec{k} \in BZ }   e^{ i \vec{k} \cdot \vec{r}} S(\vec{k}, :,:,:).
\end{equation}
In the above,   the unspecified dimensions of   structure factor can refer  to imaginary time,  and orbital indices. 


\subsection{The observable types \texttt{Obser\_Vec} and \texttt{Obser\_Latt}}\label{sec:obs}

Our definition  of the model includes observables [Eq.~(\ref{eqn:obs_rw})] . We have defined two observable types: \texttt{Obser\_vec}  for a array of scalar observables
such as the energy and  \texttt{Obser\_Latt}   for correlation functions that have the lattice symmetry. In the latter case, translation symmetry can be used to provide improved estimators and to reduce the size of the I/O.   
We also obtain improved estimators by taking measurements in the imaginary-time interval \texttt{[LOBS\_ST,LOBS\_EN]}  (see the parameter file in Sec.~\ref{sec:input}) thereby exploiting the invariance under translation in imaginary time.
Note that the translation symmetries  in space and in time are broken for a given  configuration $C$ but restored by the Monte Carlo sampling. 
In general, the user will define bins, each bins having a given amount of sweeps. Within a sweep we run sequentially trough the HS and Ising fields from   time slice 1 to $L_{\text{Trotter}}$ and back.  The results of each bin is written  in a file  and analyzed at the end of the run.     

To accomplish the reweighting of observables (see Sec.~\ref{sec:reweight}), for each configuration the measurement of an observable has to be multiplied by the factors \texttt{ZS} and \texttt{ZP}:
\begin{eqnarray}
\texttt{ZS} &=& \text{sign}(C)\;\\
\texttt{ZP} &=& \frac{e^{-S(C)}} {\Re \left[e^{-S(C)} \right]}\;,
\end{eqnarray}
They are computed from the Monte Carlo phase of a configuration,
\begin{equation}\label{eqn:phase}
	\texttt{phase}   =   \frac{e^{-S(C)}}{ \left| e^{-S(C) }\right| }\;,
\end{equation}
which is provided by the main program.


Note that each observable structure also includes the average sign [Eq.~(\ref{eqn:sign_rw})].

\subsubsection{Scalar observables}
This data type  is described in Table  \ref{table:Obser_vec} and  is useful to compute an array of  scalar observables.   Consider  a variable \texttt{Obs} of type  \texttt{Obser\_vec}.  At the beginning of each bin,  a call to  \texttt{Obser\_Vec\_Init} in the module \texttt{observables\_mod.f90}  will  set   \texttt{Obs\%N=0},   \texttt{Obs\%Ave\_sign =0}  and  \texttt{Obs\%Obs\_vec(:)=0}.  Each time the main  program calls the routine \texttt{Obser}  in the  \texttt{Hamiltonian} module,  the counter \texttt{Obs\%N}   is incremented by unity,   the sign  (see Eq.~\ref{Sign.eq}) is cumulated in the  variable \texttt{Obs\%Ave\_sign},  and the desired  the observables (multiplied by the sign and   $\frac{e^{-S(C)}} {\Re \left[e^{-S(C)} \right]}$, see Sec.~\ref{Observables.General})  are cumulated in the vector \texttt{Obs\%Obs\_vec}.  
\begin{table}[h]
   \begin{tabular}{@{} l l l l @{}}\toprule
    Variable  &  Type      &  Description &  Contribution of  \\
        &  & & configuration $C$ \\\midrule
    \texttt{Obs\%N}                       &  Int.        &   Number of measurements &\\
    \texttt{Obs\%Ave\_sign}               &  Real     &    Cumulated sign [Eq.~(\ref{eqn:sign_rw})] & $\text{sign}(C)$  \\
    \texttt{Obs\%Obs\_vec(:)}        & Compl.      &    Cumulated vector of observables [Eq.~(\ref{eqn:obs_rw})] &
           $ \langle \langle \hat{O}(:) \rangle \rangle_{C}\frac{e^{-S(C)}} {\Re \left[e^{-S(C)} \right]} \text{ sign }(C) $ \\
     \texttt{Obs\%File\_Vec}           &  Char.    &    Name of output file  &\\\bottomrule
   \end{tabular}
   \caption{Components of the \texttt{Obser\_vec}  type.  The table lists the data included in a variable  \texttt{Obs}  of type \texttt{Obser\_vec}.  
   % \mycomment{MB $\texttt{Obs\%Phase}$ is not $phase(C)$ but $sign(C)$. And the type of sign could in principle be reduced to integer.   }
         \label{table:Obser_vec}}
\end{table}
At the end of the bin, a call to  \texttt{Print\_bin\_Vec}   in  module \texttt{observables\_mod.f90}  will  append the result of the bin in the file  \texttt{File\_Vec}\_scal.  Note that this subroutine will automatically append the suffix  \_scal 
to the the filename \texttt{File\_Vec}.    This suffix  is important to allow automatic analysis of the data at the end of the run. 

\subsubsection{ Equal time and time-displaced correlation functions}
%
\begin{table}[h]
   \begin{tabular}{@{} l l l l @{}}\toprule
        Variable  &  Type      &  Description &  Contribution of  \\
        &  & & configuration $C$ \\\midrule
    \texttt{Obs\%N}                       &  Int.        &   Number of measurements &  \\
    \texttt{Obs\%Ave\_sign}  
    &  Real  &    Cumulated sign [Eq.~(\ref{eqn:sign_rw})] & $\text{sign}(C)$  \\
    \texttt{Obs\%Obs\_latt}        & Compl.      &    Cumul.  correl. fct. [Eq.~(\ref{eqn:obs_rw})] &  $ \langle \langle \hat{O}_{\vec{i},\alpha} (\tau) \hat{O}_{\vec{j},\beta} \rangle \rangle_{C} \; \frac{e^{-S(C)}} {\Re \left[e^{-S(C)} \right]}  \text{sign}(C) $ \\
     $(\vec{i}-\vec{j},\tau,\alpha,\beta)$ & & & \\
     \texttt{Obs\%Obs\_latt0($\alpha$)}        & Compl.      &    Cumul. expect. value [Eq.~(\ref{eqn:obs_rw})] &   $ \langle \langle \hat{O}_{\vec{i},\alpha} \rangle \rangle_{C}\frac{e^{-S(C)}} {\Re \left[e^{-S(C)} \right]}  \text{ sign }(C) $ \\
     \texttt{Obs\%File\_Latt}           &  Char.    &    Name of output file  &\\\bottomrule
   \end{tabular}
   \caption{Components of the \texttt{Obser\_latt}  type.  The table lists the data included in a variable  \texttt{Obs}  of type \texttt{Obser\_latt}  
      \label{table:Obser_latt}}
\end{table}
%
This data type (see Table~\ref{table:Obser_latt}) is useful so as to deal with  imaginary time displaced as well as equal time correlation functions of the form: 
\begin{equation}\label{eqn:s}
	S_{\alpha,\beta}(\vec{k},\tau) =   \frac{1}{N_{\text{unit cell}}} \sum_{\vec{i},\vec{j}}  e^{- \vec{k} \cdot \left( \vec{i}-\vec{j}\right) } \left( \langle \hat{O}_{\vec{i},\alpha} (\tau) \hat{O}_{\vec{j},\beta} \rangle  - 
	  \langle \hat{O}_{\vec{i},\alpha} \rangle \langle   \hat{O}_{\vec{i},\beta}  \rangle \right).
\end{equation}
Here,  translation symmetry of the Bravais lattice is explicitly taken into account. 
The correlation function splits in a correlated part $S_{\alpha,\beta}^{\mathrm{(corr)}}(\vec{k},\tau)$ and a background part $S_{\alpha,\beta}^{\mathrm{(back)}}(\vec{k})$:
\begin{eqnarray}
  S_{\alpha,\beta}^{\mathrm{(corr)}}(\vec{k},\tau)
  &=&
   \frac{1}{N_{\text{unit cell}}} \sum_{\vec{i},\vec{j}}  e^{- i\vec{k} \cdot \left( \vec{i}-\vec{j}\right) }  \langle \hat{O}_{\vec{i},\alpha} (\tau) \hat{O}_{\vec{j},\beta} \rangle\label{eqn:s_corr}\;,\\
         S_{\alpha,\beta}^{\mathrm{(back)}}(\vec{k})
  &=&
   \frac{1}{N_{\text{unit cell}}} \sum_{\vec{i},\vec{j}}  e^{- i\vec{k} \cdot \left( \vec{i}-\vec{j}\right) }  \langle \hat{O}_{\vec{i},\alpha} (\tau)\rangle \langle \hat{O}_{\vec{j},\beta} \rangle\nonumber\\
  &=& 
  N_{\text{unit cell}}\, \langle \hat{O}_{\alpha} \rangle \langle \hat{O}_{\beta} \rangle \, \delta(\vec{k})\label{eqn:s_back}\;,
\end{eqnarray}
where translation invariance in space and time has been exploited to obtain the last line. 
The background part depends only on the expectation value $\langle \hat{O}_{\alpha} \rangle$, for which we use the following estimator 
\begin{equation}\label{eqn:o}
\langle \hat{O}_{\alpha} \rangle \equiv \frac{1}{N_{\text{unit\,cell}}} \sum\limits_{\vec{i}} \langle \hat{O}_{\vec{i},\alpha} \rangle\;.
\end{equation}

Consider a variable  \texttt{Obs} of type  \texttt{Obser\_latt}. At the beginning of each bin a call to  \texttt{Obser\_Latt\_Init} in the module \texttt{observables\_mod.f90}  will  initialize  the elements of \texttt{Obs} to zero.    Each time the main program calls the   \texttt{Obser} or  \texttt{ObserT} routines one  cumulates $ \langle \langle \hat{O}_{\vec{i},\alpha} (\tau) \hat{O}_{\vec{j},\beta} \rangle \rangle_{C} \; \frac{e^{-S(C)}} {\Re \left[e^{-S(C)} \right]}  \text{sign}(C) $    in  \texttt{Obs\%Obs\_latt($\vec{i}-\vec{j},\tau,\alpha,\beta$)}   
and $ \langle \langle \hat{O}_{\vec{i},\alpha}= \rangle \rangle_{C}\frac{e^{-S(C)}} {\Re \left[e^{-S(C)} \right]}  \text{ sign }(C) $  in \texttt{Obs\%Obs\_latt0($\alpha$)}.   At the end of each bin, a call to \texttt{Print\_bin\_Latt} in the module  \texttt{observables\_mod.f90}   will append the result of the bin in the specified  file \texttt{Obs\%File\_Latt}.   Note that the routine  \texttt{Print\_bin\_Latt}  carries out the Fourier transformation and prints the results in k-space. We have adopted the following name convention.  For    equal time observables , that is  the second  dimension  of the array  \texttt{Obs\%Obs\_latt($\vec{i}-\vec{j},\tau,\alpha,\beta$)}    is equal to unity,  the routine \texttt{Print\_bin\_Latt}  attaches the suffix \_eq to \texttt{Obs\%File\_Latt}.  For  time displaced correlation functions we use the suffix \_tau. 

% We have three types of observables. 
% \begin{itemize}
% \item Scalar observables such as the energy
% \item Equal time correlation functions.  Let $\hat{O}_{\vec{i},\alpha} $ be a local observable,  with $\vec{i}$ labelling the unit cell and $\alpha$ labelling the orbital or bone emanating 
% from the unit cell.   The program will compute: 
% \begin{equation}
% 	S_{\alpha,\beta}(\vec{k}) = \frac{1}{N_{unit \;  cells}} \sum_{\vec{i},\vec{j}} e^{i \vec{k}\cdot (\vec{i} -  \vec{j} ) } \left( \langle \hat{O}_{\vec{i},\alpha}  \hat{O}_{\vec{j},\alpha} \rangle  - 
% 	  \langle \hat{O}_{\vec{i},\beta} \rangle \langle   \hat{O}_{\vec{i},\beta}  \rangle \right) 
% \end{equation}
% \item  Time displaced correlation functions. This has a very similar structure than above but now with an additional time index.
% \begin{equation}
% 	S_{\alpha,\beta}(\vec{k},\tau) = \frac{1}{N_{unit \;  cells}} \sum_{\vec{i},\vec{j}} e^{i \vec{k}\cdot (\vec{i} -  \vec{j} ) } \left( \langle \hat{O}_{\vec{i},\alpha} (\tau) \hat{O}_{\vec{j},\alpha} \rangle  - 
% 	  \langle \hat{O}_{\vec{i},\beta} \rangle \langle   \hat{O}_{\vec{i},\beta}  \rangle \right) 
% \end{equation}
% \end{itemize}

%\mycomment{mention bins, sweeps}
%\mycomment{We have to add some  more details.}
%\subsubsection{Scalar observables}
%Several scalar observables are measured and accumulated in the array \texttt{Obs\_scal} during the simulation (see table \ref{table:obs}).
%%
%\begin{table}[h]
%   \begin{tabular}{l l l}
%    Name of variable in the code & Definition & Description \\\hline
%\texttt{Obs\_scal(1)} & 
%$\rho=\sum\limits_{k=1}^{M_T}
%\sum\limits_{s=1}^{N_{\mathrm{fl}}}
%\sum\limits_{\sigma=1}^{N_{\mathrm{col}}}
%\sum\limits_{x}^{N_{\mathrm{dim}}}
%\langle \hat{c}^{\dagger}_{x \sigma   s} \hat{c}^{\phantom\dagger}_{x \sigma s}   \rangle$ &
%electronic density\\
%\texttt{Obs\_scal(2)} & 
%$E_{\mathrm{kin}}=\sum\limits_{k=1}^{M_T}
%\sum\limits_{s=1}^{N_{\mathrm{fl}}}
%\sum\limits_{\sigma=1}^{N_{\mathrm{col}}}
%\sum\limits_{x,y}^{N_{\mathrm{dim}}}
%\langle \hat{c}^{\dagger}_{x \sigma   s} T_{xy}^{(k s)} \hat{c}^{\phantom\dagger}_{y \sigma s}   \rangle$ &
%kinetic energy\\
%\texttt{Obs\_scal(3)} & 
%$E_{\mathrm{pot}}=\sum\limits_{x,y}^{N_{\mathrm{dim}}}
%\prod\limits_{s=1}^{N_{\mathrm{fl}}}
%\langle \hat{c}^{\dagger}_{x \sigma   s} \hat{c}^{\phantom\dagger}_{x \sigma s}  
%\rangle$ &
%potential energy \mycomment{need input here} \\
%\texttt{Obs\_scal(4)} & 
%$E_{\mathrm{tot}}=E_{\mathrm{kin}}+E_{\mathrm{pot}}$ &
%total energy\\
%\texttt{Obs\_scal(5)} & 
%$\langle \mathrm{phase} \rangle$ &
%phase of MC update probability
%   \end{tabular}
%   \caption{Scalar observables that are stored in the array \texttt{Obs\_scal}.
%       \label{table:obs}}
%\end{table}
%%
%
%
%
%\subsubsection{Equal-time correlation functions}
%
%Let $\hat{O}_{\vec{i},\alpha} $ be a local observable,  with $\vec{i}$ labelling the unit cell and $\alpha$ labelling the orbital or bone emanating 
%from the unit cell.   The program will compute: 
%\begin{equation}
%	S_{\alpha,\beta}(\vec{k}) = \frac{1}{N_{unit \;  cells}} \sum_{\vec{i},\vec{j}} e^{i \vec{k}\cdot (\vec{i} -  \vec{j} ) } \left( \langle \hat{O}_{\vec{i},\alpha}  \hat{O}_{\vec{j},\alpha} \rangle  - 
%	  \langle \hat{O}_{\vec{i},\beta} \rangle \langle   \hat{O}_{\vec{i},\beta}  \rangle \right) 
%\end{equation}
%\mycomment{Should it not be
%}
%
%\subsubsection{Time-displaced correlation functions}
%
%This has a very similar structure than above but now with an additional time index.
%\begin{equation}
%	S_{\alpha,\beta}(\vec{k},\tau) = \frac{1}{N_{unit \;  cells}} \sum_{\vec{i},\vec{j}} e^{i \vec{k}\cdot (\vec{i} -  \vec{j} ) } \left( \langle \hat{O}_{\vec{i},\alpha} (\tau) \hat{O}_{\vec{j},\alpha} \rangle  - 
%	  \langle \hat{O}_{\vec{i},\beta} \rangle \langle   \hat{O}_{\vec{i},\beta}  \rangle \right) 
%\end{equation}
%
%To set the  interaction part, we therefore have to specify the following:
%\begin{itemize}
%\item the matrix elements $\left[O_{V}^{(k)}\right]_{ij}$
%\item the set $[z_{1}^{(k)},\cdots  z_{N_{eff}^{(k)}}^{(k)}]$ 
%\item the interaction strengths $U_{k}$
%\item the numbers  $\alpha_{k}$.
%\end{itemize}
%\mycomment{Be more specific here what really has to specified in the actual code.}%
%The same logic also applies to the implementation of the hopping interaction \mycomment{be more specific}.






%\begin{itemize}
%\item in the coupling $g$ in the \texttt{Operator} structure (see Sec.~\ref{}).
%\item as normalization constant in the definition of observables (see Sec.~\ref{})
%\item as exponent in the calculation of the phase factor and the Monte Carlo update ratio.
%\end{itemize}
%\subsection{Structure of the hopping matrix  ${\bf T}$ and the interaction matrices ${\bf V}^{(k)}$}


%\subsection{The Hubbard-Stratonovich decomposition} 
%Consider a single-particle (in other words bilinear) operator $O_{i}$.
%One obtains an approximation to the evolution operator by the following series expansion \cite{AssaadBook08}
%\begin{equation}
%\label{eqn_2_HS}
%e^{-\Delta\tau O^{2}_{i} } = \sum\limits_{s=\pm1,\pm2} \gamma(s) e^{i \sqrt{\Delta\tau}\eta(s)O_{i}} + \mathcal{O}(\Delta\tau^{4})\;,
%\end{equation}
%with 
%
%\begin{eqnarray}
%\gamma(\pm 1) = (1+\sqrt{6}/3)/4\;,\;\gamma(\pm 2) = (1-\sqrt{6}/3)/4\;,\nonumber\\
%\eta(\pm 1) =\pm \sqrt{2(3-\sqrt{6})}\;,\;\eta(\pm 2) =\pm \sqrt{2(3+\sqrt{6})}\;.
%\end{eqnarray}
%
%Eq.~(\ref{eqn_2_HS}) can be easily proven by expanding its right hand side  to eighth order in $O_{i}$. 
%The transformation introduces therefore two Ising fields $s$ per lattice site $i$, taking the values $\pm 1$ and $\pm 2$.
%\mycomment{same label as the flavor index}
