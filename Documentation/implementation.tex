% Copyright (c) 2016 The ALF project.
% This is a part of the ALF project documentation.
% The ALF project documentation by the ALF contributors is licensed
% under a Creative Commons Attribution-ShareAlike 4.0 International License.
% For the licensing details of the documentation see license.CCBYSA.

% !TEX root = doc.tex
%------------------------------------------------------------
\subsection{Implementation of the Hamiltonian and the lattice} 
%------------------------------------------------------------

The module \path{Hamiltonian} defines the model Hamiltonian, the lattice under consideration and the desired observables (Table~\ref{table:hamiltonian}). This module can be found in the file \path{Hamiltonian_Examples_mod.F90}, which also contains a number of example Hamiltonians, lattices and observables.  The examples are described in Sec.~\ref{sec:ex}.
In order to implement a user-defined model, only the module \path{Hamiltonian} has to be set up. Accordingly, this documentation focuses almost entirely  on this module and the subprograms it includes.
The remaining parts of the code may hence be treated as a black box.  

In order to specify the Hamiltonian, one needs  an  \path{Operator} and a \path{Lattice} type as well as a type for the observables. These three data structures are described in the following sections.

%
\begin{table}[h]
    \begin{tabular}{@{} l l l @{}}\toprule
    Subprogram & Description & Section \\\midrule
    \hl{\texttt{Ham\_Set}}  & Reads in model and lattice parameters from the file \texttt{parameters}. \\
                       & And it sets the Hamiltonian by calling \texttt{Ham\_latt}, \texttt{Ham\_hop}, and \texttt{Ham\_V}. & \\
    \hl{\texttt{Ham\_hop}}  & Sets the hopping term  $\hat{\mathcal{H}}_{T}$ by calling \texttt{Op\_make} and \texttt{Op\_set}. & \ref{sec:op}, \ref{sec:specific}\\
    \hl{\texttt{Ham\_V}}    & Sets the interaction terms  $\hat{\mathcal{H}}_{V}$ and $\hat{\mathcal{H}}_{I}$ 
                         by calling \texttt{Op\_make} and \texttt{Op\_set}.& \ref{sec:op}, \ref{sec:specific}\\  
    \hl{\texttt{Ham\_Latt}} & Sets the lattice by calling \texttt{Make\_Lattice}.& \ref{sec:latt}\\
    \hl{\texttt{S0}}        & Returns an update ratio for the Ising term $\hat{\mathcal{H}}_{I,0}$. 
    & \ref{sec:s0} \\
    \hl{\texttt{Alloc\_obs}} & Assigns memory storage to the observables & \\
    \hl{\texttt{Obser}}      & Computes the scalar observables and equal-time correlation functions. & \ref{sec:obs} \\
    \hl{\texttt{ObserT}}     & Computes time-displaced correlation functions. & \ref{sec:obs}\\
    \texttt{Init\_obs}  & Initializes the observables to zero. & \\    
    \texttt{Pr\_obs}    & Writes the observables to disk by calling \texttt{Print\_bin}. \\\bottomrule    
   \end{tabular}
   \caption{Overview of the subprograms of the  module \texttt{Hamiltonian} to define the Hamiltonian, the lattice and the observables. The \hl{highlighted} subroutines may have to be modified by the user.
    \label{table:hamiltonian}}
\end{table}
%

%------------------------------------------------------------
\subsubsection{The \texttt{Operator} type}\label{sec:op}
%------------------------------------------------------------

The fundamental data structure in the code is the \path{Operator}. It is implemented as a Fortran derived data type designed to efficiently define the Hamiltonian~\eqref{eqn:general_ham}.

Let the matrix $\bm{X}$ of dimension $N_{\mathrm{dim}} \times N_{\mathrm{dim}}$ stand for any of the typically sparse, Hermitian matrices $\bm{T}^{(ks)}$, $\bm{V}^{(ks)}$ and $\bm{I}^{(ks)}$ that define the Hamiltonian.
Furthermore, let $\left\{z_{1},\cdots, z_{N} \right\}$ denote a subset of $N$ indices
for which
\begin{equation}
X_{x,y} 
\left\{\begin{matrix}  \neq 0  &  \text{ if }   x,  y  \in \left\{ z_1, \cdots z_N \right\}\\ 
                                 = 0         &  \text{ otherwise. } 
      \end{matrix}\right.
\end{equation}
Usually, we have $N\ll N_{\text{dim}}$.
 We define the $N \times N_{\mathrm{dim}}$ matrices $\bm{P}$  as
\begin{equation}
P_{i,x}=\delta_{z_{i},x}\;,
\end{equation}
where $i \in [1,\cdots, N ]$ and $ x  \in [1,\cdots, N_{\mathrm{dim}}]$. The matrix  $\bm{P}$ selects the non-vanishing entries of $\bm{X}$, 
which are contained in the rank-$N$  matrix $\bm{O}$ defined by:
\begin{equation}\label{eqn:xeqpdop}
\bm{X} =\bm{P}^{T} \bm{O} \bm{P}\;,
\end{equation}
and 
\begin{equation}
X_{x,y} = \sum\limits_{i,j}^{N}  P_{i,x}  O_{i,j} P_{j,y}=\sum\limits_{i,j}^{N} \delta_{z_{i},x}  O_{ij} \delta_{z_{j},y} \;.
\end{equation}
Since  the  $\bm{P}$ matrices have only one non-vanishing entry per column,  they can conveniently be stored as a vector $\vec{P}$, with entries
\begin{equation}
     P_i = z_i.
\end{equation}  
There are  many useful  identities which emerge from this  structure. For example: 
\begin{equation}
	e^{\bm{X}} =  e^{\bm{P}^{T} \bm{O} \bm{P}}   = \sum_{n=0}^{\infty}  \frac{\left( \bm{P}^{T} \bm{O} \bm{P} \right)^n}{n!} = \mathds{1}+ \bm{P}^{T} \left(e^{ \bm{O} }-\mathds{1} \right) \bm{P}\;,
\end{equation}
since 
\begin{equation} 
	 \bm{P} \bm{P}^{T}= \mathds{1}_{N\times N}.
\end{equation}

In the code, we define a structure called \path{Operator} to capture the above. 
This type \path{Operator} bundles several components, listed in Table \ref{table:operator}, that are needed to define and use an operator matrix in the program.  

%------------------------------------------------------------
\subsubsection{Specification of the model}\label{sec:specific}
%------------------------------------------------------------
%
\begin{table}[h]
    \begin{tabular}{@{} l l l @{}}\toprule
    Variable & Type & Description \\\midrule
    \hl{\texttt{Op\_X\%N}}       & Integer     &  Effective dimension $N$ \\
    \hl{\texttt{Op\_X\%O}}       & Complex    &  Matrix  $\bm{O}$  of dimension $N \times N$\\
    \hl{\texttt{Op\_X\%P}}       & Integer   &  Matrix $\bm{P}$  encoded as a vector of dimension $N$\\
    \hl{\texttt{Op\_X\%g}}       & Complex    &  Coupling strength $g$ \\  
    \hl{\texttt{Op\_X\%alpha}}   & Complex  &  Constant $\alpha$ \\
    \hl{\texttt{Op\_X\%type}}    & Integer   &  Sets the type of HS transformation (1: Ising; 2: discrete HS for\\
             &   &   perfect-square term; 3: continuous real field.)  \\ 
    \texttt{Op\_X\%diag}  & Logical & True if   $\bm{O}$  is  diagonal  \\
    \texttt{Op\_X\%U}            & Complex &  Matrix containing the eigenvectors of $\bm{O}$  \\
    \texttt{Op\_X\%E}            & Real &  Eigenvalues of $\bm{O}$ \\
    \texttt{Op\_X\%N\_non\_zero} & Integer &  Number of non-vanishing eigenvalues of $\bm{O}$  \\
    \texttt{Op\_X\%M\_exp}    & Complex &  Stores $ \texttt{M\_exp}(:,:,s) = e^{g  \phi(s,\texttt{type}) \bm{O}(:,:)} $  \\
    \texttt{Op\_X\%E\_exp}    & Complex &  Stores $ \texttt{E\_exp}(:,s) = e^{g  \phi(s,\texttt{type}) \bm{E}(:)} $ 
     \\\bottomrule
   \end{tabular}
   \caption{Member variables of the \texttt{Operator}  type. 
   In the left column, the letter \texttt{X} is a placeholder for the letters \texttt{T} and \texttt{V}, 
   indicating hopping and interaction operators, respectively.
   The \hl{highlighted} variables have to be specified by the user.   \texttt{M\_exp} and \texttt{E\_exp}  are allocated only if  $\texttt{type}=1,2$. 
  %  One will have to specify $N$, $O$, $P$, $g$, $\alpha$ and the type.  The other variables will be automatically generated in the routine \texttt{Op\_Set}.  
    \label{table:operator}}
\end{table}
%
In this section we show how to specify the  Hamiltonian (\ref{eqn:general_ham}) in the code. 
More precisely, we have to set the matrix representation of the imaginary-time propagators,
$ e^{-\Delta \tau {\bm T}^{(ks)}}$, $e^{  \sqrt{- \Delta \tau  U_k} \eta_{k\tau} {\bm V}^{(ks)} }$ and $e^{  -\Delta \tau s_{k\tau}  {\bm I}^{(ks)}}$, that appear in the 
partition function (\ref{eqn:partition_2}).  For each pair of indices $(k,s)$, these terms have the general form
\begin{equation}\label{eqn:exponent_mat}
\text{Matrix Exponential}=
e^{g \,\phi(\texttt{type})\,\bm{X} }\;.
\end{equation}
In case of the  perfect-square term,  we additionally have to set the constant $\alpha$, see the definition of the operators $\hat{V}^{(k)}$ in Eq.~(\ref{eqn:general_ham_v}).
The data structures which hold all the above information are variables of the type \path{Operator} (see Table \ref{table:operator}). 
For each pair of indices $(k,s)$, we store the following parameters in an \path{Operator} variable:
\begin{itemize}
\item $\vec{P}$ and   $ \bm{O}$   defining the matrix $\bm{X}$ [see Eq.~(\ref{eqn:xeqpdop})]
\item the constants $g$, $\alpha$
\item optionally: the type \texttt{type} of the discrete fields $\phi$
\end{itemize}
In case of the Ising term, we store \path{type=1} which sets $\phi_{k\tau}=s_{k\tau}$. 
In case of the perfect-square term, the field results from the discrete HS transformation (\ref{HS_squares}) and we store \texttt{type=2} which sets $\phi_{k\tau}=\eta_{k\tau}$. For continuous real fields $f$ we store \path{type=3} which sets $\phi_{k\tau}(f)=f$.
Note that we have dropped the color index $\sigma$, since the implementation uses the $SU(N_{\mathrm{col}})$ invariance of the Hamiltonian. 

Accordingly, the following data structures fully describe the  Hamiltonian (\ref{eqn:general_ham}):
\begin{itemize}
\item For the hopping Hamiltonian (\ref{eqn:general_ham_t}), we have to set the exponentiated hopping matrices $ e^{-\Delta \tau {\bm T}^{(ks)}}$: \\
In this case $\bm{X}^{(ks)}=\bm{T}^{(ks)}$, and a single variable  \texttt{Op\_T}  describes the operator matrix
\begin{equation}
            \left( \sum_{x,y}^{N_{\mathrm{dim}}} \hat{c}^{\dagger}_x T_{xy}^{(ks)} \hat{c}^{\phantom{\dagger}}_{y}  \right)  \;,
\end{equation} 
where $k=[1, M_{T}]$ and $s=[1, N_{\mathrm{fl}}]$. In the notation of the general expression (\ref{eqn:exponent_mat}), we set $g=-\Delta \tau$ (and $\alpha = 0$).
In case of the hopping matrix, the type variable $\texttt{Op\_T\%type}$  is neglected by the code. 
All in all, the corresponding array of structure variables is  \texttt{Op\_T(M$_T$,N$_{fl}$)}.

\item For the interaction Hamiltonian (\ref{eqn:general_ham_v}), which is of perfect-square type, we have to set the exponentiated matrices $e^{  \sqrt{ -  \Delta \tau  U_k} \eta_{k\tau} {\bm V}^{(ks)} }$:\\
In this case, $\bm{X}  = \bm{V}^{(ks)}$ and a single variable  \texttt{Op\_V}  describes the operator matrix:
\begin{equation}
             \left[ \left( \sum_{x,y}^{N_{\mathrm{dim}}} \hat{c}^{\dagger}_x V_{x,y}^{(ks)} \hat{c}^{\phantom{\dagger}}_{y}  \right)  + \alpha_{ks} \right]  \;,
\end{equation} 
where $k=[1, M_{V}]$ and $s=[1, N_{\mathrm{fl}}]$, $g = \sqrt{-\Delta \tau  U_k}$ and  $\alpha = \alpha_{ks}$. 
The discrete Hubbard-Stratonovich decomposition which is used for the perfect-square interaction, is selected by setting the type variable to $\texttt{Op\_V\%type}=2$.
All in all, the required structure variables \texttt{Op\_V} are defined  using the array \texttt{Op\_V(M$_V$,N$_\mathrm{fl}$)}.


\item For the Ising interaction Hamiltonian (\ref{eqn:general_ham_i}), we have to set the exponentiated matrices $e^{  -\Delta \tau s_{k\tau}  {\bm I}^{(ks)}}$:\\
In this case, $\bm{X}  = \bm{I}^{(k,s)} $ and a single variable  \texttt{Op\_V} then  describes the operator matrix:
\begin{equation}
            \left( \sum_{x,y}^{N_{\mathrm{dim}}} \hat{c}^{\dagger}_x I_{xy}^{(ks)} \hat{c}^{\phantom{\dagger}}_{y}  \right)  \;,
\end{equation} 
where $k=[1, M_{I}]$ and $s=[1, N_{\mathrm{fl}}]$ and $g = -\Delta \tau$ (and $\alpha = 0$).
The Ising interaction is specified by setting the type variable  $\texttt{Op\_V\%type=1}$. 
All in all, the required structure variables are contained in the array \texttt{Op\_V(M$_{I}$,N$_\mathrm{fl}$)}.

\item In case of a full interaction [perfect-square term (\ref{eqn:general_ham_v}) and Ising term (\ref{eqn:general_ham_i})],
we  define  the corresponding doubled array \texttt{Op\_V(M$_V$+M$_I$,N$_\mathrm{fl}$)} and set the variables separately for both ranges of the array according to the above.  

\end{itemize}

%------------------------------------------------------------
\subsubsection{The \texttt{Lattice} and \texttt{Unit\_cell} types}\label{sec:latt}
%------------------------------------------------------------

ALF's lattice module can generate one- and two-dimensional Bravais lattices.
Both the lattice and the unit cell are defined in the Predefined structures module and their components are detailed in Tables \ref{table:lattice} and \ref{table:unit_cell}.
Note that the orbital structure of each unit cell has to be specified by the user in the Hamiltonian module. 
The user also has to specify unit vectors $\vec{a}_1$ and $\vec{a}_2$, as well as the size of the  lattice, characterized by the vectors $\vec{L}_1$ and $\vec{L}_2$. The lattice is placed on a torus (periodic boundary conditions):
\begin{equation}
	\hat{c}_{\vec{i} + \vec{L}_1 }  = \hat{c}_{\vec{i} + \vec{L}_2 }  = \hat{c}_{\vec{i}}
\end{equation}
The function call 
\lstset{style=fortran}
\begin{lstlisting} 
Call Make_Lattice( L1, L2, a1, a2, Latt )
\end{lstlisting}
generates the lattice \texttt{Latt} of type \texttt{Lattice}.   Note again that  the orbital structure of the unit cell has to be provided by the user.    The reciprocal lattice vectors are defined by: 
\begin{equation}
\label{Latt.G.eq}
	\vec{a}_i  \cdot \vec{g}_i = 2 \pi \delta_{i,j}, 
\end{equation}
and the Brillouin zone $BZ$ corresponds to the Wigner-Seitz cell of the lattice. 
With $\vec{k} = \sum_{i} \alpha_i  \vec{g}_i $, the  k-space quantization follows from: 
\begin{equation}
\begin{bmatrix}
	\vec{L}_1 \cdot \vec{g}_1  &  \vec{L}_1 \cdot \vec{g}_2  \\
	\vec{L}_2  \cdot \vec{g_1} & \vec{L}_2 \cdot  \vec{g}_2  
\end{bmatrix}
\begin{bmatrix}
   \alpha_1 \\
   \alpha_2
\end{bmatrix}
=
2 \pi 
\begin{bmatrix}
   n \\
   m
\end{bmatrix}
\end{equation}
such that 
\begin{eqnarray}
\label{k.quant.eq}
     \vec{k} =  n \vec{b}_1  + m \vec{b}_2 \text{,  with  }   & &   \vec{b}_1 = \frac{2 \pi}{ (\vec{L}_1 \cdot \vec{g}_1)  (\vec{L}_2 \cdot  \vec{g}_2 )  - (\vec{L}_1 \cdot \vec{g}_2) (\vec{L}_2  \cdot \vec{g_1} ) }   \left[  (\vec{L}_2 \cdot  \vec{g}_2) \vec{g}_1 -   (\vec{L}_2  \cdot \vec{g_1} ) \vec{g}_2 \right] \text{   and  } \nonumber \\ 
        & & \vec{b}_2 = \frac{2 \pi}{ (\vec{L}_1 \cdot \vec{g}_1)  (\vec{L}_2 \cdot  \vec{g}_2 )  - (\vec{L}_1 \cdot \vec{g}_2) (\vec{L}_2  \cdot \vec{g_1} ) }   
           \left[  (\vec{L}_1 \cdot  \vec{g}_1) \vec{g}_2 -   (\vec{L}_1  \cdot \vec{g_2} ) \vec{g}_1 \right] 
\end{eqnarray}

%
\begin{table}[h]
   \begin{tabular}{@{} l l l @{}}\toprule
    Variable  & Type & Description \\\midrule
     \hl{\texttt{Latt\%a1\_p}, \texttt{Latt\%a2\_p}}   & Real     & Unit vectors $\vec{a}_1$,  $\vec{a}_2$ \\ 
     \hl{\texttt{Latt\%L1\_p}, \texttt{Latt\%L2\_p}}   & Real     & Vectors $\vec{L}_1$, $\vec{L}_2$ that define the topology of the  lattice. \\
     									  &              &  Tilted lattices are  thereby possible to implement.  \\
    \texttt{Latt\%N}                                                 &   Integer &  Number of lattice points, $N_{\text{unit cell}}$   \\
    \texttt{Latt\%list}                                               & Integer &  Maps each lattice point $i=1,\cdots, N_{\text{unit cell}}$ to a real space vector\\ 
                                                                             &   &  denoting the position of the unit cell: \\
                                                                             &   & $\vec{R}_i$ = \texttt{list(i,1)}$\vec{a}_1$ + \texttt{list(i,2)}$\vec{a}_2$  $  \equiv i_1  \vec{a}_1 + i_2  \vec{a}_2 $ \\
    \texttt{Latt\%invlist}                                        &  Integer &   \texttt{Invlist}$(i_1,i_2) = i $ \\
    \texttt{Latt\%nnlist}                                         &  Integer &   $j = \texttt{nnlist} (i, n_1, n_2) $,  $n_1, n_2 \in [-1,1] $ \\
                                                                           &              &    $\vec{R}_j = \vec{R}_i + n_1 \vec{a}_1  + n_2 \vec{a}_2 $ \\
   \texttt{Latt\%imj}                                             &   Integer  &  $ \vec{R}_{\mathrm{imj}(i,j)}  =  \vec{R}_i -  \vec{R}_j$.        $\mathrm{imj}, i, j \in  1,\cdots, N_{\text{unit cell}}$\\
    \texttt{Latt\%BZ1\_p}, \texttt{Latt\%BZ2\_p}  &   Real     & Reciprocal space vectors $\vec{g}_i$   (See Eq.~\ref{Latt.G.eq})\\
    \texttt{Latt\%b1\_p}, \texttt{Latt\%b1\_p}       &   Real     &  $k$-quantization (See Eq.~\ref{k.quant.eq}) \\
    \texttt{Latt\%listk}                                           &  Integer &  Maps each reciprocal lattice point $k=1,\cdots, N_{\text{unit cell}}$\\
                                                                          &    & to a reciprocal space vector\\
                                                                          &     & $\vec{k}_k= \texttt{listk(k,1)} \vec{b}_1 +  \texttt{listk(k,2)} \vec{b}_2  \equiv k_1  \vec{b}_1 +   k_2  \vec{b}_2 $\\
    \texttt{Latt\%invlistk}                                     &    Integer    &   \texttt{Invlistk}$(k_1,k_2) = k $ \\
   \texttt{Latt\%b1\_perp\_p},  \\ 
   \texttt{Latt\%b2\_perp\_p}                             &    Real         &  Orthonormal vectors to $\vec{b}_i$.  For internal use. \\\bottomrule
   \end{tabular}
   \caption{Components of the \texttt{Lattice} type for two-dimensional lattices using as example the default lattice name \texttt{Latt}.
   The \hl{highlighted} variables have to be specified by the user.  Other components of the Lattice are generated upon calling: \texttt{ Call Make\_Lattice( L1, L2, a1,  a2, Latt )}. 
    \label{table:lattice}}
\end{table}
%

The \path{Lattice} module also handles the Fourier transformation.  For example,  the  subroutine  \path{Fourier_R_to_K}   carries out the  transformation: 
\begin{equation}
	S(\vec{k}, :,:,:) =  \frac{1}{N_\mathrm{unit \,cell}}  \sum_{\vec{i},\vec{j}}   e^{-i \vec{k} \cdot \left( \vec{i}-\vec{j} \right)} S(\vec{i}  - \vec{j}, :,:,:)
\end{equation}
and  \path{Fourier_K_to_R}  the  inverse Fourier transform 
 \begin{equation}
	S(\vec{r}, :,:,:) =  \frac{1}{N_\mathrm{unit \,cell}}  \sum_{\vec{k} \in BZ }   e^{ i \vec{k} \cdot \vec{r}} S(\vec{k}, :,:,:).
\end{equation}
In the above,   the unspecified dimensions of  the structure factor can refer  to imaginary-time  and orbital indices. 

%\subsubsection{The \texttt{Unit\_cell} type}\label{sec:unit}
%This type defines the unit cell.   

\begin{table}[h]
   \begin{tabular}{@{} l l l @{}}\toprule
    Variable  & Type & Description \\\midrule
     \hl{\texttt{Norb}}   & Integer    & Number of orbitals  \\
      \hl{\texttt{N\_coord}} & Integer & Coordination number  \\
       \hl{\texttt{Orb\_pos(1..Norb,2) }} & Real & Positions of the orbitals as measured from the lattice site.  \\\bottomrule
   \end{tabular}
     \caption{Components of an instance \texttt{Latt\_unit} of the \texttt{Unit\_cell} type.
   The \hl{highlighted} variables have to be specified by the user. } 
    \label{table:unit_cell}
\end{table}


%--------------------------------------------------------------------------------------------
\subsection{The observable types \texttt{Obser\_Vec} and \texttt{Obser\_Latt}}\label{sec:obs}
%--------------------------------------------------------------------------------------------

Our definition  of the model includes observables [Eq.~(\ref{eqn:obs_rw})]. We have defined two observable types: \texttt{Obser\_vec}  for an array of scalar observables
such as the energy, and  \texttt{Obser\_Latt}   for correlation functions that have the lattice symmetry. In the latter case, translation symmetry can be used to provide improved estimators and to reduce the size of the output.   
We also obtain improved estimators by taking measurements in the imaginary-time interval \texttt{[LOBS\_ST,LOBS\_EN]}  (see the parameter file in Sec.~\ref{sec:input}) thereby exploiting the invariance under translation in imaginary-time.
Note that the translation symmetries  in space and in time are \textit{broken} for a given  configuration $C$ but restored by the Monte Carlo sampling. 
In general, the user defines size and number of bins in the parameter file, each bin containing a given amount of sweeps. Within a sweep we run sequentially through the HS and Ising fields, from time slice $1$ to time slice $L_{\text{Trotter}}$ and back.  The results of each bin are written to a file  and analyzed at the end of the run.     

To accomplish the reweighting of observables (see Sec.~\ref{sec:reweight}), for each configuration the measured value of an observable is multiplied by the factors \texttt{ZS} and \texttt{ZP}:
\begin{eqnarray}
\texttt{ZS} &=& \text{sign}(C)\;,\\
\texttt{ZP} &=& \frac{e^{-S(C)}} {\Re \left[e^{-S(C)} \right]}\;.
\end{eqnarray}
They are computed from the Monte Carlo phase of a configuration,
\begin{equation}\label{eqn:phase}
	\texttt{phase}   =   \frac{e^{-S(C)}}{ \left| e^{-S(C) }\right| }\;,
\end{equation}
which is provided by the main program.
Note that each observable structure also includes the average sign [Eq.~(\ref{eqn:sign_rw})].

%---------------------------------
\subsubsection{Scalar observables}
%---------------------------------

This data type  is described in Table  \ref{table:Obser_vec} and  is suited to store an array of  scalar observables.   Consider  a variable \texttt{Obs} of type  \texttt{Obser\_vec}.  At the beginning of each bin,  a call to  \texttt{Obser\_Vec\_Init} in the module \texttt{observables\_mod.F90}  will  set   \texttt{Obs\%N=0},   \texttt{Obs\%Ave\_sign =0}  and  \texttt{Obs\%Obs\_vec(:)=0}.  Each time the main  program calls the routine \texttt{Obser}  in the  \texttt{Hamiltonian} module,  the counter \texttt{Obs\%N}   is incremented by one, the sign  (see Eq.~\ref{Sign.eq}) is accumulated in the  variable \texttt{Obs\%Ave\_sign},  and the desired observables (multiplied by the sign and   $\frac{e^{-S(C)}} {\Re \left[e^{-S(C)} \right]}$, see Sec.~\ref{Observables.General})  are accumulated in the vector \texttt{Obs\%Obs\_vec}.  
%
\begin{table}[h]
   \begin{tabular}{@{} l l l l @{}}\toprule
    Variable  &  Type      &  Description &  Contribution of  \\
        &  & & configuration $C$ \\\midrule
    \texttt{Obs\%N}                       &  Int.        &   Number of measurements &\\
    \texttt{Obs\%Ave\_sign}               &  Real     &    Cumulated sign [Eq.~\eqref{eqn:sign_rw}] & $\text{sign}(C)$  \\
    \texttt{Obs\%Obs\_vec(:)}        & Comp.      &    Cumulated vector of observables [Eq.~\eqref{eqn:obs_rw}] &
           $ \langle \langle \hat{O}(:) \rangle \rangle_{C}\frac{e^{-S(C)}} {\Re \left[e^{-S(C)} \right]} \text{ sign }(C) $ \\
     \texttt{Obs\%File\_Vec}           &  Char.    &    Name of output file  &\\\bottomrule
   \end{tabular}
   \caption{Components of the \texttt{Obser\_vec}  type.  The table lists the data included in a variable  \texttt{Obs}  of type \texttt{Obser\_vec}.  
   % \mycomment{MB $\texttt{Obs\%Phase}$ is not $phase(C)$ but $sign(C)$. And the type of sign could in principle be reduced to integer.   }
         \label{table:Obser_vec}}
\end{table}
%
At the end of the bin, a call to  \texttt{Print\_bin\_Vec}   in  module \texttt{observables\_mod.F90}  will  append the result of the bin in the file  \texttt{File\_Vec}\emph{\_scal}.  Note that this subroutine will automatically append the suffix \emph{\_scal}
to the the filename \texttt{File\_Vec}.
This suffix  is important to allow automatic analysis of the data at the end of the run. 

%------------------------------------------------------------
\subsubsection{ Equal time and time displaced correlation functions}
%------------------------------------------------------------
%
\begin{table}[h]
   \begin{tabular}{@{} l l l l @{}}\toprule
        Variable  &  Type      &  Description &  Contribution of  \\
        &  & & configuration $C$ \\\midrule
    \texttt{Obs\%N}                       &  Int.        &   Number of measurements &  \\
    \texttt{Obs\%Ave\_sign}  
    &  Real  &    Cumulated sign [Eq.~(\ref{eqn:sign_rw})] & $\text{sign}(C)$  \\
    \texttt{Obs\%Obs\_latt}        & Compl.      &    Cumul.  correl. fct. [Eq.~(\ref{eqn:obs_rw})] &  $ \langle \langle \hat{O}_{\vec{i},\alpha} (\tau) \hat{O}_{\vec{j},\beta} \rangle \rangle_{C} \; \frac{e^{-S(C)}} {\Re \left[e^{-S(C)} \right]}  \text{sign}(C) $ \\
     $(\vec{i}-\vec{j},\tau,\alpha,\beta)$ & & & \\
     \texttt{Obs\%Obs\_latt0($\alpha$)}        & Compl.      &    Cumul. expect. value [Eq.~(\ref{eqn:obs_rw})] &   $ \langle \langle \hat{O}_{\vec{i},\alpha} \rangle \rangle_{C}\frac{e^{-S(C)}} {\Re \left[e^{-S(C)} \right]}  \text{ sign }(C) $ \\
     \texttt{Obs\%File\_Latt}           &  Char.    &    Name of output file  &\\\bottomrule
   \end{tabular}
   \caption{Components of the \texttt{Obser\_latt}  type.  The table lists the data included in a variable  \texttt{Obs}  of type \texttt{Obser\_latt}.
      \label{table:Obser_latt}}
\end{table}
%

This data type (see Table~\ref{table:Obser_latt}) is useful so as to deal with  equal time as well as imaginary-time displaced correlation functions of the form: 
\begin{equation}\label{eqn:s}
	S_{\alpha,\beta}(\vec{k},\tau) =   \frac{1}{N_{\text{unit cell}}} \sum_{\vec{i},\vec{j}}  e^{- \vec{k} \cdot \left( \vec{i}-\vec{j}\right) } \left( \langle \hat{O}_{\vec{i},\alpha} (\tau) \hat{O}_{\vec{j},\beta} \rangle  - 
	  \langle \hat{O}_{\vec{i},\alpha} \rangle \langle   \hat{O}_{\vec{j},\beta}  \rangle \right).
\end{equation}
Here,  translation symmetry of the Bravais lattice is explicitly taken into account. 
The correlation function splits in a correlated part $S_{\alpha,\beta}^{\mathrm{(corr)}}(\vec{k},\tau)$ and a background part $S_{\alpha,\beta}^{\mathrm{(back)}}(\vec{k})$:
\begin{eqnarray}
  S_{\alpha,\beta}^{\mathrm{(corr)}}(\vec{k},\tau)
  &=&
   \frac{1}{N_{\text{unit cell}}} \sum_{\vec{i},\vec{j}}  e^{- i\vec{k} \cdot \left( \vec{i}-\vec{j}\right) }  \langle \hat{O}_{\vec{i},\alpha} (\tau) \hat{O}_{\vec{j},\beta} \rangle\label{eqn:s_corr}\;,\\
         S_{\alpha,\beta}^{\mathrm{(back)}}(\vec{k})
  &=&
   \frac{1}{N_{\text{unit cell}}} \sum_{\vec{i},\vec{j}}  e^{- i\vec{k} \cdot \left( \vec{i}-\vec{j}\right) }  \langle \hat{O}_{\vec{i},\alpha} (\tau)\rangle \langle \hat{O}_{\vec{j},\beta} \rangle\nonumber\\
  &=& 
  N_{\text{unit cell}}\, \langle \hat{O}_{\alpha} \rangle \langle \hat{O}_{\beta} \rangle \, \delta(\vec{k})\label{eqn:s_back}\;,
\end{eqnarray}
where translation invariance in space and time has been exploited to obtain the last line. 
The background part depends only on the expectation value $\langle \hat{O}_{\alpha} \rangle$, for which we use the following estimator 
\begin{equation}\label{eqn:o}
\langle \hat{O}_{\alpha} \rangle \equiv \frac{1}{N_{\text{unit\,cell}}} \sum\limits_{\vec{i}} \langle \hat{O}_{\vec{i},\alpha} \rangle\;.
\end{equation}

Consider a variable  \texttt{Obs} of type  \texttt{Obser\_latt}. At the beginning of each bin a call to  \texttt{Obser\_Latt\_Init} in the module \texttt{observables\_mod.F90}  will  initialize  the elements of \texttt{Obs} to zero.    Each time the main program calls the   \texttt{Obser} or  \texttt{ObserT} routines one accumulates $ \langle \langle \hat{O}_{\vec{i},\alpha} (\tau) \hat{O}_{\vec{j},\beta} \rangle \rangle_{C} \; \frac{e^{-S(C)}} {\Re \left[e^{-S(C)} \right]}  \text{sign}(C) $    in  \texttt{Obs\%Obs\_latt($\vec{i}-\vec{j},\tau,\alpha,\beta$)}   
and $ \langle \langle \hat{O}_{\vec{i},\alpha} \rangle \rangle_{C}\frac{e^{-S(C)}} {\Re \left[e^{-S(C)} \right]}  \text{ sign }(C) $  in \texttt{Obs\%Obs\_latt0($\alpha$)}.   At the end of each bin, a call to \texttt{Print\_bin\_Latt} in the module  \texttt{observables\_mod.F90}   will append the result of the bin in the specified  file \texttt{Obs\%File\_Latt}.   Note that the routine  \texttt{Print\_bin\_Latt}  carries out the Fourier transformation and prints the results in $k$-space. 
We have adopted the following naming conventions.
For equal time observables,
defined by having the second  dimension  of the array  \texttt{Obs\%Obs\_latt($\vec{i}-\vec{j},\tau,\alpha,\beta$)}   set to unity, 
the routine \texttt{Print\_bin\_Latt}  attaches the suffix \emph{\_eq} to \texttt{Obs\%File\_Latt}.  For  time displaced correlation functions we use the suffix \emph{\_tau}.

