% Copyright (c) 2016-2020 The ALF project.
% This is a part of the ALF project documentation.
% The ALF project documentation by the ALF contributors is licensed
% under a Creative Commons Attribution-ShareAlike 4.0 International License.
% For the licensing details of the documentation see license.CCBYSA.

% !TEX root = doc.tex

%{\color{red}   We have to restructure this  section.   As it  stands it is a  chaos. We have to separate more clearly data structures, specification of the model, and  the Input output. }
%\red{Also mention:}\\
%Also available is the \texttt{Hopping\_Matrix\_type}, defined in Sec.~\ref{sec:hopping_type}, used for defining hopping matrices.


To specify a general model,  we define a set of data types.    The \texttt{Operator} (see Sec.~\ref{sec:op}) type  is used  to specify the interaction as well as the hopping.  The handling of the fields is taken care of by the \texttt{Fields}  type. To define a Bravais lattice as well as a unit cell we 
introduce the \texttt{Lattice}  and \texttt{Unit\_cell}  types.  General  scalar,  equal time and time displaced correlation functions are  handled by the  \texttt{Observable} type.   For the projective code, we provide a  \texttt{WaveFunction}  type to define the  left and right trial wave functions.  
The Hamiltonian is then specified in the  \texttt{Hamiltonian}  module   that allocates  the aforementioned types. 



%------------------------------------------------------------
\subsection{The \texttt{Operator} type}\label{sec:op}
%------------------------------------------------------------

The fundamental data structure in the code is the \path{Operator}. It is implemented as a Fortran derived data type designed to efficiently define the Hamiltonian~\eqref{eqn:general_ham}.

Let the matrix $\bm{X}$ of dimension $N_{\mathrm{dim}} \times N_{\mathrm{dim}}$ stand for any of the typically sparse, Hermitian matrices $\bm{T}^{(ks)}$, $\bm{V}^{(ks)}$ and $\bm{I}^{(ks)}$ that define the Hamiltonian.
Furthermore, let $\left\{z_{1},\cdots, z_{N} \right\}$ denote a subset of $N$ indices
for which
\begin{equation}
X_{x,y} 
\left\{\begin{matrix}  \neq 0  &  \text{ if }   x,  y  \in \left\{ z_1, \cdots z_N \right\}\\ 
                                 = 0         &  \text{ otherwise. } 
      \end{matrix}\right.
\end{equation}
Usually, we have $N\ll N_{\text{dim}}$.
 We define the $N \times N_{\mathrm{dim}}$ matrices $\bm{P}$  as
\begin{equation}
P_{i,x}=\delta_{z_{i},x}\;,
\end{equation}
where $i \in [1,\cdots, N ]$ and $ x  \in [1,\cdots, N_{\mathrm{dim}}]$. The matrix  $\bm{P}$ selects the non-vanishing entries of $\bm{X}$, 
which are contained in the rank-$N$  matrix $\bm{O}$ defined by:
\begin{equation}\label{eqn:xeqpdop}
\bm{X} =\bm{P}^{T} \bm{O} \bm{P}\;,
\end{equation}
and 
\begin{equation}
X_{x,y} = \sum\limits_{i,j}^{N}  P_{i,x}  O_{i,j} P_{j,y}=\sum\limits_{i,j}^{N} \delta_{z_{i},x}  O_{ij} \delta_{z_{j},y} \;.
\end{equation}
Since  the  $\bm{P}$ matrices have only one non-vanishing entry per column,  they can conveniently be stored as a vector $\vec{P}$, with entries
\begin{equation}
     P_i = z_i.
\end{equation}  
There are  many useful  identities which emerge from this  structure. For example: 
\begin{equation}
	e^{\bm{X}} =  e^{\bm{P}^{T} \bm{O} \bm{P}}   = \sum_{n=0}^{\infty}  \frac{\left( \bm{P}^{T} \bm{O} \bm{P} \right)^n}{n!} = \mathds{1}+ \bm{P}^{T} \left(e^{ \bm{O} }-\mathds{1} \right) \bm{P}\;,
\end{equation}
since 
\begin{equation} 
	 \bm{P} \bm{P}^{T}= \mathds{1}_{N\times N}.
\end{equation}

In the code, we define a structure called \path{Operator} to capture the above. 
This type \path{Operator} bundles several components, listed in Table \ref{table:operator}, that are needed to define and use an operator matrix in the program.  

\begin{table}[h]
    \begin{tabular}{@{} l l l @{}}\toprule
    Variable & Type & Description \\\midrule
    \hl{\texttt{Op\_X\%N}}       & Integer     &  Effective dimension $N$ \\
    \hl{\texttt{Op\_X\%O}}       & Complex    &  Matrix  $\bm{O}$  of dimension $N \times N$\\
    \hl{\texttt{Op\_X\%P}}       & Integer   &  Matrix $\bm{P}$  encoded as a vector of dimension $N$\\
    \hl{\texttt{Op\_X\%g}}       & Complex    &  Coupling strength $g$ \\  
    \hl{\texttt{Op\_X\%alpha}}   & Complex  &  Constant $\alpha$ \\
    \hl{\texttt{Op\_X\%type}}    & Integer   &  Sets the type of HS transformation (1: Ising; 2: discrete HS for\\
             &   &   perfect-square term; 3: continuous real field.)  \\ 
    \texttt{Op\_X\%diag}  & Logical & True if   $\bm{O}$  is  diagonal  \\
    \texttt{Op\_X\%U}            & Complex &  Matrix containing the eigenvectors of $\bm{O}$  \\
    \texttt{Op\_X\%E}            & Real &  Eigenvalues of $\bm{O}$ \\
    \texttt{Op\_X\%N\_non\_zero} & Integer &  Number of non-vanishing eigenvalues of $\bm{O}$  \\
    \texttt{Op\_X\%M\_exp}    & Complex &  Stores $ \texttt{M\_exp}(:,:,s) = e^{g  \phi(s,\texttt{type}) \bm{O}(:,:)} $  \\
    \texttt{Op\_X\%E\_exp}    & Complex &  Stores $ \texttt{E\_exp}(:,s) = e^{g  \phi(s,\texttt{type}) \bm{E}(:)} $ 
     \\\bottomrule
   \end{tabular}
   \caption{Member variables of the \texttt{Operator}  type. 
   In the left column, the letter \texttt{X} is a placeholder for the letters \texttt{T} and \texttt{V}, 
   indicating hopping and interaction operators, respectively.
   The \hl{highlighted} variables must be specified by the user. \texttt{M\_exp} and \texttt{E\_exp}  are allocated only if  $\texttt{type}=1,2$. 
  %  One will have to specify $N$, $O$, $P$, $g$, $\alpha$ and the type.  The other variables will be automatically generated in the routine \texttt{Op\_Set}.  
    \label{table:operator}}
\end{table}
%



%------------------------------------------------------------
\subsection{ Handling of the fields: the \texttt{Fields}  type } \label{sec:fields}
%------------------------------------------------------------

The partition function (see Sec.~\ref{sec:method}) consists of terms which, in general, can be written as $\gamma e^{g \phi \bm{X} }$, where $\bm{X}$ denotes an arbitrary operator, $g$ is a constant, and $\gamma$ and $\phi$ are fields. 
The ALF includes three different types of fields. 
\begin{itemize}
\item[\texttt{t=1}]     This type is for an  Ising field such that $\gamma= 1$    and $ \phi = \pm 1$
\item[\texttt{t=2}]     This type is for the generic HS  transformation of Eq.~\ref{HS_squares}     such that  $\gamma \equiv \gamma(l) $    and $ \phi = \eta(l)$    with $l = \pm 1, \pm 2$  (see Eq.~\ref{eta_gamma_fields}.
\item[\texttt{t=3}]     This type is for continuous fields such that $\gamma= 1$    and $ \phi  \in \mathbb{R}$.
\end{itemize}
%For an Ising term, we store type \texttt{t=1}, which sets $\gamma_{k,\tau} = 1$ and $\phi_{k,\tau}=s_{k,\tau} = \pm 1$. 
%In the case of a perfect-square term, the fields results from the discrete HS transformation (\ref{HS_squares}) and we store \texttt{t=2}, which sets $\gamma_{k,\tau} = s_{k,\tau}$ and $\phi_{k,\tau}=\eta_{k,\tau}$ ). For continuous real fields $f$ we store \texttt{t=3}, which sets $\gamma_{k,\tau} = 1$ and $\phi_{k,\tau}=f$.

For such auxiliary fields a dedicated type \texttt{Fields} is defined, whose components, listed in Table~\ref{table:Fields}, include the variables \texttt{Field\%f} and \texttt{Field\%t}, which store the field values and types, respectively, and functions such as \texttt{Field\%flip}, which flips the field values randomly.  


%
\begin{table}[h]
	\begin{center}
	\begin{tabular}{@{} p{0.25\columnwidth} p{0.08\columnwidth} p{0.6\columnwidth} @{}}\toprule
		Component                 &                & Description    \\ \midrule
		\textbf{Variable}         &  \textbf{Type} &  \\ \midrule
		\texttt{Field\%t(1:n\_op)}      &  Integer       & Sets the HS transformation type (1: Ising; 2: discrete HS for perfect-square term; 3: continuous real field). The index denotes the operator on a given time slice. \\
		\texttt{Field\%f(1:n\_op,1:Ltrot)}    &  Real          & Defines the auxiliary fields. The first index runs through the operator sequence and the second through the time slices.  For \texttt{t=1} $ f = \pm 1$,  \texttt{t=2} $ f = \pm 1, \pm 2$, and for  \texttt{t=2} $ f  \in \mathbb{R} $ \\
		\texttt{del}       &      Real          & Width $\Delta x$ of box distribution for inititial \texttt{t=3} fields, with a default value of \texttt{1}. \\
		\texttt{amplitude} &  Real          & Width of a random flip for fields of type \texttt{t=3}, defaults to \texttt{1}. \vspace{7pt} \\ 
		\multicolumn{2}{@{}l@{}}{ \textbf{Method(arguments)}}          &    \\ \midrule
		\multicolumn{2}{@{}l@{}}{ \texttt{Fields\_init(del)}}          & Initializes internal variables such as  $\eta(l)$, $\gamma(l)$  (see Eq.~\ref{eta_gamma_fields}) the variable $\texttt{del}=\Delta x$ (see above) is optional.  \\
		\multicolumn{2}{@{}l@{}}{ \texttt{Field\%make(n\_op,Ltrot)}}  & Reserves memory for the field.    \\
		\multicolumn{2}{@{}l@{}}{ \texttt{Field\%clear()}}             & Clears field from memory.    \\
		\multicolumn{2}{@{}l@{}}{ \texttt{Field\%set()}}               & Sets a random configuration.    \\
		\multicolumn{2}{@{}l@{}}{ \texttt{Field\%flip(n,nt)}}  & Flips the field values randomly  for  field \texttt{n} on time slice \texttt{nt}.    For  \texttt{t=1} it flips the sign of the Ising spin.  For \texttt{t=2} it randomly choose one of the 
		three other values of $l$. For  \texttt{t=3},  \texttt{f = f + amplitude*(ranf() -1/2)} \\
		\multicolumn{2}{@{}l@{}}{ \texttt{Field\%phi(n,nt)}}   & Returns $\phi$ for the \texttt{n}-th operator at the time slice  \texttt{nt}. \\
		\multicolumn{2}{@{}l@{}}{ \texttt{Field\%gamma(n,nt)}} & Returns $\gamma$ for the  \texttt{n}-th operator at the time slice  \texttt{nt}.    \\
		\multicolumn{2}{@{}l@{}}{ \texttt{Field\%i(n,nt)}}     & Returns \texttt{Field\%f} rounded to nearest integer (for \texttt{t=1} or \texttt{2}). \\
		\multicolumn{2}{@{}l@{}}{ \texttt{Field\%in(Group\_Comm,In\_field)}} &      If  the file  \texttt{confin\_np} exists  it reads the field configuration  from this file.   Otherwise if \texttt{In\_field}  is present it sets the fields to   \texttt{In\_field}.   
		 If both  \texttt{confin\_np}  and  \texttt{In\_field}  are not provided it sets a random field by calling \texttt{Field\%set()}.    Here np is the rank number  of the process.\\
		\multicolumn{2}{@{}l@{}}{ \texttt{Field\%out(Group\_Comm)}}    & Writes out the field configuration.    \\\bottomrule
	\end{tabular}
	\caption{Components of a variable of type \texttt{Fields} named \texttt{Field}.    \texttt{del}  and \texttt{amplitude}   are private variables of the fields module. \texttt{n\_op} and \texttt{Ltrot} (integers) are the number of  interacting operators  per time slice and time slices, respectively, \texttt{Group\_Comm} (integer) defines an MPI communicator, and the optional \texttt{In\_field} stores the initial field configuration.}
    \end{center}
	\label{table:Fields}
\end{table}
%



%------------------------------------------------------------
\subsection{The \texttt{Lattice} and \texttt{Unit\_cell} types}\label{sec:latt}
%------------------------------------------------------------

ALF's lattice module can generate one- and two-dimensional Bravais lattices.
Both the lattice and the unit cell are defined in the module \texttt{Lattices\_v3\_mod.F90} and their components are detailed in Tables \ref{table:lattice} and \ref{table:unit_cell}. 
As  its name suggest the module  \texttt{Predefined\_Latt\_mod.F90} also provides predefined lattices as  described in Sec.~\ref{sec:predefined_lattices}.
The user who wishes to define his/her own lattice also has to specify unit vectors $\vec{a}_1$ and $\vec{a}_2$,  the size of the  lattice, characterized by the vectors $\vec{L}_1$ and $\vec{L}_2$  as well as the  unit cell characterized be the number of orbitals and their positions.  The coordination number of the lattice is equally specified in the \texttt{Unit\_cell}   data type.   The lattice is placed on a torus (periodic boundary conditions):
\begin{align}
	\hat{c}_{\vec{i} + \vec{L}_1 }  =  \hat{c}_{\vec{i} + \vec{L}_2 }  = \hat{c}_{\vec{i}}\;.
\end{align}
The function call 
\begin{lstlisting}[style=fortran]
Call Make_Lattice( L1, L2, a1, a2, Latt )
\end{lstlisting}
generates the lattice \texttt{Latt} of type \texttt{Lattice}.   The reciprocal lattice vectors $\vec{g}_i$ are defined by:
\begin{equation}
\label{Latt.G.eq}
	\vec{a}_i  \cdot \vec{g}_i = 2 \pi \delta_{i,j}, 
\end{equation}
and the Brillouin zone $BZ$ corresponds to the Wigner-Seitz cell of the lattice. 
With $\vec{k} = \sum_{i} \alpha_i  \vec{g}_i $, the  k-space quantization follows from: 
\begin{equation}
\begin{bmatrix}
	\vec{L}_1 \cdot \vec{g}_1  &  \vec{L}_1 \cdot \vec{g}_2  \\
	\vec{L}_2  \cdot \vec{g_1} & \vec{L}_2 \cdot  \vec{g}_2  
\end{bmatrix}
\begin{bmatrix}
   \alpha_1 \\
   \alpha_2
\end{bmatrix}
=  2 \pi 
\begin{bmatrix}
   n \\
   m
\end{bmatrix}
\end{equation}
such that 
\begin{eqnarray} \label{k.quant.eq}
     \vec{k} =  n \vec{b}_1  + m \vec{b}_2,\; \text{with}   & &   \vec{b}_1 = \frac{2 \pi}{ (\vec{L}_1 \cdot \vec{g}_1)  (\vec{L}_2 \cdot  \vec{g}_2 )  - (\vec{L}_1 \cdot \vec{g}_2) (\vec{L}_2  \cdot \vec{g_1} ) }   \left[  (\vec{L}_2 \cdot  \vec{g}_2) \vec{g}_1 -   (\vec{L}_2  \cdot \vec{g_1} ) \vec{g}_2 \right], \nonumber \\ 
        & & \vec{b}_2 = \frac{2 \pi}{ (\vec{L}_1 \cdot \vec{g}_1)  (\vec{L}_2 \cdot  \vec{g}_2 )  - (\vec{L}_1 \cdot \vec{g}_2) (\vec{L}_2  \cdot \vec{g_1} ) }   
           \left[  (\vec{L}_1 \cdot  \vec{g}_1) \vec{g}_2 -   (\vec{L}_1  \cdot \vec{g_2} ) \vec{g}_1 \right].
\end{eqnarray}

%
\begin{table}[h]
	\begin{center}
   \begin{tabular}{@{} l l l @{}}\toprule
    Variable  & Type & Description \\\midrule
     \hl{\texttt{Latt\%a1\_p}, \texttt{Latt\%a2\_p}}   & Real     & Unit vectors $\vec{a}_1$,  $\vec{a}_2$. \\ 
     \hl{\texttt{Latt\%L1\_p}, \texttt{Latt\%L2\_p}}   & Real     & Vectors $\vec{L}_1$, $\vec{L}_2$ that define the topology of the  lattice. \\
     									  &              &  Tilted lattices are  thereby possible to implement.  \\
    \hl{\texttt{Latt\%N}}                                                 &   Integer &  Number of lattice points, $N_{\text{unit-cell}}$.   \\
    \texttt{Latt\%list}                                               & Integer &  Maps each lattice point $i=1,\cdots, N_{\text{unit-cell}}$ to a real space vector\\ 
                                                                             &   &  denoting the position of the unit cell: \\
                                                                             &   & $\vec{R}_i$ = \texttt{list(i,1)}$\vec{a}_1$ + \texttt{list(i,2)}$\vec{a}_2$  $  \equiv i_1  \vec{a}_1 + i_2  \vec{a}_2 $. \\
    \texttt{Latt\%invlist}                                        &  Integer &   Return lattice point from position: \texttt{Invlist}$(i_1,i_2) = i $. \\
    \texttt{Latt\%nnlist}                                         &  Integer &  Nearest neighbor indices: $j = \texttt{nnlist} (i, n_1, n_2) $,  $n_1, n_2 \in [-1,1] $, \\
                                                                           &              &    $\vec{R}_j = \vec{R}_i + n_1 \vec{a}_1  + n_2 \vec{a}_2 $. \\
   \texttt{Latt\%imj}                                             &   Integer  &  $\vec{R}_{\mathrm{imj}(i,j)}  =  \vec{R}_i -  \vec{R}_j$, with  $\mathrm{imj}, i, j \in  1,\cdots, N_{\text{unit-cell}}$.\\
    \texttt{Latt\%BZ1\_p}, \texttt{Latt\%BZ2\_p}  &   Real     & Reciprocal space vectors $\vec{g}_i$   (See Eq.~\ref{Latt.G.eq}).\\
    \texttt{Latt\%b1\_p}, \texttt{Latt\%b1\_p}       &   Real     &  $k$-quantization (See Eq.~\ref{k.quant.eq}). \\
    \texttt{Latt\%listk}                                           &  Integer &  Maps each reciprocal lattice point $k=1,\cdots, N_{\text{unit-cell}}$\\
                                                                          &    & to a reciprocal space vector\\
                                                                          &     & $\vec{k}_k= \texttt{listk(k,1)} \vec{b}_1 +  \texttt{listk(k,2)} \vec{b}_2  \equiv k_1  \vec{b}_1 +   k_2  \vec{b}_2 $.\\
    \texttt{Latt\%invlistk}                                     &    Integer    &   \texttt{Invlistk}$(k_1,k_2) = k $. \\
   \texttt{Latt\%b1\_perp\_p},  \\ 
   \texttt{Latt\%b2\_perp\_p}                             &    Real         &  Orthonormal vectors to $\vec{b}_i$.  For internal use. \\\bottomrule
   \end{tabular}
   \caption{Components of the \texttt{Lattice} type for two-dimensional lattices using as example the default lattice name \texttt{Latt}.
   The \hl{highlighted} variables must be specified by the user.  Other components of the Lattice are generated upon calling: \texttt{ Call Make\_Lattice( L1, L2, a1,  a2, Latt )}. 
    \label{table:lattice}}
\end{center}
\end{table}
%

The \path{Lattice} module also handles the Fourier transformation.  For example,  the  subroutine  \path{Fourier_R_to_K}   carries out the  transformation: 
\begin{equation}
	S(\vec{k}, :,:,:) =  \frac{1}{N_\text{unit-cell}}  \sum_{\vec{i},\vec{j}}   e^{-i \vec{k} \cdot \left( \vec{i}-\vec{j} \right)} S(\vec{i}  - \vec{j}, :,:,:)
\end{equation}
and  \path{Fourier_K_to_R}  the  inverse Fourier transform 
 \begin{equation}
	S(\vec{r}, :,:,:) =  \frac{1}{N_\text{unit-cell}}  \sum_{\vec{k} \in BZ }   e^{ i \vec{k} \cdot \vec{r}} S(\vec{k}, :,:,:).
\end{equation}
In the above,   the unspecified dimensions of  the structure factor can refer  to imaginary-time  and orbital indices. 

The position of an orbital  $i$  is given by   $\vec{R}_i +   \ve{\delta}_i $.   $\vec{R}_i $ is a point of the Bravais lattice that defines a unit cell,  and  $ \ve{\delta}_i $  labels the orbital in the unit cell. This information is stored in the array
\texttt{Unit\_cell\%Orb\_pos}    detailed in Table~\ref{table:unit_cell}. 

%\subsubsection{The \texttt{Unit\_cell} type}\label{sec:unit}
%This type defines the unit cell.   

\begin{table}[h]
	\begin{center}
   \begin{tabular}{@{} l l l @{}}\toprule
    Variable  & Type & Description \\\midrule
     \hl{\texttt{Norb}}   & Integer    & Number of orbitals.  \\
      \hl{\texttt{N\_coord}} & Integer & Coordination number.  \\
       \hl{\texttt{Orb\_pos(1..Norb,2[3]) }} & Real & Positions of the orbitals as measured from the lattice site.  \\\bottomrule
   \end{tabular}
     \caption{Components of an instance \texttt{Latt\_unit} of the \texttt{Unit\_cell} type.
   The \hl{highlighted} variables have to be specified by the user. Note that for bilayer lattices the second index of the \texttt{Orb\_pos} array  ranges from $1$ to $3$.  } 
    \label{table:unit_cell}
\end{center}
\end{table}

The total  number of orbitals  is then given by \texttt{Ndim=Lattice\%N*Unit\_cell\%Norb}.  To keep track of the orbital and  unit cell structure, it is useful to define arrays 
\texttt{List(Ndim,2)}  and \texttt{Inv\_list(Latt\%N, Unit\_cell\%Norb)}.  For a superindex $x = (i,n) $ labeling the unit cell, i,  and the orbital, n, of a site on the lattice, we have 
\texttt{List(x,1) = i}, \texttt{List(x,2)=n}  and \texttt{ Inv\_list(i,n) = x }. 


%--------------------------------------------------------------------------------------------
\subsection{The observable types \texttt{Obser\_Vec} and \texttt{Obser\_Latt}}\label{sec:obs}
%--------------------------------------------------------------------------------------------

Our definition  of the model includes observables [Eq.~(\ref{eqn:obs_rw})]. We have defined two observable types: \texttt{Obser\_vec}  for an array of \emph{scalar} observables
such as the energy, and  \texttt{Obser\_Latt}   for correlation functions that have the lattice symmetry. In the latter case, translation symmetry can be used to provide improved estimators and to reduce the size of the output.   
We also obtain improved estimators by taking measurements in the imaginary-time interval $[\texttt{LOBS\_ST},\texttt{LOBS\_EN\texttt{}}]$ (see the parameter file in Sec.~\ref{sec:input}) thereby exploiting the invariance under translation in imaginary-time.
Note that the translation symmetries  in space and in time are \emph{broken} for a given  configuration $C$ but restored by the Monte Carlo sampling. 
In general, the user defines size and number of bins in the parameter file, each bin containing a given amount of sweeps. Within a sweep we run sequentially through the HS and Ising fields, from time slice $1$ to time slice $L_{\text{Trotter}}$ and back.  The results of each bin are written to a file  and analyzed at the end of the run.     

To accomplish the reweighting of observables (see Sec.~\ref{sec:reweight}), for each configuration the measured value of an observable is multiplied by the factors \texttt{ZS} and \texttt{ZP}:
\begin{align}
\texttt{ZS} &= \text{sign}(C)\;, \\
\texttt{ZP} &= \frac{e^{-S(C)}} {\Re \left[e^{-S(C)} \right]}\;.
\end{align}
They are computed from the Monte Carlo phase of a configuration,
\begin{equation}\label{eqn:phase}
	\texttt{phase}   =   \frac{e^{-S(C)}}{ \left| e^{-S(C) }\right| }\;,
\end{equation}
which is provided by the main program.
Note that each observable structure also includes the average sign [Eq.~(\ref{eqn:sign_rw})].

%---------------------------------
\subsubsection{Scalar observables}
%---------------------------------

Scalar observables are stored in the data type \texttt{Obser\_vec}, described in Table \ref{table:Obser_vec}. Consider  a variable \texttt{Obs} of type  \texttt{Obser\_vec}.  At the beginning of each bin,  a call to  \texttt{Obser\_Vec\_Init} in the module \texttt{observables\_mod.F90}  will  set   \texttt{Obs\%N=0},   \texttt{Obs\%Ave\_sign=0}  and  \texttt{Obs\%Obs\_vec(:)=0}.  Each time the main  program calls the routine \texttt{Obser}  in the  \texttt{Hamiltonian} module,  the counter \texttt{Obs\%N}   is incremented by one, the sign  (see Eq.~\ref{Sign.eq}) is accumulated in the  variable \texttt{Obs\%Ave\_sign},  and the desired observables (multiplied by the sign and   $\frac{e^{-S(C)}} {\Re \left[e^{-S(C)} \right]}$, see Sec.~\ref{Observables.General})  are accumulated in the vector \texttt{Obs\%Obs\_vec}.  
%
\begin{table}[h]
	\begin{center}
   \begin{tabular}{@{} l l l l @{}}\toprule
    Variable  &  Type      &  Description &  Contribution of configuration $C$ \\\midrule
    \texttt{Obs\%N}                       &  Int.       &    Number of measurements &    $+1$ \\
    \texttt{Obs\%Ave\_sign}               &  Real     	&    Cumulated average sign [Eq.~\eqref{eqn:sign_rw}] & $\text{sign}(C)$  \\
    \texttt{Obs\%Obs\_vec(:)}  		      &  Comp.      &    Cumul. vector of observables [Eq.~\eqref{eqn:obs_rw}] &
           $ \langle \langle \hat{O}(:) \rangle \rangle_{C}\frac{e^{-S(C)}} {\Re \left[e^{-S(C)} \right]} \text{ sign }(C) $ \\
     \texttt{Obs\%File\_Vec}           &  Char.    &    Name of output file  &\\\bottomrule
   \end{tabular}
   \caption{Components of a variable of type \texttt{Obser\_vec} named \texttt{Obs}.}
   % \mycomment{MB $\texttt{Obs\%Phase}$ is not $phase(C)$ but $sign(C)$. And the type of sign could in principle be reduced to integer.   }
         \label{table:Obser_vec}
     \end{center}
\end{table}
%
At the end of the bin, a call to  \texttt{Print\_bin\_Vec}   in  module \texttt{observables\_mod.F90}  will  append the result of the bin in the file  \texttt{File\_Vec}\emph{\_scal}.  Note that this subroutine will automatically append the suffix \emph{\_scal}
to the the filename \texttt{File\_Vec}.
This suffix  is important to allow automatic analysis of the data at the end of the run. 

%------------------------------------------------------------
\subsubsection{ Equal-time and time-displaced correlation functions}
%------------------------------------------------------------

The data type \texttt{Obser\_latt} (see Table~\ref{table:Obser_latt}) is useful for dealing with both equal-time and imaginary-time-displaced correlation functions of the form: 
\begin{align}\label{eqn:s}
	S_{\alpha,\beta}(\vec{k},\tau) =   \frac{1}{N_{\text{unit-cell}}} \sum_{\vec{i},\vec{j}}  e^{-i \vec{k} \cdot \left( \vec{i}-\vec{j}\right) } \left( \langle \hat{O}_{\vec{i},\alpha} (\tau) \hat{O}_{\vec{j},\beta} \rangle  - 
	  \langle \hat{O}_{\vec{i},\alpha} \rangle \langle   \hat{O}_{\vec{j},\beta}  \rangle \right),
\end{align}
where $\alpha$ and $\beta$ are orbital indices and $\vec{i}$ and $\vec{j}$ lattice positions.
%
\begin{table}[h]
	\begin{center}
	\begin{tabular}{@{} l l l l @{}}\toprule
		Variable  &  Type      &  Description &  Contribution of configuration $C$ \\\midrule
		\texttt{Obs\%N}                       &  Int.        &   Number of measurements &    $+1$\\
		\texttt{Obs\%Ave\_sign}  
		&  Real  &    Cumulated sign [Eq.~(\ref{eqn:sign_rw})] & $\text{sign}(C)$  \\
		\texttt{Obs\%Obs\_latt($\vec{i}-\vec{j},$}        & Compl.      &    Cumul.  correl. funct. [Eq.~(\ref{eqn:obs_rw})] &  $ \langle \langle \hat{O}_{\vec{i},\alpha} (\tau) \hat{O}_{\vec{j},\beta} \rangle \rangle_{C} \; \frac{e^{-S(C)}} {\Re \left[e^{-S(C)} \right]}  \text{sign}(C) $ \vspace{-4pt} \\
		\hfill \texttt{$\tau,\alpha,\beta$)} & & & \\
		\texttt{Obs\%Obs\_latt0($\alpha$)}        & Compl.      &    Cumul. expect. value [Eq.~(\ref{eqn:obs_rw})] &   $ \langle \langle \hat{O}_{\vec{i},\alpha} \rangle \rangle_{C}\frac{e^{-S(C)}} {\Re \left[e^{-S(C)} \right]}  \text{ sign }(C) $ \\
		\texttt{Obs\%File\_Latt}           &  Char.    &    Name of output file  &
		\\
		\texttt{Obs\%Latt} & Lattice & Bravais lattice [Tab.~\ref{table:lattice}] \\
		& pointer
		\\
		\texttt{Obs\%Latt\_unit} & Unit\_cell & Unit cell [Tab.~\ref{table:unit_cell}] \\
		& pointer
		\\
		\texttt{Obs\%dtau} & Real & Imaginary time step.
		\\
		\texttt{Obs\%Channel} & Char. & Channel for Maximum Entropy.
		\\\bottomrule
	\end{tabular}
	\caption{Components of a variable of type \texttt{Obser\_latt} named \texttt{Obs}.
	Be aware: When creating the observable with the subroutine \texttt{Obser\_Latt\_make},
	\texttt{Latt} and \texttt{Latt\_unit} don't get copied but linked, meaning changing them after making the 
	observable still affects the observable.}
	\label{table:Obser_latt}
\end{center}
\end{table}
%
Here,  translation symmetry of the Bravais lattice is explicitly taken into account. 
The correlation function splits in a correlated part $S_{\alpha,\beta}^{\mathrm{(corr)}}(\vec{k},\tau)$ and a background part $S_{\alpha,\beta}^{\mathrm{(back)}}(\vec{k})$:
\begin{align}
S_{\alpha,\beta}^{\mathrm{(corr)}}(\vec{k},\tau)
&=
\frac{1}{N_{\text{unit-cell}}} \sum_{\vec{i},\vec{j}}  e^{- i\vec{k} \cdot \left( \vec{i}-\vec{j}\right) }  \langle \hat{O}_{\vec{i},\alpha} (\tau) \hat{O}_{\vec{j},\beta} \rangle \label{eqn:s_corr} \;,\\
\begin{split}
S_{\alpha,\beta}^{\mathrm{(back)}}(\vec{k})
&=
\frac{1}{N_{\text{unit-cell}}} \sum_{\vec{i},\vec{j}}  e^{- i\vec{k} \cdot \left( \vec{i}-\vec{j}\right) }  \langle \hat{O}_{\vec{i},\alpha} \rangle \langle \hat{O}_{\vec{j},\beta} \rangle\\
&=
N_{\text{unit-cell}}\, \langle \hat{O}_{\alpha} \rangle \langle \hat{O}_{\beta} \rangle \, \delta(\vec{k}) \label{eqn:s_back}\;,
\end{split}
\end{align}
where translation invariance in space and time has been exploited to obtain the last line. 
The background part depends only on the expectation value $\langle \hat{O}_{\alpha} \rangle$, for which we use the following estimator 
\begin{equation}\label{eqn:o}
\langle \hat{O}_{\alpha} \rangle \equiv \frac{1}{N_{\text{unit-cell}}} \sum\limits_{\vec{i}} \langle \hat{O}_{\vec{i},\alpha} \rangle\;.
\end{equation}
Consider a variable  \texttt{Obs} of type  \texttt{Obser\_latt}. At the beginning of each bin a call to  \texttt{Obser\_Latt\_Init} in the module \texttt{observables\_mod.F90}  will  initialize  the elements of \texttt{Obs} to zero.    Each time the main program calls the   \texttt{Obser} or  \texttt{ObserT} routines one accumulates $ \langle \langle \hat{O}_{\vec{i},\alpha} (\tau) \hat{O}_{\vec{j},\beta} \rangle \rangle_{C} \; \frac{e^{-S(C)}} {\Re \left[e^{-S(C)} \right]}  \text{sign}(C) $    in  \texttt{Obs\%Obs\_latt($\vec{i}-\vec{j},\tau,\alpha,\beta$)}   
and $ \langle \langle \hat{O}_{\vec{i},\alpha} \rangle \rangle_{C}\frac{e^{-S(C)}} {\Re \left[e^{-S(C)} \right]}  \text{ sign }(C) $  in \texttt{Obs\%Obs\_latt0($\alpha$)}.   At the end of each bin, a call to \texttt{Print\_bin\_Latt} in the module  \texttt{observables\_mod.F90}   will append the result of the bin in the specified  file \texttt{Obs\%File\_Latt}.   Note that the routine  \texttt{Print\_bin\_Latt}  carries out the Fourier transformation and prints the results in $k$-space. 
We have adopted the following naming conventions.
For equal-time observables, defined by having the second dimension of the array  \texttt{Obs\%Obs\_latt($\vec{i}-\vec{j},\tau,\alpha,\beta$)}   set to unity, 
the routine \texttt{Print\_bin\_Latt}  attaches the suffix \emph{\_eq} to \texttt{Obs\%File\_Latt}.  For time-displaced correlation functions we use the suffix \emph{\_tau}.

\subsection{The \texttt{WaveFunction} type} \label{sec:wave_function}

The projective algorithm (Sec.~\ref{sec:defT0}) requires a pair of trial wave functions,  $| \Psi_{T,L/R} \rangle $, for which there is the dedicated  \texttt{WaveFunction} type, defined in the module \texttt{WaveFunction\_mod} as described in Table~\ref{table:wavefunction}.
%
\begin{table}[h]
	\begin{center}
		\begin{tabular}{@{} p{0.13\columnwidth} p{0.1\columnwidth} p{0.71\columnwidth}  @{}}\toprule
			Variable               &  Type     &  Description \\\midrule
			\texttt{WF\%P(:,:)}    &  Complex  &  P is an $\texttt{Ndim}\times\texttt{N\_part}$ matrix, where \texttt{N\_part} is the number of particles\\
			\texttt{WF\%Degen}     &  Real     &  It stores the energy difference between the last occupied and fist un-occupied single particle state and can be used for checking for degeneracy\\\bottomrule
		\end{tabular}
		\caption{Components of a variable of type \texttt{WaveFunction} named \texttt{WF}.}
		\label{table:wavefunction}
	\end{center}
\end{table}
%

The module \texttt{WaveFunction\_mod} also includes the routine \path{WF_overlap(WF_L, WF_R, Z_norm)} for normalizing the right trial wave function \path{WF_R} by the factor \path{Z_norm}, such that $\langle \Psi_{T,L} | \Psi_{T,R} \rangle = 1$.


%------------------------------------------------------------
\subsection{Specification of the Hamiltonian: the \texttt{Hamiltonian} module} 
\label{sec:hamiltonian}
%------------------------------------------------------------


%{\color{red}   This subsection has to be  rewritten the file \path{Hamiltonian_Examples_mod.F90} is legacy. }

The modules \path{Hamiltonian} in the directory \path{$ALF/Prog/Hamiltonians}  define  specific Hamiltonians. This module must contain a set of subroutines that define  the lattice,  the hopping,  the interaction, the observables,   the  trial wave function, and optionally  updating schemes (see Sec.~\ref{sec:updating}). 
In order to implement a user-defined model, only the module \path{Hamiltonian} has to be set up. Accordingly, this documentation focuses almost entirely  on this module and the subprograms it includes.
The remaining parts of the code may hence be treated as a black box.  The  mandatory  elements of the Hamiltonian module   are defined in Table~\ref{table:hamiltonian}).
To simplify  the implementation of a new Hamiltonian, ALF comes with a set of predefined structures (Sec.~\ref{sec:predefined}) which the user can combine together or use as templates.

%
\begin{table}[h]
\begin{center}
 \begin{tabular}{  p{0.25\columnwidth} p{0.5\columnwidth}p{0.15\columnwidth}} 
    Subprogram & Description & Section \\\midrule
    \hl{\texttt{Ham\_Set}}  & Reads in model and lattice parameters from the file \texttt{parameters}.
                                  Sets the Hamiltonian calling the necessary subprograms: \texttt{Ham\_Latt},   \texttt{Ham\_Hop}, \texttt{Ham\_V}  and \texttt{Ham\_Trial}  & 
                                  \ref{sec:hamiltonian},  \ref{sec:model_classes} \\
   \hl{\texttt{Ham\_Latt}}  &  Sets the  \texttt{Lattice}   and the \texttt{Unit\_cell}    as well as the the arrays \texttt{List}  and \texttt{Inv\_list} required
                                            for multiorbital problems    &   \ref{sec:latt}, \ref{U_PV_Ham_latt}  \ref{sec:predefined_lattices} \\
   \hl{\texttt{Ham\_hop}}  & Sets the hopping term  $\hat{\mathcal{H}}_{T}$ (i.e., operator \texttt{Op\_T} ) by calling \texttt{Op\_make} and \texttt{Op\_set}. &
                                          \ref{sec:op},  \ref{U_PV_Ham_hop},  \ref{sec:predefined_hopping_1} \\    
 \hl{\texttt{Ham\_V}}  & Sets the interaction  term  $\hat{\mathcal{H}}_{V}$ (i.e., operator \texttt{Op\_V} ) by calling \texttt{Op\_make} and \texttt{Op\_set}. &
                                          \ref{sec:op},  \ref{U_PV_Ham_V},    \ref{sec:interaction_vertices} \\       
 \hl{\texttt{Ham\_Trial}}  & Sets the trial  wave function for the  projective code  $| \Psi_{T,L/R} \rangle  $ 
                                         specified by the \texttt{Wavefunction} type.  & 
                                          \ref{sec:wave_function},  \ref{U_PV_Ham_Trial},    \ref{sec:predefined_trial_wave_function}\\    
\hl{\texttt{Alloc\_obs}} & Assigns memory storage to the observables. & \ref{sec:obs} , \ref{Alloc_obs_sec},\\
\hl{\texttt{Obser}}      & Computes the scalar and equal-time observables. & \ref{sec:obs},  \ref{sec:EqualTimeobs}, \ref{sec:predefined_observales}   \\
\hl{\texttt{ObserT}}     & Computes time-displaced correlation functions. & \ref{sec:obs}, \ref{sec:TimeDispObs}, \ref{sec:predefined_observales} \\
  \hl{\texttt{S0}}     & Returns the ratio $e^{S_0(C')} /e^{-S_0(C)} $ for a single spin flip  & \ref{sec:S0} \\    
 \hl{\texttt{Global\_move\_tau}} & Generates a global move on a given time slice $\tau$.  This routine is only called if \texttt{Global\_tau\_moves=True} 
and \texttt{N\_Global\_tau>0}.  & \ref{sec:global_space} \\
\hl{\texttt{Overide\_global\_tau\_}} \hl{\texttt{sampling\_parameters}}     & Allows setting \texttt{global\_tau} parameters at run time.  &   \ref{sec:global_space} \\  
 \hl{\texttt{Hamiltonian\_}} \hl{\texttt{set\_nsigma}} &   Sets the initial field configuration.  This   routine is to be modified if one wants to specify the initial configuration. By default the initial configuration is assumed to be random.  &  \\
\hl{\texttt{Global\_move}} &  Handles global moves in time and space   & \ref{sec:global_slice}  \\
\hl{\texttt{Delta\_S0\_global}} & Computes $e^{S_0(C')} /e^{-S_0(C)} $ for a global move  & \ref{sec:global_slice} \\ 
\texttt{Init\_obs}  & Initializes the observables to zero. & \\    
\texttt{Pr\_obs}   & Writes the observables to disk by calling \texttt{Print\_bin} of the \texttt{observables\_mod.F90}  module. &   
 \\\bottomrule   
 \end{tabular}
\caption{Overview of the subprograms of the  module \texttt{Hamiltonian}, contained in the Hamiltonian files used to define various  Hamiltonians. The \hl{highlighted} subroutines may have to be modified by the user. \label{table:hamiltonian} }
\end{center}
\end{table}
%



In order to specify a Hamiltonian, we have to set the matrix representation of the imaginary-time propagators,
$ e^{-\Delta \tau {\bm T}^{(ks)}}$, $e^{  \sqrt{- \Delta \tau  U_k} \eta_{k\tau} {\bm V}^{(ks)} }$ and $e^{  -\Delta \tau s_{k\tau}  {\bm I}^{(ks)}}$, that appear in the 
partition function (\ref{eqn:partition_2}).  For each pair of indices $(k,s)$, these terms have the general form
\begin{equation}\label{eqn:exponent_mat}
\text{Matrix Exponential}=
e^{g \,\phi(\texttt{type})\,\bm{X} }\;.
\end{equation}
In case of the  perfect-square term,  we additionally have to set the constant $\alpha$, see the definition of the operators $\hat{V}^{(k)}$ in Eq.~(\ref{eqn:general_ham_v}).
The data structures which hold all the above information are variables of the type \path{Operator} (see Table \ref{table:operator}). 
For each pair of indices $(k,s)$, we store the following parameters in an \path{Operator} variable:
\begin{itemize}
\item $\vec{P}$ and   $ \bm{O}$   defining the matrix $\bm{X}$ [see Eq.~(\ref{eqn:xeqpdop})],
\item the constants $g$, $\alpha$,
\item optionally: the type \texttt{type} of the discrete fields $\phi$.
\end{itemize}
The latter parameter can take one of three values: Ising (1), discrete HS (2), and real (3), as detailed in Sec.~\ref{sec:fields}.
Note that we have dropped the color index $\sigma$, since the implementation uses the $SU(N_{\mathrm{col}})$ invariance of the Hamiltonian. 

Accordingly, the following data structures fully describe the  Hamiltonian (\ref{eqn:general_ham}):
\begin{itemize}
\item For the hopping Hamiltonian (\ref{eqn:general_ham_t}), we have to set the exponentiated hopping matrices $ e^{-\Delta \tau {\bm T}^{(ks)}}$: \\
In this case $\bm{X}^{(ks)}=\bm{T}^{(ks)}$, and a single variable  \texttt{Op\_T}  describes the operator matrix
\begin{equation}
            \left( \sum_{x,y}^{N_{\mathrm{dim}}} \hat{c}^{\dagger}_x T_{xy}^{(ks)} \hat{c}^{\phantom{\dagger}}_{y}  \right)  \;,
\end{equation} 
where $k=[1, M_{T}]$ and $s=[1, N_{\mathrm{fl}}]$. In the notation of the general expression (\ref{eqn:exponent_mat}), we set $g=-\Delta \tau$ (and $\alpha = 0$).
In case of the hopping matrix, the type variable $\texttt{Op\_T\%type}$  is neglected by the code. 
All in all, the corresponding array of structure variables is  \texttt{Op\_T(M$_T$,N$_{fl}$)}.

\item For the interaction Hamiltonian (\ref{eqn:general_ham_v}), which is of perfect-square type, we have to set the exponentiated matrices $e^{  \sqrt{ -  \Delta \tau  U_k} \eta_{k\tau} {\bm V}^{(ks)} }$:\\
In this case, $\bm{X}  = \bm{V}^{(ks)}$ and a single variable  \texttt{Op\_V}  describes the operator matrix:
\begin{equation}
             \left[ \left( \sum_{x,y}^{N_{\mathrm{dim}}} \hat{c}^{\dagger}_x V_{x,y}^{(ks)} \hat{c}^{\phantom{\dagger}}_{y}  \right)  + \alpha_{ks} \right]  \;,
\end{equation} 
where $k=[1, M_{V}]$ and $s=[1, N_{\mathrm{fl}}]$, $g = \sqrt{-\Delta \tau  U_k}$ and  $\alpha = \alpha_{ks}$. 
The discrete HS decomposition which is used for the perfect-square interaction, is selected by setting the type variable to $\texttt{Op\_V\%type}=2$.
All in all, the required structure variables \texttt{Op\_V} are defined  using the array \texttt{Op\_V(M$_V$,N$_\mathrm{fl}$)}.


\item For the Ising interaction Hamiltonian (\ref{eqn:general_ham_i}), we have to set the exponentiated matrices $e^{  -\Delta \tau s_{k\tau}  {\bm I}^{(ks)}}$:\\
In this case, $\bm{X}  = \bm{I}^{(k,s)} $ and a single variable  \texttt{Op\_V} then  describes the operator matrix:
\begin{equation}
            \left( \sum_{x,y}^{N_{\mathrm{dim}}} \hat{c}^{\dagger}_x I_{xy}^{(ks)} \hat{c}^{\phantom{\dagger}}_{y}  \right)  \;,
\end{equation} 
where $k=[1, M_{I}]$ and $s=[1, N_{\mathrm{fl}}]$ and $g = -\Delta \tau$ (and $\alpha = 0$).
The Ising interaction is specified by setting the type variable  $\texttt{Op\_V\%type=1}$. 
All in all, the required structure variables are contained in the array \texttt{Op\_V(M$_{I}$,N$_\mathrm{fl}$)}.

\item In case of a full interaction [perfect-square term (\ref{eqn:general_ham_v}) and Ising term (\ref{eqn:general_ham_i})],
we  define  the corresponding doubled array \texttt{Op\_V(M$_V$+M$_I$,N$_\mathrm{fl}$)} and set the variables separately for both ranges of the array according to the above.  

\end{itemize}


