% Copyright (c) 2016-2022 The ALF project.
% This is a part of the ALF project documentation.
% The ALF project documentation by the ALF contributors is licensed
% under a Creative Commons Attribution-ShareAlike 4.0 International License.
% For the licensing details of the documentation see license.CCBYSA.

% !TEX root = doc.tex

%\red{Also mention:}\\
%Also available is the \texttt{Hopping\_Matrix\_type}, defined in Sec.~\ref{sec:hopping_type}, used for defining hopping matrices.


To manipulate the relevant physical quantities in a general model, we define a set of corresponding data types. The \texttt{Operator} type (Sec.~\ref{sec:op}) is used  to specify the interaction as well as the hopping.  The handling of the fields is taken care of by the \texttt{Fields} type (Sec.~\ref{sec:fields}). To define a Bravais lattice as well as a unit cell we introduce the \texttt{Lattice}  and \texttt{Unit\_cell} types (Sec.~\ref{sec:latt}).  General  scalar,  equal-time, and time-displaced correlation functions are  handled by the  \texttt{Observable} type (Sec.~\ref{sec:obs}).  For the projective code, we provide a \texttt{WaveFunction} type (Sec.~\ref{sec:wave_function}) to specify the left and right trial wave functions.  The Hamiltonian is then specified in the  \texttt{Hamiltonian}  module (Sec.~\ref{sec:hamiltonian}), making use of the aforementioned types. 



%------------------------------------------------------------
\subsection{The \texttt{Operator} type}\label{sec:op}
%------------------------------------------------------------

The fundamental data structure in the code is the \path{Operator}. It is implemented as a Fortran derived data type designed to efficiently define the Hamiltonian~\eqref{eqn:general_ham}.

Let the matrix $\bm{X}$ of dimension $N_{\mathrm{dim}} \times N_{\mathrm{dim}}$ stand for any of the typically sparse, Hermitian matrices $\bm{T}^{(ks)}$, $\bm{V}^{(ks)}$ and $\bm{I}^{(ks)}$ that define the Hamiltonian.
Furthermore, let $\left\{z_{1},\cdots, z_{N} \right\}$ denote the subset of $N$ indices such that
\begin{equation}
X_{x,y} 
\left\{\begin{matrix}  \neq 0  &  \text{ if }   x,  y  \in \left\{ z_1, \cdots z_N \right\}\\ 
                                 = 0         &  \text{ otherwise. } 
      \end{matrix}\right.
\end{equation}
Usually, we have $N\ll N_{\text{dim}}$.
 We define the $N \times N_{\mathrm{dim}}$ matrices $\bm{P}$  as
\begin{equation}
P_{i,x}=\delta_{z_{i},x}\;,
\end{equation}
where $i \in [1,\cdots, N ]$ and $ x  \in [1,\cdots, N_{\mathrm{dim}}]$. The matrix  $\bm{P}$ selects the non-vanishing entries of $\bm{X}$, 
which are contained in the rank-$N$  matrix $\bm{O}$ defined by:
\begin{equation}\label{eqn:xeqpdop}
\bm{X} =\bm{P}^{T} \bm{O} \bm{P}\;,
\end{equation}
and 
\begin{equation}
X_{x,y} = \sum\limits_{i,j}^{N}  P_{i,x}  O_{i,j} P_{j,y}=\sum\limits_{i,j}^{N} \delta_{z_{i},x}  O_{ij} \delta_{z_{j},y} \;.
\end{equation}
Since  the  $\bm{P}$ matrices have only one non-vanishing entry per column,  they can conveniently be stored as a vector $\vec{P}$, with entries
\begin{equation}
     P_i = z_i.
\end{equation}  
There are  many useful  identities which emerge from this  structure. For example: 
\begin{equation}
	e^{\bm{X}} =  e^{\bm{P}^{T} \bm{O} \bm{P}}   = \sum_{n=0}^{\infty}  \frac{\left( \bm{P}^{T} \bm{O} \bm{P} \right)^n}{n!} = \mathds{1}+ \bm{P}^{T} \left(e^{ \bm{O} }-\mathds{1} \right) \bm{P}\;,
\end{equation}
since 
\begin{equation} 
	 \bm{P} \bm{P}^{T}= \mathds{1}_{N\times N}.
\end{equation}

In the code, we define a structure called \path{Operator} that makes use of the properties described above. This type \path{Operator} bundles the several components, listed in Table \ref{table:operator} and described in the remaining of this section, that are needed to define and use an operator matrix in the program.  

\begin{table}[h!]
	\begin{center}
    \begin{tabular}{@{} l l l @{}}\toprule
    Variable & Type & Description \\\midrule
    \hl{\texttt{Op\_X\%N}}       & \texttt{int}     &  Effective dimension $N$ \\
    \hl{\texttt{Op\_X\%O}}       & \texttt{cmplx}   &  Matrix  $\bm{O}$  of dimension $N \times N$\\
    \hl{\texttt{Op\_X\%P}}       & \texttt{int}     &  Matrix $\bm{P}$  encoded as a vector of dimension $N$\\
    \hl{\texttt{Op\_X\%g}}       & \texttt{cmplx}   &  Coupling strength $g$ \\  
    \hl{\texttt{Op\_X\%alpha}}   & \texttt{cmplx}   &  Constant $\alpha$ \\
    \hl{\texttt{Op\_X\%type}}    & \texttt{int}     &  Sets the type of HS transformation (1: Ising; 2: discrete\\
             &   &    HS for perfect-square term; 3: continuous real field)  \\ 
    \texttt{Op\_X\%diag}         & \texttt{logical} &  True if   $\bm{O}$  is  diagonal  \\
    \texttt{Op\_X\%U}            & \texttt{cmplx}   &  Matrix containing the eigenvectors of $\bm{O}$  \\
    \texttt{Op\_X\%E}            & \texttt{dble}    &  Eigenvalues of $\bm{O}$ \\
    \texttt{Op\_X\%N\_non\_zero} & \texttt{int}     &  Number of non-vanishing eigenvalues of $\bm{O}$  \\
    \texttt{Op\_X\%M\_exp}       & \texttt{cmplx}   &  Stores $ \texttt{M\_exp}(:,:,s) = e^{g  \phi(s,\texttt{type}) \bm{O}(:,:)} $  \\
    \texttt{Op\_X\%E\_exp}       & \texttt{cmplx}   &  Stores $ \texttt{E\_exp}(:,s) = e^{g  \phi(s,\texttt{type}) \bm{E}(:)} $ 
     \\\bottomrule
   \end{tabular}
   \caption{Member variables of the \texttt{Operator}  type. 
   In the left column, the letter \texttt{X} is a placeholder for the letters \texttt{T} and \texttt{V}, 
   indicating hopping and interaction operators, respectively.
   The \hl{highlighted} variables must be specified by the user. \texttt{M\_exp} and \texttt{E\_exp}  are allocated only if  $\texttt{type}=1,2$. 
  %  One will have to specify $N$, $O$, $P$, $g$, $\alpha$ and the type.  The other variables will be automatically generated in the routine \texttt{Op\_Set}.  
    \label{table:operator}}
\end{center}
\end{table}
%



%------------------------------------------------------------
\subsection{Handling of the fields: the \texttt{Fields} type} \label{sec:fields}
%------------------------------------------------------------

The partition function (see Sec.~\ref{sec:method}) consists of terms which, in general, can be written as $\gamma e^{g \phi \bm{X} }$, where $\bm{X}$ denotes an arbitrary operator, $g$ is a constant, and $\gamma$ and $\phi$ are fields. 
The ALF includes three different types of fields:
\begin{itemize}
\item[\texttt{t=1}]   This type is for an  Ising field, therefore $\gamma= 1$ and $ \phi = \pm 1$,
\item[\texttt{t=2}]   This type is for the generic HS  transformation of Eq.~\eqref{HS_squares} where  $\gamma \equiv \gamma(l) $  and $ \phi = \eta(l)$  with $l = \pm 1, \pm 2$ [see Eq.~\eqref{eta_gamma_fields}],
\item[\texttt{t=3}]   This type is for continuous fields, i.e., $\gamma= 1$  and $ \phi  \in \mathbb{R}$.
\end{itemize}
%For an Ising term, we store type \texttt{t=1}, which sets $\gamma_{k,\tau} = 1$ and $\phi_{k,\tau}=s_{k,\tau} = \pm 1$. 
%In the case of a perfect-square term, the fields results from the discrete HS transformation (\ref{HS_squares}) and we store \texttt{t=2}, which sets $\gamma_{k,\tau} = s_{k,\tau}$ and $\phi_{k,\tau}=\eta_{k,\tau}$ ). For continuous real fields $f$ we store \texttt{t=3}, which sets $\gamma_{k,\tau} = 1$ and $\phi_{k,\tau}=f$.

For such auxiliary fields a dedicated type \texttt{Fields} is defined, whose components, listed in Table~\ref{table:Fields}, include the variables \texttt{Field\%f} and \texttt{Field\%t}, which store the field values and types, respectively, and functions such as \texttt{Field\%flip}, which flips the field values randomly. Before using this variable type, the routine \texttt{Fields\_init(Amplitude)} should be called (its argument is optional and the default value is of unity (see Sec.~\ref{sec:sequential}), in order for internal variables such as  $\eta(l)$ and $\gamma(l)$  [see Eq.~\eqref{eta_gamma_fields}] to be initialized.


%
\begin{table}[h!]
	\begin{center}
	\begin{tabular}{@{} p{0.25\columnwidth} p{0.05\columnwidth} p{\dimexpr0.67\columnwidth-2\tabcolsep\relax} @{}}\toprule
		Component                          &                & Description    \\ \midrule
		\textbf{Variable}                  &  \textbf{Type} &  \\ \midrule
		\texttt{Field\%t(1:n\_op)}         &  \texttt{int}  & Sets the HS transformation type (1: Ising; 2: discrete HS for perfect-square term; 3: continuous real field). The index runs through the operator sequence \\
		\texttt{Field\%f(1:n\_op, 1:Ltrot)} & \texttt{dble}  & Defines the auxiliary fields. The first index runs through the operator sequence and the second through the time slices. For \texttt{t=1}, $f = \pm 1$; for \texttt{t=2}, $f = \pm 1, \pm 2$; and for \texttt{t=3}, $f \in \mathbb{R}$ \\
		\texttt{del}                       & \texttt{dble}  & Width $\Delta x$ of box distribution for initial \texttt{t=3} fields, with a default value of \texttt{1} \\
		\texttt{amplitude}                 & \texttt{dble}  & Random flip width for fields of type \texttt{t=3}, defaults to \texttt{1} \vspace{8pt} \\ 
		\multicolumn{2}{@{}l@{}}{ \textbf{Method(arguments)}}         &    \\ \midrule
		%\multicolumn{2}{@{}l@{}}{ \texttt{Fields\_init(del)}}         & Initializes internal variables such as  $\eta(l)$, $\gamma(l)$  [see Eq.~\eqref{eta_gamma_fields}]; the variable $\texttt{del}=\Delta x$ (see above) is optional  \\
		\multicolumn{2}{@{}l@{}}{ \texttt{Field\%make(n\_op,Ltrot)}}  & Reserves memory for the field \\
		\multicolumn{2}{@{}l@{}}{ \texttt{Field\%clear()}}            & Clears field from memory \\
		\multicolumn{2}{@{}l@{}}{ \texttt{Field\%set()}}              & Sets a random configuration \\
		\multicolumn{2}{@{}l@{}}{ \texttt{Field\%flip(n,nt)}}         & Flips the field values randomly  for  field \texttt{n} on time slice \texttt{nt}.    For  \texttt{t=1} it flips the sign of the Ising spin.  For \texttt{t=2} it randomly choose one of the 
		three other values of $l$. For  \texttt{t=3},  \texttt{f = f + amplitude*(ranf() -1/2)} \\
		\multicolumn{2}{@{}l@{}}{ \texttt{Field\%phi(n,nt)}}          & Returns $\phi$ for the \texttt{n}-th operator at the time slice  \texttt{nt} \\
		\multicolumn{2}{@{}l@{}}{ \texttt{Field\%gamma(n,nt)}}        & Returns $\gamma$ for the  \texttt{n}-th operator at the time slice  \texttt{nt}  \\
		\multicolumn{2}{@{}l@{}}{ \texttt{Field\%i(n,nt)}}            & Returns \texttt{Field\%f} rounded to nearest integer (if \texttt{t=1} or \texttt{2}) \\
		\multicolumn{2}{@{} p{0.3\columnwidth} @{}}{ \texttt{Field\%in(Group\_Comm, In\_field)}} &      If  the file  \texttt{confin\_np} (or \texttt{confin\_np.h5}) exists  it reads the field configuration  from this file.   Otherwise if \texttt{In\_field}  is present it sets the fields to   \texttt{In\_field}.   
		 If both  \texttt{confin\_np}(\texttt{.h5})  and  \texttt{In\_field}  are not provided it sets a random field by calling \texttt{Field\%set()}.
		 Here \texttt{np} is the rank number of the process  \\
		\multicolumn{2}{@{}l@{}}{ \texttt{Field\%out(Group\_Comm)}}   & Writes out the field configuration  \\\bottomrule
	\end{tabular}
	\caption{Components of a variable of type \texttt{Fields} named \texttt{Field}.  The routine \texttt{Fields\_init(del)} should be called before the use of this variable type, since it initializes necessary internal variables such as  $\eta(l)$, $\gamma(l)$  [see Eq.~\eqref{eta_gamma_fields}]. Note that \texttt{del}  and \texttt{amplitude}  are private variables of the fields module. The integers \texttt{n\_op} and \texttt{Ltrot} are the number of interacting operators  per time slice and time slices, respectively, \texttt{Group\_Comm} (integer) is an MPI communicator defined by the main program, and the optional \texttt{In\_field} stores the initial field configuration.}
    \end{center}
	\label{table:Fields}
\end{table}
%



%------------------------------------------------------------
\subsection{The \texttt{Lattice} and \texttt{Unit\_cell} types}\label{sec:latt}
%------------------------------------------------------------

ALF's lattice module can generate one- and two-dimensional Bravais lattices.
Both the lattice and the unit cell are defined in the module \texttt{lattices\_v3\_mod.F90} and their components are detailed in Tables \ref{table:lattice} and \ref{table:unit_cell}. 
As  its name suggest the module  \texttt{Predefined\_Latt\_mod.F90} also provides predefined lattices as  described in Sec.~\ref{sec:predefined_lattices}.
The user who wishes to define his/her own lattice has to specify: 1) unit vectors $\vec{a}_1$ and $\vec{a}_2$, 2) the size and shape of the lattice, characterized by the vectors $\vec{L}_1$ and $\vec{L}_2$ and 3) the  unit cell characterized be the number of orbitals and their positions.  The coordination number of the lattice is specified in the \texttt{Unit\_cell} data type.   The lattice is placed on a torus (periodic boundary conditions):
\begin{equation}
	\hat{c}_{\vec{i} + \vec{L}_1 }  =  \hat{c}_{\vec{i} + \vec{L}_2 }  = \hat{c}_{\vec{i}}\;.
\end{equation}
The function call 
\begin{lstlisting}[style=fortran]
Call Make_Lattice( L1, L2, a1, a2, Latt )
\end{lstlisting}
generates the lattice \texttt{Latt} of type \texttt{Lattice}.   The reciprocal lattice vectors $\vec{g}_j$ are defined by:
\begin{equation}
\label{Latt.G.eq}
	\vec{a}_i  \cdot \vec{g}_j = 2 \pi \delta_{i,j}, 
\end{equation}
and the Brillouin zone $BZ$ corresponds to the Wigner-Seitz cell of the lattice. 
With $\vec{k} = \sum_{i} \alpha_i  \vec{g}_i $, the  k-space quantization follows from: 
\begin{equation}
\begin{bmatrix}
	\vec{L}_1 \cdot \vec{g}_1  &  \vec{L}_1 \cdot \vec{g}_2  \\
	\vec{L}_2  \cdot \vec{g_1} & \vec{L}_2 \cdot  \vec{g}_2  
\end{bmatrix}
\begin{bmatrix}
   \alpha_1 \\
   \alpha_2
\end{bmatrix}
=  2 \pi 
\begin{bmatrix}
   n \\
   m
\end{bmatrix}
\end{equation}
such that 
\begin{equation}
\vec{k} =  n \vec{b}_1  + m \vec{b}_2,\; \text{with}
\end{equation}
\begin{align} \label{k.quant.eq}
\vec{b}_1 &= \frac{2 \pi}{ (\vec{L}_1 \cdot \vec{g}_1)  (\vec{L}_2 \cdot  \vec{g}_2 )  - (\vec{L}_1 \cdot \vec{g}_2) (\vec{L}_2  \cdot \vec{g_1} ) }   \left[  (\vec{L}_2 \cdot  \vec{g}_2) \vec{g}_1 -   (\vec{L}_2  \cdot \vec{g_1} ) \vec{g}_2 \right], \nonumber \\ 
\vec{b}_2 &= \frac{2 \pi}{ (\vec{L}_1 \cdot \vec{g}_1)  (\vec{L}_2 \cdot  \vec{g}_2 )  - (\vec{L}_1 \cdot \vec{g}_2) (\vec{L}_2  \cdot \vec{g_1} ) }   
\left[  (\vec{L}_1 \cdot  \vec{g}_1) \vec{g}_2 -   (\vec{L}_1  \cdot \vec{g_2} ) \vec{g}_1 \right].
\end{align}

%
\begin{table}[h!]
	\begin{center}
   \begin{tabular}{@{} l l @{$\;\;$} l @{}}\toprule
    Variable  & Type & Description \\\midrule
     \hl{\texttt{Latt\%a1\_p}, \texttt{Latt\%a2\_p}}   & \texttt{dble}  & Unit vectors $\vec{a}_1$,  $\vec{a}_2$ \\ 
     \hl{\texttt{Latt\%L1\_p}, \texttt{Latt\%L2\_p}}   & \texttt{dble}  & Vectors $\vec{L}_1$, $\vec{L}_2$ that define the topology of the  lattice \\
     									               &                &  Tilted lattices are  thereby possible to implement \\
    \hl{\texttt{Latt\%N}}                              & \texttt{int}   &  Number of lattice points, $N_{\text{unit-cell}}$ \\
    \texttt{Latt\%list}                                & \texttt{int}   &  Maps each lattice point $i=1,\cdots, N_{\text{unit-cell}}$ to a real\\
                                                                    &   &  space vector denoting the position of the unit cell: \\
                                                                    &   & $\vec{R}_i$ = \texttt{list(i,1)}$\vec{a}_1$ + \texttt{list(i,2)}$\vec{a}_2$  $  \equiv i_1  \vec{a}_1 + i_2  \vec{a}_2 $ \\
    \texttt{Latt\%invlist}                             & \texttt{int}   &  Return lattice point from position: \texttt{Invlist}$(i_1,i_2) = i $ \\
    \texttt{Latt\%nnlist}                              & \texttt{int}   &  Nearest neighbor indices: $j = \texttt{nnlist} (i, n_1, n_2) $, \\
                                                       &                &  $n_1, n_2 \in [-1,1] $, $\vec{R}_j = \vec{R}_i + n_1 \vec{a}_1  + n_2 \vec{a}_2 $ \\
   \texttt{Latt\%imj}                                  & \texttt{int}   &  $\vec{R}_{\mathrm{imj}(i,j)}  =  \vec{R}_i -  \vec{R}_j$, with  $\mathrm{imj}, i, j \in  1,\cdots, N_{\text{unit-cell}}$ \\
    \texttt{Latt\%BZ1\_p}, \texttt{Latt\%BZ2\_p}       & \texttt{dble}  & Reciprocal space vectors $\vec{g}_i$   [See Eq.~\eqref{Latt.G.eq}] \\
    \texttt{Latt\%b1\_p}, \texttt{Latt\%b2\_p}          & \texttt{dble}  &  $k$-quantization [See Eq.~\eqref{k.quant.eq}] \\
    \texttt{Latt\%listk}                               & \texttt{int}   &  Maps each reciprocal lattice point $k=1,\cdots, N_{\text{unit-cell}}$\\
                                                                   &    & to a reciprocal space vector\\
                                                                   &    & $\vec{k}_k= \texttt{listk(k,1)} \vec{b}_1 +  \texttt{listk(k,2)} \vec{b}_2  \equiv k_1  \vec{b}_1 +   k_2  \vec{b}_2 $ \\
    \texttt{Latt\%invlistk}                            & \texttt{int}   &   \texttt{Invlistk}$(k_1,k_2) = k $ \\
   \texttt{Latt\%b1\_perp\_p},                         &                & \\ 
   \texttt{Latt\%b2\_perp\_p}                          & \texttt{dble}  &  Orthonormal vectors to $\vec{b}_i$ (for internal use) \\\bottomrule
   \end{tabular}
   \caption{Components of the \texttt{Lattice} type for two-dimensional lattices using as example the default lattice name \texttt{Latt}.
   The \hl{highlighted} variables must be specified by the user.  Other components of \texttt{Lattice} are generated upon calling: \texttt{Call Make\_Lattice(L1, L2, a1, a2, Latt)}. 
    \label{table:lattice}}
\end{center}
\end{table}
%

The \path{Lattice} module also handles the Fourier transformation.  For example,  the  subroutine  \path{Fourier_R_to_K}   carries out the  transformation: 
\begin{equation}
	S(\vec{k}, :,:,:) =  \frac{1}{N_\text{unit-cell}}  \sum_{\vec{i},\vec{j}}   e^{-i \vec{k} \cdot \left( \vec{i}-\vec{j} \right)} S(\vec{i}  - \vec{j}, :,:,:)
\end{equation}
and  \path{Fourier_K_to_R}  the  inverse Fourier transform 
 \begin{equation}
	S(\vec{r}, :,:,:) =  \frac{1}{N_\text{unit-cell}}  \sum_{\vec{k} \in BZ }   e^{ i \vec{k} \cdot \vec{r}} S(\vec{k}, :,:,:).
\end{equation}
In the above,   the unspecified dimensions of  the structure factor can refer  to imaginary-time  and orbital indices. 

The position of an orbital  $i$  is given by   $\vec{R}_i +   \ve{\delta}_i $.   $\vec{R}_i $ is a point of the Bravais lattice that defines a unit cell,  and  $ \ve{\delta}_i $  labels the orbital in the unit cell. This information is stored in the array
\texttt{Unit\_cell\%Orb\_pos}    detailed in Table~\ref{table:unit_cell}. 

%\subsubsection{The \texttt{Unit\_cell} type}\label{sec:unit}
%This type defines the unit cell.   

\begin{table}[h!]
	\begin{center}
   \begin{tabular}{@{} l l l @{}}\toprule
    Variable                              & Type          & Description \\\midrule
    \hl{\texttt{Norb}}                    & \texttt{int}  & Number of orbitals \\
    \hl{\texttt{N\_coord}}                & \texttt{int}  & Coordination number \\
    \hl{\texttt{Orb\_pos(1..Norb,2[3]) }} & \texttt{dble} & Orbitals' positions, measured from the lattice site  \\\bottomrule
   \end{tabular}
     \caption{Components of an instance \texttt{Latt\_unit} of the \texttt{Unit\_cell} type.
   The \hl{highlighted} variables have to be specified by the user. Note that for bilayer lattices the second index of the \texttt{Orb\_pos} array  ranges from $1$ to $3$.  } 
    \label{table:unit_cell}
\end{center}
\end{table}

The total  number of orbitals  is then given by \texttt{Ndim=Lattice\%N*Unit\_cell\%Norb}.  To keep track of the orbital and  unit cell structure, it is useful to define arrays 
\texttt{List(Ndim,2)}  and \texttt{Inv\_list(Latt\%N, Unit\_cell\%Norb)}.  For a superindex $x = (i,n)$ labeling the unit cell, i,  and the orbital, n, of a site on the lattice, we have 
\texttt{List(x,1)=i}, \texttt{List(x,2)=n}  and \texttt{Inv\_list(i,n)=x}. 


%--------------------------------------------------------------------------------------------
\subsection{The observable types \texttt{Obser\_Vec} and \texttt{Obser\_Latt}}\label{sec:obs}
%--------------------------------------------------------------------------------------------

Our definition  of the model includes observables [Eq.~(\ref{eqn:obs_rw})]. We define two observable types: \texttt{Obser\_vec}  for an array of \emph{scalar} observables
such as the energy, and  \texttt{Obser\_Latt}   for correlation functions that have the lattice symmetry. In the latter case, translation symmetry can be used to provide improved estimators and to reduce the size of the output.   
We also obtain improved estimators by taking measurements in the imaginary-time interval $[\texttt{LOBS\_ST},\texttt{LOBS\_EN\texttt{}}]$ (see the parameter file in Sec.~\ref{sec:input}) thereby exploiting the invariance under translation in imaginary-time.
Note that the translation symmetries  in space and in time are \emph{broken} for a given  configuration $C$ but restored by the Monte Carlo sampling. 
In general, the user defines size and number of bins in the parameter file, each bin containing a given amount of sweeps. Within a sweep we run sequentially through the HS and bosonic fields, from time slice $1$ to time slice $L_{\text{Trotter}}$ and back.  The results of each bin are written to a file  and analyzed at the end of the run.     

To accomplish the reweighting of observables (see Sec.~\ref{sec:reweight}), for each configuration the measured value of an observable is multiplied by the factors \texttt{ZS} and \texttt{ZP}:
\begin{align}
\texttt{ZS} &= \sgn(C)\;, \\
\texttt{ZP} &= \frac{e^{-S(C)}} {\Re \left[e^{-S(C)} \right]}\;.
\end{align}
They are computed from the Monte Carlo phase of a configuration,
\begin{equation}\label{eqn:phase}
	\texttt{phase}   =   \frac{e^{-S(C)}}{ \left| e^{-S(C) }\right| }\;,
\end{equation}
which is provided by the main program.
Note that each observable structure also includes the average sign [Eq.~(\ref{eqn:sign_rw})].

%---------------------------------
\subsubsection{Scalar observables}
%---------------------------------

Scalar observables are stored in the data type \texttt{Obser\_vec}, described in Table \ref{table:Obser_vec}. Consider  a variable \texttt{Obs} of type  \texttt{Obser\_vec}.  At the beginning of each bin,  a call to  \texttt{Obser\_Vec\_Init} in the module \texttt{observables\_mod.F90}  will  set   \texttt{Obs\%N=0},   \texttt{Obs\%Ave\_sign=0}  and  \texttt{Obs\%Obs\_vec(:)=0}.  Each time the main  program calls the routine \texttt{Obser}  in the  \texttt{Hamiltonian} module,  the counter \texttt{Obs\%N}   is incremented by one, the sign  [see Eq.~\eqref{Sign.eq}] is accumulated in the  variable \texttt{Obs\%Ave\_sign},  and the desired observables (multiplied by the sign and   $\frac{e^{-S(C)}} {\Re \left[e^{-S(C)} \right]}$, see Sec.~\ref{sec:Observables.General})  are accumulated in the vector \texttt{Obs\%Obs\_vec}.  
%
\begin{table}[h!]
	\begin{center}
   \begin{tabular}{@{} p{0.19\columnwidth} @{ } p{0.06\columnwidth} p{0.39\columnwidth} l @{}}
   \toprule
    Variable                 &  Type          & Description &  Contribution \\
    \midrule
    \texttt{N}               & \texttt{int}   & Number of measurements &    $+1$ \\
    \texttt{Ave\_sign}       & \texttt{dble}  & Cumulated average sign [Eq.~\eqref{eqn:sign_rw}] & $\sgn(C)$  \\
    \texttt{Obs\_vec(:)}  	  & \texttt{cmplx} & Cumulated vector of observables [Eq.~\eqref{eqn:obs_rw}] &
           $ \langle \langle \hat{O}(:) \rangle \rangle_{C}\frac{e^{-S(C)}} {\Re \left[e^{-S(C)} \right]} \sgn(C) $ \\
     \texttt{File\_Vec}      & \texttt{char} & Name of output file  &\\
     \texttt{analysis\_mode} & \texttt{char} & How to analyze the observable &\\
                             &               & Default value: ``identity'' &\\
     \texttt{description(:)} & \texttt{char} & Optional description. Arbitrary number of 64-character lines &\\
     \bottomrule     
   \end{tabular}
   \caption{Components of a variable of type \texttt{Obser\_vec}. The contribution listed is that of each configuration $C$.}
         \label{table:Obser_vec}
     \end{center}
\end{table}
%
At the end of the bin, a call to \texttt{Print\_bin\_Vec} in module \texttt{observables\_mod.F90} will append the result of the bin in the file \texttt{File\_Vec}\emph{\_scal}.  Note that this subroutine will automatically append the suffix \emph{\_scal} to the the filename \texttt{File\_Vec}.
This suffix  is important to facilitate automatic analyses of the data at the end of the run.
Furthermore, the file \texttt{File\_Vec}\emph{\_scal\_info} is created (if it does not exist yet), which contains a string that specifies how to analyze the observable and an optional description.

%------------------------------------------------------------
\subsubsection{Equal-time and time-displaced correlation functions}
%------------------------------------------------------------

The data type \texttt{Obser\_latt} (see Table~\ref{table:Obser_latt}) is useful for dealing with both equal-time and imaginary-time-displaced correlation functions of the form: 
\begin{align}\label{eqn:s}
	S_{\alpha,\beta}(\vec{k},\tau) =   \frac{1}{N_{\text{unit-cell}}} \sum_{\vec{i},\vec{j}}  e^{-i \vec{k} \cdot \left( \vec{i}-\vec{j}\right) } \left( \langle \hat{O}_{\vec{i},\alpha} (\tau) \hat{O}_{\vec{j},\beta} \rangle  - 
	  \langle \hat{O}_{\vec{i},\alpha} \rangle \langle   \hat{O}_{\vec{j},\beta}  \rangle \right),
\end{align}
where $\alpha$ and $\beta$ are orbital indices and $\vec{i}$ and $\vec{j}$ lattice positions.
%
\begin{table}[h!]
	\begin{center}
	\begin{tabular}{@{} p{0.22\columnwidth}  l  p{0.30\columnwidth} p{0.26\columnwidth} @{}}\toprule
		Variable  &  Type      &  Description &  Contribution \\\midrule
		\texttt{Obs\%N}                       &  \texttt{int}        &   Number of measurements &    $+1$\\
		\texttt{Obs\%Ave\_sign} &  \texttt{dble}  &    Cumulated sign [Eq.~(\ref{eqn:sign_rw})] & $\sgn(C)$  \\
		\texttt{Obs\%Obs\_latt($\vec{i}$-$\vec{j},\allowbreak\tau,\alpha,\beta$}        & \texttt{cmplx}      &    Cumulated correlation function [Eq.~(\ref{eqn:obs_rw})] & $ \begin{multlined}[t] \langle \langle \hat{O}_{\vec{i},\alpha} (\tau) \hat{O}_{\vec{j},\beta} \rangle \rangle_{C} \times\\[-0.5ex] \tfrac{e^{-S(C)}} {\Re \left[e^{-S(C)} \right]}  \sgn(C) \end{multlined}$  \\
		\texttt{Obs\%Obs\_latt0($\alpha$)}        & \texttt{cmplx}      &    Cumulated expected value [Eq.~(\ref{eqn:obs_rw})] &  $ \begin{multlined}[t] \langle \langle \hat{O}_{\vec{i},\alpha} \rangle \rangle_{C} \times\\[-0.5ex] \tfrac{e^{-S(C)}} {\Re \left[e^{-S(C)} \right]} \sgn(C) \end{multlined}$ \\
		\texttt{Obs\%File\_Latt}           &   \texttt{char}   &    Name of output file  &
		\\
		\texttt{Obs\%Latt} & \texttt{Lattice}$^*$ & Bravais lattice [Tab.~\ref{table:lattice}] & \\
		\texttt{Obs\%Latt\_unit} & \texttt{Unit\_cell}$^*$ & Unit cell [Tab.~\ref{table:unit_cell}] & \\
		\texttt{Obs\%dtau} & \texttt{dble} & Imaginary time step & \\
		\texttt{Obs\%Channel} & \texttt{char} & Channel for Maximum Entropy &
		\\\bottomrule
	\end{tabular}
	\caption{Components of a variable of type \texttt{Obser\_latt} named \texttt{Obs}.
	Be aware: The types marked with asterisks, $*$, are actually pointers, i.e., when the subroutine \texttt{Obser\_Latt\_make} creates an observable \texttt{Obs}, the variables \texttt{Latt} and \texttt{Latt\_unit} do not get copied but linked, meaning modifying them after the creation of \texttt{Obs} still affects the observable.}
	\label{table:Obser_latt}
\end{center}
\end{table}
%
Here,  translation symmetry of the Bravais lattice is explicitly taken into account. 
The correlation function splits in a correlated part $S_{\alpha,\beta}^{\mathrm{(corr)}}(\vec{k},\tau)$ and a background part $S_{\alpha,\beta}^{\mathrm{(back)}}(\vec{k})$:
\begin{align}
S_{\alpha,\beta}^{\mathrm{(corr)}}(\vec{k},\tau)
&=
\frac{1}{N_{\text{unit-cell}}} \sum_{\vec{i},\vec{j}}  e^{- i\vec{k} \cdot \left( \vec{i}-\vec{j}\right) }  \langle \hat{O}_{\vec{i},\alpha} (\tau) \hat{O}_{\vec{j},\beta} \rangle \label{eqn:s_corr} \;,\\
\begin{split}
S_{\alpha,\beta}^{\mathrm{(back)}}(\vec{k})
&=
\frac{1}{N_{\text{unit-cell}}} \sum_{\vec{i},\vec{j}}  e^{- i\vec{k} \cdot \left( \vec{i}-\vec{j}\right) }  \langle \hat{O}_{\vec{i},\alpha} \rangle \langle \hat{O}_{\vec{j},\beta} \rangle\\
&=
N_{\text{unit-cell}}\, \langle \hat{O}_{\alpha} \rangle \langle \hat{O}_{\beta} \rangle \, \delta(\vec{k}) \label{eqn:s_back}\;,
\end{split}
\end{align}
where translation invariance in space and time has been exploited to obtain the last line. 
The background part depends only on the expectation value $\langle \hat{O}_{\alpha} \rangle$, for which we use the following estimator 
\begin{equation}\label{eqn:o}
\langle \hat{O}_{\alpha} \rangle \equiv \frac{1}{N_{\text{unit-cell}}} \sum\limits_{\vec{i}} \langle \hat{O}_{\vec{i},\alpha} \rangle\;.
\end{equation}
Consider a variable  \texttt{Obs} of type  \texttt{Obser\_latt}. At the beginning of each bin a call to  \texttt{Obser\_Latt\_Init} in the module \texttt{observables\_mod.F90}  will  initialize  the elements of \texttt{Obs} to zero.    Each time the main program calls the   \texttt{Obser} or  \texttt{ObserT} routines one accumulates $ \langle \langle \hat{O}_{\vec{i},\alpha} (\tau) \hat{O}_{\vec{j},\beta} \rangle \rangle_{C} \allowbreak \frac{e^{-S(C)}} {\Re \left[e^{-S(C)} \right]}  \sgn(C) $    in  \texttt{Obs\%Obs\_latt($\vec{i}-\vec{j},\tau,\alpha,\beta$)}   
and $ \langle \langle \hat{O}_{\vec{i},\alpha} \rangle \rangle_{C} \frac{e^{-S(C)}} {\Re \left[e^{-S(C)} \right]}\cdot \allowbreak \sgn(C) $  in \texttt{Obs\%Obs\_latt0($\alpha$)}.   At the end of each bin, a call to \texttt{Print\_bin\_Latt} in the module  \texttt{observables\_mod.F90}   will append the result of the bin in the specified  file \texttt{Obs\%File\_Latt}.   Note that the routine  \texttt{Print\_bin\_Latt}  carries out the Fourier transformation and prints the results in $k$-space. 
We have adopted the following naming conventions.
For equal-time observables, defined by having the second dimension of the array  \texttt{Obs\%Obs\_latt($\vec{i}-\vec{j},\tau,\alpha,\beta$)}   set to unity, 
the routine \texttt{Print\_bin\_Latt}  attaches the suffix \emph{\_eq} to \texttt{Obs\%File\_Latt}.  For time-displaced correlation functions we use the suffix \emph{\_tau}. 
Furthermore, \texttt{Print\_bin\_Latt} will create a corresponding info file with suffix \emph{\_eq\_info} or \emph{\_tau\_info}, if not already present. The info file contains the channel, number of imaginary time steps, length of one imaginary time step, unit cell and the vectors defining the Bravais lattice.

\subsection{The \texttt{WaveFunction} type} \label{sec:wave_function}

The projective algorithm (Sec.~\ref{sec:defT0}) requires a pair of trial wave functions,  $| \Psi_{T,L/R} \rangle $, for which there is the dedicated  \texttt{WaveFunction} type, defined in the module \texttt{WaveFunction\_mod} as described in Table~\ref{table:wavefunction}.
%
\begin{table}[h!]
	\begin{center}
		\begin{tabular}{@{} p{0.12\columnwidth} p{0.07\columnwidth} p{0.75\columnwidth}  @{}}\toprule
			Variable               & Type           &  Description \\\midrule
			\texttt{WF\%P(:,:)}    & \texttt{cmplx} &  P is an $\texttt{Ndim}\times\texttt{N\_part}$ matrix, where \texttt{N\_part} is the number of particles\\
			\texttt{WF\%Degen}     & \texttt{dble}  &  It stores the energy difference between the last occupied and first unoccupied single particle state and can be used to check for degeneracy\\\bottomrule
		\end{tabular}
		\caption{Components of a variable of type \texttt{WaveFunction} named \texttt{WF}.}
		\label{table:wavefunction}
	\end{center}
\end{table}
%

The module \texttt{WaveFunction\_mod} also includes the routine \texttt{WF\_overlap(WF\_L, WF\_R, Z\_norm)} for normalizing the right trial wave function \texttt{WF\_R} by the factor \texttt{Z\_norm}, such that $\langle \Psi_{T,L} | \Psi_{T,R} \rangle = 1$.


%------------------------------------------------------------
\subsection{Specification of the Hamiltonian: the \texttt{Hamiltonian} module} 
\label{sec:hamiltonian}
%------------------------------------------------------------


%{\color{red}   This subsection has to be  rewritten the file \path{Hamiltonian_Examples_mod.F90} is legacy. }

The module \path{Hamiltonian_main} in \path{Prog/Hamiltonian_main_mod.F90} defines the interface for all model-specific variables and subroutines needed by the Monte Carlo algorithm, like the hopping,  the interaction, the observables, the trial wave function, and optionally updating schemes (see Sec.~\ref{sec:updating}).
All Hamiltonians (which is the term we are using for an encapsulated model definition) are derived from this main Hamiltonian.
In order to implement a new user-defined Hamiltonian, one only has to set up a single submodule of the module \path{Hamiltonian_main}. Accordingly, this documentation focuses almost entirely on this module and how to derive a new model from it.
The remaining parts of the code may hence be treated as a black box.

Table~\ref{table:hamiltonian_variables} shows all variables declared in \path{Hamiltonian_main}, they fully define the model. Note that the procedures listed in Table~\ref{table:hamiltonian_procedures} are part of the variable \path{ham}.

To define a new Hamiltonian called \emph{New\_model}, one has to do two things:
\begin{enumerate}
\item Add a new line \emph{New\_model} to the file \path{Prog/Hamiltonians.list}
\item Write the new submodule in \texttt{Prog/Hamiltonians/Hamiltonian\_\emph{New\_model}\_smod.F90}
\end{enumerate}

In this new submodule the user can redefine the procedures listed in Table~\ref{table:hamiltonian_procedures}, those have to be bound to a new type, which is derived from the Hamiltonian object \path{ham_base}. The submodule has access to all variables defined in \path{Hamiltonian_main}, while all variables defined in the submodule are encapsulated. To expose the new Hamiltonian, the user has to define 
\begin{lstlisting}[style=fortran]
module Subroutine Ham_Alloc_New_model
  allocate(ham_New_model::ham)
end Subroutine Ham_Alloc_New_model
\end{lstlisting}
where \path{ham_New_model} is the name of the new type derived from \path{ham_base}. The rest of the linking is done automatically through the entry in \path{Prog/Hamiltonians.list}.

Hamiltonian variables to be read in through the parameters file should be written in a specific format, since they will be parsed at compile time and subroutines for reading from parameters file and writing the HDF5 file will be automatically generated.
For each namelist, there has to be block of this form:
\begin{lstlisting}[style=fortran]
!#PARAMETERS START# <namelist_name>
<var1_type> :: <var1_name> = <var1_default>  ! <var1_description>
<var2_type> :: <var2_name> = <var2_default>  ! <var2_description>
...
!#PARAMETERS END#
\end{lstlisting}

Each of those ``namelist specifications'' starts with a line containing \path{#PARAMETERS START#} and end with a line containing \path{#PARAMETERS END#}. The namelist name has to be written after \path{#PARAMETERS START#} on the same line. The variable type specification \path{<varX_type>} should be either \path{real}, \path{integer}, \path{character} or \path{logical}, declared as \path{real(Kind=Kind(0.d0))}, \path{integer}, \path{character(len=64)} or \path{logical} respectively. Each variable needs to have a default value. The description of the parameters is optional. A variable can be commented out, but will still be parsed to be read from parameters. This is to facilitate reading of variables that are already defined in \path{Hamiltonian_main}, e.g. \path{N_SUN}.

For example, a namelist called \path{my_parameter_list} containing \path{N_SUN} and \path{Beta} could look like:
\begin{lstlisting}[style=fortran]
!#PARAMETERS START# my_parameter_list
!Integer              :: N_SUN = 2
real(Kind=Kind(0.d0)) :: Beta  = 5.d0  ! Inverse temperature
!#PARAMETERS END#
\end{lstlisting}
The parsing script \path{parse_ham.py} in \path{Prog/} has the option \path{--test_file} for testing the namelist specifications, e.g. calling:
\begin{lstlisting}[style=bash]
./parse_ham.py --test_file Hamiltonians/Hamiltonian_New_model_smod.F90
\end{lstlisting}
prints out the results from parsing for manual checking. We recommend doing this after every change in the namelist.

During compilation, the file \texttt{Hamiltonian\_\emph{New\_model}\_read\_write\_parameters.F90} containing the subroutines \path{read_parameters} and \path{write_parameters_hdf5} is generated automatically. The former subroutine can be called in \path{ham_set}, while the latter has to be bound to \path{ham_New_model} through:
\begin{lstlisting}[style=fortran]
#ifdef HDF5
        procedure, nopass :: write_parameters_hdf5
#endif
\end{lstlisting}

To help creating a new Hamiltonian, we provide a template \path{Prog/Hamiltonians/Hamiltonian_##NAME##_smod.F90}, which can be copied to \texttt{Prog/Hamiltonians/Hamiltonian\_\emph{New\_model}\_smod.F90} before being modified.
To simplify  the implementation of a new Hamiltonian, ALF comes with a set of predefined structures (Sec.~\ref{sec:predefined}) which the user can combine together or use as templates.


\begin{table}[h!]
	\begin{center}
   \begin{tabular}{@{} p{0.2\columnwidth} p{0.2\columnwidth} p{0.54\columnwidth}  @{}}
%   lll}
   \toprule
    Public Variable           &  Type                     & Description \\
    \midrule
    \texttt{ham}              & \path{class(ham_base)} & Hamiltonian object. 
       All model dependent procedures are attached to this variable
       (see Table~\ref{table:hamiltonian_procedures}). \\
    \hl{\texttt{Op\_V}}       & \path{Operator}         & Interaction \\
    \hl{\texttt{Op\_T}}       & \path{Operator}         & Hopping \\
    \hl{\texttt{WF\_L}}       & \path{WaveFunction}     & Left trial wave function \\
    \hl{\texttt{WF\_R}}       & \path{WaveFunction}     & Right trial wave function \\
    \texttt{nsigma}           & \path{Fields}           & Fields \\
    \hl{\texttt{Ndim}}        & \path{int}              & Number of sites \\
    \hl{\texttt{N\_Fl}}       & \path{int}              & Number of flavors \\
    \hl{\texttt{N\_SUN}}      & \path{int}              & Number of colors \\
    \hl{\texttt{Ltrot}}       & \path{int}              & Total number of trotter silces \\
    \hl{\texttt{Thtrot}}      & \path{int}              & Number of trotter slices reserved for projection \\
    \hl{\texttt{Projector}}   & \path{logical}          & Enable projector code \\
    \texttt{Group\_Comm}      & \path{int}              & Group communicator for MPI \\
    \hl{\texttt{Symm}}        & \path{logical}          & Symmetric trotter \\
     \bottomrule
   \toprule
    Private Variable           &  Type                & Description \\
    \midrule
    \texttt{Obs\_scal}         & \path{Obser_Vec}  & 
       Storage for measured scalar observables \\
    \texttt{Obs\_eq}           & \path{Obser_Latt} & 
       Storage for measured equal time correlations \\
    \texttt{Obs\_tau}          & \path{Obser_Latt} & 
       Storage for measured time displaced correlations \\
     \bottomrule
   \end{tabular}
   \caption{List of the public and private variables declared in the module \texttt{Hamiltonian}. The \hl{highlighted} variables have to be set in the subroutine \texttt{ham\_set}.}
         \label{table:hamiltonian_variables}
     \end{center}
\end{table}

%
\begin{table}[h!]
\begin{center}
 \begin{tabular}{ p{0.25\columnwidth} p{0.51\columnwidth} p{0.15\columnwidth} } 
   Procedure & Description & Section \\\midrule
   \path{Ham_Set}  & 
      Reads in model and lattice parameters from the file \texttt{parameters}.
      Sets the Hamiltonian, which is commonly split up into subroutines \texttt{Ham\_Latt},           
      \texttt{Ham\_Hop}, \texttt{Ham\_V}  and \texttt{Ham\_Trial}  & 
         \ref{sec:hamiltonian},  \ref{sec:model_classes} \\
      & \texttt{Ham\_Latt}: \hspace{3pt} Sets the \texttt{Lattice} and the \texttt{Unit\_cell} 
      as well as the the arrays \texttt{List}  and \texttt{Inv\_list} required
      for multiorbital problems    &  
         \ref{sec:latt}, \ref{U_PV_Ham_latt}  \ref{sec:predefined_lattices} \\
      & \texttt{Ham\_hop}: \hspace{3pt} Sets the hopping term  $\hat{\mathcal{H}}_{T}$ 
      (i.e., operator \texttt{Op\_T}) by calling \texttt{Op\_make} and \texttt{Op\_set} &
         \ref{sec:op},  \ref{U_PV_Ham_hop},  \ref{sec:predefined_hopping_1} \\    
      & \texttt{Ham\_V}: \hspace{3pt} Sets the interaction term 
      $\hat{\mathcal{H}}_{V}$ (i.e., operator \texttt{Op\_V}) by calling \texttt{Op\_make}
      and \texttt{Op\_set} &
         \ref{sec:op}, \ref{U_PV_Ham_V}, \ref{sec:interaction_vertices} \\       
      & \texttt{Ham\_Trial}: \hspace{3pt}
      Sets the trial  wave function for the  projective code $| \Psi_{T,L/R} \rangle  $ 
      specified by the \texttt{Wavefunction} type  & 
         \ref{sec:wave_function}, \ref{U_PV_Ham_Trial}, 
         \ref{sec:predefined_trial_wave_function}\\    
   \texttt{Alloc\_obs} & 
      Assigns memory storage to the observable & 
         \ref{sec:obs} , \ref{Alloc_obs_sec}\\
   \texttt{Obser}      &
      Computes the scalar and equal-time observables &
         \ref{sec:obs},  \ref{sec:EqualTimeobs}, \ref{sec:predefined_observales}   \\
   \texttt{ObserT}     &
      Computes time-displaced correlation functions &
         \ref{sec:obs}, \ref{sec:TimeDispObs}, \ref{sec:predefined_observales} \\
   \texttt{S0}     &
      Returns the ratio $e^{S_0(C')} /e^{-S_0(C)} $ for a single spin flip  &
         \ref{sec:S0} \\    
   \texttt{Global\_move\_tau} &
      Generates a global move on a given time slice $\tau$.  This routine is only called
      if \texttt{Global\_tau\_moves=True} and \texttt{N\_Global\_tau>0} & 
         \ref{sec:global_space} \\
   \texttt{Overide\_global\_tau\_sampling\_parameters}     &
      Allows setting \texttt{global\_tau} parameters at run time  &
         \ref{sec:global_space} \\  
   \texttt{Hamiltonian\_set\_nsigma} &
      Sets the initial field configuration. This routine is to be modified if one wants
      to specify the initial configuration. By default the initial configuration is
      assumed to be random  &  \\
   \texttt{Global\_move} &  
      Handles global moves in time and space &
         \ref{sec:global_slice}  \\
   \texttt{Delta\_S0\_global} &
      Computes $e^{S_0(C')} /e^{-S_0(C)} $ for a global move  &
         \ref{sec:global_slice} \\ 
   \texttt{Init\_obs}  &
      Initializes the observables to zero. Usually, this doesn't have to be modified. & \\    
   \texttt{Pr\_obs}   &
      Writes the observables to disk by calling
      \texttt{Print\_bin} of the \texttt{Observables} module.
      Usually, this doesn't have to be modified. &   
 \\\bottomrule   
 \end{tabular}
\caption{Typebound procedures bound to type \texttt{ham\_base}. To define a new model, at least \texttt{Ham\_Set} has to be overloaded in the Hamiltonian submodule. For measurements \texttt{Alloc\_obs}, \texttt{Obser} (and \texttt{ObserT} for time displaced observables) are necessary.
The other procedures are needed for optional features.
\label{table:hamiltonian_procedures} }
\end{center}
\end{table}
%



In order to specify a Hamiltonian, we have to set the matrix representation of the imaginary-time propagators,
$ e^{-\Delta \tau {\bm T}^{(ks)}}$, $e^{  \sqrt{- \Delta \tau  U_k} \eta_{k\tau} {\bm V}^{(ks)} }$ and $e^{  -\Delta \tau s_{k\tau}  {\bm I}^{(ks)}}$, that appear in the 
partition function (\ref{eqn:partition_2}).  For each pair of indices $(k,s)$, these terms have the general form
\begin{equation}\label{eqn:exponent_mat}
\text{Matrix Exponential}=
e^{g \,\phi(\texttt{type})\,\bm{X} }\;.
\end{equation}
In case of the  perfect-square term,  we additionally have to set the constant $\alpha$, see the definition of the operators $\hat{V}^{(k)}$ in Eq.~(\ref{eqn:general_ham_v}).
The data structures which hold all the above information are variables of the type \path{Operator} (see Table \ref{table:operator}). 
For each pair of indices $(k,s)$, we store the following parameters in an \path{Operator} variable:
\begin{itemize}
\item $\vec{P}$ and   $ \bm{O}$   defining the matrix $\bm{X}$ [see Eq.~(\ref{eqn:xeqpdop})],
\item the constants $g$, $\alpha$,
\item optionally: the type \texttt{type} of the discrete fields $\phi$.
\end{itemize}
The latter parameter can take one of three values: Ising (1), discrete HS (2), and real (3), as detailed in Sec.~\ref{sec:fields}.
Note that we have dropped the color index $\sigma$, since the implementation uses the SU($N_{\mathrm{col}}$) invariance of the Hamiltonian. 

Accordingly, the following data structures fully describe the  Hamiltonian (\ref{eqn:general_ham}):
\begin{itemize}
\item For the hopping Hamiltonian (\ref{eqn:general_ham_t}), we have to set the exponentiated hopping matrices $ e^{-\Delta \tau {\bm T}^{(ks)}}$: \\
In this case $\bm{X}^{(ks)}=\bm{T}^{(ks)}$, and a single variable  \texttt{Op\_T}  describes the operator matrix
\begin{equation}
            \left( \sum_{x,y}^{N_{\mathrm{dim}}} \hat{c}^{\dagger}_{xs} T_{xy}^{(ks)} \hat{c}^{\phantom{\dagger}}_{ys}  \right)  \;,
\end{equation} 
where $k=[1, M_{T}]$ and $s=[1, N_{\mathrm{fl}}]$. In the notation of the general expression (\ref{eqn:exponent_mat}), we set $g=-\Delta \tau$ (and $\alpha = 0$).
In case of the hopping matrix, the type variable  takes its default value  $\texttt{Op\_T\%type}=0$. 
All in all, the corresponding array of structure variables is  \texttt{Op\_T(M$_T$,N$_{fl}$)}.

\item For the interaction Hamiltonian (\ref{eqn:general_ham_v}), which is of perfect-square type, we have to set the exponentiated matrices $e^{  \sqrt{ -  \Delta \tau  U_k} \eta_{k\tau} {\bm V}^{(ks)} }$:\\
In this case, $\bm{X}  = \bm{V}^{(ks)}$ and a single variable  \texttt{Op\_V}  describes the operator matrix:
\begin{equation}
             \left[ \left( \sum_{x,y}^{N_{\mathrm{dim}}} \hat{c}^{\dagger}_{xs} V_{x,y}^{(ks)} \hat{c}^{\phantom{\dagger}}_{ys}  \right)  + \alpha_{ks} \right]  \;,
\end{equation} 
where $k=[1, M_{V}]$ and $s=[1, N_{\mathrm{fl}}]$, $g = \sqrt{-\Delta \tau  U_k}$ and  $\alpha = \alpha_{ks}$. 
The discrete HS decomposition which is used for the perfect-square interaction, is selected by setting the type variable to $\texttt{Op\_V\%type}=2$.
All in all, the required structure variables \texttt{Op\_V} are defined  using the array \texttt{Op\_V(M$_V$,N$_\mathrm{fl}$)}.

\item For the bosonic interaction Hamiltonian (\ref{eqn:general_ham_i}), we have to set the exponentiated matrices $e^{  -\Delta \tau s_{k\tau}  {\bm I}^{(ks)}}$:\\
In this case, $\bm{X}  = \bm{I}^{(k,s)} $ and a single variable  \texttt{Op\_V} then  describes the operator matrix:
\begin{equation}
            \left( \sum_{x,y}^{N_{\mathrm{dim}}} \hat{c}^{\dagger}_{xs} I_{xy}^{(ks)} \hat{c}^{\phantom{\dagger}}_{ys}  \right)  \;,
\end{equation} 
where $k=[1, M_{I}]$ and $s=[1, N_{\mathrm{fl}}]$ and $g = -\Delta \tau$ (and $\alpha = 0$).
It this operator couples to an Ising field, we specify the  type variable  $\texttt{Op\_V\%type=1}$.    On the other hand, if it couples to a scalar field (i.e. real number)   then we 
specify $\texttt{Op\_V\%type=3}$.
All in all, the required structure variables are contained in the array \texttt{Op\_V(M$_{I}$,N$_\mathrm{fl}$)}.

\item In case of a full interaction [perfect-square term (\ref{eqn:general_ham_v}) and bosonic term (\ref{eqn:general_ham_i})],
we  define  the corresponding doubled array \texttt{Op\_V(M$_V$+M$_I$,N$_\mathrm{fl}$)} and set the variables separately for both ranges of the array according to the above.  

\end{itemize}


