% !TEX root = doc.tex
% Copyright (c) 2017 The ALF project.
% This is a part of the ALF project documentation.
% The ALF project documentation by the ALF contributors is licensed
% under a Creative Commons Attribution-ShareAlike 4.0 International License.
% For the licensing details of the documentation see license.CCBYSA.
%
%------------------------------------------------------------
\subsection{Pseudocode description}\label{sec:pseudocode}
%------------------------------------------------------------

The Monte Carlo algorithm as implemented in ALF can be summarized as follows:

%\begin{algorithm}[H]
\begin{breakablealgorithm}
	\caption{Basic structure of the QMC implementation in \texttt{Prog/main.f90}}
	\label{alg:1}
	\begin{algorithmic}[1]
		
		\LineComment{\textsc{Initialization}}
		\State{\textbf{call} Ham\_Set}\Comment{Set the Hamiltonian and the lattice}
		\State{\textbf{call} Fields\_Init}\Comment{Set the auxiliary fields}
		\State{\textbf{call} Nsigma\%in}\Comment{Read in an auxiliary-field configuration or generate it randomly}
		\For{$n=L_{\text{Trotter}}$ to $1$} \Comment{Fill the storage needed for the first actual MC sweep}
		\State{\textbf{call} Wrapul}\Comment{Compute propagation matrices and store them at stabilization points}
		\EndFor
		\vspace{1.5ex}
		
		\LineComment{\textsc{Monte Carlo run}}
		\For{$n_{\text{bc}}=1$ to $N_{\text{bin}}$}\Comment{Loop over bins. The bin defines the unit of Monte Carlo time}
		
		\For{$n_{\text{sw}}=1$ to $N_{\text{sweep}}$}\Comment{\parbox[t]{0.6\linewidth}{Loop over sweeps. Each sweep updates twice (upward and downward in imaginary time) the space-time lattice of auxiliary fields}}
		
		\If{Tempering}
		\State{\textbf{call} Exchange\_Step} \Comment{Perform exchange step in a parallel tempering run}
		\EndIf
		\If{Global\_moves}
		\State{\textbf{call} Global\_Updates} \Comment{Perform chosen global updates}
		\EndIf
		\vspace{1.5ex}
		
		\LineComment{\textsc{Upward sweep}}
		\For{$n_{\tau}=1$ to $L_{\text{Trotter}}$}
		\State{\textbf{call} Wrapgrup}\Comment{\parbox[t]{0.6\linewidth}{\textsc{Propagate} Green function from $n_{\tau}-1$ to $n_{\tau}$, and compute its new estimate at $n_{\tau}$, using sequential updates}}
		\vspace{1.5ex}
		
		\If{$n_{\tau}$ = stabilization point in imaginary time} \Comment{\textsc{Stabilize}}
		\State{\textbf{call} Wrapur}\Comment{Compute propagation from previous stabilization point to $n_{\tau}$} 
		
		\LineComment{Storage management:\\
			-- Read from storage: propagation from $L_{\text{Trotter}}$ to $n_{\tau}$\\
			-- Write to storage: the just computed propagation}
		
		\State{\textbf{call} CGR} \Comment{Recalculate the Green function at time $n_{\tau}$ in a stable way}
		\State{\textbf{call} Control\_PrecisionG}\Comment{Compare propagated and recalculated Green fct.}
		\EndIf
		\vspace{1.5ex}
		
		\If{$n_{\tau} \in [\text{Lobs\_st}, \text{Lobs\_en}]$} \Comment{\textsc{Measure} the equal-time observables}
		\State{\textbf{call} Obser}
		\EndIf
		\EndFor
		\vspace{1.5ex}
		
		\LineComment{\textsc{Downward sweep}}
		\For{$n_{\tau}=L_{\text{Trotter}}$ to $1$}
		\FirstLineComment{Same steps as for the upward sweep (propagation and estimate update, stabilization, equal-time measurements) now downwards in imaginary time}
		\EndFor
		\vspace{1.5ex}

		\LineComment{\textsc{Measure} time-displaced observables}
		\If{$n_{\tau} = 1$}
		\State{\textbf{call} Tau\_m}
		\EndIf
		
		\EndFor
		\vspace{1.5ex}
		
		\State{\textbf{call} Pr\_obs} \Comment{Calculate and write to disk measurement averages for the current bin}
		\State{\textbf{call} Nsigm\%out}\Comment{Write auxiliary field configuration to disk}
		
		\EndFor
		
	\end{algorithmic}
\end{breakablealgorithm}
%\end{algorithm}

