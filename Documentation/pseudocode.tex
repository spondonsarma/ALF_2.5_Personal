% !TEX root = doc.tex
% Copyright (c) 2017 The ALF project.
% This is a part of the ALF project documentation.
% The ALF project documentation by the ALF contributors is licensed
% under a Creative Commons Attribution-ShareAlike 4.0 International License.
% For the licensing details of the documentation see license.CCBYSA.
%
%------------------------------------------------------------
\subsection{Pseudo code description}\label{sec:pseudocode}
%------------------------------------------------------------
%
The following pseudo code describes the main structure of the quantum Monte Carlo program (see \path{Prog/main.f90}):
\lstset{style=fortran_pseudo_code}
\begin{lstlisting}

! Set the Hamiltonian and the lattice:
call ham_set

! Read in an auxiliary-field configuration or generate it randomly:
call confin

! This loop fills the storage, needed for the first actual Monte Carlo sweep:
do ntau = ltrot, 1, -1 
   ! Compute propagation matrices and store them at the stabilization points:
   call wrapul 
enddo

! Loop over bins. The bin defines the unit of Monte Carlo time:
do nbc = 1, nbin 

   ! Loop over sweeps. Each sweeps updates twice (sweeping upward and downward in time)
   ! the whole space-time lattice of auxiliary fields:
   do nsw = 1, nsweep 
   
      ! Upward sweep:
      do ntau = 1, ltrot
      
         ! Propagate the Green function from time ntau -1 to ntau, 
         ! and compute new estimate (using sequential update scheme) of Green at ntau: 
         call wrapgrup
         
         ! Stabilization:      
         if (ntau == stabilization point)
            ! Compute propagation matrix from previous stabilization point to ntau: 
            call wrapur
            ! Read from storage: propagation from ltrot to ntau
            ! Write to storage : the just computed propagation 
                        
            ! Recalculate Green function at time ntau in a stable way:
            call cgr
            
            ! Check the precision between propagated and recalculated Green function
         endif
        
         ! Measure observables:
         call obser
      enddo
      
      ! Downward sweep:
      do ntau = ltrot, 1, -1
         ! Repeat the above steps (update, propagation, stabilization, measurements) 
         ! for the downward direction in imaginary time
      enddo
      
   enddo ! Loop over sweeps
    
   ! Calculate averages of the measurements of the previous bin and write to disk
   ! Write auxiliary-field configuration to disk
   
enddo ! Loop over bins        

\end{lstlisting}
