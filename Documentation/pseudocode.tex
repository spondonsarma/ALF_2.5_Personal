% !TEX root = doc.tex
% Copyright (c) 2017 The ALF project.
% This is a part of the ALF project documentation.
% The ALF project documentation by the ALF contributors is licensed
% under a Creative Commons Attribution-ShareAlike 4.0 International License.
% For the licensing details of the documentation see license.CCBYSA.
%
%------------------------------------------------------------
\subsection{Pseudo code description}\label{sec:pseudocode}
%------------------------------------------------------------
%
\begin{mdframed}[frametitle={Basic structure of the auxiliary field QMC implementation (\path{Prog/main.f90}):}]
{\setlength{\parindent}{0pt}
Set the Hamiltonian and the lattice:\\
\textbf{Call} ham\_set\\
Read in an auxiliary-field configuration or generate it randomly:\\
\textbf{Call} confin\\

This loop fills the storage, needed for the first actual Monte Carlo sweep:\\
\textbf{Do} \texttt{ntau} from  \texttt{ltrot} to \texttt{1}\\
\hspace*{1em} Compute propagation matrices and store them at the stabilization points:\\
\hspace*{1em} \textbf{Call} wrapul\\
\textbf{Enddo}\\
 
Loop over bins: \\
\textbf{Do} \texttt{nbc} from  \texttt{1} to \texttt{nbin} \\
\hspace*{1em} Loop over sweeps. Each sweep updates twice (sweeping upward and downward in time)\\
\hspace*{1em} the whole space-time lattice of auxiliary fields.
\hspace*{1em} The sweep defines the unit of Monte Carlo time:\\
\hspace*{1em} \textbf{Do} \texttt{nsw} from  \texttt{1} to \texttt{nsweep}  \\
\hspace*{2em} Upward sweep:\\
\hspace*{2em} \textbf{Do} \texttt{ntau} from \texttt{1} to \texttt{ltrot}\\      
\hspace*{3em} Propagate the Green function from time \texttt{ntau -1} to \texttt{ntau}, and compute\\
\hspace*{3em} a new estimate (using sequential update scheme) of the Green function at \texttt{ntau}: \\
\hspace*{3em} \textbf{Call} wrapgrup\\
         
\hspace*{3em}  Stabilization: \\     
\hspace*{3em} \textbf{If} \texttt{ntau} equals stabilization point in imaginary time \textbf{then}\\
\hspace*{4em} Compute propagation matrix from previous stabilization point to \texttt{ntau}: \\
\hspace*{4em} \textbf{Call} wrapur\\
\hspace*{4em} Read from storage: propagation from \texttt{ltrot} to \texttt{ntau}\\
\hspace*{4em} Write to storage : the just computed propagation \\
\hspace*{4em} Recalculate the Green function at time \texttt{ntau} in a stable way:\\
\hspace*{4em} \textbf{Call} cgr\\            
\hspace*{4em} Check the precision between propagated and recalculated Green functions:\\
\hspace*{4em} \textbf{Call} control\_precisionG\\
\hspace*{3em} \textbf{Endif}\\
    
\hspace*{3em} Measure the equal time observables: \\
\hspace*{3em} \textbf{If} \texttt{ntau} is in the intervall \texttt{[LOBS\_ST, LOBS\_EN]} \textbf{then}\\
\hspace*{4em} \textbf{Call} obser\\
\hspace*{3em} \textbf{Endif}\\
\hspace*{2em} \textbf{Enddo}\\

\hspace*{2em} Downward sweep:\\
\hspace*{2em} \textbf{Do} \texttt{ntau} from \texttt{ltrot} to \texttt{1}\\
\hspace*{3em} Repeat the above steps (update, propagation, stabilization, equal time measurements) \\
\hspace*{3em} for the downward direction in imaginary time\\
\hspace*{2em} \textbf{Enddo}\\

\hspace*{2em} Measure the time displaced observables: \\
\hspace*{2em} \textbf{Call} tau\_m\\
\hspace*{1em} \textbf{Enddo} (loop over sweeps)\\
    
\hspace*{1em} Calculate measurement averages for current bin and write them to disk:\\
\hspace*{1em} \textbf{call} pr\_obs\\
\hspace*{1em} Write auxiliary-field configuration to disk: \\
\hspace*{1em} \textbf{call} confout\\
\textbf{Enddo} (loop over bins)\\
% 
% Write the \texttt{info} file to disk:\\
% \textbf{call} control\_print
}
\end{mdframed}
% 
% 
% \lstset{style=fortran_pseudo_code}
% \begin{lstlisting}
% 
% ! Set the Hamiltonian and the lattice:
% call ham_set
% 
% ! Read in an auxiliary-field configuration or generate it randomly:
% call confin
% 
% ! This loop fills the storage, needed for the first actual Monte Carlo sweep:
% do ntau = ltrot, 1, -1 
%    ! Compute propagation matrices and store them at the stabilization points:
%    call wrapul 
% enddo
% 
% ! Loop over bins. The bin defines the unit of Monte Carlo time:
% do nbc = 1, nbin 
% 
%    ! Loop over sweeps. Each sweep updates twice (sweeping upward and downward in time)
%    ! the whole space-time lattice of auxiliary fields:
%    do nsw = 1, nsweep 
%    
%       ! Upward sweep:
%       do ntau = 1, ltrot
%       
%          ! Propagate the Green function from time ntau -1 to ntau, 
%          ! and compute new estimate (using sequential update scheme) of Green at ntau: 
%          call wrapgrup
%          
%          ! Stabilization:      
%          if (ntau == stabilization point)
%             ! Compute propagation matrix from previous stabilization point to ntau: 
%             call wrapur
%             ! Read from storage: propagation from ltrot to ntau
%             ! Write to storage : the just computed propagation 
%                         
%             ! Recalculate Green function at time ntau in a stable way:
%             call cgr
%             
%             ! Check the precision between propagated and recalculated Green function
%          endif
%         
%          ! Measure the equal time observables, 
%          ! if ntau is in the measuring range [LOBS_ST, LOBS_EN]:
%          call obser
%       enddo
%       
%       ! Downward sweep:
%       do ntau = ltrot, 1, -1
%          ! Repeat the above steps (update, propagation, stabilization, measurements) 
%          ! for the downward direction in imaginary time
%       enddo
%       
%    enddo ! Loop over sweeps
%     
%    ! Calculate averages of the measurements of the previous bin and write to disk
%    ! Write auxiliary-field configuration to disk
%    
% enddo ! Loop over bins        
% 
% \end{lstlisting}
