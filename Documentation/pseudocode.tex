% !TEX root = doc.tex
% Copyright (c) 2017 The ALF project.
% This is a part of the ALF project documentation.
% The ALF project documentation by the ALF contributors is licensed
% under a Creative Commons Attribution-ShareAlike 4.0 International License.
% For the licensing details of the documentation see license.CCBYSA.
%
%------------------------------------------------------------
\subsection{Pseudocode description}\label{sec:pseudocode}
%------------------------------------------------------------

The Monte Carlo algorithm as implemented in ALF is summarized in Alg.~\ref{alg:1}. Key control variables include:
\begin{description}[leftmargin=!,align=right,noitemsep,labelwidth=\widthof{\bfseries Global\_moves},font=\texttt] % width: The longest label
	\item[Projector]     Uses (=true) the projective instead of finite-$T$ algorithm (see Sec.~\ref{sec:defT0})
	\item[$L_{\tau}$]    Measures (Ltau=1) time-displaced observables (see Sec.~\ref{sec:Observables.General})
	\item[Tempering]     Runs (=true) in parallel tempering mode (see Table~\ref{table:Updating_schemes})
	\item[Global\_moves] Carries out (=true) global moves in a single time slice (see Table~\ref{table:Updating_schemes}) 
	\item[Sequential]    Carries out (=true) sequential, single spin-flip updates (see Table~\ref{table:Updating_schemes})
	\item[Langevin]      Uses (=true) Langevin dynamics instead of sequential (see Table~\ref{table:Updating_schemes})
\end{description}
Per default, the finite-temperature algorithm is used, \texttt{Ltau=0}, and the updating used is \texttt{Sequential} (i.e., \texttt{Global\_moves}, \texttt{Tempering} and \texttt{Langevin} default values are all \texttt{.false.}).

%\begin{algorithm}[H]
\begin{breakablealgorithm}
	\caption{Basic structure of the QMC implementation in \texttt{Prog/main.f90}}
	\label{alg:1}
	\begin{algorithmic}[1]
		
		\LineComment{\textsc{Initialization}}
		\State{\textbf{call} Ham\_Set}\Comment{Set the Hamiltonian and the lattice}
		\State{\textbf{call} Fields\_Init}\Comment{Set the auxiliary fields}
		\State{\textbf{call} Nsigma\%in}\Comment{Read in an auxiliary-field configuration or generate it randomly}
		\For{$n=L_{\text{Trotter}}$ to $1$} \Comment{Fill the storage needed for the first actual MC sweep}
		\State{\textbf{call} Wrapul}\Comment{Compute propagation matrices and store them at stabilization points}
		\EndFor
		\vspace{1.5ex}
		
		\LineComment{\textsc{Monte Carlo run}}
		\For{$n_{\text{bc}}=1$ to $N_{\text{Bin}}$}\Comment{Loop over bins. The bin defines the unit of Monte Carlo time}
		
		\For{$n_{\text{sw}}=1$ to $N_{\text{Sweep}}$}\Comment{\parbox[t]{0.6\linewidth}{Loop over sweeps. Each sweep updates twice (upward and downward in imaginary time) the space-time lattice of auxiliary fields}}
		
		\If{Tempering}
		\State{\textbf{call} Exchange\_Step} \Comment{Perform exchange step in a parallel tempering run}
		\EndIf
		\If{Global\_moves}
		\State{\textbf{call} Global\_Updates} \Comment{Perform chosen global updates}
		\EndIf
		\If{Langevin}
			\State{\textbf{call} Langevin\_update} \Comment{\textsc{Update and measure} equal-time observables}
			\If{$L_{\tau} = 1$}
			\If{Projector}
			\State{\textbf{call} Tau\_p} \Comment{\textsc{Measure} time-displaced observables (projective code)}
			\Else
			\State{\textbf{call} Tau\_m} \Comment{\textsc{Measure} time-displaced observables (finite temperature)}
			\EndIf
			\EndIf
		\EndIf \textcolor{gray}{~\scriptsize (Langevin)}
		\vspace{1.5ex}
		
		\If{Sequential}
		\vspace{1.5ex}
		
		\FirstLineComment{\textsc{Upward sweep}}
		\For{$n_{\tau}=1$ to $L_{\text{Trotter}}$}
		\State{\textbf{call} Wrapgrup}\Comment{\parbox[t]{0.6\linewidth}{\textsc{Propagate} Green function from $n_{\tau}-1$ to $n_{\tau}$, and compute its new estimate at $n_{\tau}$, using sequential updates}}
		\vspace{1.5ex}
		
		\If{$n_{\tau}$ = stabilization point in imaginary time} \Comment{\textsc{Stabilize}}
		\State{\textbf{call} Wrapur}\Comment{Propagate from previous stabilization point to $n_{\tau}$} 
		
		\LineComment{Storage management:\\
			-- Read from storage: propagation from $L_{\text{Trotter}}$ to $n_{\tau}$\\
			-- Write to storage: the just computed propagation}
		
		\State{\textbf{call} CGR} \Comment{Recalculate the Green function at time $n_{\tau}$ in a stable way}
		\State{\textbf{call} Control\_PrecisionG}\Comment{Compare propagated and recalculated Greens}
		\EndIf
		\vspace{1.5ex}
		
		\If{$n_{\tau} \in [\text{Lobs\_st}, \text{Lobs\_en}]$}
		\State{\textbf{call} Obser} \Comment{\textsc{Measure} the equal-time observables}
		\EndIf
		\EndFor
		\vspace{1.5ex}
		
		\LineComment{\textsc{Downward sweep}}
		\For{$n_{\tau}=L_{\text{Trotter}}$ to $1$}
		\FirstLineComment{Same steps as for the upward sweep (propagation and estimate update, stabilization, equal-time measurements) now downwards in imaginary time}
		\If{Projector \textbf{and}  $L_{\tau} = 1$ \textbf{and} \\
			\hspace{15.8ex} $n_{\tau}$ = stabilization point in imaginary time \textbf{and} \\
			\hspace{15.8ex} the projection time $\theta$ is within the measurement interval}
		\State{\textbf{call} Tau\_p} \Comment{\textsc{Measure} time-displaced observables (projective code)}
		\EndIf
		\EndFor
		\vspace{1.5ex}

		\LineComment{\textsc{Measure} time-displaced observables (finite temperature)}
		\If{$L_{\tau} = 1$ \textbf{and not} Projector}
		\State{\textbf{call} Tau\_m}
		\EndIf
		\vspace{1.5ex}
		
		\EndIf  \textcolor{gray}{~\scriptsize (Sequential)} % sequential
		\vspace{1.5ex}
		
		\EndFor  \textcolor{gray}{~\scriptsize (Sweeps)} % sweeps
		\vspace{1.5ex}
		
		\State{\textbf{call} Pr\_obs} \Comment{Calculate and write to disk measurement averages for the current bin}
		\State{\textbf{call} Nsigma\%out}\Comment{Write auxiliary field configuration to disk}
		
		\EndFor  \textcolor{gray}{~\scriptsize (Bins)} % bins
		
	\end{algorithmic}
\end{breakablealgorithm}
%\end{algorithm}

