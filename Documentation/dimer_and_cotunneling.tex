% Copyright (c) 2016, 2020 The ALF project.
% This is a part of the ALF project documentation.
% The ALF project documentation by the ALF contributors is licensed
% under a Creative Commons Attribution-ShareAlike 4.0 International License.
% For the licensing details of the documentation see license.CCBYSA.

% !TEX root = predefined.tex

\subsubsection{ Cotunneling   for Kondo models}

The  Kondo lattice model (KLM), $\hat{H}_{KLM}$   is obtained by carrying out a   canonical Schrieffer-Wolf  \cite{Schrieffer66}  transformation  of the   periodic Anderson model (PAM), $\hat{H}_{PAM}$.    Hence, $e^{\hat{S}}  $ $\hat{H}_{PAM} $$ e^{-S}   = \hat{H}_{KLM}$  with  $S^\dagger = - S$.     Let $\hat{f}_{x,\sigma} $  create an  electron on the   correlation f-orbital of the   PAM.  Then, 
\begin{equation}
	e^{\hat{S}} \hat{f}^{\dagger}_{x,\sigma'} e^{-S}   \simeq  
     \frac{2V}{U}  \left( c^{\dagger}_{x,-\sigma'}  \hat{S}^{\sigma'}_{x} +  \sigma'   c^{\dagger}_{x,\sigma'} \hat{S}^{z}_x   \right)  \equiv  \frac{2V}{U}   \tilde{f}^{\dagger}_{x,\sigma'}    
\end{equation}
In the above, it is understood that $\sigma'$ takes the value $1$ ($-1$)  for up  (down) spin degrees of freedom, that  $  \hat{S}^{\sigma'}_{x} =  f^{\dagger}_{x,\sigma'} f^{}_{x,-\sigma'}  $  and that 
$ \hat{S}^{z}_{x} = \frac{1}{2} \sum_{\sigma'}  \sigma' f^{\dagger}_{x,\sigma'} f^{}_{x,\sigma'} $.  Finally, $c^{\dagger}_{x,\sigma'}$ corresponds to the conduction electron that hybridizes with $f^{\dagger}_{x,\sigma'}$.  This form matches that derived in Ref.~\cite{Costi00} and a  calculation  of the former equation can be found in Ref.~\cite{Raczkowski18}.  An equivalent, but more transparent formulation is given in  Ref.~\cite{Maltseva09}   and reads: 
\begin{equation}
	\tilde{f}^{\dagger}_{x,\sigma} = \sum_{\sigma'} \hat{c}^{\dagger}_{x,\sigma'} \ve{\sigma}_{\sigma',\sigma} \cdot  \ve{S}_x
\end{equation}
where $\ve{\sigma}$  denotes  the vector of Pauli spin matrices.
The  function 
\begin{lstlisting}[style=fortran]
Complex (Kind=Kind(0.d0)) function Predefined_Obs_Cotunneling(x_c x, y_c, y,  
                                                          GT0,G0T,G00,GTT, N_SUN, N_FL)
\end{lstlisting}
returns the   value of the time displaced correlation function: 
\begin{equation}
	\sum_{\sigma} \langle \langle   \tilde{f}^{\dagger}_{x,\sigma}(\tau)  \tilde{f}_{y,\sigma}(0)   \rangle \rangle_{C} 
\end{equation}
Here, $x_c$ and $y_c$ corresponds to the   conduction orbitals that hybridize with the  x and y f-orbitals.   The routine  works for    SU(N)  symmetric codes  corresponding to   \texttt{N\_FL=1}  and \texttt{N\_SUN = 2,4,6,8}.  For the larger N-values,  we have replaced the generators of SU(2)  with that of SU(N).  The routine also handles the case where spin-symmetry is broken by e.g. a  Zeeman field. This  corresponds to the 
case  \texttt{N\_FL=2}  and \texttt{N\_SUN=1}.    Note that the function only  carries out the Wick decomposition   and the handling  of the observable  type corresponding to this quantity   has to be done by the user.     To carry out the Wick decomposition, we use Mathematica, see  notebooks 
\texttt{Cotunneling\_SU2\_NFL\_2.nb}  and  \texttt{Cotunneling\_SUN\_NFL\_1.nb}. 