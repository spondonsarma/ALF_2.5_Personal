% Copyright (c) 2016 The ALF project.
% This is a part of the ALF project documentation.
% The ALF project documentation by the ALF contributors is licensed
% under a Creative Commons Attribution-ShareAlike 4.0 International License.
% For the licensing details of the documentation see license.CCBYSA.

% !TEX root = Doc.tex
%-------------------------------------------------------------------------------------
\subsection{Running the code}\label{sec:running}
%-------------------------------------------------------------------------------------

In this section we describe the steps how to compile and run the code, as well as how to perform the error analysis of the data.

%-------------------------------------------------------------------------------------
\subsubsection{Compilation}
%-------------------------------------------------------------------------------------

The environment variables and the directives to compile the code are set in the following makefile \texttt{Makefile}:
\lstset{style=bash}
\begin{lstlisting}

# -DMPI selects MPI.
# -DQRREF  Default  stabilization, using QR with pivoting.
           Packed form of QR factorization  is used. 
# -DSTAB1  Alternative stabilization, using the  singular value decomposition.
# -DSTAB2  Alternative stabilization, using QR  with pivoting. 
           Packed form of QR factorization  is not used.
# -DSTAB3  Alternative stabilization, using QR  with pivoting. 
           Internally, scales larger and smaller one are distinguished.
# -DLOG    Alternative stabilization, using QR  with pivoting. 
           Internally, scales are stored on log axsis to allow larger beta and
           larger and smaller have to be distinguished.
PROGRAMCONFIGURATION = -DMPI 
PROGRAMCONFIGURATION = -DQRREF
f90 = gfortran
export f90
F90OPTFLAGS = -O3 -Wconversion  -fcheck=all
F90OPTFLAGS = -O3
export F90OPTFLAGS
F90USEFULFLAGS = -cpp -std=f2003
F90USEFULFLAGS = -cpp
export F90USEFULFLAGS
FL = -c ${F90OPTFLAGS} ${PROGRAMCONFIGURATION}
export FL
DIR = ${CURDIR}
export DIR
Libs = ${DIR}/Libraries/
export Libs
LIB_BLAS_LAPACK = -llapack -lblas
export LIB_BLAS_LAPACK

all: lib ana prog
prog:
	cd Prog && $(MAKE)
lib:
	cd Libraries && $(MAKE)
ana:
	cd Analysis && $(MAKE)
clean: cleanall
cleanall: cleanprog cleanlib cleanana
cleanprog:
	cd Prog && $(MAKE) clean
cleanlib:
	cd Libraries && $(MAKE) clean
cleanana:
	cd Analysis && $(MAKE) clean
help:
	@echo "The following are some of the valid targets of this Makefile"
	@echo "all, prog, lib, ana, clean, cleanall, cleanprog, cleanlib, cleanana"

\end{lstlisting}
In the above, the GNU Fortan compiler \texttt{gfortran} is set.\footnote{A known issue with the alternative Intel Fortran compiler \texttt{ifort} is the handling of automatic, temporary arrays 
which \texttt{ifort} allocates on the stack. For large system sizes and/or low temperatures this may lead to 
a runtime error. One solution is to demand allocation of arrays above a certain size on the heap instead of the stack. 
This is accomplished by the \texttt{ifort} compiler flag \texttt{-heap-arrays [n]} where \texttt{[n]} is the minimal size (in kilobytes, for example \texttt{n=1024}) of arrays 
that are allocated on the heap.}
We provide a set of options for compilation of the QMC code. The present options are \texttt{-DMPI}, \texttt{-DQRREF}, \texttt{-DSTAB1}, and \texttt{-DSTAB2}. 
They can be included in the string variable \texttt{PROGRAMCONFIGURATION} by the user, as shown above.
The program can be compiled and ran either in single-thread mode (default) or 
in multi-threading mode (define \texttt{-DMPI}) using the MPI standard for parallelization. The remaining three compiler options select a particular stabilization scheme for the matrix multiplications (see Sec.~\ref{sec:output_prec}).
To compile the libraries, the analysis routines and the QMC program at once, just execute the single command:
\begin{verbatim}
make
\end{verbatim}
To clean up all directories and remove the object files and executables, execute the command \texttt{make clean}. As can be seen in the above makefile, there exist also rules to compile/clean up the library, the analysis routines and the QMC program separately.  


%-------------------------------------------------------------------------------------
\subsubsection{Starting a simulation}
%-------------------------------------------------------------------------------------

To start a simulation from scratch, the following files have to be present: \texttt{parameters} and \texttt{seeds}. 
To run a single-thread simulation, for example by using the parameters of one of the  Hubbard models described in Sec.~\ref{sec:ex}, issue the command
\begin{verbatim}
./Prog/Examples.out
\end{verbatim}
To restart the code using an existing simulation as a starting point, first run the script \texttt{out\_to\_in.sh} to set 
the input configuration files.

%-------------------------------------------------------------------------------------
\subsubsection{Error analysis}
%-------------------------------------------------------------------------------------

To perform an error analysis (based on the jackknife scheme) of the Monte Carlo bins for all observables run the script \texttt{analysis.sh} 
(see Sec.~\ref{sec:analysis}).

 



