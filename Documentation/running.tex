% Copyright (c) 2016 2017 The ALF project.
% This is a part of the ALF project documentation.
% The ALF project documentation by the ALF contributors is licensed
% under a Creative Commons Attribution-ShareAlike 4.0 International License.
% For the licensing details of the documentation see license.CCBYSA.

% !TEX root = doc.tex


In this section we describe the steps for compiling and running the code from the shell, and describe how to search for optimal parameter values as well as how to perform the error analysis of the data.

The source code of ALF~\ALFver{} is available at \url{https://git.physik.uni-wuerzburg.de/ALF/ALF/-/tree/\ALFbranch/} and can be cloned with git or downloaded from the repository (make sure to choose the appropriate release).

A Python interface, \textbf{pyALF}, is also available and can be found, together with a number of Jupyter notebooks exploring the interface's capabilities, at \url{https://git.physik.uni-wuerzburg.de/ALF/pyALF/-/tree/\pyALFbranch/}. This interface facilitates setting up simple runs and is ideal for setting benchmarks and getting acquainted with ALF. Some of pyALF's notebooks form the core of the introductory part of the \href{https://git.physik.uni-wuerzburg.de/ALF/ALF_Tutorial/-/blob/master/Tutorial-ALF-\tutALFver/}{ALF Tutorial}, where pyALF's usage is described in more detail.

We start out by providing step-by-step instructions that allow a first-time user to go from zero to performing a simulation and reading out their first measurement using ALF.

%-------------------------------------------------------------------------------------
\subsection{Zeroth step}
\label{sec:zeroth}
%-------------------------------------------------------------------------------------

The aim of this section is to provide a fruitful and stress-free first contact with the package. Ideally, it should be possible to copy and paste the instructions below to a Debian/Ubuntu-based Linux shell without further thought\footnote{For other systems and distributions see the package's  \href{https://git.physik.uni-wuerzburg.de/ALF/ALF/-/blob/\ALFbranch/README.md}{README}.}. Explanations and further options and details are found in the remaining sections and in the \href{https://git.physik.uni-wuerzburg.de/ALF/ALF_Tutorial/-/blob/master/Tutorial-ALF-\tutALFver/}{Tutorial}.

\textbf{Prerequisites}:
You should have access to a shell and the permissions to install -- or have already installed -- the numerical packages Lapack and Blas, a Fortran compiler, Python, and the tools \texttt{make} and \texttt{git}. 

The following commands can be executed in a Debian-based shell in order to install ALF~\ALFver{} and its dependencies, run a demonstration simulation and output one of the measurements performed:
\begin{itemize}[itemsep=0pt]
	\item \lstinline[style=bash,morekeywords={sudo}]{sudo apt-get update}
	\item \lstinline[style=bash,morekeywords={sudo}]{sudo apt-get install gfortran liblapack-dev python3 make git}
	\item \lstinline[style=bash]{git clone -b }\texttt{\ALFbranch}\lstinline[style=bash]{https://git.physik.uni-wuerzburg.de/ALF/ALF.git}
	\item \lstinline[style=bash,morekeywords={cd}]{cd ALF}
	\item \lstinline[style=bash,morekeywords={source}]{source configure.sh GNU noMPI}
	\item \lstinline[style=bash,morekeywords={make}]{make Hubbard_Plain_Vanilla ana}
	\item \lstinline[style=bash,morekeywords={cp}]{cp -r ./Scripts_and_Parameters_files/Start ./Run && cd ./Run/}
	\item \lstinline[style=bash,morekeywords={}]{$ALF_DIR/Prog/ALF.out}
	\item \lstinline[style=bash,morekeywords={}]{$ALF_DIR/Analysis/ana.out Ener_scal}
	\item \lstinline[style=bash,morekeywords={}]{cat Ener_scalJ}
\end{itemize}
The last command will output a few lines, including one similar to:
\begin{lstlisting}[style=bash]
OBS :    1      -30.009191      0.110961
\end{lstlisting}
which is listing the internal energy of the system and its error.

%-------------------------------------------------------------------------------------
\subsection{Compiling and running}
\label{sec:compilation}
%-------------------------------------------------------------------------------------

The necessary environment variables and the directives for compiling the code are set by the script \texttt{configure.sh}:
\begin{lstlisting}[style=bash]
source configure.sh [MACHINE] [MODE] [STAB]
\end{lstlisting}
If run with no arguments, it lists the available options and sets a generic, serial GNU compiler with minimal flags \texttt{-cpp -O3 -ffree-line-length-none -ffast-math}. The predefined machine configurations and parallelization modes available, as well as the options for stabilization schemes for the matrix multiplications (see Sec.~\ref{sec:stable}) are shown in Table~\ref{table:configureHPC}. The stabilization scheme choice, in particular, is critical for performance and is discussed further in Sec.~\ref{sec:optimize}.

In order to compile the libraries, the analysis routines and the QMC program at once, just execute the single command:%\lstinline[style=bash,morekeywords={make}]{make}.
\begin{lstlisting}[style=bash,morekeywords={make}]
make
\end{lstlisting}
Related auxiliary directories, object files and executables can be removed by executing the command \lstinline[style=bash,morekeywords={make}]{make clean}. The accompanying \texttt{Makefile} also provides rules for compiling and cleaning up the library, the analysis routines and the QMC program separately.  

\begin{table}[!ht]
	\begin{center}
		\begin{tabular}{@{} p{0.19\columnwidth} p{0.78\columnwidth} @{}}\toprule %16 81
			Argument & Selected feature \\\midrule
			\textbf{\texttt{MACHINE}} &  \\\midrule
			\texttt{GNU}  &   GNU compiler (\texttt{gfortran}/\texttt{mpifort}) for a generic machine (\emph{default}) \\
			\texttt{Intel}  &   Intel compiler (\texttt{ifort}/\texttt{mpiifort}) for a generic machine\footnote{A known issue with the alternative Intel Fortran compiler \texttt{ifort} is the handling of automatic, temporary arrays 
				which \texttt{ifort} allocates on the stack. For large system sizes and/or low temperatures this may lead to a runtime error. One solution is to demand allocation of arrays above a certain size on the heap instead of the stack. This is accomplished by the \texttt{ifort} compiler flag \texttt{-heap-arrays [n]} where \texttt{[n]} is the minimal size (in kilobytes, for example \texttt{n=1024}) of arrays that are allocated on the heap.} \\
			\texttt{PGI}  &   PGI compiler (\texttt{pgfortran}/\texttt{mpifort}) for a generic machine \\
			\texttt{SuperMUC-NG}  &  Intel compiler (\texttt{mpiifort}) and loads modules for SuperMUC-NG\footnote{Supercomputer at the Leibniz Supercomputing Centre.} \\
			\texttt{JUWELS}  &  Intel compiler (\texttt{mpiifort}) and loads modules for JUWELS\footnote{Supercomputer at the J\"ulich Supercomputing Centre.}
			\vspace{7pt}\\
			
			\textbf{\texttt{MODE}} &  \\\midrule
			\texttt{noMPI|Serial}  &  No parallelization \\
			\texttt{MPI}  &  MPI parallelization (\emph{default} -- if a machine is selected) \\
			\texttt{Tempering}  &  Parallel tempering (Sec.~\ref{Parallel_tempering.sec}) and the required MPI as well
			\vspace{7pt}\\ 
			
			\textbf{\texttt{STAB}} &  \\\midrule
			\texttt{STAB1}  &  Simplest stabilization: UDV (QR-, not SVD-based) decompositions\\
			\texttt{STAB2}  &  QR-based UDV decompositions with additional normalizations\\
			\texttt{STAB3}  &  Latest, additionally separates large and small scales (\emph{default})\\
			\texttt{LOG}  &  Logarithmic storage for internal scales, increases accessible ranges
			\vspace{7pt}\\
			
			\textbf{\texttt{OTHER SWITCHES}} &  \\\midrule
			\texttt{Devel}  &  Compile with additional flags for development and debugging\\
			\texttt{HDF5}  &  Compile with HDF5 -- automatically downloads and installs HDF5 if not present\\
			\texttt{NO-INTERACTIVE} & Do not ask for user confirmation during execution of this script\\
			\bottomrule
		\end{tabular}
		\caption{Available arguments for the script \texttt{configure.sh}, called before compilation of the package: predefined machines, parallelization modes, and stabilization schemes (see also Sec.~\ref{sec:optimize}).} \label{table:configureHPC}
	\end{center}
\end{table}

A suite of tests for individual parts of the code (subroutines, functions, operations, etc.) is available at the directory \path{testsuite}. The tests can be run by executing the following sequence of commands (the script \path{configure.sh} sets environment variables as described above):
\begin{lstlisting}[style=bash,morekeywords={make,cmake,ctest}]

source configure.sh GNU Devel noMPI
gfortran -v
make lib
make ana
make program
cd testsuite
cmake -E make_directory tests
cd tests
cmake -G "Unix Makefiles" -DCMAKE_Fortran_FLAGS_RELEASE=${F90OPTFLAGS} \
-DCMAKE_BUILD_TYPE=RELEASE ..
cmake --build . --target all --config Release
ctest -VV -O log.txt
\end{lstlisting}
which will output test results and total success rate.

%
%-------------------------------------------------------------------------------------
\subsection*{Starting a simulation}
%-------------------------------------------------------------------------------------
%

In order to start a simulation from scratch, the following files have to be present: \texttt{parameters} and \texttt{seeds} (see Sec.~\ref{sec:input}). 
To run serial simulation, issue the command
\begin{lstlisting}[style=bash]
$ALF_DIR/Prog/ALF.out
\end{lstlisting}
In order to run with MPI parallelization, the appropriate MPI execution command should be called. For instance, a program compiled with OpenMPI can be run in parallel by issuing  
\begin{lstlisting}[style=bash]
orterun -np <number of processes> $ALF_DIR/Prog/ALF.out
\end{lstlisting}

\noindent The environment variable \texttt{ALF\_SHM\_CHUNK\_SIZE\_GB} can be used to reduce the program's memory footprint by sharing memory between MPI processes on the same node. The variable, a positive real number, defines the chunk size of the shared memory objects in units of GB.
Typical values are 1.0 or 2.0 GB, but larger values can be used, if otherwise the total number of MPI communicators so large as to trigger MPI error messages.
If \texttt{ALF\_SHM\_CHUNK\_SIZE\_GB} is not defined or set to values smaller that one, then the memory is not shared between MPI processes, which is the default behavior.

To restart the code using the configuration from a previous simulation as a starting point, first run the script \texttt{out\_to\_in.sh}, which copies outputted field configurations into input files, before calling the ALF executable.   This file is located in the  directory \path{$ALF_DIR/Scripts_and_Parameters_files/Start/}

%
%-------------------------------------------------------------------------------------
\subsection{Error analysis}\label{sec:analysis}
%-------------------------------------------------------------------------------------
%

The ALF package includes the analysis programs \texttt{ana.out} for plain text bins and \texttt{ana\_hdf5.out} for bins in HDF5 format. They perform the same simple error analysis and correlation function calculations on the three observable types. To perform an error analysis based on the Jackknife resampling method~\cite{efron1981} (Sec.~\ref{sec:jack}) of the Monte Carlo bins for a list of plain-text observables run
\begin{lstlisting}[style=bash]
$ALF_DIR/Analysis/ana.out <list of files>
\end{lstlisting}
or run
\begin{lstlisting}[style=bash]
$ALF_DIR/Analysis/ana.out *
\end{lstlisting}
for all observables.

For analyzing observables stored in the HDF5 file \texttt{data.h5}, run
\begin{lstlisting}[style=bash]
$ALF_DIR/Analysis/ana_hdf5.out <list of observables>
\end{lstlisting}
or run
\begin{lstlisting}[style=bash]
$ALF_DIR/Analysis/ana_hdf5.out *
\end{lstlisting}
for all observables.

The programs \texttt{ana.out} and \texttt{ana\_hdf5.out} are based on the included module \texttt{ana\_mod}, which provides subroutines for reading and analyzing ALF Monte Carlo bins, that can be used to implement more specialized analysis. The three high-level analysis routines employed by \texttt{ana\_mod} are listed in Table~\ref{table:analysis_programs}. The files taken as input, as well as the output files are listed in Table~\ref{table:analysis_files}.

\begin{table}[h]
	\begin{center}
	\begin{tabular}{@{} p{0.18\linewidth} @{$\;$} p{0.8\linewidth} @{}}\toprule
		Program & Description \\\midrule
		\texttt{cov\_vec(name)}  &  
		  The bin file \texttt{name}, which should have suffix \texttt{\_scal}, is read in, and  the corresponding file with suffix \texttt{\_scalJ} is produced. It contains the result of the Jackknife rebinning analysis (see Sec.~\ref{sec:sampling}) \\
		\texttt{cov\_eq(name)}    &  
		  The bin file \texttt{name}, which should have suffix \texttt{\_eq}, is read in, and the corresponding files with suffix \texttt{\_eqJR} and \texttt{\_eqJK} are produced. They  correspond to correlation functions in real and Fourier space, respectively \\
		\texttt{cov\_tau(name)}   &  
		  The bin file \texttt{name}, which should have suffix \texttt{\_tau}, is read in, and the directories \texttt{X\_kx\_ky} are produced for all \texttt{kx} and \texttt{ky} greater or equal to zero. \\
		& Here \texttt{X}  is a place holder from \texttt{Green}, \texttt{SpinXY}, etc., as specified in \texttt{ Alloc\_obs(Ltau)} (See section \ref{Alloc_obs_sec}). Each directory contains  a  file    \texttt{g\_dat}  containing the  time-displaced correlation function traced over the  orbitals.  It also contains the covariance matrix if \texttt{N\_cov} is set to unity in the parameter file (see Sec.~\ref{sec:input}). Besides, a directory \texttt{X\_R0} for the local time displaced correlation function is generated. \\
		& For particle-hole, imaginary-time correlation functions (\texttt{Channel = "PH"}) such as spin and charge, we use the fact that these correlation functions  are symmetric around $\tau = \beta/2$ so that we can define an improved estimator by averaging over $\tau$ and $\beta - \tau$
		\\\bottomrule
	\end{tabular}
	\caption{Overview of analysis subroutines called within \texttt{ana.out} and \texttt{ana\_hdf5.out}. \label{table:analysis_programs}}
\end{center}
\end{table}
%
%Here we briefly describe the analysis programs, which read in bins and carry out the error analysis. (See Sec.~\ref{sec:sampling}  for a more detailed discussion.)
The error analysis is based on the central limit theorem, which requires bins to be statistically independent, and also the existence of a well-defined variance for the observable under consideration (see Sec.~\ref{sec:sampling}).
The former will be the case if bins are  longer than the autocorrelation time -- autocorrelation functions are computed by setting the parameter \path{N_auto} to a nonzero value -- which has to be checked by the user.  In the parameter file described in Sec.~\ref{sec:input}, the user  can specify how many initial bins should be omitted (variable \texttt{n\_skip}). This number should be comparable to the autocorrelation time.
The  rebinning  variable \texttt{N\_rebin} will merge \texttt{N\_rebin}  bins into a single new bin. 
If the autocorrelation time  is smaller than the effective bin size, the error should become independent of the bin size and thereby of the variable \texttt{N\_rebin}.
%
\begin{table}[h]
	\begin{center}
		\begin{tabular}{@{} p{0.23\linewidth} p{0.74\linewidth} @{}}\toprule
		File                 & Description  \\ \midrule
		Input                &  \\\midrule %\cmidrule{1-1}
		\texttt{parameters}  &  Includes error analysis variables \texttt{N\_skip}, \texttt{N\_rebin}, and \texttt{N\_Cov} (see Sec.~\ref{sec:input}) \\
		\texttt{X\_scal}, \texttt{Y\_eq}, \texttt{Y\_tau} & Monte Carlo bins (see Table \ref{table:output}) \vspace{7pt}\\

		Output               &  \\\midrule
		\texttt{X\_scalJ}    & Jackknife mean and error of \texttt{X}, where  \texttt{X} stands for \texttt{Kin, Pot, Part}, or \texttt{Ener}\\
		\texttt{Y\_eqJR} and \texttt{Y\_eqJK} & Jackknife mean and error of \texttt{Y}, which stands for \texttt{Green, SpinZ, SpinXY}, or \texttt{Den}. The suffixes \texttt{R} and \texttt{K} refer to real and reciprocal space, respectively\\
		\texttt{Y\_R0/g\_R0} & Time-resolved and spatially local Jackknife mean and error of \texttt{Y}, where \texttt{Y} stands for \texttt{Green, SpinZ, SpinXY}, and \texttt{Den}\\
		\texttt{Y\_kx\_ky/g\_kx\_ky} & Time resolved and $\vec{k}$-dependent Jackknife mean and error of \texttt{Y}, where \texttt{Y} stands for \texttt{Green, SpinZ, SpinXY}, and \texttt{Den}\\
		\texttt{Part\_scal\_Auto} & Autocorrelation functions $S_{\hat{O}}(t_{\textrm{Auto}})$ in the range $t_{\text{Auto}}=[0,\textrm{\path{N_auto}}]$ for the observable $\hat{O}$ \\\bottomrule
	\end{tabular}
	\caption{Standard input and output files of the error analysis program \texttt{ana.out}. \label{table:analysis_files}}
\end{center}
\end{table}
%
The analysis output files listed in Table \ref{table:analysis_files} and are formatted in the following way:
\begin{itemize}
	\item For the scalar quantities \texttt{X}, the output files  \texttt{X\_scalJ} have the following formatting:
	{\small \begin{alltt}
		Effective number of bins, and bins:  <N_bin - N_skip>/<N_rebin>  <N_bin>	
		OBS :  1    <mean(X)>      <error(X)>	
		OBS :  2    <mean(sign)>   <error(sign)>
	\end{alltt} }
	
	\item For the equal-time correlation functions \texttt{Y}, the formatting of the output files \texttt{Y\_eqJR} and \texttt{Y\_eqJK} follows the structure:
	{\small \begin{alltt}
		do i = 1, N_unit_cell
		   <k_x(i)>   <k_y(i)>
		   do alpha = 1, N_orbital
		      do beta  = 1, N_orbital
		         alpha  beta  Re<mean(Y)>  Re<error(Y)>  Im<mean(Y)>  Im<error(Y)>
		      enddo
		   enddo
		enddo
	\end{alltt} }
	where \texttt{Re} and \texttt{Im} refer to the real and imaginary part, respectively.
	
	\item The imaginary-time displaced correlation functions \texttt{Y} are written to the output files \texttt{g\_R0} inside folders \texttt{Y\_R0}, when measured locally in space; 
	and to the output files \texttt{g\_kx\_ky} inside folders \texttt{Y\_kx\_ky} when they are measured $\vec{k}$-resolved (where $\vec{k}=(\texttt{kx}, \texttt{ky})$). 
	The first line of the file contains the  number of  imaginary times, the effective number of bins,  $\beta$, the number of orbitals and  the channel.  
	Both output files have the following formatting:
	{\small \begin{alltt}
		do i = 0, Ltau
		   tau(i)   <mean( Tr[Y] )>   <error( Tr[Y])>
		enddo
	\end{alltt} }
	where \texttt{Tr} corresponds to the trace over the orbital degrees of freedom.   For particle-hole quantities at finite temperature,  $\tau$ runs from 
	$0$ to $\beta/2$.   In all other cases it runs from $0$ to $\beta$. 
	
	\item The file  \texttt{Y\_tauJK}    contains the susceptibilities defined as: 
	\begin{equation}
	  \chi(\ve{q})   =   \sum_{n,n'=1}^{\text{Norb}}  \int_{0}^{\beta}   d \tau    \left(   \left<   Y_n(\ve{q},\tau) Y_{n'}(-\ve{q},0) \right>  - \left<   Y_n(\ve{q}) \right> \left< Y_{n'}(-\ve{q}) \right> \delta_{\ve{q},\ve{0}}  \right) 
	\end{equation}
	The output file 	has the following formatting:
	{\small \begin{alltt}
		do i = 0, Ltau
		   q_x, q_y, <mean(Real(chi(q)) )>,  <error(Real(chi(q)))>, & 
		           & <mean(Im  (chi(q)) )>,  <error(lmi (chi(q)))>
		enddo
	\end{alltt} }

	
	\item Setting the parameter \path{N_auto} to a finite value triggers the computation of autocorrelation functions $S_{\hat{O}}(t_{\textrm{Auto}})$ in the range $t_{\text{Auto}}=[0,\textrm{\path{N_auto}}]$. The output is written to the file \texttt{Part\_scal\_Auto}, where the data in organized in three columns:
	{\small
		\begin{alltt}
			\(t\sb{\textrm{Auto}}\)   \(S\sb{\hat{O}}(t\sb{\textrm{Auto}})\)   error
		\end{alltt}
	}
	Since these computations are quite time consuming and require many Monte Carlo bins, our default is \path{N_auto=0}.
	
	
	
	
\end{itemize}



%
%-------------------------------------------------------------------------------------
\subsection{Parameter optimization} \label{sec:optimize}
%-------------------------------------------------------------------------------------
%


The finite-temperature, auxiliary-field QMC algorithm is known to be numerically unstable, as discussed in Sec.~\ref{sec:stable}.
The numerical instabilities arise from the imaginary-time propagation, which invariably leads to exponentially small and exponentially large scales.
As shown in Ref.~\cite{Assaad08_rev}, scales can be omitted in the ground state algorithm -- thus rendering it very stable --  but have to be taken into account in the  finite-temperature code.

Numerical stabilization of the code is a delicate procedure that has been pioneered in Ref.~\cite{White89}  for the finite-temperature algorithm and in Refs.~\cite{Sugiyama86,Sorella89} for the zero-temperature, projective algorithm.
It is important to be aware of the fragility of the numerical stabilization and that there is no guarantee that it will work for a given model. It is therefore crucial to always check the file \texttt{info}, which, apart from runtime data, contains important information concerning the stability of the code, in particular \texttt{Precision Green}.
If the numerical stabilization fails, one possible measure is to reduce the value of the parameter \texttt{Nwrap} in the parameter file, which will however also impact performance -- see Table.~\ref{table:tips} for further optimization tips for the Monte Carlo algorithm (Sec.~\ref{sec:sampling}). Typical values for the numerical precision ALF can achieve can be found in Sec.~\ref{sec:hubbard}.

\begin{table}[h!]
	\begin{center}
		\begin{tabular}{@{} p{0.22\linewidth} p{0.74\linewidth} @{}}\toprule
			Element & Suggestion  \\ \midrule
			\texttt{Precision Green}, \texttt{Precision Phase}  &  \multirow{2}{*}{Should be found to be \emph{small}, of order $< 10^{-8}$ (see Sec.~\ref{sec:stable})}\\
			\texttt{theta}       & Should be \emph{large} enough to guarantee convergence to ground state \\
			\texttt{dtau}        & Should be set \emph{small} enough to limit Trotter errors\\
			\texttt{Nwrap}       & Should be set \emph{small} enough to keep \texttt{Precision}s small\\
			\texttt{Nsweep}      & Should be set \emph{large} enough for bins to be of the order of the auto-correlation time\\
			\texttt{Nbin}        & Should be set \emph{large} enough to provide desired statistics \\
			\texttt{nskip}       & Should be set \emph{large} enough to allow for equilibration ($\sim$ autocorrelation time) \\
			\texttt{Nrebin}      & Can be set to 1 when \texttt{Nsweep} is large enough; otherwise, and for testing, larger values can be used\\
			Stabilization scheme & Use the default \texttt{STAB3} -- newest and fastest, if it works for your model; alternatives are: \texttt{STAB1} -- simplest, for reference only; \texttt{STAB2} -- with additional normalizations; and \texttt{LOG} -- for dealing with more extreme scales (see also Tab.~\ref{table:configureHPC}) \\
			Parallelism          & For some models and systems, restricting parallelism in your BLAS library can improve performance: for OpenBLAS try setting \texttt{OPENBLAS\_NUM\_THREADS=1} in the shell \\
			\texttt{ALF\_SHM\_CHUNK\_SIZE\_GB} & An environment variable that sets the chunk size in GBs for the memory shared between different MPI processes on the same computing node. By default it is zero (i.e., no sharing), but can be set to, e.g., 1.0 or 2.0 GB or larger if, for instance, the total number of MPI communicators is so large as to trigger MPI error messages. \\\bottomrule
		\end{tabular}
		\caption{Rules of thumb for obtaining best results and performance from ALF. It is important to fine tune the parameters to the specific model under consideration and perform sanity checks throughout. Most suggestions can severely impact performance and numerical stability if overdone. \label{table:tips}}
	\end{center}
\end{table}


In particular, for the stabilization of the involved matrix multiplications we rely on routines from LAPACK. Notice that results are very likely to change
%significantly
depending on the specific implementation of the library used\footnote{The linked library should implement at least the LAPACK-3.4.0 interface.}. In order to deal with this possibility, we offer a simple baseline which can be used as a quick check as tho whether results depend on the library used for linear algebra routines. Namely, we have included QR-decomposition related routines of the LAPACK-3.7.0 reference implementation from \url{http://www.netlib.org/lapack/}, which you can use by running the script \path{configure.sh}, (described in Sec.~\ref{sec:running}), with the flag \texttt{STAB1} and recompiling ALF\footnote{This flag may trigger compiling issues, in particular, the Intel ifort compiler version 10.1 fails for all optimization levels.}. The stabilization flags available are described in Tables~\ref{table:configureHPC} and~\ref{table:tips}. The performance of the package is further discussed in Sec.~\ref{sec:performance}.

