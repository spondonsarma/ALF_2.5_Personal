% Copyright (c) 2016 The ALF project.
% This is a part of the ALF project documentation.
% The ALF project documentation by the ALF contributors is licensed
% under a Creative Commons Attribution-ShareAlike 4.0 International License.
% For the licensing details of the documentation see license.CCBYSA.

% !TEX root = Doc.tex
%-------------------------------------------------------------------------------------
\section{Running the code}\label{sec:running}
%-------------------------------------------------------------------------------------

In this section we describe the steps to compile and run the code and to perform the error analysis of the data.

%-------------------------------------------------------------------------------------
\subsection{Compilation}
%-------------------------------------------------------------------------------------

The environment variables are defined in the bash script \texttt{set\_env.sh} as follows:
\lstset{style=custombash}
\begin{lstlisting}

# Description of PROGRAMMCONFIGURATION:
# -DMPI selects MPI.
# Setting nothing  compiles without mpi.
# -DQRREF selects  a reference implementation of the QR decomposition. 
# Setting nothing selects system lapack for the QR decomposition.
# -DSTAB1 selects an alternative stabilization scheme.
# Setting nothing selects the default stabilizatiion
PROGRAMMCONFIGURATION=""
f90="gfortran"
export f90
F90OPTFLAGS="-O3"
export F90OPTFLAGS
FL="-c ${F90OPTFLAGS} ${PROGRAMMCONFIGURATION}"
export FL
DIR=`pwd`
export DIR
Libs=${DIR}"/Libraries/"
export Libs
LIB_BLAS_LAPACK="-llapack -lblas"
export LIB_BLAS_LAPACK

\end{lstlisting}
In the above, the GNU Fortan compiler \texttt{gfortran} is set.\footnote{A known issue with the alternative Intel Fortran compiler \texttt{ifort} is the handling of automatic, temporary arrays 
which \texttt{ifort} allocates on the stack. For large system sizes and/or low temperatures this may lead to 
a runtime error. One solution is to demand allocation of arrays above a certain size on the heap instead of the stack. 
This is accomplished by the \texttt{ifort} compiler flag \texttt{-heap-arrays [n]} where \texttt{[n]} is the minimal size (in kilobytes, for example \texttt{n=1024}) of arrays 
that are allocated on the heap.}
The program can be compiled and ran either in single-thread mode (default) or 
in multi-threading mode (define \texttt{-DMPI}) using the MPI standard for parallelization.
To compile the libraries, the analysis programs and the quantum Monte Carlo program, the following steps should be executed:
\begin{enumerate}
\item Export  the environment variables:
\begin{verbatim}
source set_env.sh
\end{verbatim}
\item Compile the libraries and the error analysis routines
\begin{verbatim}
cd Libraries
make
cd ..
cd Analysis
make
cd ..
\end{verbatim}
\item Compile the quantum Monte Carlo code
\begin{verbatim}
cd Prog
make
cd ..
\end{verbatim}
\end{enumerate}

%-------------------------------------------------------------------------------------
\subsection{Starting a simulation}
%-------------------------------------------------------------------------------------

To start a simulation from scratch, the following files have to be present: \texttt{parameters} and \texttt{seeds}. 
To run a single-thread simulation for one of the  Hubbard model described in Sec.~\ref{sec:walk1} - \ref{sec:walk2}, issue the command
\begin{verbatim}
./Prog/Examples.out
\end{verbatim}
To restart the code using an existing simulation as a starting point, first run the script \texttt{out\_to\_in.sh} to set 
the input configuration files.

%-------------------------------------------------------------------------------------
\subsection{Error analysis}
%-------------------------------------------------------------------------------------

To perform an error analysis, based on the jackknife scheme, of the Monte Carlo bins for all observables run the script \texttt{analysis.sh} 
(see Sec.~\ref{sec:analysis}).

 



