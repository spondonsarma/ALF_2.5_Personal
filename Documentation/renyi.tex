% !TEX root = doc.tex
% Copyright (c) 2017-2020 The ALF project.
% This is a part of the ALF project documentation.
% The ALF project documentation by the ALF contributors is licensed
% under a Creative Commons Attribution-ShareAlike 4.0 International License.
% For the licensing details of the documentation see license.CCBYSA.
%
%-----------------------------------------------------------------------------------
\subsubsection{R{\'e}nyi Entropy}
%-----------------------------------------------------------------------------------
The provided module \texttt{entanglement\_mod.F90} allows to compute the $2^{\rm o}$ R\'enyi entropy $S_2$ for a subsystem. Only real-space partitions are currently supported.
The degrees of freedom defining the subsystem are identified by the lattice site, flavor, and color indexes.
Routines with different level of specialization, are provided.

\begin{lstlisting}[style=fortran]
Obs = Calc_Renyi_Ent_gen_all(GRC,List,Nsites)
\end{lstlisting}

This returns an observable \texttt{Obs} such that $\langle\texttt{Obs}\rangle=e^{-S_2}$. \texttt{List(:, N\_FL, N\_SUN)} is a three-dimensional array that contains, for every flavor and color index the list of lattice sites pertaining to the subsystem. \texttt{Nsites(N\_FL, N\_SUN)} is a bidimensional array that provides the number of lattice sites in the subsystem for every flavor and color index.

\begin{lstlisting}[style=fortran]
Obs = Calc_Renyi_Ent_gen_fl(GRC,List,Nsites,N_SUN)
\end{lstlisting}
This computes the observable for the R\'enyi entropy for a subsystem whose degrees of freedom, for a given flavor index, have a common value of color indexes. \texttt{List(:, N\_FL)} is a bidimensional array that contains, for every flavor index, the list of lattice sites of the subsystem. \texttt{Nsites(N\_FL)} contains the number of sites in the subsystem for any given flavor index. \texttt{N\_SUN(N\_FL)} contains the number of color indexes for a given flavor index.

\begin{lstlisting}[style=fortran]
Obs = Calc_Renyi_Ent_indep(GRC,List,Nsites,N_SUN)
\end{lstlisting}
Same as before, for a subsystem whose lattice degrees of freedom are flavor- and color-independent. \texttt{List(:)} is a one-dimensional array containing the lattice sites of the subsystem.
\texttt{N\_SUN} is the number of color indexes belonging to the subsystem.
For every element \texttt{I} of \texttt{List}, the subsystem contains all degrees of freedom with site index \texttt{I}, any flavor index, and \texttt{1} \ldots \texttt{N\_SUN} color index.

\begin{lstlisting}[style=fortran]
Call Calc_Mutual_Inf_gen_all(GRC,List_c,Nsites_c,List_f,Nsites_f,
                             Renyi_c,Renyi_f,Renyi_cf)
\end{lstlisting}
This routine computes the second R\'enyi entropy observables needed to extract the mutual information between a subsystem $c$ and $f$. \texttt{List\_c}, and \texttt{Nsites} are input parameters describing the subsystem $c$, with the same convention as for \texttt{Calc\_Renyi\_Ent\_gen\_all}. \texttt{List\_f} and \texttt{Nsites\_f} are the corresponding input parameters for the subsystem $f$. \texttt{Renyi\_c}, \texttt{Renyi\_f}, and \texttt{Renyi\_cf} are the output variables, containing the R\'enyi observables for the subsystem $c$, $f$, and $c\cup f$. The mutual information between $c$ and $f$ is $I_2=-\ln \langle \texttt{Renyi\_c}\rangle -\ln \langle \texttt{Renyi\_f}\rangle +\ln \langle \texttt{Renyi\_cf}\rangle$.

\begin{lstlisting}[style=fortran,breaklines=true]
Calc_Mutual_Inf_gen_fl(GRC,List_c,Nsites_c,List_f,Nsites_f,N_SUN,Renyi_c,Renyi_f,Renyi_cf)
\end{lstlisting}
This is the analog of \texttt{Calc\_Mutual\_Inf\_gen\_all} for two subsytems with the same assumptions and convention of \texttt{Calc\_Renyi\_Ent\_gen\_fl}.

\begin{lstlisting}[style=fortran,breaklines=true]
Call Calc_Mutual_Inf_indep(GRC,List_c,Nsites_c,List_f,Nsites_f,N_SUN,Renyi_c,Renyi_f,Renyi_cf)
\end{lstlisting}
This is the analog of \texttt{Calc\_Mutual\_Inf\_gen\_all} for two subsytems with the same assumptions and convention of \texttt{Calc\_Renyi\_Ent\_indep}.

For convenience, we also provide the interfaces \texttt{Calc\_Renyi\_Ent} and \texttt{Calc\_Mutual\_Inf} to the specialization listed above, respectively. The argument list provided in the call determines the specialization that is executed.

Finally, the following wrappers can be used to add the R{\'e}nyi entropy and mutual information measurments to the according Observables \texttt{Obs}, with \texttt{N\_SUN} as an optional argument as explained in detail above:
\begin{lstlisting}[style=fortran,breaklines=true]
Call Predefined_Obs_scal_Renyi_Ent(GRC, List, Nsites, N_SUN, ZS, ZP, Obs )
Call Predefined_Obs_scal_Mutual_Inf_indep(GRC, List_c, Nsites_c, List_f, Nsites_f,
                                          N_SUN, ZS, ZP, Obs )
\end{lstlisting}
