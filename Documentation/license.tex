% Copyright (c) 2016, 2017 The ALF project.
% This is a part of the ALF project documentation.
% The ALF project documentation by the ALF contributors is licensed
% under a Creative Commons Attribution-ShareAlike 4.0 International License.
% For the licensing details of the documentation see license.CCBYSA.

% !TEX root = doc.tex

%-------------------------------------------------------------------------------------
\section{Licenses and Copyrights}
%-------------------------------------------------------------------------------------

The  ALF code  is provided as an open source software  such that it is  available  to all and we  hope that  it 
will be useful.  If you benefit from this code  we ask that you acknowledge  the ALF collaboration  as mentioned on our
homepage \url{alf.physik.uni-wuerzburg.de}.   The git repository at   \url{alf.physik.uni-wuerzburg.de} gives us the tools to 
create a small but vibrant community around the code and provides a suitable entry point for future contributors  and future developments. 
The homepage is also the place where the original source files can be found.
With the coming public release it was necessary to add copyright headers to our source files.
The Creative Commons licenses are a good way to share our documentation and it is also well accepted by publishers. Therefore this documentation is licensed to you under a CC-BY-SA license.
This means you can share it and redistribute it as long as you cite the original source and license your changes under the same license. The details are in the file \texttt{license.CCBYSA}, which you should have received with this documentation.
The source code itself is licensed under a GPL license to keep the source as well as any future work in the community.
To express our desire for a proper attribution we decided to make this a visible part of the license.
To that end we have exercised the rights of section 7 of GPL version 3 and have amended the license terms with an additional paragraph that expresses our wish that if an author has benefited from this code
that he/she should consider giving back a citation as specified on \url{alf.physik.uni-wuerzburg.de}.
This is not something that is meant to restrict your freedom of use, but something that we strongly expect to be good scientific conduct.
The original GPL license can be found in the file \texttt{license.GPL} and the additional terms can be found in \texttt{license.additional}.
In favour to our users, the ALF code contains part of the Lapack implementation version 3.6.1 from \url{http://www.netlib.org/lapack}.
Lapack is licensed under the modified BSD license whose full text can be found in \texttt{license.BSD}.\\
With that being said, we hope that the ALF code will prove to you to be a suitable and high-performance tool that enables
you to perform quantum Monte Carlo studies of solid state models of unprecedented complexity.\\
\\
The ALF project's contributors.\\
                        
%-------------------------------------------------------------------------------------
\subsection*{COPYRIGHT}
%-------------------------------------------------------------------------------------

Copyright \textcopyright ~2016-2020, The \textit{ALF} Project.\\
The ALF Project Documentation 
is licensed under a Creative Commons Attribution-ShareAlike 4.0 International License.
You are free to share and benefit from this documentation as long as this license is preserved and proper attribution to the authors is given. For details see the ALF project website \url{alf.physik.uni-wuerzburg.de} and the file \texttt{license.CCBYSA}.
