\documentclass[12pt]{article}
\usepackage{a4}
%%% DFG prefers Arial =>
% substitute Helvetica (called "Arial" by Windows people)
% for the default \sf font
% use with \bfseries instead of \bf and \slshape instead of \it
\usepackage{cmbright}           % makes math fonts cmbright instead of CM.
\usepackage{helvet}
\renewcommand{\familydefault}{\sfdefault}
\usepackage{eurosym}
%%% Copied from the FRG-FG-Vorantrag
\setlength{\textheight}{23.8cm}
\setlength{\textwidth}{15.5cm} %previously 14.5
\setlength{\evensidemargin}{0.25cm}
\setlength{\oddsidemargin}{0.25cm}
\setlength{\topmargin}{0.3cm}
\setlength{\parindent}{0.5cm}
\headheight=0.0cm
\headsep=0.0cm

\hfuzz=5pt
\parindent=0pt
\parskip=5pt
\setcounter{page}{1}

\pagestyle{plain}

\begin{document}

%\subsection*{P2 Weak coupling continuous time QMC and its application to 
%             the electron-phonon problem.} 


\subsection*{\begin{center} Sustainability of research software, declaration of intention \end{center}}
\subsection*{\begin{center} {\huge A}lgorithms for {\huge L}attice  {\huge F}ermions \end{center} } 


{\bf Prof. Dr. Fakher F.\ Assaad} on behalf of the ALF-collaboration \\ 
Institut f\"ur Theoretische Physik und Astrophysik \\
Universit\"at W\"urzburg, 97074 W\"urzburg, Germany 
\\ \\

%\end{center}
% Motivation and present status

{\bf Background: the prototype ALF-05 } \\
Material science is a  complex multi-scale problem such that  complementary  methods have to be used to describe physics  at  different energy scale.  Low energy collective and emergent phenomena  such as superconductivity, topological phases of matter or  magnetism,  are described by  models of interacting fermions on a lattice.   The aim of the ALF-package is to provide a unified framework to tackle a number of fermion models within the so called auxiliary field  quantum Monte Carlo approach (see Ref.~\cite{Assaad08_rev} for a review).  It provides standards   so as to specify  very general models, standards to define the lattice structure, and standards to compute different types  observables.  Being a stochastic approach, it also comes with an error analysis library.  The motivation for  such a project is to secure our knowhow in  a structured, robust and  well documented  program package.  This  allows  new generations of PhD students and Postdoctoral researchers to profit from past algorithmic development and optimization  so as to efficiently  investigate new directions.  It also allows to reproduce published results and to  play with new ideas within minutes.  

Our project is at the prototype level.  Our code is to be found  on our Git server  \texttt{http://alf.physik.uni-wuerzburg.de/} which is presently only accessible  upon invitation since we are still at a testing level.  We have adopted Git since it allows for open communication within the user pool,  tracks issues and  documents them.   Furthermore the history of the code development is stored in the Git such that older versions of the code can be tested against newer ones.   At every commit out git server carries as set of tests so as to automatically scan for errors.   Clearly, such a development framework  has  become a prerequisite for sustainable code development.  ALF-05 has already been used to generate publication quality results. In particular  Ref.~\cite{Assaad16}   is one of the first applications of the package.  

Quantum Monte Carlo methods are  computationally very demanding. One central aspect of the project is to maintain  efficiency for general models.  Our code  heavily relies on BLAS, and comes with a parallel MPI-based implementation. We are in contact with KONWIHR   (Kompetenznetzwerk f\"ur Wissenschaftliches H\"ochstleistungsrechnen in Bayern)  and with the JSC (J\"ulich Supercomputing Centre)  so as to optimize further the code and to bring  it to ever higher programming standards.  

{\bf User pool } \\ 
The ALF project  is unique. To the best of our knowledge  there is no package as general as ours for lattice fermions. The potential pool of users are solid states physicists interested in low  temperature collective phenomena.    Researches in the domain of cold atoms {\it build} models that can be simulated with the ALF-package. We hence foresee that the ALF-package, if easy to use, will find interest in this community. Finally  the high energy lattice gauge  researchers work with very similar algorithms and models such that we can equally  foresee  a substantial overlap with this community.   To date,  our user pools counts researchers at the following institutions:  IOP (Institute of Physics), Beijing,  Beijing Normal University, University of California at San-Diego,  University of Iowa,  RWTH Aachen, University of Erlangen, J\"ulich Supercomputing Centre and of course the  University of W\"urzburg.   As mentioned above,  we are still testing the software. Once we believe that it as user friendly as possible it will be available as an open-source package. 

{\bf Licence} \\ 
We have opted to licence the code under  the terms of the GNU General Public License as published by
the Free Software Foundation, either version 3 of the License, or any later version.   The documentation (which we have included here)  is licensed under a Creative Commons Attribution-ShareAlike 4.0 International License.  This choice of licenses allows for free distribution of the source code. 


 
{\bf Goals } \\
The success of such a project relies on dedicated man-power. One has to continually develop and  maintain the package, provide user support, communicate with the user pool so as to identify  future developments and issues.   Our long term  goals  are  to enhance  our toolbox of algorithms  and to  make sure that the package is user friendly so as to allow  non-specialists  in quantum Monte Carlo methods to  use it and literally play with model systems of correlated fermions.  We will equally  enhance the pool of predefined models, and provide an ever growing set of test cases and reference material. 

{\bf Requested funding } \\
As it stands this project has essentially been carried out in our free time albeit with the help of  Dr. F. Goth who is funded by the SFB1170   in the aim of providing support for program development and optimization. To pursue our effort we will have to create a small team around our program package. We thus would like to apply for two positions, one senior and one junior. In other words: \\ 

One Postdoctoral researcher for 3 years  and \\
One PhD student for 3 years \\

To provide training for the program package, we also plan to organize a workshop. For this we  will need a total of 25 K\euro.  \\

Our Git server will have to be updated  and extended. For this we foresee hardware costs of  25 K\euro. \\


%\bibliographystyle{/Users/fassaad/Dropbox/TeX_Globals/prsty} 
%\bibliographystyle{/Users/fassaad/Dropbox/TeX_Globals/prXsty} 
%%references in Phys. Rev. Style.
%\bibliography{/Users/fassaad/Dropbox/TeX_Globals/fassaad}
\small 
\begin{thebibliography}{1}

\bibitem{Assaad08_rev}
F. Assaad and H. Evertz,  in {\em Computational Many-Particle Physics},
  Vol.~739 of {\em Lecture Notes in Physics}, edited by H. Fehske, R.
  Schneider, and A. Wei{\ss}e (Springer, Berlin Heidelberg, 2008), pp.\
  277--356.

\bibitem{Assaad16}
F.~F. Assaad and T. Grover, Phys. Rev. X {\bf 6},  041049  (2016).

\end{thebibliography}
\normalsize

\end{document}


